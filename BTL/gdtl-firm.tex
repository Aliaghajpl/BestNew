%%%%%%%%%%%%%%%%%%%%%%%%%%%%%%%%%%%%%%%%%%%%%%%
%2345678901234567890123456789012345678901234567890123456789012345678901234567890
%        1         2         3         4         5         6         7         8

\documentclass[letterpaper, 10 pt, conference]{ieeeconf}  % Comment this line out if you need a4paper

%\documentclass[a4paper, 10pt, conference]{ieeeconf}      % Use this line for a4 paper

\IEEEoverridecommandlockouts                              % This command is only needed if 
                                                          % you want to use the \thanks command

\overrideIEEEmargins                                      % Needed to meet printer requirements.
%\documentclass[conference]{IEEEtran}

\let\proof\relax
\let\endproof\relax
%% Additional packages
\usepackage{times}
\usepackage{amsmath}
\usepackage{amsthm}
\usepackage{amssymb}
\usepackage{amsfonts}
\usepackage{mathtools}
%\usepackage{calc}
\usepackage{subfigure}
\usepackage{graphicx}
\usepackage{color}
\usepackage{url}
%\usepackage{lineno}
%\usepackage{ulem} % for underlining and strike through
%\normalem % reset emph to normal
%\usepackage{setspace} % for line spacing - e.g. 1.5, 2
\usepackage[usenames,x11names]{xcolor}
%\usepackage{xspace} % for inserting a space in TeX commands if needed
%\usepackage{caption}
\usepackage[bookmarks=true]{hyperref}
%\renewcommand{\thesubfigure}{\relax} % no subfig counters
%\let\chapter\section % algorithm2e natbib compatibility
%\usepackage{accents}
%\usepackage[titletoc,toc,title]{appendix}
%\usepackage{longtable}
%\usepackage{setspace}
\usepackage{multicol}

%% Packages
%\usepackage[margin=1in]{geometry}
%\usepackage[abs]{overpic}
\usepackage[linesnumbered,vlined,ruled]{algorithm2e}
%\usepackage{multirow} % for cell tables spanning multiple rows
\usepackage{tikz,pgf}%,tikz-3dplot}
\usetikzlibrary{arrows,automata,shapes,calc,backgrounds,spy,positioning}
%\usetikzlibrary{external}
%\tikzexternalize
%\usepackage[numbers]{natbib}
%\usepackage[sort,round]{natbib}
%\usepackage{lipsum}
\usepackage{epstopdf}

\theoremstyle{plain}
\newtheorem{theorem}{Theorem}
\newtheorem{cor}{Corollary}
\newtheorem{prop}{Proposition}
\newtheorem{lemma}{Lemma}

\theoremstyle{definition}
%\newtheorem{definition}{Definition}
\newtheorem{remark}{Remark}
\newtheorem{cond}{Condition}

\newtheorem{example}{Example}
\newtheorem{problem}{Problem}
%\newtheorem{assumption}{Assumption}


%%% figure path
\graphicspath{{figures/}}

%% Remove footnote mark
%\renewcommand{\footnotemark}{}

%%% Redefine qed symbol
%\renewcommand{\qedsymbol}{$\blacksquare$}

%% Projection symbol
%\newcommand{\project}[1]{\! \upharpoonright_{#1}}


%% Theorems 
%\newtheorem{theorem}{Theorem}[section]
%\newtheorem{proposition}[theorem]{Proposition}
%\newtheorem{corollary}[theorem]{Corollary}
\newtheorem{definition}[theorem]{Definition}
%\newtheorem{lemma}[theorem]{Lemma}
%\newtheorem{remark}[theorem]{Remark}
%\newtheorem{remarks}[theorem]{Remarks}
%\newtheorem{example}[theorem]{Example}
%\newtheorem{algo}[theorem]{Algorithm}
%\newtheorem{problem}[theorem]{Problem}
%\newtheorem{Procedure}[theorem]{Procedure}
%\newcommand{\exampler}[2]{\medskip \hskip -\parindent {\bf Example #1 Revisited.~}{\it #2}\medskip}

%% Percent
%\newcommand\oprocendsymbol{\hbox{$\square$}}
%\newcommand\oprocend{\relax\ifmmode\else\unskip\hfill\fi\oprocendsymbol}
%\def\eqoprocend{\tag*{$\bullet$}}

%% Enumerate environment
%\renewcommand{\labelenumi}{(\roman{enumi})}
%\renewcommand{\labelenumii}{(\alph{enumii})}

%% Breakable comma
%\mathchardef\breakingcomma\mathcode`\,
%{\catcode`,=\active
%  \gdef,{\breakingcomma\discretionary{}{}{}}
%}
%\newcommand{\breqn}[1]{\mathcode`\,=\string"8000 #1}

%% Other Stuff
%\newcommand{\margin}[1]{\marginpar{\tiny\color{blue} #1}}
%%\addtolength{\marginparwidth}{-0.3in}
\newcommand{\todo}[1]{\vskip 0.05in \colorbox{yellow}{$\Box$ \ttfamily\bfseries\small#1}\vskip 0.05in}
%\newcommand{\todo}[1]{}
%\newcommand{\vers}{\operatorname{vers}}

%% Roman, calligraphic, boldface, double barred letters
\newcommand{\RM}[1]{\mathrm{#1}}
\newcommand{\CA}[1]{\mathcal{#1}}
\newcommand{\BF}[1]{\mathbf{#1}}
\newcommand{\IT}[1]{\mathit{#1}}
\newcommand{\BB}[1]{\mathbb{#1}}
\newcommand{\TT}[1]{\mathtt{#1}}
\newcommand{\FK}[1]{\mathfrak{#1}}
\newcommand{\BS}[1]{\boldsymbol{#1}}

%%% Temporal logic symbols
\newcommand{\notltl}{\neg}
\newcommand{\andltl}{\wedge}
\newcommand{\orltl}{\vee}
\newcommand{\Next}{\ensuremath{\bigcirc}}
\newcommand{\Always}{\ensuremath{\ \square\ }}
\newcommand{\Event}{\ensuremath{\ \diamondsuit\ }}
\newcommand{\Until}{\ \CA{U}\ }
\newcommand{\Implies}{\Rightarrow}
\newcommand{\Equiv}{\Leftrightarrow}
%\newcommand{\Not}{\lnot}
\newcommand{\True}{\top}
\newcommand{\False}{\perp}
%\def\prop{\TT{data}}
%\def\popt{\pi}
\newcommand{\AP}{AP}
\newcommand{\pred}{\xi}

\newcommand{\Real}{\BB{R}}

%% Abbreviations
%\def\eg{e.g.\xspace}
%\def\Eg{E.g.\xspace}
%\def\ie{i.e.\xspace}
%\def\Ie{I.e.\xspace}
%\def\etc{etc.\xspace}
%\def\vs{vs.\xspace}
%\def\wrt{w.r.t.\xspace}
%\def\etal{et al.\xspace}

%% Exotic words
\newcommand{\buchi}{B\"uchi\ }

%% Symbols of automata
\newcommand{\PA}{\mathcal{P}}
\newcommand{\BA}{\mathcal{B}}
%\newcommand{\FA}{\mathcal{F}}

\newcommand{\TS}{\mathcal{F}}
\newcommand{\LA}{\mathcal{L}}
\newcommand{\KA}{\mathcal{K}}
\newcommand{\MDP}{\mathcal{M}}
\newcommand{\RA}{\mathcal{R}}
\newcommand{\FSA}{\mathcal{A}}

\newcommand{\TSX}{\BB{V}_\TS}
\newcommand{\TSE}{\BB{E}_\TS}
\newcommand{\TSEE}{\BB{E}}

\newcommand{\DTL}{DTL\xspace}


%% Short macros for arrows
\newcommand{\la}{\leftarrow}
\newcommand{\ra}{\rightarrow}
\newcommand{\ras}[1]{\stackrel{#1}{\rightarrow}}
\newcommand{\asgn}{\la}
\newcommand{\proj}[2]{{#1}{\downharpoonright_{#2}}}

\newcommand{\df}{\xspace\RM{d}}

%% Names of the Algorithms
%\newcommand{\optrun}{\textsc{Optimal-Run}\ }
%\newcommand{\exactmultioptrun}{\textsc{Exact-Multi-Robot-Optimal-Run}\ }
%\newcommand{\constR}{\textsc{Construct-Region-Automaton}\ }
%\newcommand{\constT}{\textsc{Construct-Team-TS}\ }
%\newcommand{\syncT}{\textsc{Sync-Team-TS}\ }
%\newcommand{\boundOpt}{\textsc{Bound-Optimality}\ }

% Custom operators
%\DeclareMathOperator*{\argmin}{arg\,min}
\newcommand{\norm}[1]{\left\| {#1} \right\|}
\newcommand{\norminf}[1]{\left\| {#1} \right\|_{\infty}}
\newcommand{\normeucl}[1]{\left\| {#1} \right\|_{2}}
\newcommand{\abs}[1]{\left| {#1} \right|}
\newcommand{\card}[1]{\left| {#1} \right|}
\newcommand{\spow}[1]{2^{#1}}
%\newcommand{\interior}[1]{\accentset{\smash{\raisebox{-0.12ex}{$\scriptstyle\circ$}}}{#1}\rule{0pt}{2.3ex}}
\newcommand{\interior}[1]{\mathring{#1}}
\DeclareMathOperator{\diag}{diag}
\newcommand{\lift}{\upharpoonright}
%
%% Display a grid to help align images
%\beamertemplategridbackground[1cm]
%\usepackage[style=numeric-comp]{biblatex}
%\usepackage{cite}


\tikzset{/tikz/arrows = ->, /tikz/> = triangle 45, /tikz/shorten > = 0.5pt}
\tikzset{every loop/.style={min distance=2mm,out=110,in=70,looseness=3}}
\tikzset{every state/.style={circle,thick,draw=colorBlue,minimum size=7mm}}
\tikzset{/tikz/initial text=,/tikz/initial distance=0.2}
\tikzset{/tikz/node distance=2cm}
\tikzset{/tikz/auto=left}
\tikzset{every picture/.style={scale=3}}

% Draw the helper grid in 3-d
\newcommand{\largegrid}[3]
{
    \foreach \x in {0,#1}
        \draw[#3] (\x,0) -- (\x,#2);
    \foreach \y in {0,#2}
        \draw[#3] (0,\y) -- (#1,\y);
}

% Draw the coordinate axes
\newcommand{\coordaxes}
{
    \begin{scope}
    \tikzset{axis/.style={draw,thick,->,>=triangle 45,scale=2}}
    \path [axis] (0,0) -> (1, 0) node [below] {x};
    \path [axis] (0,0) -> (0, 1) node [above left] {y};
    \end{scope}
}

% Draw a robot at position
\newcommand{\robot}[2]
{
    % Draw the quad
    \tikzset{robot/.style={very thick, circle,fill=magenta}}
    \node [robot](pos) at (#1, #2) {};
}

% Draw cameras
\newcommand{\cameras}{
    % Draw cameras
    \tikzset{camera/.style={very thick, circle,fill=red}}
    \node[camera] at (0, 0) {};
    \node at (0.3, 0.4) {$C_1$};
    \node[camera] at (8, 0) {};
    \node at (7.7, 0.4) {$C_2$};
    \node[camera] at (8, 5.1) {};
    \node at (7.7, 4.7) {$C_3$};
    \node[camera] at (0, 5.1) {};
    \node at (0.3, 4.7) {$C_4$};
    % Draw its FoV
    \tikzset{fov/.style = {very thick, cyan, opacity=.2}};
    \tikzset{fov line/.style = {very thick, cyan, opacity=.5}};
    \fill [fov] (0, 0) -- ++(4.5,1.7) -- ++(-1,0) -- ++(0,1.7) -- ++(1.454,1.7) -- ++(-4.954,0);
    \draw [fov line] (0, 0) -- ++(4.5,1.7) -- ++(-1,0) -- ++(0,1.7) -- ++(1.454,1.7) -- ++(-4.954,0) -- ++(0, -5.1);
    \fill [fov] (8, 0) -- ++(-4.5,1.7) -- ++(1,0) -- ++(0,1.7) -- ++(-1.454,1.7) -- ++(4.954,0);
    \draw [fov line] (8, 0) -- ++(-4.5,1.7) -- ++(1,0) -- ++(0,1.7) -- ++(-1.454,1.7) -- ++(4.954,0) -- ++(0, -5.1);
    \fill [fov] (8, 5.1) -- ++(-4.5,-1.7) -- ++(1,0) -- ++(0,-1.7) -- ++(-1.454,-1.7) -- ++(4.954,0);
    \draw [fov line] (8, 5.1) -- ++(-4.5,-1.7) -- ++(1,0) -- ++(0,-1.7) -- ++(-1.454,-1.7) -- ++(4.954,0) -- ++(0, 5.1);
    \fill [fov] (0, 5.1) -- ++(4.5,-1.7) -- ++(-1,0) -- ++(0,-1.7) -- ++(1.454,-1.7) -- ++(-4.954,0);
    \draw [fov line] (0, 5.1) -- ++(4.5,-1.7) -- ++(-1,0) -- ++(0,-1.7) -- ++(1.454,-1.7) -- ++(-4.954,0) -- ++(0, 5.1);
}

% Draw camera
\newcommand{\camera}{
    % Draw cameras
    \tikzset{camera/.style={very thick, circle,fill=red}}
    \node[camera] at (0, 0) {};
    \node at (0.3, 0.4) {$C_1$};
    % Draw its FoV
    \tikzset{fov/.style = {very thick, cyan, opacity=.2}};
    \tikzset{fov line/.style = {very thick, cyan, opacity=.5}};
    \fill [fov] (0, 0) -- ++(4.5,1.7) -- ++(-1,0) -- ++(0,1.7) -- ++(1.454,1.7) -- ++(-4.954,0);
    \draw [fov line] (0, 0) -- ++(4.5,1.7) -- ++(-1,0) -- ++(0,1.7) -- ++(1.454,1.7) -- ++(-4.954,0);
}

\newcommand{\regions}
{
    % Region A
    \tikzset{region A/.style = {thick, green, opacity=.8}};
    \fill [region A] (0,0) -- (2,0) -- (2,2) -- (0,2);
    \node at (1,1) {\Large $A$};
    \draw [black, thick] (0,0) -- (2,0) -- (2,2) -- (0,2) -- (0, 0);
    % Region B
    \tikzset{region B/.style = {thick, blue, opacity=.8}};
    \fill [region B] (6,3.1) -- (8,3.1) -- (8,5.1) -- (6,5.1);
    \node at (7,4.05) {\Large $B$};
    \draw [black, thick] (6,3.1) -- (8,3.1) -- (8,5.1) -- (6,5.1) -- (6,3.1);
    % Region D1
    \tikzset{region D1/.style = {thick, orange, opacity=.8}};
    \fill [region D1] (3.7,0) -- (4.3,0) -- (4.3,1.7) -- (3.7,1.7);
    \node at (4,0.85) {\Large $D_1$};
    \draw [black, thick] (3.7,0) -- (4.3,0) -- (4.3,1.7) -- (3.7,1.7) -- (3.7,0);
    % Region D2
    \tikzset{region D2/.style = {thick, pink, opacity=.8}};
    \fill [region D2] (3.7,3.4) -- (4.3,3.4) -- (4.3,5.1) -- (3.7,5.1);
    \node at (4,4.25) {\Large $D_2$};
    \draw [black, thick] (3.7,3.4) -- (4.3,3.4) -- (4.3,5.1) -- (3.7,5.1) -- (3.7,3.4);
    % Region O
    \tikzset{region O/.style = {thick, gray, opacity=.8}};
    \fill [region O] (3.5,1.7) -- (4.5,1.7) -- (4.5,3.4) -- (3.5,3.4);
    \node at (4,2.55) {\Large $Obs$};
    \draw [black, thick] (3.5,1.7) -- (4.5,1.7) -- (4.5,3.4) -- (3.5,3.4) -- (3.5,1.7);
}

\newcommand{\PRM}[2]{
    \tikzset{prm/.style = {thin, circle, fill=black, inner sep=0pt,minimum size=5pt}};
    \node [prm](s0) at (#1, #2) {};
    \node [prm](s1) at (2, 3.5) {};
    \node [prm](s2) at (1.8, 1.7) {};
    \node [prm](s3) at (3, 2.6) {};
    \node [prm](s4) at (2.75, 1) {};
    \node [prm](s5) at (2.6, 4.3) {};
    \node [prm](s6) at (4.1, 4.7) {};
    \node [prm](s7) at (5.2, 3.9) {};
    \node [prm](s8) at (6.7, 4) {};
    \node [prm](s9) at (5.9, 2.8) {};
    \node [prm](s10) at (4.9, 2.2) {};
    \node [prm](s11) at (6.5, 1.8) {};
    \node [prm](s12) at (5.3, 1.3) {};
    \node [prm](s13) at (4.8, 0.5) {};
    \node [prm](s14) at (4, 0.6) {};

    \tikzset{trans/.style = {thin, opacity=.8}};
    \draw[->] [trans] (s0) -- (s1);
    \draw[<->] [trans] (s1) -- (s2);
    \draw[<->] [trans] (s1) -- (s3);
    \draw[<->] [trans] (s1) -- (s5);
    \draw[<->] [trans] (s2) -- (s3);
    \draw[<->] [trans] (s2) -- (s4);
    \draw[<->] [trans] (s3) -- (s4);
    \draw[<-] [trans] (s5) -- (s6);
    \draw[<-] [trans] (s6) -- (s7);
    \draw[<->] [trans] (s7) -- (s8);
    \draw[<->] [trans] (s7) -- (s9);
    \draw[<->] [trans] (s8) -- (s9);
    \draw[<->] [trans] (s9) -- (s10);
    \draw[<->] [trans] (s9) -- (s11);
    \draw[<->] [trans] (s9) -- (s12);
    \draw[<->] [trans] (s10) -- (s11);
    \draw[<->] [trans] (s10) -- (s12);
    \draw[<->] [trans] (s11) -- (s12);
    \draw[<->] [trans] (s13) -- (s12);
    \draw[<-] [trans] (s13) -- (s14);
    \draw[<-] [trans] (s14) -- (s4);
}

\newcommand{\arena}[2]
{
    \tikzset{/tikz/arrows = -}
    \tikzset{every picture/.style={scale=1}}
    \begin{tikzpicture}%[spy using outlines={rectangle, magnification=1.75, size=9cm, connect spies}]
    \begin{scope}
    % Draw the regions
    \regions;
    % Coordinate axes
    \coordaxes;
    % Draw the helper grid
    \largegrid{8}{5.1}{black!50};
    % Draw the robot
    \robot{#1}{#2};
    \end{scope}
    \end{tikzpicture}
}

\newcommand{\arenacam}[2]
{
    \tikzset{/tikz/arrows = -}
    \tikzset{every picture/.style={scale=1}}
    \begin{tikzpicture}%[spy using outlines={rectangle, magnification=1.75, size=9cm, connect spies}]
    \begin{scope}
    % Draw the regions
    \regions;
    % Coordinate axes
    \coordaxes;
    % Draw the helper grid
    \largegrid{8}{5.1}{black!50};
    % Draw the robot
    \robot{#1}{#2};
    % Draw camera
    \camera;
    \end{scope}
    \end{tikzpicture}
}

\newcommand{\arenacams}[2]
{
    \tikzset{/tikz/arrows = -}
    \tikzset{every picture/.style={scale=1}}
    \begin{tikzpicture}%[spy using outlines={rectangle, magnification=1.75, size=9cm, connect spies}]
    \begin{scope}
    % Draw the regions
    \regions;
    % Coordinate axes
    \coordaxes;
    % Draw the helper grid
    \largegrid{8}{5.1}{black!50};
    % Draw the robot
    \robot{#1}{#2};
    % Draw cameras
    \cameras;
    \end{scope}
    \end{tikzpicture}
}

\newcommand{\arenaprm}[2]
{
    \tikzset{/tikz/arrows = -}
    \tikzset{every picture/.style={scale=1}}
    \begin{tikzpicture}%[spy using outlines={rectangle, magnification=1.75, size=9cm, connect spies}]
    \begin{scope}
    % Draw the regions
    \regions;
    % Coordinate axes
    \coordaxes;
    % Draw the helper grid
    \largegrid{8}{5.1}{black!50};
    % Draw the robot
    \robot{#1}{#2};
    % Draw PRM
    \PRM{#1}{#2};
    \end{scope}
    \end{tikzpicture}
}


% \pdfinfo{
%    /Author (Homer Simpson)
%    /Title  (Robots: Our new overlords)
%    /CreationDate (D:20101201120000)
%    /Subject (Robots)
%    /Keywords (Robots;Overlords)
% }

% paper title
\title{\LARGE \bf
Control in Belief Space with Temporal Logic Specifications
}

\author{Cristian-Ioan Vasile$^{1}$, Kevin Leahy$^{2}$, Eric Cristofalo$^{2}$, Austin Jones$^{3}$, Mac Schwager$^{4}$ and Calin Belta$^{2}$
\thanks{*This work was supported by NSF NRI-1426907, NSF CMMI-1400167.}% <-this % stops a space
\thanks{$^{1}$Cristian-Ioan Vasile is with the Division of Systems Engineering,
    Boston University, 15 Saint Mary's Street, Brookline, MA 02446, USA
    {\tt\small cvasile@bu.edu}
}%
\thanks{$^{2}$Kevin Leahy, Eric Cristofalo and Calin Belta are with the
    Department of Mechanical Engineering, Boston University,
    110 Cummington Mall, Boston, MA 02215, USA
    {\tt\small \{kjleahy, emc73, cbelta\}@bu.edu}
}%
\thanks{$^{3}$Austin Jones is with Mechanical Engineering and
    Electrical Engineering at Georgia Institute of Technology, North Ave NW, Atlanta, GA 30332, USA 
    {\tt\small austinjones@gatech.edu}}
%
\thanks{$^{4}$Mac Schwager is with the
    Department of Aeronautics and Astronautics, Stanford University,
    Durand Building, 496 Lomita Mall,  Stanford, CA 94305, USA
    {\tt\small schwager@stanford.edu}
}%
}

\begin{document}

\maketitle
\thispagestyle{empty}
\pagestyle{empty}


%\begin{abstract}
%One of the greatest challenges of motion planning in robotics 
%is dealing with uncertainty in the robot's location and motion.
%In this paper, we present a sampling-based planner for
%persistent missions with temporal and uncertainty constraints.
%We introduce a specification language called
%{\em Gaussian Distribution Temporal Logic (GDTL)},
%an extension of Boolean logic that allows us to incorporate
%temporal evolution and noise mitigation directly into the task
%specifications,
%e.g. "Go to region A and reduce the variance of your position
%estimate below 0.1 $m^2$."
%Our motion planning algorithm
%%adapts the computationally efficient feedback information roadmap planner to
%generates a transition system in the belief space.
%The planner breaks the {\em curse of history} associated
%with belief space planning using local feedback controllers,
%which makes automata-based methods tractable.
%Control policies are then computed using a product
%Markov Decision Process (MDP) between
%the transition system and the Rabin automaton encoding 
%the task specification.
%We present algorithms to translate a GDTL formula to a Rabin
%automaton and to efficiently construct the product MDP
%by leveraging recent results from incremental computing.
%Our approach is evaluated in simulation and hardware
%experiments using a camera network and ground robot.
%\end{abstract}

\begin{abstract}
In this paper, we present a sampling-based algorithm
to synthesize control policies with temporal and
uncertainty constraints.
We introduce a specification language called
{\em Gaussian Distribution Temporal Logic (GDTL)},
an extension of Boolean logic that allows us to incorporate
temporal evolution and noise mitigation directly into the task
specifications,
e.g. ``Go to region A and reduce the variance of your state
estimate below 0.1 $m^2$."
Our algorithm generates a transition system in the belief space
and uses local feedback controllers to break the {\em curse of history}
associated with belief space planning.
%Another benefit is that automata-based methods become tractable.
Furthermore, conventional automata-based methods become tractable.
Switching control policies are then computed using a product
Markov Decision Process (MDP) between
the transition system and the Rabin automaton encoding 
the task specification.
We present algorithms to translate a GDTL formula to a Rabin
automaton and to efficiently construct the product MDP
by leveraging recent results from incremental computing.
Our approach is evaluated in hardware experiments using
a camera network and ground robot.
\end{abstract}

%\IEEEpeerreviewmaketitle


\input{Introduction}
%\input{Preliminaries}
\section{Gaussian Distribution Temporal Logic}
\label{sec:gdtl}
In this section, we define Gaussian Distribution Temporal Logic (GDTL),
a predicate temporal logic defined over the space of Gaussian
distributions with fixed dimension.

\smallskip
\noindent {\bf Notation:}
Let $\Sigma$ be a finite set. The cardinality,
power set, Kleene- and $\omega$-closures
of $\Sigma$ are denoted by $\card{\Sigma}$,
$\spow{\Sigma}$, $\Sigma^*$ and $\Sigma^\omega$,
respectively.
$A \subseteq \BB{R}^n$ and $B \subseteq \BB{R}^m$,
$n, m \geq 0$, we denote by $\CA{M}(A, B)$ the set of
functions with domain $A$ and co-domain $B$, where $A$ has positive measure with
respect to the Lebesgue measure of $\BB{R}^n$.
The set of all positive semi-definite matrices of size
$n \times n$, $n \geq 1$, is denoted by $S^n$.
$\BB{E}[\cdot]$ is the expectation operator.
The $m \times n$ zero matrix and
the $n \times n$ identity matrix are denoted by
$\BF{0}_{m, n}$ and $\BF{I}_n$, respectively.
The supremum and Euclidean norms are denoted by
$\norminf{\cdot}$ and $\normeucl{\cdot}$, respectively.

Let $\CA{G}$ denote the Gaussian belief space
of dimension $n$, i.e. the space of Gaussian
probability measures over $\BB{R}^n$.
For brevity, we identify the Gaussian measures
with their finite parametrization, mean and
covariance matrix. Thus,
$\CA{G} =  \BB{R}^n \times  S^n$.
If $\BF{b} = b^0b^1 \ldots \in \CA{G}^{\omega}$,
we denote the suffix sequence $b^i b^{i+1} \ldots$ by
$\BF{b}^i$, $i \geq 0$.

\begin{definition}[GDTL Syntax]
\label{def:gdtl-syntax}
The {\em syntax} of Gaussian Distribution
Temporal Logic is defined as
\begin{equation*}
 \phi :=  \True \ |\ f \leq 0 \ |\ \notltl \phi \ |\ \phi_1 \andltl \phi_2 \ |\ \phi_1 \LTLUNTIL \phi_2,% \ |\ \LTLNEXT \phi, 
\end{equation*}
where $\True$ is the Boolean constant ``True'',
$f \leq 0$ is a predicate over $\CA{G}$, where
$f \in \CA{M}(\CA{G}, \BB{R})$,
$\notltl$ is negation (``Not''), $\andltl$ is conjunction (``And''),
%$\LTLNEXT$ is ``Next'',
and $\LTLUNTIL$ is ``Until''.
\end{definition}

For convenience, we define the additional operators:
$\phi_1 \orltl \phi_2 \equiv  \notltl (\notltl \phi_1 \andltl \notltl \phi_2)$,
$\LTLEVENTUALLY \phi \equiv \True \LTLUNTIL \phi$, and
$\LTLALWAYS \phi \equiv \notltl \LTLEVENTUALLY \notltl \phi$,
%\begin{align*}
%\phi_1 \orltl \phi_2 & \equiv  \notltl (\notltl \phi_1 \andltl \notltl \phi_2) \\
%\LTLEVENTUALLY \phi & \equiv \True \LTLUNTIL \phi \\
%\LTLALWAYS \phi & \equiv \notltl \LTLEVENTUALLY \notltl \phi
%\end{align*}
where $\equiv$ denotes semantic equivalence.

\begin{definition}[GDTL Semantics]  
\label{def:gdtl-semantics}
Let $\BF{b} = b^0b^1 \ldots \in \mathcal{G}^{\omega}$
be an infinite sequence of belief states.
The {\em semantics} of GDTL is defined recursively as
\begin{align*}
&\BF{b}^i \models  \top  & \\
&\BF{b}^i \models f \leq 0 & \Equiv\quad & f(b^i) \leq 0\\ % \forall (x,P) \in b^i \\
&\BF{b}^i \models \notltl \phi & \Equiv\quad & \notltl (\BF{b}^i \models \phi) \\
&\BF{b}^i \models \phi_1 \andltl  \phi_2  & \Equiv\quad & ( \BF{b}^i \models \phi_1 ) \andltl ( \BF{b}^i \models \phi_2 ) \\
&\BF{b}^i \models \phi_1 \orltl  \phi_2  & \Equiv\quad & ( \BF{b}^i \models \phi_1 ) \orltl ( \BF{b}^i \models \phi_2 ) \\
&\BF{b}^i \models  \phi_1 \LTLUNTIL \phi_2 & \Equiv\quad & \exists j \geq i \text{ s.t. } ( \BF{b}^j \models \phi_2 ) \\
& & & \andltl (\BF{b}^k \models \phi_1, \forall k \in \{i, \ldots j-1\})\\
&\BF{b}^i \models \LTLEVENTUALLY \phi  & \Equiv\quad & \exists j \geq i \text{ s.t. } \BF{b}^j \models \phi \\
&\BF{b}^i \models \LTLALWAYS \phi  & \Equiv\quad & \forall j \geq i \text{ s.t. } \BF{b}^j \models \phi
\end{align*}

The word $\BF{b}$ satisfies $\phi$, denoted $\BF{b} \models \phi$,
if and only if $\BF{b}^0 \models \phi$.
\end{definition}

By allowing the definition of the atomic predicates used in GDTL to be quite general,
we can potentially enforce interesting and relevant properties on the evolution of a system
through belief space.  Some of these properties include

\begin{itemize}
 \item Bounds on determinant of covariance matrix $det(P)$.  This is used when 
 we want to bound the overall uncertainty about the system's state.
 \item Bounds on trace of covariance matrix $Tr(P)$.  This is used when we
 want to bound the uncertainty about the system's state in any direction.
 %\item Bound on projection of covariance matrix $\Pi P$.  This is used when
 %we want to bound the uncertainty about the robot's state in a specific direction
 \item Bounds on state mean $\hat{x}$.  This is used when we want to specify
 where in state space the system should be.
% \item Bounds on Mahalanobis distance $\mathcal{M}(\hat{x},P,x) = (\hat{x}-x)^TP^{-1}(\hat{x}-x)$.
% The Mahalanobis distance describes the distance from a point to a Gaussian distribution.  
% It is used when we want to specify a desired state (or region) in the state space and describe how
% certain we are that the agent achieves it.
\end{itemize}

\begin{example}
\label{ex:running}
Let $R$ be a system evolving along a straight line
with state denoted by $x \in \BB{R}$.
The belief space for this particular robot is thus
$(\hat{x}, P) \in \BB{R} \times [0, \infty)$,
where $\hat{x}$ and $P$ are its state estimate
and covariance obtained from its sensors.  
The system is tasked with going back and forth
between two goal regions (denoted as $\pi_{g,1}$
and $\pi_{g,2}$ in the top of Fig.~\ref{fig:OneDExample}).  
It also must ensure that it never overshoots the goal
regions or lands in obstacle regions $\pi_{o,1}$
and $\pi_{o,2}$.
The system must also %ensure that it maintains its
maintain a
covariance $P$ of less than 0.5 m$^2$ at all times
and less than 0.3 m$^2$ when in one of the goal regions.
These requirements can be described by the GDTL formula
\begin{equation}
\label{eq:onedformula}
\begin{array}{lcl}
 \phi_{1d} &=& \phi_{avoid}  \wedge \phi_{reach}  \wedge \phi_{u,1} \wedge \phi_{u,2}\:\mbox{, where}   \\
 \phi_{avoid}  &=& \LTLALWAYS \neg ((box(\hat{x},-4,0.35)\leq 1) \\
 & &\vee (box(\hat{x},4,0.35)\leq 1) ) \\
 \phi_{reach} &=& \LTLALWAYS \LTLEVENTUALLY (box(\hat{x},-2,0.35)\leq 1)  \\ 
 & & \wedge \LTLALWAYS \LTLEVENTUALLY (box(\hat{x},2,0.35)\leq 1) \\
 \phi_{u,1} &=& \LTLALWAYS ( P < 0.5) \\
 \phi_{u,2} &=&  \LTLALWAYS ( (box(\hat{x},-2,0.35)\leq 1)  \\ & & 
 \wedge (box(\hat{x},2,0.35)\leq 1)) \Rightarrow (P < 0.3)\:,
\end{array}
 \end{equation}
where $box\left(\hat{x},x_c,a\right)=\norminf{a^T\left(\hat{x}-x_c\right)}$
is a function bounding $\hat{x}$ inside an interval of size $2\abs{a}$ centered at $x_c$.
Subformula $\phi_{avoid}$ encodes keeping the system away from the obstacle regions.
Subformula $\phi_{reach}$ encodes periodically visiting the goal regions.  
Subformula $\phi_{u,1}$ encodes maintaining the uncertainty below 0.5 m$^2$ 
globally and subformula $\phi_{u,2}$ encodes maintaining the uncertainty below
0.3 m$^2$ in the goal regions.

The belief space associated with this problem is shown in the bottom of Fig.~\ref{fig:OneDExample}.
The curves in the figure correspond to the borders between the satisfaction and
violation of predicates in~\eqref{eq:onedformula}, e.g. the level sets that
are induced by the predicates when inequalities are replaced with equality. In
the figure, + denotes that the predicate is satisfied in that region and - indicates that it is not.
An example trajectory that satisfies~\eqref{eq:onedformula} is shown in black. Note
that every point in this belief trajectory has covariance $P$ less than 0.5, which
satisfies $\phi_{u,1}$. Further, the forbidden regions in $\phi_{avoid}$ (marked with red stripes)
are always avoided while each of the goal regions in $\phi_{reach}$ 
(marked with green stars) are each visited. Further, whenever the belief 
is in a goal region, it has covariance $P$ less than 0.3, which means $\phi_{u,2}$
is satisfied.
\endproof
\end{example}

%\begin{figure}
% \begin{center}
%  \includegraphics[width=0.7\columnwidth]{OneDStateSpace}
%  \caption{The state space of a robot moving along one dimension.}
%  \label{fig:OneDExample}
% \end{center}
%\end{figure}
%
%\begin{figure}
% \begin{center}
% \includegraphics[width=\columnwidth]{newFig}
% %\includegraphics[width=\columnwidth]{OneDBelief}
% \caption{The predicates from~\eqref{eq:onedformula} as functions of the belief of the robot from Example!\ref{ex:running}.}
% %The belief space  of  the robot in Example~\ref{ex:running}. 
%% The curves correspond to the predicates in~\eqref{eq:onedformula}.}
% \label{fig:GDTLExample}
% \end{center}
%\end{figure}

\begin{figure}[!htb]
\centering
%\subfigure[]{
%  \includegraphics[width=.6\columnwidth]{OneDStateSpace_new}
%  \label{fig:OneDExample}
%}
%\subfigure[]{
%  \includegraphics[width=.5\columnwidth]{OneDBelief_new}
%  \label{fig:GDTLExample}
%}
\includegraphics[width=.9\columnwidth]{OneDBeliefSpace}
%\caption{The state space of a system evolving along one dimension is shown in Fig.~\subref{fig:OneDExample} and the predicates from~\eqref{eq:onedformula} as functions of the belief of the system from Ex.~\ref{ex:running} are shown in Fig.~\subref{fig:GDTLExample}.}
\caption{(Top) The state space of a system evolving along one dimension and (Bottom) the predicates from~\eqref{eq:onedformula} as functions of the belief of the system from Ex.~\ref{ex:running}.}
\label{fig:OneDExample}
\end{figure}

\section{Problem Formulation}\label{sec:prob}
In this section, we define the problem of controlling a system to satisfy a given
GDTL formula with maximum probability.  
%This
%corresponds to steering a robot to satisfy a rich, temporally layered navigation
%task that explicitly depends on its localization uncertainty.  

\subsection{Motion and sensing models}
\label{sec:motion}
We assume the system has noisy linear time invariant (LTI) dynamics given by
\begin{equation}
\label{eq:LTI}
\begin{aligned}
x_{k+1} &= A x_k +B u_k + w_k,\\
\end{aligned}
\end{equation}
%
%{\color{blue}[We need to update this notation!] where $x \in \CA{M}( \BB{Z}_{\geq 0}, \CA{X})$ is
%a trajectory of the system},
where $x_k\in\CA{X}$ is the state of the system,
$\CA{X} \subseteq \BB{R}^n$ is the state space,
$A \in \BB{R}^{n \times n}$ is the dynamics matrix,
$B \in \BB{R}^{n \times p}$ is the control matrix,
$u_k \in \CA{U}$ is a control signal,
$\CA{U} \subseteq \BB{R}^p$ is the control space,
and $w_k$ is a zero-mean Gaussian process with covariance
$Q \in \BB{R}^{n \times n}$.
The state is observed indirectly 
%$y_k \in \CA{Y}$ 
according to the
linear observation model
\begin{equation}
\label{eq:linearObservations}
 y_k = C x_k + v_k,
\end{equation}
%
where $y_k \in \CA{Y}$ is a measurement, $\CA{Y} \subseteq \BB{R}^m$ is the observation space,
$C \in \BB{R}^{m \times n}$ is the observation matrix
and $v_k$ is a zero-mean Gaussian process with covariance
$R \in \BB{R}^{m \times m}$.
%In this paper, we make the following assumptions:
We assume the LTI system~\eqref{eq:LTI},~\eqref{eq:linearObservations}
is controllable and observable, i.e., $(A,B)$ is a controllable pair
and $(A,C)$ is an observable pair.
Moreover, we assume that $C$ is full rank.
%\footnote{
These assumptions apply to many systems, including nonlinear systems that can be linearized to satisfy the assumptions.
%For details of a system and observation model that fit this framework, see Sec.~\ref{sec:caseStudy}.}
 
%The state of the system is recursively estimated using a Bayesian filter, which provides a belief about the system's state at each moment in time.
%For linear systems with linear observation models and Gaussian noise, optimal estimation can be performed with Kalman Filters (KF).
%The Kalman observer is defined by the following dynamics:
The belief state at each time step is characterized by
the {\it a posteriori} state and error covariance estimates, $\hat{x}_k$ and
%the {\it a posteriori} error covariance 
$P_k$, i.e.,
$b^k = (\hat{x}_k, P_k)$. The belief state  is maintained via
a Kalman filter~\cite{Bertsekas2012}, which we denote compactly as 
%
%{\small
%\begin{equation}
%\label{eq:kf}
%\begin{aligned}
%\hat{x}_{k+1}& = A \hat{x}_k + B u_k + K_{k+1} \Big(y_{k+1} - C  (A \hat{x}_k + B u_k) \Big)\\
%P_{k+1} &= (I - K_{k+1 }C ) (A P_k A^T + Q)\\
%K_{k+1} &= (A P_k A^T + Q) C^T (C (A P_k A^T + Q) C^T + R)^{-1}
%\end{aligned},
%\end{equation}
%}%
%where $K_k$ is called the Kalman gain at time step $k$.
%We write the Kalman filter dynamics compactly as
\begin{equation}
\label{eq:kf-compact}
b^{k+1} = \tau(b^k, u_k, y_{k+1}),\ \ b^0 = (\hat{x}_0, P_0)\:,
\end{equation}
%
%\begin{align}
%b^{k+1} &= \tau(b^k, u_k, z_{k+1})\\
%b^0 &= (\hat{x}_0, P_0)\:,
%\label{eq:kf-compact}
%\end{align}
where $b^0$
is the known initial belief about the system's state
centered at $\hat{x}_0$ with covariance $P_0$.
For a belief state $(x, P) \in \CA{G}$ 
we denote by $N_\delta(x, P) = \{ b \in \CA{G} \,|\, \norm{b - (x, P)}_\CA{G} \leq \delta \}$ the uncertainty ball of radius $\delta$ in the belief space centered at $(x, P)$, where $\norm{\cdot}_\CA{G}$ over $\CA{G}$ is a suitable norm in $\CA{G}$.

The robot model together with the Kalman filter may be
interpreted as a POMDP~\cite{Kaelbling98,puterman2014,Pineau03a}.

%\subsubsection{Linear Quadratic Gaussian (LQG) Control}

\subsection{Problem definition}
%In this section, we define the problem of searching for an optimal policy that 
%drives the system to satisfy a given  GDTL formula.
%Since we intend to solve the problem over a feedback
%information roadmap (FIRM), we must select a search space of policies that
%is consistent with this problem.

\begin{definition}[Policy]
A control policy for the system is a feedback function from the belief space
$\CA{G}$ to the control space, e.g., $\mu : \CA{G} \to \CA{U}$.
Denote the space of all policies by $\BB{M} = \CA{M}(\CA{G}, \CA{U})$.
\end{definition}

\noindent
%We consider the following probability maximization problem:
We now introduce the main problem under consideration in this work:

\begin{problem}[Maximum Probability Problem]
\label{pb:mpp}
Let $\phi$ be a given GDTL formula and let the system
evolve according to dynamics~\eqref{eq:LTI},
with observation dynamics~\eqref{eq:linearObservations},
and using a Kalman filter defined by~\eqref{eq:kf-compact}.
Find a policy
$\mu^*$ such that 

\begin{equation}
\label{eq:mpp}
\begin{aligned}
&\mu^* = \underset{\mu \in \BB{M}}{\arg \max}Pr[\BF{b} \models \phi] \\
&\text{subject to~\eqref{eq:LTI},~\eqref{eq:linearObservations},~\eqref{eq:kf-compact}.}  \\
%&\ \ x_{k+1} = A x_k +B \mu(b^k) + w_k, \text{\small motion dynamics~\eqref{eq:LTI}}\\
%&\ \ z_k = C x_k + v_k, \text{\small observation dynamics~\eqref{eq:linearObservations}}\\
%&\ \ b^{k+1} = \tau(b^k, \mu(b^k), y_{k+1}), \text{\small KF dynamics~\eqref{eq:kf},~\eqref{eq:kf-compact}}\\
%&\ \ b^k = (\hat{x}_k, P_k)
%%&\ \ u_k = 
\end{aligned}
\end{equation}

\end{problem}




%\section{Preliminaries}
%\label{sec:preliminaries}
%In this section, we provide a brief overview of
%the main ideas and methods from sampling-based planning,
%control theory, and formal methods used in the solution proposed
%in Section~\ref{sec:solution} to solve Problem~\ref{pb:mpp}.
%For detailed information on these notions,
%see~\cite{Baier08,Lav06,Bertsekas2012}.% and the references therein.

%\subsection{Sampling-based planners}
%\label{sec:prelim-rrg}
%Sampling-based algorithms are a class of randomized algorithms
%developed for path and motion planning~\cite{Lav06}.
%In short, a sampling-based algorithm 
%iteratively grows a graph $\TS$ in the state space in which nodes are individual states and edges correspond to motion primitives that drive the robot from state to state.
%%grows a graph by adding
%%randomly sampled states. 
%However, the extension procedure
%is biased towards exploration of uncovered regions of the
%state space.
%Sampling-based algorithms are built using a set of primitive
%functions that are assumed to be available.
%The primitive functions we assume to be available are:
%\begin{itemize}
%  \item $sample(\CA{X})$ generates random states
%from a distribution over the state space $\CA{X}$,
%  \item $nearest(x_r, \TS)$ returns the closest state
%in $\TS$ to the state $x_r$ using the metric defined on
%$\CA{X}$,
%  \item $near(B_n, \TSX, \#Nbrs)$ returns the closest $\#Nbrs$
%states in $\TSX$ to $B_n$, and
%  \item $steer(x_i, x_t)$ returns a state obtained by
%attempting to drive the system from $x_i$ towards $x_t$.
%\end{itemize}
%For simplicity, in this paper we assume that the $steer()$
%function is always able to produce a state that is closer
%to $x_t$ that $x_i$ with respect to the metric defined on
%$\CA{X}$.
%Using these primitive functions, an extension procedure
%$extend(\CA{X}, \TS)$ of the motion abstraction graph
%can be defined as:
%\begin{enumerate}
%  \item generate a new sample $x_r \asgn sample(\CA{X})$,
%  \item find nearest state $x_u \asgn nearest(x_r, \TS)$, and
%  \item drive the system towards the random sample
%$x_n \asgn steer(x_u, x_r)$.
%%
%\end{enumerate}
%For more details about the sampling-based algorithms
%and the primitive functions and their implementations
%see~\cite{Lav06,KF-IJRR11,VaBe-IROS-2013}.
%
%\subsection{Linear quadratic Gaussian control}
%\label{sec:lqg}
%
%Given an LTI system~\eqref{eq:LTI} with linear observation model~\eqref{eq:linearObservations},
%we can design an optimal controller
%called Linear Quadratic Regulator (LQR) of the form
%\begin{align}
%u = -L (\hat{x}_k - x^d)
%\end{align}
%%
%where $L$ is the stationary feedback gain,
%$\hat{x}_k$ is the {\it a posteriori} state estimate
%computed by the Kalman filter,
%$x^d$ is the fixpoint of the LTI system.
%The LQR controller minimizes the following cost function
%\begin{equation}
%\label{eq:lqr-cost}
%J(u) = \BB{E}\left[ \sum_{k \geq 0} (\hat{x}_k - x^d)^T W_x (\hat{x}_k - x^d) + u_k^T  W_u u_k\right],
%\end{equation}
%where $W_x$ and $W_u$ are positive-definite weight matrices.
%The feedback gain $L$ is generated by solving a discrete
%algebraic Riccatti equation (DARE).
%The feedback gain $L$ is generated by solving the discrete
%algebraic Riccatti equation (DARE)
%\begin{equation}
%\label{eq:lqr-dare}
%S = W_x + A^T S A = A^T S B (B^T S B + W_u)^{-1} B^T S A
%\end{equation}
%and setting the gain as
%\begin{equation}
%\label{eq:lqr-gain}
%L = (B^T S B + W_u)^{-1} B^T S A
%\end{equation}

%If the process and observation noise are
%zero-mean Gaussian processes, then the LQR
%controller is optimal with respect to the cost
%function~\eqref{eq:lqr-cost}. Moreover,
%the controller can be designed independently from
%the observer.
%The LQR controller together with the KF
%form a Linear Quadratic Gaussian (LQG) controller.
%
%The stationary covariance matrix may be computed
%as follows~\cite[Lemma 2]{Agha14},~\cite{Bertsekas2012}:
%\begin{lemma}
%%If Assumption~\ref{assump:lti} holds, then the stationary
%If the LTI system~\eqref{eq:LTI}-\eqref{eq:linearObservations} is observable and $C$ is full-rank, then the stationary
%covariance matrix can be computed as
%\begin{equation}
%\label{eq:cov-infty}
%P^\infty = P^\infty_- -  P^\infty_- C^T ( C P^\infty_- C^T + R )^{-1} C P^\infty_-,
%\end{equation}
%where $P^\infty_-$ is the unique symmetric positive-definite
%solution of the DARE
%\begin{equation*}
%P^\infty_- = Q + A P^\infty_- A^T - A P^\infty_- C^T (C P^\infty_- C^T + R)^{-1} C P^\infty_- A^T .
%\end{equation*}
%\end{lemma}

%\subsection{Linear Temporal Logic and Rabin Automata}
%\subsection{Rabin Automata}

%A Linear Temporal Logic (LTL) formula over a set of atomic propositions $\Pi$
%is defined using standard Boolean operators, $\notltl$ (negation),
%$\andltl$ (conjunction) and $\orltl$ (disjunction), and
%temporal operators, $\Next$ (next), $\Until$ (until), $\Event$ (eventually),
%$\Always$ (always).
%The semantics of LTL formulae over $\Pi$ are given with respect to infinite words over $2^\Pi$.
%In this paper, we consider a particular fragment of LTL, called LTL$_{-\Next}$~\cite{Baier08},
%which does not include the $\Next$ (next) operator.
%LTL$_{-\Next}$ is useful in defining specifications which are independent of the number of
%consecutive repetitions of symbols or stutter-invariant formulae~\cite{Baier08}.
%Formal definitions for the LTL syntax, semantics, and model checking can be found in~\cite{Baier08}.

%\begin{definition}[Rabin Automaton]
%A (deterministic) Rabin automaton is a tuple $\RA = (S_\RA, s_0^\RA, \Sigma, \delta, \Omega_\RA)$, where %:
%$S_\RA$ is a finite set of states, $s_0^\RA \in S_\RA$ is the initial state,
%$\Sigma$ is the input alphabet,
%$\delta : S_\RA \times \Sigma \ra S_\RA$ is the transition function, and
%$\Omega_\RA$ is a set of tuples $(\CA{F}_i, \CA{B}_i)$ of disjoint subsets
%of $S_\RA$ which correspond to good ($\CA{F}_i$) and bad ($\CA{B}_i$) states.
%%\begin{itemize}
%%    \item $S_\RA$ is a finite set of states;
%%    \item $s_0^\RA \in S_\RA$ is the initial state;
%%    \item $\Sigma$ is the input alphabet;
%%    \item $\delta : S_\RA \times \Sigma \ra S_\RA$ is the transition function;
%%    \item $\Omega_\RA$ is a set of tuples $(\CA{F}_i, \CA{B}_i)$ of disjoint subsets of $S_\RA$ which correspond to good ($\CA{F}_i$) and bad ($\CA{B}_i$) states.
%%\end{itemize}
%\end{definition}
%
%A transition $s' = \delta(s, \sigma)$ is also denoted by $s \ras{\sigma}_\RA s'$.
%A trajectory of the Rabin automaton $\BF{s} = s_0 s_1 \ldots$ is generated by
%an infinite sequence of symbols $\BS{\sigma} = \sigma_0 \sigma_1 \ldots$ if
%$s_0 = s_0^\RA$ is the initial state of $\RA$ and $s_k \ras{\sigma_k}_\RA s_{k+1}$
%for all $k \geq 0$.
%Given a state trajectory $\BF{s}$ we define $\vartheta_\infty(\BF{s}) \subseteq S_\RA$
%as the set of states which appear infinitely many times in $\BF{s}$.
%An infinite input sequence over $\Sigma$ is said to be accepted by a Rabin automaton
%$\RA$ if there exists a tuple $(\CA{F}_i, \CA{B}_i) \in \Omega_\RA$ of good and bad
%states such that the state trajectory $\BF{s}$ of $\RA$ generated by $\BS{\sigma}$
%intersects the set $\CA{F}_i$ infinitely many times and the set $\CA{B}_i$ only finitely
%many times.
%Formally, this means that $\vartheta_\infty(\BF{s}) \cap \CA{F}_i \neq \emptyset$
%and $\vartheta_\infty(\BF{s}) \cap \CA{B}_i = \emptyset$.
%
%%It is shown in~\cite{Baier08} that for every LTL formula $\phi$
%%over $\Pi$ there exists a DRA $\RA$ over alphabet $\Sigma = \spow{\Pi}$ such that
%%$\RA$ accepts all and only those infinite sequences over $\Pi$ that satisfy $\phi$.
%{\color{blue} [Move to GDTL or solution sections] 
%There exist efficient algorithms that translate LTL formulae into Rabin automata~\cite{klein2006}.}


\section{Solution}
\label{sec:solution}

In our approach, we use sampling-based  techniques to generate paths throughout the state space.
Local controllers drive the systems along these paths and stabilize at key points.
The closed-loop behavior of the system induces paths in the belief space.
The FIRM describes the stochastic process that generates these paths.
%By combining the FIRM with a Rabin automaton, we are able to build
%an MDP that allows us to evaluate whether sample paths satisfy a GDTL formula.
We build an MDP by combing the FIRM with a Rabin automaton which then allows us to check if sample paths satisfy a GDTL formula.
We compute transition probabilities and intersection probabilities
(probability of intersecting a good or bad set from the Rabin automaton's
acceptance condition) for each edge in this structure.
We use dynamic programming to find the policy in this structure that maximizes the probability of satisfying the formula.
The resulting policy can then be translated to a non-stationary switched local controller that approximates the solution to Pb.~\ref{pb:mpp}.
An important property of the proposed solution is that all operations
are incremental with respect to the size of the FIRM.
Note that the proposed solution may be applied to nonlinear
systems whose linearizations around random samples in the state
space satisfy the assumptions in Sec.~\ref{sec:motion}.
The details of our solution Alg.~\ref{alg:compute-mdp} are presented below.

%{\color{blue}[This is only used in Alg. 1. If we trim that algorithm, we should check if this text is still necessary.] }
%Let $(x, P) \in \CA{G}$ be a belief state. 
%We denote by $N_\delta(x, P) = \{ b \in \CA{G} \,|\, \norm{b - (x, P)}_\CA{G} \leq \delta \}$ the uncertainty ball of radius $\delta$ in the belief space centered at $(x, P)$, where $\norm{\cdot}_\CA{G}$ over $\CA{G}$ is a suitable norm in $\CA{G}$.

%In the following sections, the system's dynamics will be denoted by
%the pair $(f, h)$, where $f(x, u, w) = Ax + Bu + w$ is the motion model,
%and $h(x, v) = Cx + v$ is the observation model.
%$(f, h)$

\subsection{Sampling-based algorithm}

We propose a sampling-based algorithm to solve Pb.~\ref{pb:mpp}
that overcomes the curse of dimension and history generally associated with POMDPs.
In short, a sampling-based algorithm iteratively grows a graph $\TS$
in the state space, where nodes are individual states, and edges 
correspond to motion primitives that drive the system from state to state~\cite{Lav06}.
The extension procedure is biased towards exploration of
uncovered regions of the state space.
Similar to~\cite{Agha14}, we adapt sampling-based
methods to produce finite abstractions (e.g., graphs) of the belief space.
Alg.~\ref{alg:compute-mdp} incrementally constructs
a transition system $\TS = (\TSX, B_0, \Delta_\TS, \CA{C}_\TS)$,
where the state space $\TSX$ is composed of
belief nodes, i.e., bounded hyper-balls in $\CA{G}$,
$ \Delta_\TS$ is the set of transitions, and $\CA{C}_\TS$ is a set
of controllers associated with edges.
%subsets of $\CA{G}$.
%In the proposed algorithm, belief nodes are hyper-balls in the
%belief space.
The center of a belief node is a belief state $b=(x, P^\infty)$,
where the mean $x$ is obtained through random sampling of
the system's state space, and $P^\infty$ is the stationary covariance.
The initial belief node is denoted by $B_0$.

Sampling-based algorithms are built using a set of primitive
functions that are assumed to be available:
%The primitive functions we assume to be available are:
\begin{itemize}
  \item $sample(\CA{X})$ generates random states
from a distribution over the state space $\CA{X}$,
  \item $nearest(x^r, \TS) = \arg\min_{x^u}\{ \normeucl{x^r-x^u} \,|\, \exists P^u \wedge N_\delta(x^u, P^u) \in \TSX \}$
  returns the mean $x^u$ of a belief node's center in $\TS$ such that $x^u$ is closest
  to the state $x^r$ using the metric defined on $\CA{X}$,
  \item $near(B_n, \TSX, \gamma)$ returns the closest $\gamma$
belief nodes in $\TSX$ to $B_n$ with respect to the distance between their centers
induced by $\norm{\cdot}_\CA{G}$, and
  \item $steer(x^i, x^t)$ returns a state obtained by
attempting to drive the system from $x^i$ towards $x^t$.
\end{itemize}
%For simplicity, in this paper we assume that the $steer()$
%function is always able to produce a state that is closer
%to $x_t$ that $x_i$ with respect to the metric defined on
%$\CA{X}$.
Using these primitive functions, an extension procedure
$extend(\CA{X}, \TS)$ of the transition system $\TS$
can be defined as:
\begin{enumerate}
  \item generate a new sample $x^r \asgn sample(\CA{X})$,
  \item find nearest state $x^u \asgn nearest(x^r, \TS)$, and
  \item drive the system towards the random sample
$x^n \asgn steer(x^u, x^r)$.
\end{enumerate}
For more details about sampling-based algorithms,
primitive functions and their implementations
see~\cite{Lav06,KF-IJRR11,VaBe-IROS-2013}.

Transitions are enforced using local controllers which are stored
in $\CA{C}_\TS$. i.e., we assign to each edge
$e \in \Delta_\TS$ a local controller $ec_e \in \CA{C}_\TS$.
Under the assumptions of our model~\cite{Agha14}, the local controllers
are guaranteed to stabilize the system to belief nodes along a path in finite time.
Thus we abstract the roadmap to a deterministic system.
In Alg.~\ref{alg:compute-mdp}, local controllers are generated
using the method $localController()$.
The design of the node controllers is presented Sec.~\ref{sec:caseStudy}. 
%However, when we plan to compute a policy to enforce the specification,
%we must compute the probability of invalidating the specification
%before stabilizing at a subsequent belief node.
%For this reason, we annotate the edges of the product automaton with
%``failure probabilities" (intersection) and transform it to an MDP.
%The intersection probabilities at the same time as the transition
%probabilities between the states of the MDP, i.e., pairs of belief nodes
%and DRA states.

The algorithm checks for the presence of a satisfying path using
a deterministic Rabin automaton (DRA) $\RA$ that is computed from
the GDTL specification using an intermediate linear temporal logic (LTL)
construction~\cite{JonesDTL2013}.
There exist efficient algorithms that translate LTL formulae into Rabin automata~\cite{klein2006}.
We denote the set of predicates in GDTL formula $\phi$ as $F_\phi$.

\begin{definition}[Rabin Automaton]
A (deterministic) Rabin automaton is a tuple $\RA = (S_\RA, s_0^\RA, \Sigma, \delta, \Omega_\RA)$, where %:
$S_\RA$ is a finite set of states, $s_0^\RA \in S_\RA$ is the initial state,
$\Sigma\subseteq 2^{F_\phi}$ is the input alphabet,
$\delta : S_\RA \times \Sigma \ra S_\RA$ is the transition function, and
$\Omega_\RA$ is a set of tuples $(\CA{F}_i, \CA{B}_i)$ of disjoint subsets
of $S_\RA$ which correspond to good ($\CA{F}_i$) and bad ($\CA{B}_i$) states.
%\begin{itemize}
%    \item $S_\RA$ is a finite set of states;
%    \item $s_0^\RA \in S_\RA$ is the initial state;
%    \item $\Sigma$ is the input alphabet;
%    \item $\delta : S_\RA \times \Sigma \ra S_\RA$ is the transition function;
%    \item $\Omega_\RA$ is a set of tuples $(\CA{F}_i, \CA{B}_i)$ of disjoint subsets of $S_\RA$ which correspond to good ($\CA{F}_i$) and bad ($\CA{B}_i$) states.
%\end{itemize}
\end{definition}

A transition $s' = \delta(s, \sigma)$ is also denoted by $s \ras{\sigma}_\RA s'$.
A trajectory of the Rabin automaton $\BF{s} = s_0 s_1 \ldots$ is generated by
an infinite sequence of symbols $\BS{\sigma} = \sigma_0 \sigma_1 \ldots$ if
$s_0 = s_0^\RA$ is the initial state of $\RA$ and $s_k \ras{\sigma_k}_\RA s_{k+1}$
for all $k \geq 0$.
Given a state trajectory $\BF{s}$ we define $\vartheta_\infty(\BF{s}) \subseteq S_\RA$
as the set of states which appear infinitely many times in $\BF{s}$.
An infinite input sequence over $\Sigma$ is said to be accepted by a Rabin automaton
$\RA$ if there exists a tuple $(\CA{F}_i, \CA{B}_i) \in \Omega_\RA$ of good and bad
states such that the state trajectory $\BF{s}$ of $\RA$ generated by $\BS{\sigma}$
intersects the set $\CA{F}_i$ infinitely many times and the set $\CA{B}_i$ only finitely
many times.
Formally, this means that $\vartheta_\infty(\BF{s}) \cap \CA{F}_i \neq \emptyset$
and $\vartheta_\infty(\BF{s}) \cap \CA{B}_i = \emptyset$.

%A product MDP $\PA$
%between the TS $\TS$ and the DRA $\RA$ is
%maintained in an incremental fashion.
%The MDP captures motion and satisfaction at the same time, and 
%is used to compute the maximum probability satisfying policy $\mu^*$
%on the finite abstraction $\TS$.

\begin{algorithm}[!htb]
\caption{$ConstructTS(x_0,\phi,\varepsilon)$}
\label{alg:compute-mdp}
\DontPrintSemicolon
\KwIn{initial state $x^0$, GDTL specification $\phi$, and lower bound $\varepsilon$}
%\KwIn{\color{blue} $(f, h)$ -- model of the robot (motion and sensing)}
%\KwIn{$\phi$ -- GDTL specification}
%\KwIn{$\varepsilon$ -- lower bound probability of satisfying policy}
\KwOut{belief transition system $\TS$, product MDP $\PA$, and satisfying policy $\mu^*$}
%\KwOut{$\PA$ -- product MDP}
%\KwOut{$\mu^* : S_\PA \to S_\PA$ -- satisfying feedback policy on $\PA$ w/ probability at least $\varepsilon$}
\BlankLine

convert GDTL formula $\phi$ to LTL formula $\varphi$ over the set of atomic propositions $\AP=F_\phi$\;
compute DRA $\RA=(S_\RA, s_0^\RA, \spow{\AP}, \delta, \Omega_\RA)$ from $\varphi$\;
$ec_0, P^\infty_0 \asgn localController(x^0)$\;
$B_0 \asgn N_\delta(x^0, P^\infty_0)$\;
$e_0 = (B_0, B_0)$\;
$\pi^{S_\RA}_0, \pi^{\Omega_\RA}_0 \asgn computeProb(e_0, s_0, ec_0, \RA)$\;
initialize belief TS $\TS = (\TSX = \{B_0\}, B_0, \Delta_\TS = \{e_0\}, \CA{C}_\TS = \{ (e_0, ec_0)\})$\;
construct product MDP $\PA = \TS \times \RA = (S_\PA = \TSX \times S_\RA, (B_0, s_0),
Act=\TSX, \delta_\PA = \{ \pi^{S_\RA}_0 \}, \Omega_\PA = \{ \pi^{\Omega_\RA}_0 \})$\;

\For{$index = 1$ \KwTo $N$}{
     $x^n \asgn extend(\CA{X}, \TS)$  \;
%    $x_r \asgn sample(\CA{X})$\;
%    $x_u \asgn nearest(x_r, \TS)$\;
%    $x_n \asgn steer(x_u, x_r)$\;
    $ec_n, P^\infty_n \asgn localController(x^n)$\;
    $B_n \asgn N_\delta(x^n, P^\infty_n)$\;
    $\CA{N}_n \asgn near(B_n, \TSX, \gamma)$\;
    $\begin{aligned}\Delta_n &\asgn \{ (B_i, B_n) | x^n = steer(x^i, x^n), B_i \in \CA{N}_n\}\\
           &\quad {} \cup \{ (B_n, B_i) | x^i = steer(x^n, x^i), B_i \in \CA{N}_n \} \end{aligned}$\;
    $\TSX \asgn \TSX \cup \{B_n\}$, $\Delta_\TS \asgn \Delta_\TS \cup \Delta_n$\;
    $S_\PA \asgn S_\PA \cup (\{ B_n \} \times S_\RA)$\;
    \ForEach{$e=(B_u, B_v) \in \Delta_n$}{
%        $ec_e \asgn edgeController(B_u, B_v, nc_v)$\;
        $\CA{C}_\TS \asgn \CA{C}_\TS \cup \{ (e, ec_v)\}$\;
        \ForEach{$s_u \in S_\RA$ s.t. $(B_u, s_u) \in S_\PA$}{
            $\pi^{S_\RA}_e, \pi^{\Omega_\RA}_e \asgn computeProb(e, s_u, ec_v, \RA)$\;
            $\delta_\PA \asgn \delta_\PA \cup \{ \pi^{S_\RA}_e\}$\;
            $\Omega_\PA \asgn \Omega_\PA \cup \{ \pi^{\Omega_\RA}_e \}$\;
        }
%        {\color{blue} compute edge cost (?)}\;
    }
    $\Delta^n_\PA = \{ (p, p') \in \Delta_\PA \,|\, \proj{(p, p')}{\TS} \in \Delta_n \}$\;
    \ForEach(\tcp*[h]{update ECs}){$(\CA{F}_i, \CA{B}_i) \in \Omega_\RA$}{
        $\begin{aligned}\Gamma_i &= \{ (p, p') \in \Delta^n_\PA \,|\, \pi^{\Omega_\RA}(e, \CA{F}_i) = 0 \\
        &\qquad {} \wedge \pi^{\Omega_\RA}(e, \CA{B}_i) > 0, e=\proj{(p, p')}{\TS} \} \end{aligned}$\;
        $c_i.update(\Delta^n_\PA \setminus \Gamma_i)$\;
    }

%    {\color{blue} update SCCs' reach and stay probabilities}\;
    \If{$existsSatPolicy(\PA)$}{
        solve DP~\eqref{eq:dp} and compute policy $\mu^*$ with probability of satisfaction $p$\;
        \lIf{$p \geq \varepsilon$}{
            \Return $(\TS, \PA, \mu^*)$
        }
    }
}
\Return $(\TS, \PA, \emptyset)$
\end{algorithm}

%\begin{algorithm}
%\caption{$extend(\TS)$}
%\label{alg:extend}
%\DontPrintSemicolon
%\KwIn{$\TS$ -- belief transition system}
%\KwOut{$x_n$ -- random sample}
%\KwOut{$\Delta_n$ -- set of transitions}
%\BlankLine
%
%{\color{blue} TODO:}\;
%$done \asgn \False$\;
%\Repeat{$done$}{
%	
%}
%\Return $(x_n, \Delta_n)$
%\end{algorithm}

%\subsection{Local controllers}
%
%Similar to FIRM~\cite{Agha14}, we use local controllers
%to drive the system between the belief nodes of the
%transition system $\TS$, and to stabilize the belief about
%the system's state around the destination nodes.
%We assign to each node $B_i \in \TSX$ in the transition
%system a local controller $nc_i$, which drives the belief
%into $B_i$.
%We also associate with each transition
%$e = (B_u, B_v) \in \Delta_\TS$ an edge controller $ec_e$.
%The edge controller may simply be the node
%controller $nc_v$ associated with the destination node $B_v$.
%However, as suggested in~\cite{Agha14},
%two-stage local controllers are used instead.
%In the first stage, a
%pre-computed nominal trajectory is tracked until the system's state
%gets close to the destination belief node.
%Then, the controller switches to the node controller
%associated with the destination belief node.
%
%The design of the node controllers is presented Sec.~\ref{sec:caseStudy}. 
%For brevity, we omit the details of the tracking controller.
%{\color{blue} For more details about local controllers and examples using
%linear quadratic Gaussian techniques see~\cite{Agha14}.
%}
%It is shown in~\cite[Lemma 3]{Agha14} that these
%controllers can reach their destination belief nodes
%in finite (mean) time.

%\subsection{GDTL to LTL}
%\label{sec:gdtl2ltl}
%
%The predicates considered in GDTL
%are defined over the belief space $\CA{G}$ and
%describe sets in this space. However, because the
%uncertainty is separate from the temporal ordering
%of the satisfaction of the predicates, the sets
%determined by the predicates are independent
%of the position of a belief state $b^i$ in
%a belief word $\BF{b}$.
%As a consequence, we can convert a GDTL formula
%into an LTL formula.
%
%Denote $\CA{G}_f = \{ b\in \CA{G} \ |\ f(b) \leq 0\}$,
%where $f \in \CA{F}(\CA{G}, \BB{R})$.
%
%\begin{definition}[LTL Equivalent]
%\label{def:gdtl2ltl}
%Let $\phi$ be a GDTL formula and $F_\phi$ be the set
%of all predicates in $\phi$. Let $\AP$ be a finite set such
%that $\card{\AP} = \card{F_\phi}$ and a bijective
%map $\widetilde{\ }: F_\phi \to \AP$.
%Consider the LTL formula $\varphi$, where
%each predicate in $F_\phi$ is substituted by its associated
%atomic proposition in $\AP$ using the map
%$\widetilde{\ }$.
%The semantics of $\varphi$ are given with
%respect to infinite words in $\CA{G}^\omega$.
%Satisfaction of an atomic proposition
%$\BF{b} \models \widetilde{p}$ is interpreted as
%$b^0 \in \CA{G}_f$, where $p=(f \leq 0)\in F_\phi$.
%The Boolean and temporal operators retain
%their usual meaning.
%\end{definition}

\subsection{Computing transition and intersection probability}

Given a transition $e=(B_u, B_v)$ and its associated local controller $ec_e$,
Alg.~\ref{alg:compute-prob} computes the transition distribution from
an initial DRA state $s_u$ to a some random DRA state, and a set
of intersection distributions associated with each pair $(\CA{F}_i, \CA{B}_i)$
of the acceptance set of $\RA$.
These distributions are hard to compute analytically. Therefore, we
estimate them from sample trajectories of the closed-loop system
enforcing edge $e$.
In Alg.~\ref{alg:compute-prob}, the function $sampleBeliefSet(S)$
returns a random sample from a uniform distribution over the
belief set $S$.

\begin{algorithm}
\caption{$computeProb(e=(B_u, B_v), s_u, ec_e, \RA)$}
\label{alg:compute-prob}
\DontPrintSemicolon
\SetKwInOut{KwParam}{Parameter}
\KwIn{transition between belief nodes $e=(B_u, B_v)$, starting DRA state $s_u$, controller enforcing $e$ $ec_e$, and deterministic Rabin automaton $\RA$}
%\KwIn{$s_u$ -- starting DRA state}
%\KwIn{$ec_e$ -- controller enforcing $e$}
%\KwIn{$\RA$ -- deterministic Rabin automaton}
\KwOut{transition distribution $\pi^{S_\RA}$, and intersection distribution $\pi^{\Omega_\RA}$ }
%\KwOut{$\pi^{\Omega_\RA}$ -- reach/avoid distribution}
\KwParam{$NP$ -- number of particles}
\BlankLine

$t \asgn \BF{0}_{\card{S_\RA}, 1}$\;
$ra_i \asgn \BF{0}_{3, 1}$, $\forall (\CA{F}_i, \CA{B}_i) \in \Omega_\RA$\;
\For{$p=1:NP$}{
    $b_u \asgn sampleBeliefSet(B_u)$\;
    $b^{0:T} \asgn ec_e(b_u)$\;
    \For{$k = 0$ \KwTo $T-1$}{
        $\sigma_k \asgn \{ f \,|\, f(b^k) \leq 0, \forall f \in F_\phi \}$\;
    }
    $\BF{s} = s_{0:T} \asgn (s_u \ras{\sigma_{0:T-1}} s_T)$\;
    $t[s_T] \asgn t[s_T] + 1$\;
    \For{$(\CA{F}_i, \CA{B}_i) \in\card{\Omega_\RA}$}{
        \lIf{$\CA{F}_i \cap \BF{s} \neq \emptyset$}{
            $ra_i[1] \asgn ra_i[1] + 1$
        }
        \lIf{$\CA{B}_i \cap \BF{s} \neq \emptyset$}{
            $ra_i[2] \asgn ra_i[2] + 1$
        }
        \lIf{$(\CA{F}_i \cup \CA{B}_i)\cap \BF{s} = \emptyset$}{
            $ra_i[3] \asgn ra_i[3] + 1$
        }
    }
}
\Return $\left( \pi^{S_\RA} = \frac{t}{NP}, \pi^{\Omega_\RA} = \left\{ \frac{ra_i}{NP} \,|\, 1\leq i\leq \card{\Omega_\RA} \right\} \right)$
\end{algorithm}

The distribution $\pi^{S_\RA}$ captures the probability that
$s_v$ is the state of $\RA$ at the end of closed-loop
trajectory generated by controller $ec_e$ to steer the system
from belief node $B_u$ and DRA state $s_u$ to belief node $B_v$:
$\pi^{S_\RA} = Pr[s_v \, |\, e, s_u, ec_e]$,
where $s_v \in S_\RA$, $s_{u} \ras{\sigma_{0:T-1}} s_v$,
$b^{0:T} = ec_e(b_u)$, $b_u \in B_u$, and
$\sigma_k \asgn \{ f \,|\, f(b^k) \leq 0, \forall f \in F_\phi \}$.

Each intersection distribution represents the probability that edge $e$
intersects $\CA{F}_i$, $\CA{B}_i$ or neither, where
$(\CA{F}_i, \CA{B}_i) \in \Omega_\RA$, and the controller $ec_e$
was used to drive the system along the edge $e$ starting from
the DRA state $s_u$:

{\footnotesize
\begin{equation}
\pi^{\Omega_\RA} = \left\{\left.
\begin{cases}
Pr[\BF{s}\cap \CA{F}_i \, |\, e, s_u, ec_e]\\
Pr[\BF{s}\cap \CA{B}_i \, |\, e, s_u, ec_e]\\
Pr[\BF{s}\cap (\CA{F}_i \cup \CA{B}_i) \, |\, e, s_u, ec_e]
\end{cases}
\,\right|\, \forall (\CA{F}_i, \CA{B}_i) \in \Omega_\RA \right\}
\end{equation}
}%
For convenience, we use the following notation
$\pi^{\Omega_\RA}(e, X) = Pr[\BF{s}\cap X \, |\, e, s_u, ec_e]$,
where $X \in \{\CA{F}_i, \CA{B}_i, \CA{F}_i \cup \CA{B}_i\}$.

\subsection{GDTL-FIRM Product MDP}

In this section, we define a construction procedure of
the product MDP between the (belief) TS $\TS$ and
the specification DRA $\RA$.

\begin{definition}[GDTL-FIRM MDP]
\label{def:pa}
Given a DTS $\TS = (\TSX, B_0, \Delta_\TS, \CA{C}_\TS)$,
a Rabin automaton $\RA = (S_\RA, s_0^\RA, \Sigma=\spow{\AP}, \delta, \Omega_\RA)$,
and the transition and intersection probabilities $\pi^{S_\RA}$, $\pi^{\Omega_\RA}$,
their product MDP, denoted by $\PA = \TS \times \RA$, is a tuple
$\PA = (S_\PA, s_0^\PA, Act, \delta_\PA, \Omega_\PA)$ where %:
$s_0^\PA = (B_0 , s_0^\RA)$ is the initial state;
$S_\PA \subseteq \TSX \times S_\RA $ is a finite set of states
which are reachable from the initial state by run of positive probability (see below);
$Act = \TSX$ is the set of actions available at each state;
$\delta_\PA : S_\PA \times Act \times S_\PA \ra [0, 1]$ is the transition probability
defined by $\delta_\PA((B_i, s_i), B_j, (B_j, s_j)) = \pi^{S_\RA}(s_j ; e_{ij}, s_i, \CA{C}_\TS(e_{ij}))$, $e_{ij} = (B_i, B_j)$;
and $\Omega_\PA$ is the set of tuples of good and bad transitions in the product automaton.

%\begin{itemize}
%    \item $s_0^\PA = (B_0 , s_0^\RA)$ is the initial state;
%    \item $S_\PA \subseteq \TSX \times S_\RA $ is a finite set of states
%    which are reachable from the initial state by run of positive probability (see below);
%    \item $Act = \TSX$ is the set of actions available at each state;
%    \item $\delta_\PA : S_\PA \times Act \times S_\PA \ra [0, 1]$ is the transition probability
%    defined by $\delta_\PA((B_i, s_i), B_j, (B_j, s_j)) = \pi^{S_\RA}(s_j ; e_{ij}, s_i, \CA{C}_\TS(e_{ij}))$, $e_{ij} = (B_i, B_j)$;
%    \item $\Omega_\PA$ is the set of tuples of good and bad transitions in the product automaton.
%\end{itemize}
\end{definition}

%    : for every $(x^*, s^*) \in S_\PA$ there exists a sequence of $\BF{x} = x_0 x_1 \ldots x_n x^*$, with $x_k \ra_\TS x_{k+1}$ for all $0 \leq k < n$ and $x_n \ra_\TS x^*$, and a sequence $\BF{s} = s_0 s_1 \ldots s_n s^*$ such that $s_0 = s_0^\RA$, $s_k \ras{h(x_k)}_\RA s_{k+1}$ for all $0 \leq k < n$ and $s_n \ras{h(x_n)}_\RA s^*$;

Denote the set of edges of positive probability by
$\Delta_\PA = \left\{ \big((B_i, s_i), (B_j, s_j) \big) \,|\, \delta_\PA((B_i, s_i), B_j, (B_j, s_j)) > 0 \right\}$.
A transition in $\PA$ is also denoted by
$p_i \ra_\PA p_j$ if $(p_i, p_j) \in \Delta_\PA$.
A trajectory (or run) of {\em positive probability} of $\PA$
is an infinite sequence $\BF{p} = p_0 p_1 \ldots$, where
$p_0 = s^\PA_0$ and $p_k \ra_\PA p_{k+1}$ for all $k \geq 0$.

The acceptance condition for a trajectory of $\PA$ is encoded in
$\Omega_\PA$, and is induced by the acceptance condition of
$\RA$. Formally, $\Omega_\PA$ is a set of pairs
$(\CA{F}^\PA_i, \CA{B}^\PA_i)$, where
$\CA{F}^\PA_i = \{ e \in \Delta_\PA \,|\, \pi^{\Omega_\RA}(e, \CA{F}_i) > 0\}$,
$\CA{B}^\PA_i = \{ e \in \Delta_\PA \,|\, \pi^{\Omega_\RA}(e, \CA{B}_i) > 0\}$,
and $(\CA{F}_i, \CA{B}_i) \in \Omega_\RA$.

A trajectory of $\PA = \TS \times \RA$ is said to be accepting
if and only if there is a tuple $(\CA{F}^\PA_i, \CA{B}^\PA_i) \in \Omega_\PA$
such that the trajectory intersects the sets $\CA{F}^\PA_i$ and $\CA{B}^\PA_i$
infinitely and finitely many times, respectively.
It follows by construction that a trajectory $\BF{p} = (B_0, s_0) (B_1, s_1) \ldots$ of $\PA$
is accepting if and only if the trajectory $\BF{s}^0_{0:T_0-1} \BF{s}^1_{0:T_1-1} \ldots$
is accepting in $\RA$,
where $\BF{s}^i_{0:T_i}$ is the random trajectory of $\RA$ obtained
by traversing the transition $e = (B_i, B_{i+1})$ using the controller
$\CA{C}_\TS(e)$ and $s^i_{0} = s_i$ for all $i \geq 0$. Note that $\BF{s}^i_{T_i} = \BF{s}^{i+1}_0$.
As a result, a trajectory of $\TS$ obtained from an accepting trajectory of $\PA$
satisfies the given specification encoded by $\RA$ with positive probability.
%
%For $x \in X$, we define $\beta_\PA(x) = s \in S_\RA$ such that  $(x, s) \in S_\PA$ as the Rabin automaton state that corresponds to $x$ in $\PA$.
We denote the projection of a trajectory
$\BF{p} = (B_0, s_0) (B_1, s_1) \ldots$ onto $\TS$ by
$\proj{\BF{p}}{\TS} = B_0 B_1 \ldots$.
A similar notation is used for projections of finite trajectories.

\begin{remark}
Note that the product MDP in Def.~\ref{def:pa} is defined to be amenable to
incremental operations with respect to the growth of the DTS, i.e., updating and
checking for a solution of positive probability.
This property is achieved by requiring the states of $\PA$ to be reachable
by transitions in $\Delta_\PA$.
The incremental update can be performed using a recursive procedure
similar to the one described in~\cite{VaBe-IROS-2013}.
\end{remark}
\begin{remark}
The acceptance condition for $\PA$ is defined by its transitions and not in the usual way in terms
of its states, % the usual way.
%The choice is
due to the stochastic nature of transitions between belief
nodes in $\TS$. We only record the initial and end DRA states of the
DRA trajectories induced by the sample paths obtained using the local
controllers.
%Our construction is conservative, but avoids the need to store a (possibly large)
%number of intermediate states in $\PA$ for spurious sample paths % which deviate
%deviating
%from the nominal one.
%Thus, the burden of the search is shifted from the verification component
%of the solution, to the sampling-based search procedure.
%A heuristic argument for our choice is that most of the time sample paths
%swill generate the same DRA trajectory as the nominal path.
\end{remark}


%For brevity, we omit its description and refer the reader to~\cite{VaBe-IROS-2013}.

\subsection{Finding satisfying policies}%Checking existence for satisfying paths}

The existence of a satisfying policy with positive probability can be checked efficiently
on the product MDP $\PA$ by maintaining end components EC\footnote{An
EC of an MDP is a sub-MDP  such that there exists a policy such that each
node in the EC can be reached from each other node in the EC with positive
probability.}
for induced subgraphs of $\PA$ determined by the pairs in the acceptance
condition $\Omega_\PA$.
%An SCC is a subgraph such that each node in the SCC can be reached from each other node in the SCC.
For each pair $\CA{F}^\PA_i, \CA{B}^\PA_i$, let $c_i$ denote the
ECs associated with the graphs $G^\PA_i = (S_\PA,\Delta_\PA \setminus \Gamma_i)$, where
$\Gamma_i =  \{ (p, p') \in \Delta_\PA \,|\, \pi^{\Omega_\RA}(e, \CA{F}_i) = 0 \wedge \pi^{\Omega_\RA}(e, \CA{B}_i) > 0, e=\proj{(p, p')}{\TS} \}$.
%Let $dag_i$ be the directed acyclic graphs (DAGs) associated with $scc_i$.
Given $c_i$, checking for a satisfying trajectory in procedure $existsSatPolicy(\PA)$
becomes trivial.
We test if there exists an EC %reachable in $dag_i$ from the SCC containing $s^\PA_0$
that contains a transition $(p, p')$ such that
$\pi^{\Omega_\RA}(e, \CA{F}_i) > 0$, where $e=\proj{(p, p')}{\TS}$.
Note that we do not need to maintain $\Omega_\PA$ explicitly, we only
need to maintain the  $c_i$.
Efficient incremental algorithms to maintain these ECs were proposed
in~\cite{Haeupler2012}. %,Bender2015

\subsection{Dynamic program for Maximum Probability Policy}
Given a \DTL-FIRM MDP, we can compute the optimal switching policy to maximize the probability that the given formula $\phi$ is satisfied.  
%%
%This is done by solving the following optimization problem
%
%\begin{equation}
%\label{probMax}
%	\underset{m \in \mathcal{F}(S_{\mathcal{P}},Act)}{\arg \max} \bigwedge_{i\in \Omega_{\mathcal{R}}}\bigwedge_{j=1}^\infty Pr_{m}( \bigvee_{k=1}^{\infty} s_{j+k} \cap \mathcal{F}_i \wedge s_{j+k} \in S_{\mathcal{P}} \setminus B_i) 
%\end{equation}
%
In other words, we %need to 
find a policy that maximizes the probability of visiting the states in $Acc_\PA$ and avoiding $Trap_\PA$.
% To find this policy, we first decompose $\mathcal{P}$ into a set of end components and find the accepting components. Since any sample path that satisfies $\phi$ must end in an accepting component, maximizing the probability of satisfying $\phi$ is equivalent to maximizing the probability of reaching such a component.
The optimal policy is thus given by the relationship
%{\footnotesize
% \begin{equation}
% \label{eq:dp}
% \resizebox{0.90\columnwidth}{!}{
% $\begin{array}{lcl}
% J^{\infty}(s) &=& \left \{ \begin{array}{ll}
% 1, &  s \in c_i \\ 
% \underset{a \in Act(s)}{\max} \sum_{s'} \delta(s,a,s')J^{\infty}(s') & \text{else} 
% \end{array} \right .  \\
% m(s) &=& \underset{a \in Act(s)}{\arg \max} \sum_{s'} \delta(s,a,s')J^{\infty}(s') \\
% \end{array}$}
% \end{equation}

\begin{align}
\label{eq:dp}
\begin{array}{lcl}
J^{\infty}(s) &= \left \{ \begin{array}{ll}
1, &  s \in Acc_\PA \\ 
\underset{\mu \in \mathbb{M}(s)}{\max} \sum_{s'} \delta(s,\mu,s')J^{\infty}(s') & \text{else} 
\end{array} \right .  \\ \\
\mu(s) &= \underset{\mu \in \mathbb{M}(s)}{\arg \max} \sum_{s'} \delta(s,\mu,s')J^{\infty}(s')
\end{array}
\end{align} 

%}%
This can be solved by a variety of methods, including approximate value iteration and linear programming~\cite{Bertsekas2012}.
%[Cite Bertsekas]


%\subsection{Querying the FIRM}

\subsection{Complexity}
The overall complexity of maintaining the ECs used for checking
for satisfying runs in $\PA$ is $O(\card{\Omega_\RA} \card{S_\PA}^{\frac{3}{2}})$.
The complexity bound is obtained using the algorithm described
in~\cite{Haeupler2012} and is better by a polynomial factor
$\card{S_\PA}^{\frac{1}{2}}$ than computing the ECs at each
step using a linear algorithm.
Thus, checking for the existence of a satisfying run of positive probability
can be done in $O(\card{\Omega_\RA})$ time.
The dynamic programming algorithm is polynomial in
$\card{S_\PA}$~\cite{papadimitriou1987}.

%{\color{blue} [To be written by Cristi]
%
%For all TS, DRA and MDP, we use $\card{\cdot}$ to denote size,
%which is the cardinality of the corresponding set of states.
%
%\begin{itemize}
%  \item the overall complexity of maintaining the SCCs is $O(\card{\Omega_\RA} n^{\frac{3}{2}})$, where $\card{\TS} \leq n = \card{\PA} \leq \card{\TS} \cdot \card{\RA}$, if:
%  \begin{itemize}
%    \item $\TS$ is a sparse graph; and
%    \item the incremental SCC algorithm used to update the SCCs has overall complexity $O(n^{\frac{3}{2}})$.
%  \end{itemize}
%  \item the first part of the test can be checked in $O(1)$ if one maintains $F_\PA$, which resembles the case of a product with a \buchi
%  \item the complexity of checking for the existence of satisfying path is $O(\card{\Omega_\RA})$
%\end{itemize}
%
%}

%\subsection{Conclusion of Running Example}

\section{Case Studies}
\label{sec:caseStudy}

In this section, we apply our algorithm to control a unicycle robot
moving in a bounded planar environment.
To deal with the non-linear nature of the robot model,
we locally approximate the robot's dynamics using LTI systems
with Gaussian noise around samples in the workspace.
This heuristic is very common, since the non-linear and
non-Gaussian cases yield recursive filters that do not
in general admit finite parametrization.
Moreover, the control policy is constrained to satisfy a
rich temporal specification.
The proposed sampling-based solution opvercomes %may overcome
these difficulties due to its randomized and incremental
nature.
As the size of the GDTL-FIRM increases, we expect
the algorithm to return a policy, if one exists, with
increasing satisfaction probability.
Since it is very difficult to obtain analytical bounds on
the satisfaction probability, we demonstrate the
performance of our solution in experimental trials.

\subsubsection{Motion model}

%The motion model for a unicycle is
%\begin{equation}
%\label{eq:unicycle-mm-cont}
%\begin{bmatrix} \dot{p}^x \\ \dot{p}^y \\ \dot{\theta} \end{bmatrix} =
%\begin{bmatrix} \cos(\theta) & 0 \\ \sin(\theta) & 0 \\ 0 & 1 \end{bmatrix} \cdot
% \begin{bmatrix} v \\ \omega \end{bmatrix} +
%\RM{d}\,w^{\CA{X}},
%\end{equation}
%where $p^x$, $p^y$ and $\theta$ are the
%position and orientation of the robot
%in the global reference frame,
%$v$ is the forward velocity,
%$\omega$ is the angular velocity
%and $\RM{d}\,w^{\CA{X}}$ and $\RM{d}\,w^{\CA{U}}$
%are Gaussian white noise process with
%covariance matrices $Q_\CA{X}$ and $Q_\CA{U}$,
%respectively.

The motion model for our system is a unicycle.
We discretize the system dynamics
using Euler's approximation. The motion
model becomes:
{\small
\begin{align}
\label{eq:unicycle-mm-disc}
x_{k+1} &= f(x_k, u_k, w_k) = x_k +
\begin{bmatrix} \cos(\theta_k) & 0 \\ \sin(\theta_k) & 0 \\ 0 & 1 \end{bmatrix} \cdot
u_k + w_k
\end{align}
}%
where
$x_k = \begin{bmatrix} p^x_k\  p^y_k\ \theta_k \end{bmatrix}^T$,
$p^x_k$, $p^y_k$ and $\theta_k$ are the position and orientation of the robot
in a global reference frame,
$u_k = \begin{bmatrix} v'_k\ \omega'_k \end{bmatrix}^T = \Delta t \begin{bmatrix} v_k \ \omega_k \end{bmatrix}^T$,
$v_k$ and $\omega_k$ are the linear and rotation velocities
of the robot,
$\Delta t$ is the discretization step, and
$w_k$ is a zero-mean Gaussian process with covariance matrix
%$Q = \sqrt{\Delta t} Q_{\CA{X}}$, where $Q_{\CA{X}}\in\mathbb{R}^{3\times 3}$.
$Q \in \BB{R}^{3\times 3}$.
%
Next, we linearize the system around a nominal
operating point $(x^d, u^d)$ without noise,
\begin{equation}
\label{eq:unycycle-lti}
x_{k+1}  =  f(x^d, u^d, 0) + A \ (x_k - x^d) + B \ (u_k - u^d) + w_k,
\end{equation}
where $A = \frac{\partial f}{\partial x_k} (x^d, u^d, 0)$
and $B = \frac{\partial f}{\partial u_k} (x^d, u^d, 0)$
are the process and control Jacobians,
$x^d = \begin{bmatrix} p^{x\,d} \  p^{y\,d}\ \theta^d \end{bmatrix}^T$,
and $u^d = \begin{bmatrix} v'^d_k\ \omega'^d_k \end{bmatrix}^T$.
%\begin{align}
%A &= \frac{\partial f}{\partial x_k} (x^d, u^d, 0) =
%\begin{bmatrix} 1 & 0 & - v'^d \cos(\theta^d) \\ 0 & 1 & v'^d \sin(\theta^d) \\ 0 & 0 & 1 \end{bmatrix}\\
%B &= \frac{\partial f}{\partial u_k} (x^d, u^d, 0) =
%\begin{bmatrix} \cos(\theta^d) & 0 \\ \sin(\theta^d) & 0 \\ 0 & 1 \end{bmatrix}\\
%G &= \frac{\partial f}{\partial w_k} (x^d, u^d, 0) =
%\begin{bmatrix} 1 & 0 & 0 \\ 0 & 1 & 0 \\ 0 & 0 & 1 \end{bmatrix}
%\end{align}

In our framework, we associate with each belief node
$B_g \in \TSX$ centered at $(\hat{x}^g, P)$ an LTI
system obtained by linearization~\eqref{eq:unycycle-lti}
about $(\hat{x}^g, u^g)$, where $u^g = [0.1, \, 0]^T$
corresponds to 0.1\;m/s linear velocity and 0
angular velocity.

%%%%%%%%%%
\subsubsection{Observation Model}
\label{sec:bayes}
%%%%%%%%%%%%%%%%%%%%%%%%%%%%%%%%%%%%%%%%%%%%%%%%%%%%%%%%%%%%%%
% Observation Model - Last Modified: 2016/01/25 by Eric Cristofalo
%%%%%%%%%%%%%%%%%%%%%%%%%%%%%%%%%%%%%%%%%%%%%%%%%%%%%%%%%%%%%%

%{\color{blue} [To be expanded by Eric]}
We localize the robot with a multiple camera network. 
This reflects the real world constraints of sensor networks, e.g. finite coverage, finite resolution, and improved accuracy with the addition of more sensors. 
The network was implemented using four TRENDnet Internet Protocol (IP) cameras with known pose with respect to the global coordinate frame of the experimental space. 
Each $640\times400$ RGB image is acquired and segmented, yielding multiple pixel locations that correspond to a known pattern on the robot. 
%The pixel locations and camera poses are used to solve common least squares problem from the Structure from Motion community that estimates the planar position and orientation of the robot in the global frame~\cite{YM-SS-JK-SSS:04}.
The estimation of the planar position and orientation of the robot in the global frame is formulated as a least squares problem
({\em structure from motion})~\cite{YM-SS-JK-SSS:04}.
%as it is commonly done in the {\em structure from motion} community~\cite{YM-SS-JK-SSS:04}.
The measurement, $y_k \in \CA{Y}$, is given by the discrete observation model:
$y_k = Cx_k+v_k$.
%\begin{equation}
%\label{eq:unicycle-on-disc}
%z_k = Cx_k+v_k
%\, .
%\end{equation}
The measurement error covariance matrix is defined as $R = \mathrm{diag}(r_x,r_y,r_{\theta})$, where the value of each scalar is inversely proportional to the number of cameras used in the estimation, i.e. the number of camera views that identify the robot. These values are generated from a camera coverage map (Fig.~\ref{fig:simple-arena-cover}) of the experimental space.  


%The measurement, $z_k \in \mathbb{R}^3$, is calculated by solving a least squares problem given $m$ cameras, their known relative positions and orientations, and the pixel positions of identifiable points on the robot in each camera's frame~\cite{YM-SS-JK-SSS:04}. 
%The resulting measurement includes the position and orientation of the robot on the ground plane with respect to the global coordinate system. 

%%%The continuous time observation model of the system is the following:
%%%\begin{equation}
%%%\label{eq:unicycle-om-cont}
%%%z = C
%%%\begin{bmatrix} p^x \\ p^y \\ \theta \end{bmatrix} +
%%%\RM{d}\,v,
%%%\end{equation}
%%%where $C = \begin{bmatrix} 1 & 0 & 0 \\ 0 & 1 & 0 \\ 0 & 0 & 1 \end{bmatrix}$
%%%is the observation matrix corresponding to the camera
%%%localization system, and
%%%$\RM{d}\,v$ is a Gaussian white noise process
%%%with covariance matrix $R$.
%%%
%%The discrete 
%%%version of the 
%%observation model is
%%\begin{equation}
%%\label{eq:unicycle-om-disc}
%%z_k = Cx_k+
%%%\begin{bmatrix} p^x_k \\ p^y_k \\ \theta_k \end{bmatrix} +
%%v_k.
%%\end{equation}
%%%where $C\in\mathbb{R}^{3\times3}$ is the observation matrix corresponding to the camera localization system, and $v_k$ is a zero-%mean Gaussian process with covariance matrix $R\in\mathbb{R}^{3\times 3}$.
%%%Note that 
%%The robot's orientation is not directly measured and its estimation is left for the filter. The measurement error covariance matrix, $R$, is inversely proportional to the number of cameras used in the robot's pose estimation. 
%%% and was estimated via sample covariance with multiple robot ground truth measurements. 

%The observation model of the system is the following:
%\begin{equation}
%\label{eq:unicycle-om-cont}
%z = C
%\begin{bmatrix} p^x \\ p^y \\ \theta \end{bmatrix} +
%\RM{d}\,v,
%\end{equation}
%where $C = \begin{bmatrix} 1 & 0 & 0 \\ 0 & 1 & 0 \end{bmatrix}$
%is the observation matrix corresponding to the camera
%localization system, and
%$\RM{d}\,v$ is a Gaussian white noise process
%with covariance matrix $R$.
%%
%The discrete version of the observation model is
%\begin{equation}
%\label{eq:unicycle-om-disc}
%z_k = C
%\begin{bmatrix} p^x_k \\ p^y_k \\ \theta_k \end{bmatrix} +
%v_k,
%\end{equation}
%where $v_k$ is a zero-mean Gaussian process
%with covariance matrix $R$.




%%%%%%%%%%
%\subsubsection{Observation model}
%{\color{blue} [To be expanded by Eric]}
%
%The observation model of the system is the following:
%\begin{equation}
%\label{eq:unicycle-om-cont}
%z = C
%\begin{bmatrix} p^x \\ p^y \\ \theta \end{bmatrix} +
%\RM{d}\,v,
%\end{equation}
%where $C = \begin{bmatrix} 1 & 0 & 0 \\ 0 & 1 & 0 \end{bmatrix}$
%is the observation matrix corresponding to the camera
%localization system, and
%$\RM{d}\,v$ is a Gaussian white noise process
%with covariance matrix $R$.
%%
%The discrete version of the observation model is
%\begin{equation}
%\label{eq:unicycle-om-disc}
%z_k = C
%\begin{bmatrix} p^x_k \\ p^y_k \\ \theta_k \end{bmatrix} +
%v_k,
%\end{equation}
%where $v_k$ is a zero-mean Gaussian process
%with covariance matrix $R$.

\subsubsection{Specification}
The specification is given over belief states associated
with the measurement $y$ of the robot as follows:
``Visit regions $A$ and $B$ infinitely many times.
If region $A$ is visited, then only corridor $D_1$
may be used to cross to the right side of the environment.
Similarly, if region $B$ is visited, then only corridor
$D_2$ may be used to cross to the left side of the
environment. The obstacle $Obs$ in the center must
always be avoided. The uncertainty must always
be less than $0.9$. When passing through the
corridors $D_1$ and $D_2$ the uncertainty must
be at most $0.6$.''

The corresponding GDTL formula is:
\begin{align}
\label{eq:simple-spec}
\phi_1 =\ & \phi_{avoid}  \andltl \phi_{reach}  \andltl \phi_{u,1} \andltl \phi_{u,2} \andltl \phi_{bounds} \\
\phi_{avoid} = & \LTLALWAYS \neg \phi_{Obs} \nonumber\\
\phi_{reach} = & \LTLALWAYS \big( \LTLEVENTUALLY (\phi_A \andltl \notltl \phi_{D_2} \LTLUNTIL \phi_B) \LTLEVENTUALLY (\phi_B \andltl \notltl \phi_{D_1} \LTLUNTIL \phi_A) \big) \nonumber\\
% &\qquad \andltl \LTLEVENTUALLY (\phi_B \andltl \notltl \phi_{D_1} \LTLUNTIL \phi_A) \big)\\
\phi_{u,1} = & \LTLALWAYS ( tr(P) \leq 0.9) \nonumber\\
\phi_{u,2} =&  \LTLALWAYS \big( ( \phi_{D_1} \orltl \phi_{D_2}) \Implies ( tr(P) \leq 0.6 ) \big) \nonumber\\
\phi_{bounds} =& \LTLALWAYS (box(\hat{x},x_c,a)\leq 1) , \nonumber
\end{align}
where $(\hat{x}, P)$ is a belief state associated with $y$,
%\begin{equation*}
$
a = \begin{bmatrix}
\frac{2}{l} & \frac{2}{w} & 0
\end{bmatrix} $
%\end{equation*}
%where
so that
$\hat{x}$ must remain within a rectangular $l\times w$ region
with center $x_c= \begin{bmatrix}
\frac{l}{2} & \frac{w}{2} & 0
\end{bmatrix}$, $l=4.13\,m$ and $w=3.54\,m$.
The 5 regions in the environment are defined by GDTL predicate
formulae $\phi_{Reg} = (box(\hat{x},x_{Reg},r_{Reg})\leq 1)$,
%as follows:
%\begin{align*}
%\phi_A &=& (\CA{M}(\hat{x},P,x_A) \leq r_A) \\
%\phi_B &=& (\CA{M}(\hat{x},P,x_B) \leq r_B) \\
%\phi_{D_1} &=& (\CA{M}(\hat{x},P,x_A) \leq r_{D_1}) \\
%\phi_{D_2} &=& (\CA{M}(\hat{x},P,x_A) \leq r_{D_2}) \\
%\phi_{Obs} &=& (\CA{M}(\hat{x},P,x_{Obs}) \leq r_{Obs}),
%\end{align*}
%\begin{equation*}
%\phi_{Reg} = (box(\hat{x},x_{Reg},r_{Reg})\leq 1),
%\end{equation*}
where $x_{Reg}$ and $r_{Reg}$ are the center and the dimensions
of region $Reg \in \{A, B, D_1, D_2, Obs\}$, respectively.
%Fig.~\ref{fig:simple-arena} shows the planar environment
%and the 5 regions.

%The LTL formula $\varphi_1$ corresponding to $\phi_1$ is:
%\begin{align*}
%\phi_1 = & \LTLALWAYS \big( \LTLEVENTUALLY (\pi_A \andltl \notltl \pi_{D_2} \LTLUNTIL \pi_B)
%\andltl \LTLEVENTUALLY (\pi_B \andltl \notltl \pi_{D_1} \LTLUNTIL \pi_A) \big)\\
%& \andltl \LTLALWAYS \neg \pi_{Obs} \andltl \LTLALWAYS \pi_{u,1}
%\andltl \LTLALWAYS \big( ( \pi_{D_1} \orltl \pi_{D_2}) \Implies \pi_{u,2} \big),
%\end{align*}
%where $\AP=\{ \pi_A, \pi_B, \pi_{D_1}, \pi_{D_2}, \pi_{Obs}, \pi_{u,1}, \pi_{u,2} \}$,
%and the map from the predicates of $\phi_1$ to $\AP$ is
%$\widetilde{\phi_{Reg}} = \pi_{Reg}$, $Reg \in \{A, B, D_1, D_2, Obs\}$,
%$\widetilde{tr(P) \leq 0.5} = \pi_{u,1}$ and
%$\widetilde{tr(P) \leq 0.3} = \pi_{u,2}$.

\begin{figure}[!htb]
\centering
\subfigure[Environment]{
%  \resizebox{0.46\columnwidth}{!}{\arena{1}{4}}
  \includegraphics[width=0.45\columnwidth]{exp_setup_1}
  \label{fig:simple-arena}
}
\subfigure[Camera coverage map]{
%  \resizebox{0.46\columnwidth}{!}{\arenacams{1}{4}}
  \includegraphics[width=0.45\columnwidth]{exp_setup_2}
  \label{fig:simple-arena-cover}
}
\subfigure[Pose estimation]{
%  \resizebox{0.46\columnwidth}{!}{\arenacams{1}{4}}
  \includegraphics[width=0.45\columnwidth]{exp_setup_3}
  \label{fig:simple-arena-pose}
}
\subfigure[Transition system ]{
%  \resizebox{0.46\columnwidth}{!}{\arenaprm{1}{4}}
  \includegraphics[width=0.45\columnwidth]{exp_setup_4}
  \label{fig:simple-arena-ts}
}
\caption{Fig.~\subref{fig:simple-arena}
shows an environment with
two regions $A$ and $B$, two corridors
$D_1$ and $D_2$ and an obstacle $Obs$.
%The magenta disk is the initial position of
%the robot.
Fig.~\subref{fig:simple-arena-cover}
shows the coverage of the cameras.
Fig.~\subref{fig:simple-arena-pose}
shows the pose of the robot computed
from the images taken by the 4 cameras.
Fig.~\subref{fig:simple-arena-ts}
shows the transition system computed
by Alg.~\ref{alg:compute-mdp}.
%Fig.~\subref{fig:simple-arena-cover}
%shows the overall coverage of the four
%cameras placed at the corners of the
%arena.
%Fig.~\subref{fig:simple-arena-prm}
%shows a possible MDP obtained by
%a run of the FIRM-based algorithm.
}
\label{fig:simple-case}
\end{figure}


\subsubsection{Local controllers}
We used the following simple switching controller
to drive the robot towards belief nodes:
\begin{equation*}
\resizebox{0.95\columnwidth}{!}{$u_{k+1} = \begin{cases}
\begin{bmatrix}k_D \normeucl{\alpha^T(x^g - \hat{x}_k)} &  k_\theta (\theta^{los}_k - \hat{\theta}_k) \end{bmatrix}^T & \mbox{if } \abs{\theta^{los}_k - \hat{\theta}_k} < \frac{\pi}{12}\\
\begin{bmatrix}0 & k_\theta (\theta^{los}_k - \hat{\theta}_k) \end{bmatrix}^T, & \mbox{otherwise}
\end{cases}$,}
\end{equation*}
where $k_D>0$ and $k_\theta>0$ are proportional scalar gains,
$x^g$ is the goal position, $\theta^{los}_k$ is the line-of-sight angle
and $\alpha=[1\  1\  0]^T$.
We assume, as in~\cite{Agha14}, that the controller
is able to stabilize the system state and uncertainty
around the goal belief state $(x^g, P^\infty)$,
where $P^\infty$ is the stationary covariance matrix.

\subsubsection{Experiments}

The algorithms in this paper were implemented in Python2.7 using
{\em LOMAP}~\cite{ulusoy-ijrr2013} and {\em networkx}~\cite{nx} libraries.
The {\em ltl2star} tool~\cite{klein2006} was used to convert the LTL specification
into a Rabin automaton.
All computation was performed on a Ubuntu 14.04 machine with
an Intel Core i7 CPU at 2.4 Ghz and 8GB RAM.

\begin{figure}[!htb]
\centering
\includegraphics[width=0.8\columnwidth]{experiment_trajectories}
\caption{The figure shows the trajectory of the robot over 10 surveillance cycles.
At each time step, the pose of the robot is marked by an arrow.
The true trajectory of the robot is shown in green. The trajectory obtained from
the camera network is shown in yellow, while the trajectory estimated by the
Kalman filter is shown in black.}
\label{fig:exp-traj}
\end{figure}

A switched feedback policy was computed for the ground
robot described by~\eqref{eq:unicycle-mm-disc} operating
in the environment shown in Fig.~\ref{fig:simple-arena}
with mission specification~\eqref{eq:simple-spec} using
Alg.~\ref{alg:compute-mdp}.
The overall computation time to generate the policy was
32.739 seconds and generated a transition system and product
MDP of sizes (23, 90) and (144, 538), respectively.
The Rabin automaton obtained from the GDTL formula has
7 states and 23 transitions operating over a set of atomic
propositions of size 8.
The most computationally intensive operation in
Alg.~\ref{alg:compute-mdp} is the computation of the
transition and intersection probabilities.
To speed up the execution, we generated trajectories for
each transition of the TS and reused them whenever
Alg.~\ref{alg:compute-prob} is called for a transition
of the product MDP.
The mean execution time for the probability computation
was 0.389 seconds for each transition of $\TS$.

We executed the computed policy on the ground vehicle
over 9 experimental trials for a total of 24 surveillance cycles.
The specification was met in all of surveillance cycles.
A trajectory of the ground robot over 10 surveillance cycles
(continuous operation) is shown in Fig.~\ref{fig:exp-traj}.

\section{Conclusion}
\label{sec:conclusion}

In this paper, we presented a sampling-based algorithm that
generates feedback policies for stochastic systems with
temporal and uncertainty constraints.
The desired behavior of the system is specified using
{\em Gaussian Distribution Temporal Logic}
such that the generated policy satisfies the task
specification with maximum probability.
The proposed algorithm generates a transition system
in the belief space of the system.
A key step towards the scalability of the automata-based
methods employed in the solution
was breaking the {\em curse of history} for POMDPs.
Local feedback controllers that drive the system within
belief sets were employed to achieve history independence
for paths in the transition system.
%The second component that contributes 
Also contributing
to the scalability
of our solution is a construction procedure for an annotated
product Markov Decision Process called GDTL-FIRM,
where each transition is associated with a ``failure probability''.
GDTL-FIRM captures both satisfaction and the stochastic
behavior of the system.
Switching feedback policies were computed over the product MDP.
Lastly, we showed the performance of the computed policies in
experimental trials with a ground robot %operating in an planar
%environment 
%equipped with a 
tracked via
camera network.
%
The case study %also 
shows that properties specifying
the temporal and stochastic behavior of systems can be
expressed using GDTL and our algorithm is able to
compute control policies that satisfy the %task
specification with a given probability.

%GDTL was used to formulate a maximum probability (MP) problem.
%Our approach to solve the MP problem adapts sampling-based
%and automata-based methods.
%We use an off-the shelf tool~\cite{klein2006} to translate
%a GDTL formula to a deterministic Rabin automaton.
%Inspired by the feedback information roadmap planner~\cite{Agha14},
%we propose a planning algorithm that generates finite transition
%systems in the belief space of the robot.
%These transition systems are combined with Rabin automata
%encoding the GDTL task specifications into product
%Markov Decision Process (MDP) models.
%In order to mitigate the state space explosion problem,
%the planner generates product MDPs with transition-based
%acceptance conditions.
%Switching feedback policies were computed over the product MDPs
%based on dynamic programming.
%Lastly, we showed the performance of the computed policies in
%experimental trials with a ground robot operating in an planar
%environment equipped with a camera network.


%% Use plainnat to work nicely with natbib. 

%\bibliographystyle{plainnat}
\bibliographystyle{IEEEtran}
\bibliography{references}

\end{document}



\documentclass{ifacconf}
\usepackage{times}
\usepackage{float}
\usepackage{amsmath}
\usepackage{graphicx}      % include this line if your document contains figures
\usepackage{natbib}        % required for bibliography
%===========================================
%\documentclass[letterpaper, 11pt, onecolumn]{TemplateFiles/ieeeconf}
%\IEEEoverridecommandlockouts \overrideIEEEmargins 
%\pagestyle{plain}

%\usepackage[ruled,vlined,linesnumbered,boxruled]{algorithm2e}
%\usepackage{mathrsfs}
%\usepackage{graphicx}
%\usepackage{amsfonts}
%\usepackage{amsmath}
%\usepackage{amssymb}
%\usepackage{array}
%\usepackage{flafter}
%\usepackage{tabu}
%\usepackage{cite}
%\usepackage{subfigure}
%\usepackage{verbatim}
%\usepackage{bbm}
%\usepackage[usenames]{color}
%\usepackage[svgnames]{xcolor}

%\usepackage{balance}

%\usepackage{nicefrac}
%\usepackage{psfrag}
%\usepackage{umoline}
%\usepackage{hyperref}
%\usepackage{appendix} %[2009/09/02 v1.2b extra appendix facilities]
%\let\proof\relax
%\let\endproof\relax
%\usepackage{amsthm}	% This is needed for \newtheorem and proof environment.
% \usepackage{natbib} % for citing the papers with auther-year format in parantheses.
%\usepackage[numbers, sort]{natbib}

%\usepackage{etoolbox}

%\providecommand{\citet}[1]{\citeauthor{#1}\,[\citeyear{#1}]}
%\providecommand{\citep}[1]{\cite{#1}}

%\newtheorem{thm}{Theorem}
%\newtheorem{Def}{Definition}
%\newtheorem{Problem}{Problem}
%\newtheorem{Lem}{Lemma}
%\newtheorem{Cor}{Corollary}
%\newtheorem{assumption}{Assumption}
%\newtheorem{proposition}{Proposition}
%\newtheorem{definition}{Definition}
%\newtheorem{property}{Property}
%\newtheorem{Remark}{Remark}
%\newtheorem{exmp}{Example}[section]
%\newenvironment{proof}[1][Proof]{\begin{trivlist}
%\item[\hskip \labelsep {\bfseries #1}]}{\end{trivlist}}

%\newcommand{\TypeOfDoc}{IJRR} % This either can be TR or Conf or IJRR
%\newcommand{\FinalFlag}{No} % This either can be No or Accepted
%\newtoggle{finalpaper}
%\toggletrue{finalpaper}
%\togglefalse{finalpaper}

\graphicspath{{./figs/}}

%\allowdisplaybreaks[1]

%\newcommand{\kXX}[1]{\color{blue} XX #1 XX \color{black}}
%newcommand{\AXX}[1]{\color{purple} XX #1 XX \color{black}}
\newcommand{\aXX}[1]{\color{orange} #1  \color{black}}
\newcommand{\axx}[1]{\aXX{#1}}


\newcommand{\pr}[1]{\textbf{#1:} }

\newcommand{\codeline}[1]{\par{\ttfamily #1 \par}}  % This line is added because sth like this \newcommand{\initeali}{\verb|InitializeEdge|} does not work. For further details please check out this link % http://tex.stackexchange.com/questions/86071/newcommand-for-verbatim
% please do not delete above comments as it can be really confsing


%%%%%%%%%%%%%%%% symbols
\newcommand{\tv}{\varpi} % tube volume (edge tube volume)
\newcommand{\td}{\Gamma} % tube distance (edge tube distance)

%%%%%%%%%%%%%%%%%%%%%%%%%%%%%%%%%%%%%%%%%%%%%%%%%% To reduce pages
%\textfloatsep = 0pt
%\renewcommand{\baselinestretch}{0.96}
%%%%%%%%%%%%%%%%% Make bibs smaller
%\renewcommand{\IEEEbibitemsep}{0pt plus 2pt}
%\makeatletter
%\IEEEtriggercmd{\reset@font\normalfont\footnotesize}
%\makeatother
%\IEEEtriggeratref{1}

%%%%%%%%%%%%%%%%% margins
%\usepackage{geometry}
% \newcommand{\papermargin}{0.97in} % IEEE asks for 0.75 on all pages, but the first page
% %\newgeometry{top=0.75in,bottom=.75in,right=.75in,left=.75in}
% \newgeometry{top=\papermargin,bottom=\papermargin,right=\papermargin,left=\papermargin}
% \newgeometry{top=1.in,bottom=1.in,right=1.in,left=1.in}

\let\labelindent\relax
\usepackage{enumitem}

%% Additional packages
\usepackage{amsmath,amssymb,amsfonts}
\usepackage{mathtools}
% \usepackage{subfigure}
\usepackage{graphicx}
\usepackage{color}
\usepackage{url}
%\usepackage[usenames,x11names]{xcolor}

\ifdefined\algorithm
% Don't load theorem style
\else
\usepackage[linesnumbered,vlined,ruled]{algorithm2e}
\fi

\usepackage{tikz}
\usepackage{pgfplots} 
\pgfplotsset{compat=1.14}
\usetikzlibrary{arrows,automata,shapes,calc,backgrounds,spy,positioning}
\usetikzlibrary{fit}

\usepackage{epstopdf}


 

%%% figure path
\graphicspath{{figures/}}
 
 
%% Roman, calligraphic, boldface, double barred letters
\newcommand{\RM}[1]{\mathrm{#1}}
\newcommand{\CA}[1]{\mathcal{#1}}
\newcommand{\BF}[1]{\mathbf{#1}}
\newcommand{\IT}[1]{\mathit{#1}}
\newcommand{\BB}[1]{\mathbb{#1}}
\newcommand{\TT}[1]{\mathtt{#1}}
\newcommand{\FK}[1]{\mathfrak{#1}}
\newcommand{\BS}[1]{\boldsymbol{#1}}


%% spaces 
\newcommand{\Real}{\BB{R}}
\newcommand{\Symb}{\mathcal S}
\newcommand{\borel}[1]{\mathcal{B}\left(#1\right)}


%%% Probability
\newcommand{\Ex}{\mathbf{E}}     % Probability of an event

\newcommand{\po}{\mathbf{P}}     % Probability of an event
\newcommand{\p}[1]{\po\left(#1\right)}     % Probability of an event
\renewcommand{\P}{\BF{P}}
\newcommand{\pd}[1]{p\left(#1\right)}     % Probability density  



%% Modelling symbols
%-------------MDP----------------------------
\newcommand{\MDP}{\mathsf{M}}
\newcommand{\POMDP}{\MDP_{\Z}}
\newcommand{\C}{\mathbf{C}}
\newcommand{\MB}{\mathsf{B}}

 
\newcommand{\X}{{\mathbb{X}}}  % State
\newcommand{\Z}{{\mathbb{Z}}}	% Observation space
\newcommand{\A}{{\mathbb{U}}} % Action space
\newcommand{\init}{\rho}
\newcommand{\tr}{t}


% product MDP
\renewcommand{\S}{{\mathbb{S}}}	% Observation space

\newcommand{\polb}{{\boldsymbol{\mu}}}
\renewcommand{\pol}{{\mu}}

\newcommand{\Y}{{\mathbb{Y}}}  % State

%-------------POMDP-----------------------------
\newcommand{\Hist}{{\mathbb{H}}}  %  History
\newcommand{\I}{{\mathbb{I}}}  %  History
\newcommand{\Belief}{b}
\newcommand{\trb}{\tr_\Belief}
\newcommand{\Xb}{\mathbb{B}}  % State
\newcommand{\initb}{\init_\Belief}


% -----------------Refinement relation----------
\newcommand{\InF}{\mathcal{U}_{v}}
\newcommand{\Wt}{\mathbb{W}_{\tr}}

\newcommand{\grid}{{\boldsymbol{\eta}}}

%%% Logic 
%------------------Predicates-----------------------------------
\newcommand{\Fpred}{{\mathcal F}}
\newcommand{\Lab}{\mathsf{L}}
\newcommand{\Labset}{\mathcal{L}}

\newcommand{\alphabeth}{\Sigma}
\newcommand{\word}{{\boldsymbol{\pi}}} % words formed from the alphabeth
\newcommand{\letter}{\pi} % words formed from the alphabeth

\newcommand{\Bel}{{\mathbf {T}}}
\newcommand{\BelR}{{\mathbf {R}}}
\newcommand{\trunc}[2]{\operatorname{trunc}_{#1}\left(#2\right)}


%%% Temporal logic symbols
\newcommand{\notltl}{\neg}
\newcommand{\andltl}{\wedge}
\newcommand{\orltl}{\vee}
\newcommand{\Next}{\ensuremath{\bigcirc}}
\newcommand{\Always}{\ensuremath{\ \square\ }}
\newcommand{\Event}{\ensuremath{\ \diamondsuit\ }}
\newcommand{\Until}{\ \CA{U}\ }
\newcommand{\Implies}{\Rightarrow}
\newcommand{\Equiv}{\Leftrightarrow}
\newcommand{\True}{\top}
\newcommand{\False}{\perp}
\newcommand{\AP}{{AP}}
\newcommand{\pred}{\xi}


\newcommand{\eps}{\epsilon} \newcommand{\rel}{\mathcal{R}} % numbers option provides compact numerical references in the text. 

%----- Exotic words----- 
\newcommand{\buchi}{B\"uchi\ }

%% Symbols of automata
\newcommand{\PA}{\mathcal{P}} 
\newcommand{\BA}{\mathcal{B}}
\newcommand{\TS}{\mathcal{F}}
\newcommand{\Language}{\mathbf{Lang}} % Language?
\newcommand{\KA}{\mathcal{K}}
\newcommand{\RA}{\mathcal{R}}
\newcommand{\FSA}{\mathcal{A}}

\newcommand{\TSX}{\BB{V}_\TS}
\newcommand{\TSE}{\BB{E}_\TS}
\newcommand{\TSEE}{\BB{E}}

\newcommand{\DTL}{DTL~}

 % Custom operators
\newcommand{\norm}[1]{\left\| {#1} \right\|}
\newcommand{\norminf}[1]{\left\| {#1} \right\|_{\infty}}
\newcommand{\normeucl}[1]{\left\| {#1} \right\|_{2}}
\newcommand{\abs}[1]{\left| {#1} \right|} 
\DeclareMathOperator{\diag}{diag}

\DeclareMathOperator*{\argmin}{arg\,min} 
\DeclareMathOperator*{\argmax}{arg\,max} 

\ifdefined\theoremstyle
% Don't load theorem style:
%% There are a number of predefined theorem-like environments in
%% ifacconf.cls:
%%
%% \begin{thm} ... \end{thm}            % Theorem
%% \begin{lem} ... \end{lem}            % Lemma
%% \begin{claim} ... \end{claim}        % Claim
%% \begin{conj} ... \end{conj}          % Conjecture
%% \begin{cor} ... \end{cor}            % Corollary
%% \begin{fact} ... \end{fact}          % Fact
%% \begin{hypo} ... \end{hypo}          % Hypothesis
%% \begin{prop} ... \end{prop}          % Proposition
%% \begin{crit} ... \end{crit}          % Criterion
\newtheorem{theorem}[thm]{Theorem}
\newtheorem{lemma}[thm]{Lemma}
\newtheorem{definition}[thm]{Definition}
\newtheorem{problem}{Problem}


\newtheorem{example}{Example}

\else
\usepackage{amsthm}
\theoremstyle{plain}
\newtheorem{theorem}{Theorem}
\newtheorem{cor}{Corollary}
\newtheorem{prop}{Proposition}
\newtheorem{lemma}{Lemma}
\newtheorem{remark}{Remark}
\newtheorem{cond}{Condition}

\newtheorem{example}{Example}
\newtheorem{theorem}{Theorem}
\newtheorem{definition}[theorem]{Definition}
\fi


\pdfinfo{
   /Author (S.Haesaert et al.)
   /Title  (Formal abstraction of POMDPs for Distribution LTL)
   /CreationDate (D:20101201120000)
   /Subject (Formal abstraction)
   /Keywords (abstraction;POMDP)
}

% Table caption wrangling
\usepackage{etoolbox}
\usepackage[utf8]{inputenc}


\allowdisplaybreaks[1]
%% commenting
\newcommand{\red}[1]{{\color{red} #1}}
\renewcommand{\axx}[1]{{\color{orange} Ali: #1}}

\newcommand{\new}[1]{{\color{blue}#1}}
\newcommand{\ind}{\mathbf{1}}

\newcommand{\cristi}[1]{{\color{orange}#1}}


\begin{document}


\begin{frontmatter}

\title{\huge Refinement-based temporal logic control of partially observable Markov decision processes }
%\thanks[footnoteinfo]{Sponsor and financial support acknowledgment
%goes here. Paper titles should be written in uppercase and lowercase
%letters, not all uppercase.}

\author[cal]{S. Haesaert}
\author[cal]{P. Nilsson}
\author[mit]{C.I. Vasile}
\author[jpl]{R. Thakker}
\author[jpl]{A. Agha}
\author[cal]{A.D.  Ames}
\author[cal]{R. M. Murray}



\address[cal]{California Institute of Technology,
   Pasadena, CA 91125 USA} % (e-mail: \{haesaert,pettni,ames,murray\}@caltech ).}
\address[mit]{Massachusetts Institute of Technology,
   Cambridge, MA 02139 USA}% (e-mail:  cvasile@mit.edu)}
\address[jpl]{Jet Propulsion Laboratory,
   Pasadena, CA 91109 USA}% (e-mail: rohan.a.thakker@jpl.nasa.gov)}
\maketitle
\begin{abstract}
The stochastic evolution of a partially observable Markov decision process (POMDP) can be modeled by an associated Markov decision process (MDP) in belief space.
In this work, we consider synthesis of controllers guaranteeing  specifications given in linear temporal logic on such belief models. The computational issues associated with correct-by-construction synthesis on continuous state spaces are circumvented via construction of a finite-state abstraction, on which a control policy is synthesized and refined back to the original model. We leverage the notion of approximate stochastic simulations to quantify the deviation of the approximate model. By compensating {\it a priori} for these deviations in the control synthesis, correctness guarantees are inherited by the refined control policy.
\end{abstract}
\begin{keyword} Belief space models,
correct-by-construction control synthesis, Markov decision processes, partially observable
\end{keyword}

\end{frontmatter}
%%%%%%%%%%%%%%%%%%%%%%%%%%%%%%%%%%%%%%%%%%%%%%%%
%%%%%%%%%%%%%%%%%%%%%%%%%%%%%%%%%%%%%%%%%%%%%%%%

\section{Introduction}\label{subsec:intro}
Emerging applications of robotics systems necessitate control systems capable of autonomously performing complex tasks in a safe manner. Therefore, systems deployed in (partially) unknown environments have to make decisions in the face of uncertainty that maximize the probability of task completion. Temporal logics have emerged as a principled formalism for expressing behavior, and associated correct-by-construction synthesis techniques for the control policy have been developed \citep{Murray2009}. However, work to date in this area has focused on specifications expressed in terms of the system state; such specifications encompass many relevant behaviors but are incapable of expressing properties about system uncertainty. For instance, in a Mars rover exploration mission the accuracy of pose estimates must be of much higher quality during the crucial sample extraction phase, as compared to when the rover is traversing a ``safe'' area. In this work we are concerned with correct-by-construction synthesis for properties that quantify such notions of uncertainty.

% Given a specification $\psi$  written in linear temporal logic with propositions over the belief state of a POMDP, we are interested in the design of a policy $\pol$ such that  $\psi$ is satisfied with probability at least $p$. In this paper, we approach the synthesis problem as follows. Firstly, for a given belief MDP $\MB$ a finite-state abstraction  $\tilde \MB$ is computed. Secondly, we compute a labeling-based $\delta$-approximate stochastic  simulation relation between the abstract belief space model $\tilde{\MB}$ and the concrete model $\MB$. Then, a $\delta$-robust policy for the abstract $\tilde{\MB}$ is computed  together with a stationary value function for the associated stochastic optimal control problem. Finally, a policy for the concrete belief space model is obtained as its refinement.

For probabilistic temporal logic properties over finite-state Markov decision processes (MDPs), there exist several tools for policy synthesis and verification, such as PRISM \citep{KNP11} and  Storm \citep{dehnert2017storm}. However, when the MDP state space is uncountable the characterization of these properties, expressed in terms of the system state, cannot in general be attained analytically \citep{Abate1}, so an alternative is to approximate these models by simpler processes that can be analyzed and verified algorithmically, such as finite-state MDPs \citep{soudjani2015faust}, or deterministic transition systems \citep{Zamani2014}. \new{By quantifying the approximation accuracy via (approximate) simulation relations \citep{Zamani2014,haesaert2017verification,tech_report_TACAS}, the resulting controller can be shown to still be correct-by-construction.}

Control and decision making under motion and sensing uncertainty is best captured by the class of partially observable Markov decision processes (POMDPs) \citep{Kaelbling98,Smallwood73}. Without full-state observations, formal synthesis becomes more challenging. For finite-state POMDPs verification and policy synthesis has been considered for PCTL properties \citep{Norman2017, Chatterjee2014}. Results for specifications defined on the continuous states of a POMDP are rather preliminary and have been focused mainly on reachability and safety problems, where reachability and safety is defined with respect to the hidden state of the POMDP. In \citep{ding2013optimal} the optimal control of partially observable systems under safety specifications is analyzed, whereas \citep{LESSER20141989} studies reachability over partially observable stochastic systems.

Like previously done in  \citep{Vasile2016,JonesDTL2013}, in this paper we consider properties defined on the belief space of a POMDP---the space of probability distributions over states. Belief space properties express requirements on the estimation quality or uncertainty. We novelly solve this synthesis problem in a  correct-by-construction fashion by leveraging recent results on  approximate simulation relations \citep{haesaert2017verification, tech_report_TACAS}.
More precisely, since the temporal logic problem cannot be solved directly over the original model, we compute a policy with guarantees over an approximate model. Leveraging similarity, the policy can then be refined together with  its guaranties  to the original model.

\new{Next to the fact that correct-by-construction policies are novel within the realm of belief space temporal logics, we contribute to the state-of-the art as follows.
 Firstly, we reinterpret the notion of approximate stochastic simulation relation via the non-determinism in the proposition labeling. Secondly, we newly construct the associated policy refinement via the value function. We also compute the approximate simulation relation for a non-stationary Kalman filtered  Linear Time-Invariant (LTI) system.   Finally,  we give a case study in which a finite environment POMDP is combined with a LTI POMDP.}

%More precisely, we approach the synthesis problem as follows. Firstly, for a given belief MDP $\MB$ a finite-state abstraction  $\tilde \MB$ is computed. Secondly, we compute a labeling-based $\delta$-approximate stochastic  simulation relation between the abstract belief space model $\tilde{\MB}$ and the concrete model $\MB$. Then, a $\delta$-robust policy for the abstract $\tilde{\MB}$ is computed  together with a stationary value function for the associated stochastic optimal control problem. Finally, a policy for the concrete belief space model is obtained as its refinement.


\subsubsection{Notation and preliminaries:}

For a metric space $\Y$, we denote by $\borel{\Y}$ its Borel $\sigma$-field. Then $\mathcal P(\Y)$ is the set of probability measures on the Borel-measurable space $(\Y,\borel{\Y})$. A probability measure $\po \in \mathcal P(\Y)$ together with $(\Y,\borel{\Y})$ defines a probability space $(\Y,\mathcal{B}(\Y),\po)$ with realizations $s{\,\sim\,}\po$. Given a probability measure, the corresponding expectation operator is denoted as $\Ex[\cdot]$. In this work, we restrict attention to the case when $\Y$ is a Polish space, i.e, a complete and separable metric space \citep{bogachev2007measure}.

% For two sets $\X_1$ and $\X_2$, a relation $\rel\subset \X_1 \times \X_2$ is a subset of the Cartesian product that relates $x_1 \in \X_1$ to $x_2 \in \X_2$ if $(x_1, x_2)\in\rel$, equivalently written $x_1 \rel x_2$. Examples of relations include the diagonal relation $\{(x_1,x_2){\mid} x_1=x_2\}$, and norm based relations $\{(x_1,x_2){\mid} \|Px_1-x_2\|_2\leq \epsilon\}$ where $P$ is a projection matrix.
For a vector $\grid  \in \mathbb{R}^n$ we write the $\grid$-ball in infinity norm as $\ball_\grid(x_0) = \{ x : |x_i - x_{0,i}| \leq \grid_i \}$. The indicator function for a set $A$ is written $\ind_A(x)$ and is equal to $1$ if $x \in A$, and to 0 otherwise. For a relation $\rel \subset \X_1 \times \X_2$ we introduce the associated mappings $\rel(X_1):=\{x_2: x_1\rel x_2, x_1 \in \X_1\}$ and  $\rel^{-1}( X_2 ):=\{x_1: x_1\rel x_2, x_2 \in \X_2 \}$ for $X_1 \subseteq \X_1$ and $X_2 \subseteq \X_2$.


\subsubsection{Paper layout:}

In the next section, the POMDPs and associated belief models are introduced together followed by temporal logic concepts. Afterwards, in Section \ref{sec:exact_synth} we consider the control synthesis problem for temporal logic properties over belief MDPs, before suggesting a robust synthesis procedure for abstractions in Section \ref{sec:refinement}. In Section \ref{sec:gaussian}, we study the special case of Gaussian POMDPs in more detail, before concluding the paper with an illustrative case study in Section~\ref{sec:case} and conclusions in Section \ref{sec:conclusions}.


\section{POMDPs and temporal logic specifications}

We first give the definition of a Partially Observable Markov Decision Process (POMDP). Then we define a linear temporal logic (LTL) equivalent to the formulation of distribution temporal logic for POMDPs as originally given in \citep{JonesDTL2013}.

\subsection{Partially Observable Markov  Decision Processes}

We first define a Markov decision process as follows \citep{hll1996}.
% \citep{mt1993,bertsekas2004stochastic}.
\begin{definition}
\label{def:MDP}
  A discrete-time \emph{Markov decision process} (MDP) is a tuple $\MDP = (\X, \init, \tr, \A)$ where
  \begin{itemize}
    \item $\X$ is a \new{(Polish)} state space with states $x\in\X$; % as its elements;
    \item $\init \in \mathcal P(\X)$ is an initial probability distribution;
    \item $\A$ is a \new{(Polish)} input space with inputs $u\in\A$;
    \item $\tr:\X\times\A\times\mathcal B(\X)\rightarrow[0,1]$ is a conditional stochastic kernel that assigns to each state $x\in \X$ and control $u\in \A$ a probability measure $\tr(\cdot\mid x,u)$ over $(\X,\mathcal B(\X))$.
  \end{itemize}
\end{definition}

An execution of $\MDP$ is a state-input sequence $(x_0, u_0)(x_1, u_1)\ldots$ where $x_0 \sim \rho$ and $x_k \sim t(\cdot \mid x_k, u_k)$ for inputs $u_k \in \A$. While MDPs capture uncertainty in state transitions, full knowledge is assumed about the state of the system. We therefore augment MDPs with a model for observation.

% Given a string of inputs $u(0),$ $u(1), $ $\ldots, $ $u(N)$, over a finite time horizon $0,1,\ldots, N$ and an initial state $x_0$ sampled from $\rho$, the states $x_{k+1}$ with $k\in \{0,1,\ldots, N\}$ are obtained as a realization of the controlled Borel-measurable stochastic kernel $\tr\left(\cdot{\,\mid\,t} x_k, u_k \right)$ \cristi{C: $\tr\left(\cdot\mid x_k, u_k \right)$ ?} -- these semantics induce paths (or executions) of the MDP.

\begin{definition}
\label{def:POMDP}

A \emph{partially observable Markov decision process} (POMDP) $\POMDP$ is an MDP $\MDP = (\X, \init, \tr, \A)$ together with an observation model $(\Z, r)$ where
\begin{itemize}
	\item $\Z$ is a (Polish) output space with outputs $z \in \Z$;
  \item $r : \X \times \mathcal B(\Y) \rightarrow [0,1]$ is an observation kernel that to each state $x \in \X$ assigns an output $z \in \Z$ according to $z\sim r(\cdot|x)$.
\end{itemize}
\end{definition}

An execution of the POMDP up to time $K$ is a sequence
\begin{equation}
\label{eq:history}
  (x_0,u_0,z_0) (x_1,u_1,z_1) \ldots (x_K,u_K,z_K)
\end{equation}
where $(x_0, u_0) \ldots (x_K, u_K)$ is an execution of $\MDP$ and $z_k \sim r(\cdot | x_k)$. The execution \eqref{eq:history} grows with the number of observations $K$ and takes values in the \emph{history space} $\Hist_K = (\X \times \A \times \Z)^{K+1}$.

In a POMDP control actions $u_k$ can be chosen as a function of the available information on the history of inputs and observed outputs. For this purpose, define the \emph{$k$-th information space} as $\I_k = (\Z \times \A)^{k} \times \Z$ with elements $i_k = (z_0,u_0,z_1,u_1,\ldots,z_{k-1}, u_{k-1}, z_k)$ referred to as the \emph{$k$-th information vector}. Based on the notion of information, an \emph{observation-based} policy for $\POMDP$ is a sequence $\polb=(\pol_0,\ldots,\pol_{K-1})$ such that for each $k$, $\pol_k( \mathrm{d} u_k|\init, i_k)$ is a universally measurable stochastic kernel on $\A$  given $\mathcal{P}(\X)\times \I_k$. We say that $\polb$ is \emph{non-randomized} if for all $\init$, $k$, and $i_k$, $\pol_k(\cdot|\init, i_k)$ is a Dirac distribution. % Unless otherwise mentioned, we will in the following assume that a policy represents a sequence of stochastic kernels on $\A$.

Given an observation-based policy $\pol$ and an initial distribution $\init$, the theorem of Ionescu Tulcea \citep{hll1996} implies the existence of a unique probability measure $\P_\pol^\init$ on $\Hist_\infty$, or, equivalently, a stochastic process on the probability space $(\Hist_\infty, \borel{\Hist_\infty}, \P_\pol^\init)$. For an information vector $i_k$, state knowledge can be expressed as a conditional probability distribution $b_k( \mathrm{d} x )=\P(x_k\in \mathrm{d} x |\init,i_k)\in \mathcal P (\X)$. The distribution $b_k$ is called the \emph{belief state} and the set of all beliefs is the \emph{belief space} $\Xb\subset \mathcal P(\X)$. It can be shown that the belief state evolves based on a fixed stochastic kernel
\begin{align}
\label{eq:trb}
	 b_{k+1} \sim \trb(\cdot|b_k,u_k).
\end{align}

Since \eqref{eq:trb} is a completely observable system it follows that a POMDP $\POMDP$ can equivalently be expressed as an MDP $\MB(\POMDP) = (\Xb, \initb, \trb, \A)$ in belief space \citep{bertsekas2004stochastic}. In the sequel we will omit $\POMDP$ and simply write $\MB = \MB(\POMDP)$ for the belief model. The information vector for $\MB$ at time $k$ is given as $i_k=(b_0, u_0, b_1, u_1, \ldots, b_{k-1}, u_{k-1}, b_k)$. Thus a policy for $\MB$ is a sequence $\polb=(\pol_0,\ldots,\pol_{K-1})$ such that for all $k$, $\pol_k(\mathrm{d}u_k|\initb, i_k)$ is a universally measurable stochastic kernel on $\A$.	We say that a policy $\polb$ is a \emph{Markov policy} if for each $k$ it depends only on the current state, i.e., $\pol_k(\mathrm{d} u_k|\initb, i_k) = \pol_k(\mathrm{d} u_k| b_k)$. Furthermore $\polb$ is a \emph{stationary Markov policy} if $\pol_k(\mathrm{d} u| b)=  \pol(\mathrm{d} u| b)$  for all $k$ and $b$ with $\pol:\mathbb B\rightarrow \mathcal{P}(\A)$ a universally measurable stochastic kernel. %\pol_k( \cdot |b)$ for all $k$ and $b$.


\subsection{Linear Temporal Logic for Belief MDPs}

In this work we are interesting in controlling state evolution under uncertainty to guarantee properties of the belief state. First we introduce atomic propositions in belief space to give examples of the types of behavior that can be quantified. These propositions are the basic building blocks from which temporal specifications are then constructed.

\subsubsection{Atomic propositions in belief space:}

An atomic proposition $p_i$ \red{[S: introduced subscript $i$ to avoid notational conflict]} is associated with a subset of the belief space $\Xb$. While belief spaces are generally infinite-dimensional, Gaussian distributions are uniquely characterized by the mean and variance. That is, the belief state is $b = (\hat x, P)$ and takes values in the finite-dimensional belief space $\mathbb{R}^n \times \mathbb{S}^n$. In this case, examples of atomic propositions in belief space are:
\begin{enumerate}
  \item a position-based proposition $p_1 = \{ (\hat x, P) : \hat x \in A \}$ that evaluates to true when the state mean is in a set $A \subset \mathbb{R}^n$;
  \item an uncertainty-based proposition $p_2 = \{ (\hat x, P) : \det(P) \leq c \}$ that is true if the determinant of the variance is small;
  \item a proposition $p_3 = \{ (\hat x, P) : \int_{A} \mathbf{P}( \mathrm{d} x \mid \hat x, P) \geq c \}$ which lower bounds the probability of an event $A$.
\end{enumerate}
\red{[S: I am not sure whether it is smart to define AP as sets over the state space. Normally one considers the AP to be indicators for these sets.]}

% Several atomic propositions can simultaneously hold at a state $(\hat x, P)$. For example, the mean $\hat x$ can belong to a given set as in (1), while the variance is also low as in (2).
Consider a set $AP = \{ p_1, \ldots, p_L \}$ of atomic propositions; it defines an \emph{alphabet} $\alphabeth := 2^{AP}$ where each \emph{letter} $\letter$ of the alphabet is defined as a set of atomic propositions. An infinite string of letters is a \emph{word} $\word=\letter_0\letter_1\letter_2\ldots\in\alphabeth^{\mathbb{N}}$.

To quantify the dynamic behavior of a system, we define the word associated to an MDP execution. Consider a labeling function $\Lab:\Xb\rightarrow \alphabeth$ that maps belief states to letters in the alphabet. We require that $\Lab$ is measurable, i.e. that $\{b|\Lab(b)\vDash p \}\in \borel{\Xb}$ for all $p \in AP$. Since Borel measurability is preserved by standard linear operations \citep{azoff1974borel}%\citep[page 116]{lang1993real}
, the above properties are all Borel measurable. Under these assumptions, the words generated by a belief trajectory $\mathbf{b} = b_0 b_1 b_2 \ldots$ can be defined as the word $\word:=\Lab(b_0)\Lab(b_1) \Lab(b_2) \ldots$. System properties can now be expressed via temporal logic formulas over the generated words.

\subsubsection{Linear temporal logic formulas:}

Properties are formulas composed of atomic propositions and operators. In this work we restrict attention to a fragment of linear temporal logic.
\begin{definition}
  \label{def:gdtl-syntax}
  Formulas in the \emph{syntactically co-safe LTL} (scLTL) fragment are constructed according to the grammar
  \begin{equation*}
    \label{eq:scLTL}
    \psi :=  \True \ |\ p \ |\ \notltl p \ |\ \psi_1 \vee\psi_2  \ |\ \psi_1 \andltl \psi_2 \ |\ \psi_1 \Until \psi_2 \ |\ \Next \psi
  \end{equation*}
  where $p\in \AP$ is an atomic proposition.
\end{definition}

% The syntax defines the symbols and their correct ordering to form a formulae.
To define the interpretation of a formula, i.e. the \emph{semantics}, consider a word $\word$ with the suffix sequences $\word_i:= \letter_i \letter_{i+1} \letter_{i+2} \ldots$

\begin{definition}
 The \emph{semantics} of scLTL are defined recursively  over $\word_i$ as
    $\word_i \models \True$,
    $\word_i \models p$ iff $p \in \letter_i$,
    $\word_i \models \psi_1 \andltl  \psi_2  $ iff $ ( \word_i \models \psi_1 ) \andltl ( \word_i \models \psi_2 ) $,
    $\word_i \models \psi_1 \orltl  \psi_2  $ iff $ ( \word_i \models \psi_1 ) \orltl ( \word_i \models \psi_2 ) $,
    $\word_i \models  \psi_1 \Until \psi_2 $ iff $\exists j \geq i \text{ s.t. } (\word_j \models \psi_2 ) $ and $\word_k \models \psi_1, \forall k \in \{i, \ldots j-1\}$,
    $\word_i \models \Next \psi$ iff $\word_{i+1} \models \psi$.
\end{definition}

We say that a belief trajectory $\mathbf{b} = b_0 b_1 b_2 \ldots$ satisfies a specification $\psi$, written $\mathbf{b} \models \psi$, if the generated word $\word =\Lab(b_0) \Lab(b_1) \Lab(b_2) \ldots$ satisfies $\psi$ at time 0, i.e. $\word_0 \models \psi$.


\subsection{Problem statement}

With the definition above, a probability measure $\P_\init^\polb$ over executions $b_0 b_1 b_2 \ldots\in \Xb^{\mathbb N}$ also implies a probability measure over formula satisfaction. The objective of this work is to design a policy $\polb$ such that a specification $\psi$ is satisfied with a given probability.

\begin{problem}
\label{prob:main}
  Consider a belief model $\MB = (\Xb, \rho, \trb, \A)$ generating trajectories $\mathbf{b} = b_0 b_1 b_2 \ldots$, a labeling function $\Lab : \Xb \rightarrow AP$, and a scLTL formula $\psi$ defined over $AP$. Construct a policy $\polb$ such that
  \begin{equation}
    \P_\init^\polb ( \mathbf{b} \models \psi )\geq p,
  \end{equation}
  where $p$ is either given or to be maximized.
\end{problem}


\section{Control synthesis for scLTL formulae}
\label{sec:exact_synth}

An important property of the scLTL fragment is that the satisfaction of a formula can be expressed as a reachability property in a deterministic finite-state automaton.

\begin{definition}
  A \emph{deterministic finite-state automaton} (DFSA) is a tuple $\FSA = (Q, q_0, \Sigma, \delta_\FSA, Q_f)$, where $Q$ is a finite set of states, $q_0 \in Q$ is a initial state, $\Sigma$ is a input alphabet, $\delta_\FSA : Q \times \Sigma \rightarrow Q$ is a transition function, and $Q_f\subseteq Q$ is a set of accepting states.

  A word $\word = \letter_0 \letter_1 \letter_2 \ldots$ \textbf{is accepted} by the DFSA if there exists a sequence $q_0 q_1 q_2 \ldots q_f$ with $q_f\in Q_f$, that starts with the initial state $q_0$ and for which $q_{i+1}=\delta_{\CA A}(q_i,\letter_i)$. In other words, a sequence of letters is accepted if the corresponding trace in the DFSA reaches the set of accepting states. We denote the set of words accepted by a DFSA $\CA A$ as $\Language (\mathcal A)$.
\end{definition}

For every property $\psi$ expressed as a syntactically co-safe LTL formula \eqref{eq:scLTL}, there exists a DFSA  $\FSA_\psi$ that models the same property \citep{Belta2017}. In particular, $\word\models\psi$ if and only if $\word\in \Language(\FSA_\psi)$. We can therefore reason about satisfaction of properties on $\MB$ by analyzing a product system $\MB \otimes \FSA_\psi$. As shown in \citep{tmka2013}, also this system is an MDP.

\begin{definition}
\label{def:product}
  Given an MDP $\MDP = (\X, \init, \tr, \A)$, a finite alphabet $\Sigma$, a labeling function $\Lab : \X\rightarrow\Sigma$, and a DFSA  $\FSA_\psi = (Q, q_0, \Sigma, \delta_\FSA, Q_f)$, the \textbf{product} between $\MDP$ and $\FSA_\psi$ is another MDP $\MDP\otimes\FSA_\psi = (\X \times Q, \bar\init, \bar{\tr}, \A)$, where $\bar\init(dx,q) = \init(dx)$ if $q= \delta_\FSA(q_0,\mathsf L(x))$ and $ \bar\init(dx,q) =0$ otherwise, and the transition kernel is similarly given as
  \begin{equation*}
    \bar{\tr}(d x'\times\{q'\}|x,q,u) = \begin{cases}\tr(dx'|x,u)& \text{if } q' =\delta_{\FSA}(q,\Lab(x')),\\ 0 & \text{otherwise.}  \end{cases}
  \end{equation*}
\end{definition}

From this construction it follows that a policy $\polb$ for $\MB$ enforces a specification $\psi$ with probability $p$ if and only if $\polb$ for $\MB \otimes \FSA_\psi$ reaches the set $\Xb \times Q_f$ with probability $p$. In other words, Problem \ref{prob:main} can be converted into the problem of constructing a reachability-enforcing policy on $\MB \otimes \FSA_\psi$. In the following we review the reachability problem on MDPs.

\subsection{MDP reachability via value iteration}

Consider an MDP $\MDP = (\X, \rho, t, \A)$ and a set of accepting states $X_f \subset \X$. We are interested in constructing a policy $\pol$ that maximizes the possibility of reaching $X_f$. To this end, we define for a stationary policy $\polb$ the time-dependent value function $\mathbf V_\polb^K:\X\rightarrow [0,1]$ as
\begin{equation*}
\label{eq:Valfunc}
  \mathbf V^K_\polb(x)=\mathbb{E}^\polb \left[ \sum\limits_{ i=0 }^K \ind_{X_f}( x_i) \prod \limits_{\mathclap{j = 0}}^{i-1}\ind_{ \X \setminus X_f}( x_j ) \bigg| x_0 = x \right].
\end{equation*}
\textcolor{red}{As shown in \citep{Abate1}}, $\mathbf V_\polb^K (x)$ expresses the probability that a trajectory generated by $\polb$ starting from $x$ will reach the target set $X_f$ within a time horizon $K$.

Next express the associated Bellman operator $\Bel_\pol$ as
\begin{align}
\label{eq:V_recopt}
  & \Bel_\pol (\mathbf  V)(x) = \int_\X \max \left( \mathbf 1_{X_f}( x'), \mathbf V(x') \right) {\tr}(d  x' | x,\pol( x)),
\end{align}
from where it follows that $\mathbf V_\polb^{K} = \Bel_\pol \mathbf V_\polb^{K-1}$. Thus if $\mathbf V_\polb^{K-1} $ expresses the probability of reaching $X_f$ within $K-1$ steps, then $ \Bel_\pol \mathbf V_\polb^{K} $ expresses the probability of reaching $X_f$ within $K$ steps. It follows that the infinite-horizon reachability probability distribution can be computed as $\lim_{K\rightarrow \infty}\mathbf T_\mu^{K} \mathbf V^0$ with $\mathbf{V}^0 \equiv 0$.

Instead of defining the recursions for a given policy $\polb$, we can also optimize with respect to the set $\mathbf D_{\pol}$ of universally measurable deterministic policies. This yields the policy-optimal Bellman recursion as
\begin{equation}
  \Bel_{\!\ast} \!(\mathbf V)(x) \!=\!\!\sup_{\pol \in \mathbf D_{\pol}}\! \int_{\X}\! \max\! \left( \mathbf 1_{X_f}\!( x'),\! \mathbf V(x') \right)\! {\tr}(d  x' | x,\pol( x)).\!\label{eq:opt_bell}
\end{equation}
From \cite{Abate1} we know that the policy $\polb^\ast$ optimizing  the reachability recursions \eqref{eq:opt_bell}  is a stationary, universally measurable, and deterministic policy.


\subsection{Policy synthesis}
Policy synthesis for a scLTL specification over a belief state space $\Xb$ is equivalent to reachability of $\Xb\times Q_f$ on the product system $\MB \otimes \FSA_\psi$. In this case, the Bellman operator \eqref{eq:V_recopt} reduces to
\begin{equation}
\label{eq:V_recopt_inf_mu}
%\begin{aligned}
  \Bel_\pol (\mathbf V)(b,q)\! = \hspace{-1mm} \int_{\Xb} \hspace{-1mm} \max \left( \mathbf 1_{Q_f}(q'), \!\mathbf V( b'\!, q') \right)\! {\tr}(\mathrm{d} b' |b,\pol(b,q))\!
%\end{aligned}
\end{equation}
with {\color{orange}the implicit DFSA transitions  $q' =\delta_\FSA(q,\Lab(b'))$}, %\red{$\mathrm{d} b'= d b'\times\{q'\}$}, 
and similarly for the policy-optimal version $\Bel_\ast$. The stationary policy $\pol$ is defined for the product MDP, i.e, it defines a control action as a function of the state $(b,q) \in \Xb \times Q$.  A stationary policy %$\pol$ 
for the product MDP $\MB \otimes \FSA_\psi$ can be translated to a time-dependent policy $\polb$ for the original belief MDP $\MB$ that includes $\FSA_\psi$ as a memory model.

Even though this section outlines a solution method to Problem \ref{prob:main}, the required recursions can only be computed for a limited set of systems.
Computation is especially challenging for high-dimensional continuous-state belief MDPs. In the sequel, we tackle this issue by constructing an approximate model $\tilde \MB$ that is close to the concrete model $\MB$ in a quantified sense, but which is more amenable to policy synthesis.


\section{Refinement-based approximate control synthesis}
\label{sec:refinement}

Let a belief MDP $\MB$ and an abstract model $\tilde \MB$ be given. In this section we show how we can leverage a policy for the abstract model to control the concrete model with inherited guarantees. For this we first quantify the difference between $\MB$ and $\tilde \MB$ and design a control policy for $\tilde \MB$ that satisfies a property robustly with respect to the difference. Finally, we show a novel way of constructing a refined policy for the concrete belief MDP.

% We approach the control synthesis of the belief MDP in three steps:
% \begin{enumerate}
%   \item We quantify the similarity of the models via the novel concept of labeling-based $\delta$-stochastic simulation relations that quantify the difference between $\tilde{\MB}$ and $\MB$.
%   \item We augment the synthesis procedure with robustness properties that compensate for the difference, which allows a robust policy constructed for $\tilde \MB$ to be refined to a correct policy for$\MB$.
%   \item We give a novel construction of control refinement based on a robustified value function from step (2).
% \end{enumerate}


\subsection{Lifting-based simulation and approximate similarity}

Let $\X_1,\X_2$ be two sets with associated measurable spaces $(\X_1,\mathcal B(\X_1)),$ $(\X_2,\mathcal B(\X_2))$, and let $\mathbf{P}_1 \in \mathcal{P}(\X_1,\mathcal B(\X_1)) $ and $\mathbf{P}_2 \in \mathcal{P}(\X_2,\mathcal B(\X_2)) $ be two probability measures. Suppose that $\rel \subset \X_1 \times \X_2$ defines a set that captures pairwise similarity between $x_1 \in \X_1$ and $x_2 \in \X_2$, then also similarity between $\mathbf{P}_1$ and $\mathbf{P}_2$ can be quantified as in \cite{haesaert2017verification}.
%
\begin{definition}
\label{def:dellifting}
  For a given relation $\rel\in \mathcal B(\X_1\times \X_2)$, we say that  $\mathbf{P}_1$ and $\mathbf{P}_2$ are in the corresponding \textbf{$\delta$-lifted relation} $\bar \rel_\delta$, written $\mathbf{P}_1 \bar \rel_\delta \mathbf{P}_2$, if there exists a \textbf{lifting} $\mathbb{W}$ which is a probability distribution on $(\X_1\times \X_2,\mathcal B(\X_1\times \X_2))$ such that { \setlength{\parskip}{-1pt}\setlength{\parsep}{0pt}
		\begin{description}
			\item[\textbf{L1.}] for all $X_1\in \mathcal{B}(\X_1)$: $\mathbb W(X_1\times \X_2)=\mathbf{P}_1(X_1)$;
			\item [\textbf{L2.}] for all $X_2\in \mathcal{B}(\X_2)$:  $\mathbb W(\X_1\times X_2)=\mathbf{P}_2(X_2)$;
			\item[\textbf{L3.}] for the probability space  $(\X_1\times \X_2,\mathcal B(\X_1\times \X_2), \mathbb W)$ it holds that
			$x_1\rel x_2$ with probability at least $1-\delta$, or, equivalently, that $\mathbb{W}\left(\rel\right)\geq1-\delta$.
	\end{description}}
\end{definition}
% \subsubsection{Preservation of labels:} As a first requirement, we require that $\rel$ is proposition-preserving, i.e. that if an atomic propositions holds for $\MB$ it should also hold for all related states in $\tilde \MB$. To allow for approximate relations where atomic proposition can be ambiguous, we consider set-valued labellings for the abstract model via a set-valued function $ \tilde{\Labset}(\tilde b):\Xb\rightarrow2^\alphabeth$. We require that for all $(\tilde b, b) \in \rel$,
% \begin{equation}
% \label{req:lab}
%   \Lab(b)\in \tilde{\Labset}(\tilde b),
% \end{equation}
% where  $\Lab:\Xb\rightarrow\alphabeth$ is the labeling function of the concrete belief MDP $\MB$. That is, for every concrete state we require that the correct labeling is present in every associated abstract state.
% Consider the trivial set-valued extension of the labeling function $\Lab : \Xb \rightarrow \alphabeth$, that is, $\Labset: \Xb \rightarrow2^\alphabeth$ with $\Labset(b)=\{\Lab(b)\}$.
% Requirement \eqref{req:lab} can then be equivalently expressed as
% \begin{equation}
%   \forall (\tilde b,b )\in \rel:  \Labset(b)\subset \tilde{\Labset}(\tilde b).
% \end{equation}
% \subsubsection{Initial condition similarity:} Secondly, we require that there exists a $\delta$-lifting for the initial distributions $\initb$ and $\tilde\init$, that is,
% \begin{equation}
% \tilde\init \bar \rel_\delta \init.
% 	\tag{\textbf{SR 1.}}
% \end{equation}
%
% \subsubsection{Transition similarity:} Finally, we also require that the stochastic transitions of the belief MDPs are approximately similar.  We require that any control action of the abstract MDP can be \emph{refined} to the concrete MDP in a way such that the stochastic transitions are approximately similar. Levering the $\delta$-lifting  we can  express this as follows:
% \begin{equation}
%   \label{req:translifting}
% 	\forall (\tilde x, x)\in \rel, \forall \tilde u \in \tilde A, \exists u \in \A\; \text{s.t.} \; \tilde \trb(\cdot| \tilde x, \tilde u)\ \bar \rel_\delta \  \trb(\cdot| x, u).
% \end{equation}
%
% To make sure that requirement \eqref{req:translifting} is constructive for the control refinement, we need a more strict condition on the measurability.  Hence, we require that there exists a Borel measurable stochastic kernel $\Wt(\,\cdot\,{\mid} \tilde u,\tilde x,x)$ on $\tilde \Xb\times\Xb$ such that $\forall (\tilde x,x)\in \rel$, $\forall \tilde u\in\tilde \A$:
% \begin{equation}
%   \tilde \trb(\cdot| \tilde x, \tilde u)\ \bar \rel_\delta \  \trb(\cdot| x, \InF(\tilde u,\tilde x,x)),\tag{\textbf{SR 2.}}
% \end{equation}
% with respect to $\Wt$ and a given Borel-measurable interface function $\InF: \tilde \A\times \tilde \Xb\times\Xb \rightarrow \mathcal{P}(\A,\mathcal B(\A))$ that maps the abstract action and state pair to refined actions for the concrete system.
%
% Based on these requirements we can define the concept of a labeling-based $\delta$-stochastic simulation relation.
%
The difference between two probability distributions can be quantified with respect to a relation $\rel$ based on Definition \ref{def:dellifting}. For stochastic kernels we add a measurability condition.
\begin{definition}\label{def:kernellift}
		\new{Borel measurable stochastic kernels $\tr_1(\cdot {\mid} x)$ and $\tr_2(\cdot {\mid} x)$ are in a \ $\delta$-lifted relation $\bar \rel_\delta$ for a set $x\in A$, i.e.,  $\tr_1(\cdot \mid x) \bar \rel_\delta \tr_2(\cdot {\mid} x)\, \forall x\in A$,  if there exists a lifting $\mathbb W_t$ that is a  Borel measurable stochastic kernel for which Def. \ref{def:dellifting} holds. }
\end{definition}
We next apply this concept to quantify the difference between two belief MDPs $\tilde \MB$ and $\MB$. In particular, we give requirements for the two systems  to be approximately stochastically simulated with respect to a relation $\rel\subset \tilde \Xb \times \Xb$. In the next section, we will give a concrete example of such a relation for the special case of Gaussian LTI POMDP systems. To allow for approximate relations where atomic proposition can be ambiguous, we consider set-valued labelings $\Xb\rightarrow2^\alphabeth$. The trivial set-valued extension of the labeling function $\Lab : \Xb \rightarrow \alphabeth$ for the concrete MDP $\MB$ is $\Labset: \Xb \rightarrow2^\alphabeth$ with $\Labset(b)=\{\Lab(b)\}$.

\begin{definition}
\label{def:apbsim}
Consider a concrete belief MDP $\MB = (\Xb, \rho, \trb, \A)$ and an abstract MDP $\tilde\MB = (\tilde \Xb, \tilde \rho, \tilde \trb, \tilde \A)$, with (set-valued) labeling maps $\Labset$ and  $\tilde{\Labset}$. We say that	$\tilde \MB$ is \textbf{$\delta$-stochastically simulated} by $ \MB$ with respect to $(\tilde{\Labset},\Labset)$, written
\begin{equation}
 \tilde {\MB}\preceq^{\delta}_{\tilde{\Labset},\Labset}  \MB,
\end{equation}
\red{[Order was wrong?,  Control defines the non-determinism wrt which we order the relation.]}
if there exists a \red{Borel measurable interface function $\InF : \tilde \A \times \tilde \Xb \times \Xb \rightarrow \mathcal P(\A)$} and a Borel-measurable relation $\rel\subseteq \tilde \Xb\times \Xb$, for which for all  $ \tilde u\in \tilde \A$, and for all $ (\tilde b,b)\in \rel$
\begin{align}
  \label{eq:sr-1} & \tilde\init \bar \rel_\delta \init, \tag{\textbf{SR 1}} \\
  \label{eq:sr-2} &  \tilde \trb(\cdot| \tilde b, \tilde u)\ \bar \rel_\delta \  \trb(\cdot| b, \InF(\tilde u,\tilde b,b)), \tag{\textbf{SR 2}} \\
  \label{eq:sr-l} & %\forall (\tilde b,b)\in \rel:
   \Labset(b)\subseteq \tilde{\Labset}(\tilde b).\tag{\textbf{SR$\,\boldsymbol{\CA{L}}$}}
\end{align}

\end{definition}

Condition \eqref{eq:sr-1} enforces $\delta$-probabilistic similarity between the initial distributions, while \eqref{eq:sr-2} ensures that the transition kernels of the two MDPs are approximately matched via the interface. Finally, \eqref{eq:sr-l} guarantees that any label $\Labset(b)$ of a concrete state $b$ is also present in the abstract label $\Labset(\tilde b)$. Thus, any word generated by $\MB$ will also be present in the possible behaviors of $\tilde \MB$.
\\
In \citet{haesaert2017verification} the difference between models is quantified based both on a probabilistic difference as above, and on a metric difference in an output space. To facilitate working with belief space models this paper implements this novel alternative notion of simulation relation that instead of a metric error incorporates non-determinism in the labeling.
\new{Next, we show that a  robust policy synthesis, introduced in \cite{tech_report_TACAS}, can used for the new labeling based approximate stochastic simulation relations.}

\subsection{Robust policy synthesis}

As a means of compensating for the differences between a concrete system and its abstraction, we proceed by introducing a robust quantification of the probability that an scLTL property is satisfied. Policies synthesized for the abstraction with the robustified procedure can be refined to the concrete system while preserving probabilistic guarantees. A robustified synthesis procedure must incorporate both the non-determinism in the labeling function, as well as the difference $\delta$ in probability. The former can be dealt with by introducing an additional minimization in the Bellman operator \eqref{eq:V_recopt_inf_mu} that effectively considers the worst-case resolution of non-determinism
\begin{align}
  \Bel_\pol^\Labset (\mathbf V)(b,q) = \hspace{-1mm} \int_\Xb
  \min_{q' \in \delta_\FSA(q, \Labset( b'))} \hspace{-3mm} \max \left( \mathbf 1_{Q_f}(q'), \mathbf V( b',q') \right)  \notag\\[-.4em]
   \times {\tr}(d b'|b,\pol(b,q)).\label{eq:V_recopt_nondet}
\end{align}
The policy-optimal Bellman operator $\Bel_\ast^\Labset$ is constructed analogously. Secondly, we compensate for the difference in probability $\delta$ by introducing the robust Bellman operator
\begin{equation}
  \label{eq:V_recopt_nondet_rob}
  \BelR_\pol^{(\Labset,\delta)} (\mathbf V)(b,q) = \max \left( 0, \Bel_\pol^\Labset (V)(b,q) - \delta  \right),
\end{equation}
and similarly for the policy-optimal Bellman operator $\BelR_\ast^{(\Labset, \delta)}$. The robustified optimal probability that an execution $\tilde{\mathbf{b}}$ of $\tilde \MB$ reaches $Q_f$, and hence satisfies $\psi$, is therefore given as
\begin{equation}
\label{eq:prob:robust_optimal}
\begin{aligned}
  &\mathbf{P_{\! rob}}_\init^{\!\ast} (\tilde{\mathbf{b}} \vDash \psi; \Labset,\delta) =\\ & \int_{\Xb}\  \min_{\ \ \bar q \in \delta_\FSA(q_0,\Labset(b_0))}  \max \left( \mathbf 1_{Q_f}(\bar q), \mathbf W^\infty_\ast(b_0,\bar q) \right) \initb(\mathrm{d} b_0) -\delta
\end{aligned}
\end{equation}
with
$\mathbf W^\infty_\ast \!:=  \lim_{N\rightarrow\infty}[\BelR_\ast^{(\Labset,\delta)}]^N\! (\mathbf W^0)$ and $\mathbf W^0\equiv0$. Remark this robust probability is parameterized with the set-valued labeling and the $\delta$-difference in the stochastic transitions.
\\
From the robust synthesis results in \citep{tech_report_TACAS},
we know  that there exists a refined control policy with inherited guarantees.\begin{prop}\label{prop:1}
  Suppose that $\MB$ is $\delta$-stochastically simulated by $\tilde \MB$ with respect to $({\tilde{\Labset},\Labset})$, i.e., ${\MB}\preceq^{\delta}_{\tilde{\Labset},\Labset} \tilde \MB$. Then, a control policy ${\tilde \polb}^\ast$ for $\tilde \MB$ can be refined to a control policy $\polb$ for $\MB$ such that %for all joint executions $\mathbf{\tilde b}$ and $\mathbf{b}$
  \begin{equation}
  \label{eq:prob_result}
    \mathbf{P_{\! rob}}_{\tilde \init}^{\!\ast}(\mathbf{\tilde b} \vDash\psi; \tilde\Labset,\delta)\leq  \mathbf P _{\init}^{\polb}( \mathbf{b} \vDash\psi).
  \end{equation}
\end{prop}



\subsection{Control refinement}
\label{sec:control}

Based on Proposition \ref{prop:1}, we already know that a refined policy for the concrete model can be constructed. In \cite{tech_report_TACAS}, this refined policy is constructed via the composition of the conditional lifting %of the $\delta$-stochastic simulation relation
 (see Def. \ref{def:kernellift}) and based on the abstract policy $\tilde  \polb^\ast$. Here we improve this construction in a way that is independent of the  lifting. Given $\mathbf{W}^\infty_\ast$ and abstract policy $\tilde \polb^\ast$, a refined concrete policy over the product MDP $\MB\otimes\FSA_\psi$ is obtained as
\begin{equation*}
		\polb_\ast(x,q): = \InF( \tilde{\polb}(\tilde x^*, q), \tilde x^*, x), \quad \tilde x^*:=\argmax_{ \tilde x\in
		\rel^{-1}(x)} \mathbf  W^\infty_\ast(\tilde x,q).
\end{equation*}
\new{We remark that this construction is allowed, as it effectively implements a one step improvement of the lower bound on the reachability probability.
Additionally, we can assume without loss of generality that no measurability issues are introduced as the maximization is performed over a subset of the state space of the abstracted model, which in general  has a finite domain.} Remark that as a corollary of Proposition \ref{prop:1}, it follows that the probability that an scLTL property is satisfied is still lower bounded as in inequality \eqref{eq:prob_result}.


\section{Abstractions for Gaussian LTI POMDPs}
\label{sec:gaussian}

We now consider the special case of POMDPs that can be modeled as Linear Time Invariant (LTI) models and give a concrete model abstraction. Consider an LTI system
\begin{equation}
  \label{eq:LTI}
    \begin{aligned}
    x_{k+1}&=A x_{k} + B u_k+ w_k, & w_k\sim \mathcal N(0,\CA W), \\
    z_k&=Cx_k+v_k, & v_k\sim \mathcal N(0,\CA V).
  \end{aligned}
\end{equation}
Equation \eqref{eq:LTI} defines a \new{POMDP} with state space $\X\subseteq\mathbb R^n$, initial distribution $\init:=\mathcal N(\hat x_\init,P_\init)$, control inputs $\A \subseteq \mathbb R^m$, and a transition kernel $t(\cdot | x, u ) = \mathcal N(Ax + Bu, \CA W)$, together with an observation model $r(\cdot \mid x) = \mathcal N(Cx, \CA V)$ generating noisy measurements $z_k$.
\\
It is well known that the belief state $(\hat x_{k \mid k}, P_{k \mid k})$ for this type of system evolves over the space of Gaussian distributions, and is given by the Kalman filter equations. In particular,
\begin{align}
  \label{eq:beliefx} \hat x_{k|k}&=A\hat x_{k-1|k-1}+Bu_{k-1}+P_{k|k-1}C^Ts_k \\
  \label{eq:beliefP} P_{k|k}&= (I - L_k C) P_{k|k-1},
\end{align}
where
\begin{equation}
\begin{aligned}
  &P_{k|k-1}=AP_{k-1|k-1}A^T+\mathcal W, \\
  &L_k = P_{k|k-1}C^T\left(CP_{k|k-1}C^T+\mathcal V\right)^{-1}, \\
 &\quad s_k\sim \mathcal N (0, S_k ), \quad \red{S_k = CP_{k|k-1}C^T+\mathcal V} .
\end{aligned}
\end{equation}

These equations together define a belief model $\MB$. Since the computation of the backwards recursions \eqref{eq:V_recopt_inf_mu} is intractable over the continuous space of this system we construct a finite-state abstraction that is related to $\MB$ in the sense of Definition \ref{def:apbsim}. As a first step, we choose to remove the dynamics of the covariance matrix $P$. Furthermore, we replace the stochastic transitions in equation \eqref{eq:beliefx} by
\begin{equation}
		\hat x_k  = A\hat x_{k-1} +B\hat u_{k-1} + \bar P  C^T  \hat{s}_k,\label{eq:abstract}
\end{equation}
with $ \hat{s}_k\sim \CA N (0,\hat{S}_{inv})$ and $\hat{S}_{inv}\preceq S_k^{-1}$ for all $k$, where $\bar P$ defines the steady state $P_{k|k-1}$, i.e., the solution of the Kalman equations. We say that system \eqref{eq:abstract} has state space $\Xb_{x}$ and abstract the system further to a finite-state model $\tilde \MB_{grid}$ by gridding $\Xb_x$. As illustrated in Figure \ref{fig:grid}, the abstract model $\tilde \MB_{grid}$ has a finite set of states $s \in \mathbb{S}$, each associated with a representative state $x_s \in \Xb_x$ distributed equidistantly over the state space $\Xb_x$. Each finite state $s$ is the representative state for a cell $\Delta_{s}= \ball_\eta(x_s)$ where $\grid$ is a vector-valued precision parameter that defines the coarseness of the gridding. In addition, we require that the cells cover the whole state space, that is, $\Xb_x \subseteq \bigcup_{s \in \mathbb{S}} \Delta_{s}$.

The transition kernel of $\hat\MB_{grid}$ is given as $t_{grid}(s'|s,u)=\hat t \left(\Delta_{s'}\mid x_s, u\right)$, where $\hat t$ is the stochastic transition kernel associated with \eqref{eq:abstract}. Thus, the probability of transitioning from $s$ to $s'$ is given by the probability of transitioning from the representative \emph{state} $x_s$ of $s$ to the representative \emph{cell} $\Delta_{s'}$ of $s'$.

\begin{figure}[htp]
\centering
	\includegraphics[width = .6\columnwidth]{figs/grid}
	\caption{Depiction of the gridded state space. Depicted are the mean state $\hat x$ of the concrete belief MDP ($\bullet$), and the representative states $x_s$ of the abstract MDP ($\boldsymbol{+}$), together with an illustration of a stochastic transition. \textcolor{red}{should add $x_s, s, \Delta_s$, etc to figure} }
  \label{fig:grid}
\end{figure}

We consider simulation relations $\sim$ between states $(\hat x, P)$ of $\MB$ and states $s$ of $\hat \MB_{grid}$ and interfaces $\InF$ defined on the forms
\begin{align}
  \label{eq:rel}
    & \rel = \left\{(s, (\hat x, P ) ) {\mid}\begin{aligned}
      (\hat x-x_s)^T M(\hat x-x_s)\leq \eps^2 \\
      P^-\preceq P \preceq   P^+
    \end{aligned} \right\}, \\
    & \InF(\tilde u,  x_s, \hat x)  =K( \hat x - x_s)+\tilde u,
\end{align}
for some matrices $M, P^-, P^+$ and $K$. We now derive the conditions under which $\rel$ and $\InF$ define a labeling-based $\delta$-stochastic simulation relation between $\hat \MB_{grid}$ and $\MB$. To do so, we construct a set-valued labeling $\hat \Labset: \mathbb{S} \rightarrow 2^\alphabeth$ and compute $\delta$ such that $\MB\preceq^\delta_{\hat\Labset,\Labset}\hat \MB_{grid}$.

\subsubsection{Labeling requirement:}

We construct a set-valued labeling function $\hat \Labset: \mathbb{S} \rightarrow 2^\alphabeth$ satisfying \eqref{eq:sr-l}.

For a position-based proposition $p_i$ consider, without loss of generality, a labeling $\Lab_{p_i}: \Xb \rightarrow \{\{p_i\},\emptyset\}$ for the concrete belief MDP defined by $p_i\in\Lab_p \left( (\hat x,P) \right)  \Leftrightarrow \hat x \in A$. The set-valued extension to $2^{\{\{p_i\},\emptyset\}}$ for the abstract MDP is defined as
 \begin{align}
 	\hat \Labset_{p_i}(s) =\left\{\begin{array}{ll} \{\{p_i\}\} & \mbox{ if } \forall \hat x \in \rel (s):p\in\Lab_{p_i}(b), \\
 	  \{\emptyset \} & \mbox{ if } \forall \hat x \in \rel (s):{p_i}\not\in\Lab_{p_i}(b),\\
 	  \{\{p_i\},\emptyset\}&\mbox{ otherwise.}\end{array} \right.
 \end{align}
The labeling $\hat \Labset_{p_i}$ can be easily computed by shrinking or expanding the set $A$ and the extension to multiple atomic propositions is straightforward. Similarly, for propositions involving the variance of the current belief state any atomic property that is \textcolor{orange}{monotonic\footnote{\color{orange}Here monotonicity of a function in $P$ is defined based on the preorder $\succeq $ fo the matrices $P$ with $A\succeq B$ if $x^T(A-B)x\geq0\, \forall x\in\mathbb R^n$. } in $P$} can be mapped to the abstract model. However, atomic propositions that include probability require caution since the quantification of probability of an event is in general not \red{monotonic} with respect to variance.


\subsubsection{Probability requirements:}

In the interest of satisfying \eqref{eq:sr-1}, remark that the initial condition for the concrete system $\MB$ is given deterministically as $(\hat x_\init, P_\init)$. Hence we only need to show that there exists an initial state $x_{s,0}$ for $\tilde \MB_{grid}$ such that \red{$x_{s,0} \sim (\hat x_\init, P_\init)$}, which is true if  $P^-\preceq  P_\init \preceq P^+$ and $x_{s,0}\in \rel^{-1}( \hat x_\init )$. For the latter not to be an empty set it is sufficient if for all $x$ with $x \in \ball_\grid(0)$ it holds that $x^T M x \leq \eps^2$.

To show satisfaction of \eqref{eq:sr-2}, we need to show that there exists a $\delta$-lifting. First we require that $P^+$ (resp. $P^-$) is an upper (resp. lower) bound for $P_{k|k}$ of the belief MDP \eqref{eq:beliefx}-\eqref{eq:beliefP}.  We say that $P^-$ is a lower bound if it is a lower bound for the initial condition (see above) and   if it is monotonically increasing with respect to the Riccati equations \citep{bitmead1985monotonicity}. For the upper bound, we require, mutatis mutandis, a monotonically decreasing $P^+$.

We can quantify the difference between $\MB$ and \eqref{eq:abstract} by verifying that for all  $(\hat x_k,\hat x_{k|k})\in \rel$, with probability at least $1-\delta$ it holds that $(\hat x_{k+1},\hat x_{k+1|k+1})\in \rel$. The difference between the states in \eqref{eq:abstract} and \eqref{eq:beliefx} evolves according to
\begin{equation}
\begin{aligned}
  \hat x_{k+1|k+1}-  \hat x_{k+1}=(A+BK)(\hat x_{k|k}-\hat x_{k-1}) \\
  +  \bar P   C^T s^\Delta_{k+1} +\Delta_{k+1}( \red{\hat{s}}_{k+1}+ \red{s}^\Delta_{k+1}),
\end{aligned}
\end{equation}
with $\Delta_{k+1}:=(P_{k+1|k}C^T-  \bar P C^T)$, $ \hat{\red{s}}_{k+1}\sim \CA N (0,\hat{S}_{inv})$, and $ \red{s}^\Delta_{k+1}\sim  \CA N (0,\  S_{k+1}^{-1}-\hat{S}_{inv})$.
Furthermore, for all $\hat x_{k+1}$, there exists $x_\grid \in \ball_\grid (0)$ such that  $\hat x_{k+1}+x_\grid = x_s$ for some $s \in \mathbb{S}$. Therefore we can write the update of the difference expression in \eqref{eq:rel} as
\begin{equation}
\begin{aligned}
 \hat x_{k+1|k+1} - x_{s_{k+1}}=(A+BK)(\hat x_{k|k}- x_{s,k})+x_\grid \\ + \bar P   C^T s^\Delta_{k+1} +\Delta_{k+1}( \hat{\red{s}}_{k+1}+ \red{s}^\Delta_{k+1}).
\end{aligned}
\end{equation}
Given that $(\hat x_{k|k}-\hat x_{s|k})$ and $x_\grid$ belong to bounded sets, we can bound the influence of the noise terms $\red{s}^\Delta_{k+1}$ and $ \hat{\red{s}}_{k+1}$   to arrive at a probability $1-\delta$ that lower bounds the probability of remaining in $\rel$.
\begin{remark}
  Here we did not explicitly write out the lifted probability space $\mathbb{W}$ of Definition \ref{def:dellifting}. It is obtained from the probability measure induced by joint evolution of \eqref{eq:beliefx} and \eqref{eq:abstract} and a compensation for the gridding error.
\end{remark}

{\color{red}
\begin{itemize}
  \item Mention interface and control refinement?
  \item Give expression for $\delta$? \color{orange} Expression for $\delta$ vs $\rel$ doesnt exist, as it is computated numerically using convex optimizations.  Instead it is possible to give an expression  of the bounded set to which $\hat s_{k+1}$ and  $\hat s^{\Delta}_{k+1}$ belong. 
\end{itemize}
}

\section{Case study}
\label{sec:case}

Motivated by space exploration applications we consider an autonomous rover tasked with identifying and collecting scientific samples in a partially unknown Mars environment.

\textbf{Rover model.} Without loss of generality, we consider a simple rover model as a point mass $x_k \in \mathbb{R}^2$ affected by stochastic disturbances. We also assume that the position of the rover cannot be measured exactly and model its dynamics as
\begin{equation*}
\begin{aligned}
  x_{k+1}&= x_{k} + u_k+ w_k,  & w_k \sim \CA N \left(0,\begin{bsmallmatrix}.4&-.2\\-.2&.4\end{bsmallmatrix}\right), \\
  z_k&= x_k+v_k, &  v_k \sim \CA N \left(0,\begin{bsmallmatrix}1&.1\\.1&1\end{bsmallmatrix}\right).
\end{aligned}
\end{equation*}

A finite abstraction $\MDP_{syst}$ is constructed as outlined in the previous section by discretizing the state space $[-10, 10]^2$ with a discretization parameter $\grid = (0.76481, 0.64426)$, and by representing the input space $[-1,1]^2$ with nine discrete inputs $\{0, 1, -1\}^2$.

{\color{red}
Say something about Fig. \ref{fig:tradeoff}}

\begin{figure}
  % This file was created by matplotlib2tikz v0.6.14.
\centering
\begin{tikzpicture}
\newlength\figurewidth
\newlength\figureheight
\setlength\figurewidth{\columnwidth}
\setlength\figureheight{.4\columnwidth}
\definecolor{color0}{rgb}{0.12156862745098,0.466666666666667,0.705882352941177}

\begin{axis}[scale =.8,
xlabel={$\delta$},
ylabel={$\epsilon$},
xmin=-0.01395, xmax=0.25,
ymin=0.8, ymax=1.45222764450879,
width=\figurewidth,
height=\figureheight,
% tick align=outside,
% tick pos=left,
x grid style={lightgray!92.026143790849673!black},
y grid style={lightgray!92.026143790849673!black},
axis y line=left,
axis x line=middle,
every axis x label/.style={
    at={(ticklabel* cs:1.05)}
}
]
\addplot [semithick, color0, forget plot]
table {%
0.001 1.4272384844875
0.0167368421052632 1.22339465219428
0.0324736842105263 1.16617411196529
0.0482105263157895 1.13191279574885
0.0639473684210526 1.10601589042324
0.0796842105263158 1.08400668945514
0.0954210526315789 1.06721503359466
0.111157894736842 1.05265893349788
0.126894736842105 1.04157018327801
0.142631578947368 1.02417176675944
0.158368421052632 1.01254639660019
0.174105263157895 1.00120158889137
0.189842105263158 0.991557539021032
0.205578947368421 0.982277989301703
0.221315789473684 0.97353496742902
0.237052631578947 0.965239656823681
0.252789473684211 0.960158902624749
0.268526315789474 0.951512372393061
0.284263157894737 0.927455284061691
0.3 0.935781247731614
};
\end{axis}

\end{tikzpicture}
  \caption{Trade-off between $\epsilon$ and $\delta$.  }
  \label{fig:tradeoff}
\end{figure}

\textbf{Environment model.} We consider sample collection in a partially unknown environment. In particular, there are two potential target regions $T_1$ and $T_2$ where the probability of encountering a sample has been assessed via satellite imagery as 0.5 and 0.6, respectively. In addition, there are two potential obstacle regions $O_1$ and $O_2$ where estimates give probabilities of 0.9 and 0.7 that the rover can traverse the area safely. For both target and obstacle regions, we assume that the true nature of the region can be determined when the rover is within a certain distance of the regions. The regions are illustrated in Figs. \ref{fig:exp1} and \ref{fig:exp2}.

The dynamics of the uncertainty in $T_1$ can be modeled as an MDP with three states \texttt{init}, \texttt{sample} and \texttt{empty}, and two actions \texttt{measure} and \texttt{none}. Initially, the state is \texttt{init}. If the \texttt{measure} action is performed, the state transitions to \texttt{sample} with probability $p$ \red{[previously used notation][Give value?]} and to \texttt{empty} with probability $1-p$. After a measurement, no further transitions occur. This type of MDP can be constructed for each region of uncertainty, and the overall environment model $\MDP_{env}$ is their parallel composition. \red{Furthermore, we model a failure probability of 0.01 at each step with an additional MDP $\MDP_{fail}$ which discourages solutions that take long time to execute.[This does not make sense to me[sofie]! Actually this delta error should pop up from the error incurred in the approximation of the Kalman filter]}

\red{[Define aggregate system, comment on approx. simulation relation. State that the parallel composition of environment with the system does  not influence the delta accuracy. ]}
\textbf{Specification.} The objective is to collect a sample while avoiding unsafe regions. To this end, we consider the scLTL specification
\begin{equation}
  \psi = \lozenge \texttt{sample} \land \left( \lnot \texttt{fail} \; \mathcal {U} \; \texttt{sample} \right),
\end{equation}
\red{[above is equivalent to $\left( \lnot \texttt{fail} \; \mathcal {U} \; \texttt{sample} \right),$ ]}
where the atomic proposition \texttt{sample} is true if the rover is in a target region that contains a sample, and the atomic proposition \texttt{fail} is defined as being true if 1) the rover is in an obstacle region that contains an obstacle, or 2) $\MDP_{fail}$ is in failure mode.

\textbf{Results.} We consider the aggregate system $\MDP_{syst} \times \MDP_{env} \red{\times \MDP_{fail}}$ defined as a product system analogously to Definition \ref{def:product}, which is an MDP with 9 inputs and 277182 states. Via formula \eqref{eq:prob:robust_optimal}, we synthesize a control policy that is robust to the difference between the concrete and abstract models. Figs. \ref{fig:exp1} and \ref{fig:exp2} depict the evolution of the estimated position for executions generated by the control policies, and the estimated probability that the specification can be satisfied, under two different resolutions of the unknown environment. In both experiments only $T_2$ contains a sample, and $O_1$ contains an obstacle. The difference is that in Fig. \ref{fig:exp1} also $O_2$ turns out to contain an obstacle but in Fig. \ref{fig:exp2} this region can be safely traversed. As can be seen, the policy synthesis technique elicits intelligent behaviors where the agents explore the most promising regions first, while remaining robust to noise effects.

\begin{figure}
  \footnotesize
  \setlength\figurewidth{\columnwidth}
  \setlength\figureheight{0.6\columnwidth}

  % This file was created by matplotlib2tikz v0.6.14.
\begin{tikzpicture}

\definecolor{color1}{rgb}{1,0.498039215686275,0.0549019607843137}
\definecolor{color0}{rgb}{0.12156862745098,0.466666666666667,0.705882352941177}
\definecolor{color2}{rgb}{0.172549019607843,0.627450980392157,0.172549019607843}

\begin{axis}[
xmin=-11, xmax=11,
ymin=-10.3867095347535, ymax=10.9707956921311,
width=\figurewidth,
height=\figureheight,
mark size = 0.5,
tick align=outside,
tick pos=left,
ticks=none,
x grid style={lightgray!92.026143790849673!black},
y grid style={lightgray!92.026143790849673!black}
]
\addplot [only marks, draw=black, fill=black, colormap/viridis]
table{%
x                      y
-8.000000000000000e+00 -9.000000000000000e+00
-8.281791193811282e+00 -6.953277059756937e+00
-6.609947231673338e+00 -7.304755895777043e+00
-6.447420151999264e+00 -6.984253859219061e+00
-5.930025319876885e+00 -6.296279125763103e+00
-5.476123458484502e+00 -6.049868453584115e+00
-4.680375628377898e+00 -4.982516422756326e+00
-3.356608221836718e+00 -4.767445283086258e+00
-3.850337386959414e+00 -3.882953851781119e+00
-3.888519145722516e+00 -3.480146659512000e+00
-5.320305970691908e+00 -2.277119535242806e+00
-4.379722627893722e+00 -3.023297381626749e+00
-2.966831046751210e+00 -2.008805754832921e+00
-4.696058526749307e+00 +1.019091306738940e-01
-4.628219306657025e+00 +1.300100090541607e+00
-5.641564937919989e+00 +1.796635145522382e+00
-6.307184225641469e+00 +2.288130625455811e+00
-7.217828733789998e+00 +3.143921812081303e+00
-7.301911031188041e+00 +4.255956608402661e+00
-6.401344954223247e+00 +5.270680166902570e+00
-8.171856419883056e+00 +6.563080734161671e+00
};
\addplot [only marks, draw=black, fill=black, colormap/viridis]
table{%
x                      y
+0.000000000000000e+00 -9.000000000000000e+00
+2.985780663708154e-02 -8.155463278156050e+00
+1.071409254737555e+00 -8.202590105544974e+00
+2.130577217930679e+00 -8.147034783291767e+00
+3.046653997506764e+00 -7.814782310273142e+00
+2.993503950729483e+00 -6.732513426782055e+00
+3.364980364937721e+00 -5.919601911047740e+00
+4.577267269183741e+00 -5.442531058948178e+00
+4.641924364448064e+00 -4.277469389688367e+00
+4.066522550812239e+00 -2.261074508296681e+00
+4.900884132823335e+00 -1.747839224197386e+00
+5.242948546817821e+00 -1.658932790996332e+00
+4.951650027020791e+00 -5.458682191996806e-01
+5.160727075484096e+00 +1.786499818804538e-02
+4.951600531141017e+00 +3.796585040088483e-01
+5.548111410812660e+00 +9.093914656273503e-01
+5.998289710281388e+00 +1.845859092339660e+00
+5.586468819813884e+00 +2.958893844938534e+00
+6.773322394656686e+00 +4.341573264476058e+00
+6.244801362810041e+00 +3.255339248138960e+00
+5.386058184709245e+00 +2.832977405695359e+00
+3.382447161289477e+00 +5.131303079675625e+00
+2.874337596801116e+00 +4.495954941056568e+00
+8.693708078920757e-01 +4.402472806183647e+00
-1.833677935774903e-01 +4.671535259512195e+00
-2.290906405270350e+00 +4.881507277565658e+00
-4.280927629432187e+00 +6.338912021995005e+00
-4.626650566950111e+00 +6.233071174567763e+00
-5.024226278687982e+00 +7.396015589088654e+00
-6.083092777439494e+00 +7.089329608744675e+00
};
\addplot [only marks, draw=black, fill=black, colormap/viridis]
table{%
x                      y
+8.000000000000000e+00 -9.000000000000000e+00
+7.712479407981174e+00 -9.068560218314534e+00
+7.684857446710697e+00 -9.261842777911905e+00
+9.091886701481306e+00 -9.116022915545141e+00
+7.920264049554019e+00 -8.627351053042807e+00
+7.836372046718704e+00 -7.993013882387533e+00
+6.924714511084391e+00 -7.711316244642425e+00
+6.270580096654860e+00 -7.113203236528710e+00
+4.261157395109652e+00 -5.405610734506721e+00
+4.549567306091522e+00 -4.854291172967754e+00
+5.603414469156846e+00 -4.801669493342254e+00
+5.663344946347504e+00 -4.411676549805895e+00
+6.055644152177520e+00 -4.710990344437167e+00
+5.399991387428658e+00 -2.248963208448632e+00
+4.079080783625948e+00 -3.641899917871495e-01
+3.920155202311550e+00 +8.691698572170817e-01
+3.370749568978111e+00 +3.132663534104976e+00
+3.568671650336753e+00 +3.232289113753480e+00
+4.510831005190408e+00 +3.354928317636965e+00
+5.435609201437062e+00 +3.518835438870448e+00
+5.534818929272653e+00 +4.406944686880900e+00
+3.207427357074954e+00 +6.678357019964662e+00
+1.772408135280159e+00 +5.221828658975668e+00
-2.171849729529804e-01 +5.918511271530562e+00
-1.407705408136741e+00 +7.142892522473158e+00
-2.578687393117541e+00 +5.885635198531411e+00
-4.001703186615487e+00 +7.335332717542842e+00
-4.767609190082132e+00 +7.291363002593066e+00
-6.458371658930611e+00 +8.009045427964324e+00
};
\path [draw=red, fill=red, opacity=0.5] (axis cs:2,-5)
--(axis cs:2,2)
--(axis cs:0,2)
--(axis cs:0,-5)
--cycle;

\path [draw=red, fill=red, opacity=0.5] (axis cs:2,3)
--(axis cs:2,10)
--(axis cs:0,10)
--(axis cs:0,3)
--cycle;

\path [draw=red, fill opacity=0, dashed] (axis cs:3.9,-6.9)
--(axis cs:3.9,3.1)
--(axis cs:-1.9,3.1)
--(axis cs:-1.9,-6.9)
--cycle;

\path [draw=red, fill opacity=0, dashed] (axis cs:3.9,1.1)
--(axis cs:3.9,10)
--(axis cs:-1.9,10)
--(axis cs:-1.9,1.1)
--cycle;

\path [draw=blue, fill=blue, opacity=0.5] (axis cs:-6,6)
--(axis cs:-6,9)
--(axis cs:-9,9)
--(axis cs:-9,6)
--cycle;

\path [draw=blue, fill=blue, opacity=0.5] (axis cs:9,6)
--(axis cs:9,9)
--(axis cs:6,9)
--(axis cs:6,6)
--cycle;

\path [draw=blue, fill opacity=0, dashed] (axis cs:-3.9,4.1)
--(axis cs:-3.9,10)
--(axis cs:-10,10)
--(axis cs:-10,4.1)
--cycle;

\path [draw=blue, fill opacity=0, dashed] (axis cs:10,4.1)
--(axis cs:10,10)
--(axis cs:4.1,10)
--(axis cs:4.1,4.1)
--cycle;

\addplot [semithick, color0, forget plot]
table {%
-8 -9
-8.28179119381128 -6.95327705975694
-6.60994723167334 -7.30475589577704
-6.44742015199926 -6.98425385921906
-5.93002531987688 -6.2962791257631
-5.4761234584845 -6.04986845358412
-4.6803756283779 -4.98251642275633
-3.35660822183672 -4.76744528308626
-3.85033738695941 -3.88295385178112
-3.88851914572252 -3.480146659512
-5.32030597069191 -2.27711953524281
-4.37972262789372 -3.02329738162675
-2.96683104675121 -2.00880575483292
-4.69605852674931 0.101909130673894
-4.62821930665702 1.30010009054161
-5.64156493791999 1.79663514552238
-6.30718422564147 2.28813062545581
-7.21782873379 3.1439218120813
-7.30191103118804 4.25595660840266
-6.40134495422325 5.27068016690257
-8.17185641988306 6.56308073416167
};
\addplot [semithick, color1, forget plot]
table {%
0 -9
0.0298578066370815 -8.15546327815605
1.07140925473755 -8.20259010554497
2.13057721793068 -8.14703478329177
3.04665399750676 -7.81478231027314
2.99350395072948 -6.73251342678205
3.36498036493772 -5.91960191104774
4.57726726918374 -5.44253105894818
4.64192436444806 -4.27746938968837
4.06652255081224 -2.26107450829668
4.90088413282334 -1.74783922419739
5.24294854681782 -1.65893279099633
4.95165002702079 -0.545868219199681
5.1607270754841 0.0178649981880454
4.95160053114102 0.379658504008848
5.54811141081266 0.90939146562735
5.99828971028139 1.84585909233966
5.58646881981388 2.95889384493853
6.77332239465669 4.34157326447606
6.24480136281004 3.25533924813896
5.38605818470924 2.83297740569536
3.38244716128948 5.13130307967563
2.87433759680112 4.49595494105657
0.869370807892076 4.40247280618365
-0.18336779357749 4.6715352595122
-2.29090640527035 4.88150727756566
-4.28092762943219 6.33891202199501
-4.62665056695011 6.23307117456776
-5.02422627868798 7.39601558908865
-6.08309277743949 7.08932960874468
};
\addplot [semithick, color2, forget plot]
table {%
8 -9
7.71247940798117 -9.06856021831453
7.6848574467107 -9.26184277791191
9.09188670148131 -9.11602291554514
7.92026404955402 -8.62735105304281
7.8363720467187 -7.99301388238753
6.92471451108439 -7.71131624464243
6.27058009665486 -7.11320323652871
4.26115739510965 -5.40561073450672
4.54956730609152 -4.85429117296775
5.60341446915685 -4.80166949334225
5.6633449463475 -4.41167654980589
6.05564415217752 -4.71099034443717
5.39999138742866 -2.24896320844863
4.07908078362595 -0.36418999178715
3.92015520231155 0.869169857217082
3.37074956897811 3.13266353410498
3.56867165033675 3.23228911375348
4.51083100519041 3.35492831763697
5.43560920143706 3.51883543887045
5.53481892927265 4.4069446868809
3.20742735707495 6.67835701996466
1.77240813528016 5.22182865897567
-0.21718497295298 5.91851127153056
-1.40770540813674 7.14289252247316
-2.57868739311754 5.88563519853141
-4.00170318661549 7.33533271754284
-4.76760919008213 7.29136300259307
-6.45837165893061 8.00904542796432
};
\node at (axis cs:-0.5,-3)[
  anchor=base west,
  text=black,
  rotate=0.0
]{ $R_1$};
\node at (axis cs:-0.5,7.5)[
  anchor=base west,
  text=black,
  rotate=0.0
]{ $R_2$};
\node at (axis cs:-9,7.5)[
  anchor=base west,
  text=black,
  rotate=0.0
]{ $T_1$};
\node at (axis cs:6.5,7.5)[
  anchor=base west,
  text=black,
  rotate=0.0
]{ $T_2$};
\end{axis}

\end{tikzpicture}
  \setlength\figureheight{0.3\columnwidth}

  % This file was created by matplotlib2tikz v0.6.14.
\begin{tikzpicture}

\definecolor{color1}{rgb}{1,0.498039215686275,0.0549019607843137}
\definecolor{color0}{rgb}{0.12156862745098,0.466666666666667,0.705882352941177}
\definecolor{color3}{rgb}{0.83921568627451,0.152941176470588,0.156862745098039}
\definecolor{color2}{rgb}{0.172549019607843,0.627450980392157,0.172549019607843}
\definecolor{color5}{rgb}{0.549019607843137,0.337254901960784,0.294117647058824}
\definecolor{color4}{rgb}{0.580392156862745,0.403921568627451,0.741176470588235}

\begin{axis}[
xlabel={$t$},
ylabel={$\mathbf{P}(\varphi)$},
xmin=-1.8, xmax=37.8,
ymin=0.38853456067633, ymax=1.02911740187259,
width=\figurewidth,
height=\figureheight,
tick align=outside,
tick pos=left,
x grid style={lightgray!92.026143790849673!black},
y grid style={lightgray!92.026143790849673!black}
]
\addplot [semithick, color0, forget plot]
table {%
0 0.566291945691099
1 0.566291945691099
2 0.605882976959095
3 0.612354132971019
4 0.615146535917152
5 0.620313388350838
6 0.623169341037894
7 0.632266923722546
8 0.640795190330476
9 0.64428856744584
10 0.650176429635556
11 0.660779548866506
12 0.664373678411201
13 0.667953572290617
14 0.421487160337076
15 0.419564608493884
16 0.417651962548887
17 0.424169566255033
18 0.426345484650421
19 0.431108756242508
20 0.430751120445197
21 0.433029533909929
22 0.436207218324974
23 0.436207218324974
24 0.448504360104309
25 0.462109503911388
26 0.466372172152465
27 0.468495540003818
28 0.474972821467856
29 0.480216364807438
30 0.979868029536619
31 0.988089397434187
32 0.983895865110246
33 1.00000000000003
};
\addplot [semithick, color1, forget plot]
table {%
0 0.598873106893363
1 0.598873106893363
2 0.611938797531012
3 0.620313388350838
4 0.632266923722546
5 0.641338529177884
6 0.651286575651904
7 0.653667661427169
8 0.664373678411201
9 0.667389245077004
10 0.663807726695736
11 0.670472509073791
12 0.671058796741362
13 0.680876876002895
14 0.423299602729996
15 0.428005120445258
16 0.430751120445197
17 0.431108756242508
18 0.436564779737749
19 0.448714044552832
20 0.455361601380584
21 0.454662990626655
22 0.459229983640598
23 0.463841336213989
24 0.463444874596063
25 0.471062027085172
26 0.482793941967823
27 0.479801326577619
28 0.969453775583137
29 0.983895865110246
30 1.00000000000003
};
\addplot [semithick, color2, forget plot]
table {%
0 0.601199049439663
1 0.601199049439663
2 0.611373258769312
3 0.610825977206025
4 0.61950569968835
5 0.61950569968835
6 0.628810791270583
7 0.637985226843664
8 0.641338529177884
9 0.64428856744584
10 0.64428856744584
11 0.65073129659484
12 0.657238925992285
13 0.673118973708607
14 0.680502261669169
15 0.423299602729996
16 0.425630605759241
17 0.428005120445258
18 0.432928711351226
19 0.444768945274215
20 0.442429168485533
21 0.445763301335716
22 0.452612154776499
23 0.455361601380584
24 0.459622816160484
25 0.458418097195606
26 0.460938662424354
27 0.458418097195606
28 0.461333021386082
29 0.468094526343635
30 0.475377924614331
31 0.480216364807438
32 0.970297252430312
33 0.963929119062114
34 0.979868029536619
35 0.984943443987629
36 1.00000000000003
};
\addplot [semithick, color3, forget plot]
table {%
0 0.60221316758681
1 0.60221316758681
2 0.603095656982768
3 0.612754055717539
4 0.624538495801648
5 0.62791668893401
6 0.633081343456916
7 0.639922816020793
8 0.646852485781285
9 0.652756759496805
10 0.658726606328146
11 0.655977449212481
12 0.665293062315518
13 0.671265524930602
14 0.673823637438148
15 0.974478764836217
16 1.00000000000003
};
\addplot [semithick, color4, forget plot]
table {%
0 0.601395515122941
1 0.601395515122941
2 0.61392554430982
3 0.630267741952442
4 0.630807082939585
5 0.633670305343681
6 0.639922816020793
7 0.649819559965179
8 0.659285200283324
9 0.652182261074434
10 0.658796362546444
11 0.661342844708322
12 0.673823637438148
13 0.969766480712513
14 0.979000636609888
15 0.984943431279268
16 1.00000000000003
};
\addplot [semithick, color5, forget plot]
table {%
0 0.511047337255817
1 0.511047337255817
2 0.612978700250407
3 0.620063167202848
4 0.626830204231395
5 0.633081343456916
6 0.640464676396021
7 0.649268206152165
8 0.649819559965179
9 0.659844019602223
10 0.665920200487818
11 0.67605034228333
12 0.681786379389656
13 0.684269702304902
14 0.974797775257462
15 1.00000000000003
};
\end{axis}

\end{tikzpicture}
  \caption{Above: trajectories of six noise state estimates starting at initial conditions $(-10, -9)$ $(-6, -9)$ \ldots $(10,-9)$. Below: estimated probability to satisfy the specification over time for the same trajectories. No sample is found in $T_1$, and $O_2$ can be safely traversed. Jumps occur in the probabilities when (non-)existence of samples are measured when the rover is close to the regions.}
  \label{fig:exp1}
\end{figure}

\begin{figure}
  \footnotesize
  \setlength\figurewidth{\columnwidth}
  \setlength\figureheight{0.6\columnwidth}

  % This file was created by matplotlib2tikz v0.6.14.
\begin{tikzpicture}

\definecolor{color1}{rgb}{1,0.498039215686275,0.0549019607843137}
\definecolor{color0}{rgb}{0.12156862745098,0.466666666666667,0.705882352941177}
\definecolor{color2}{rgb}{0.172549019607843,0.627450980392157,0.172549019607843}

\begin{axis}[
mark size = 0.5,
ticks=none,
xmin=-11, xmax=11,
ymin=-10.6290882408405, ymax=10.9823375352781,
width=\figurewidth,
height=\figureheight,
tick align=outside,
tick pos=left,
x grid style={lightgray!92.026143790849673!black},
y grid style={lightgray!92.026143790849673!black}
]
\addplot [only marks, draw=black, fill=black, colormap/viridis]
table{%
x                      y
-8.000000000000000e+00 -9.000000000000000e+00
-8.281791193811282e+00 -6.953277059756937e+00
-6.609947231673338e+00 -7.304755895777043e+00
-6.447420151999264e+00 -6.984253859219061e+00
-5.930025319876885e+00 -6.296279125763103e+00
-5.476123458484502e+00 -6.049868453584115e+00
-4.680375628377898e+00 -4.982516422756326e+00
-3.356608221836718e+00 -4.767445283086258e+00
-3.850337386959414e+00 -3.882953851781119e+00
-3.888519145722516e+00 -3.480146659512000e+00
-5.320305970691908e+00 -2.277119535242806e+00
-4.379722627893722e+00 -3.023297381626749e+00
-2.966831046751209e+00 -2.008805754832921e+00
-4.696058526749306e+00 +1.019091306738935e-01
-4.628219306657023e+00 +1.300100090541606e+00
-5.641564937919985e+00 +1.796635145522380e+00
-6.307184225641461e+00 +2.288130625455808e+00
-7.217828733789983e+00 +3.143921812081298e+00
-7.301911031188013e+00 +4.255956608402649e+00
-6.401344954223192e+00 +5.270680166902547e+00
-8.171856419882948e+00 +6.563080734161627e+00
};
\addplot [only marks, draw=black, fill=black, colormap/viridis]
table{%
x                      y
+0.000000000000000e+00 -9.000000000000000e+00
+2.985780663708160e-02 -8.155463278156050e+00
+1.071409254737555e+00 -8.202590105544974e+00
+2.130577217930679e+00 -8.147034783291767e+00
+3.046653997506764e+00 -7.814782310273142e+00
+2.993503950729483e+00 -6.732513426782055e+00
+3.364980364937721e+00 -5.919601911047740e+00
+4.577267269183741e+00 -5.442531058948178e+00
+4.641924364448064e+00 -4.277469389688367e+00
+4.066522550812240e+00 -2.261074508296681e+00
+4.900884132823337e+00 -1.747839224197387e+00
+5.242948546817824e+00 -1.658932790996334e+00
+4.951650027020797e+00 -5.458682191996835e-01
+5.160727075484108e+00 +1.786499818804027e-02
+4.951600531141041e+00 +3.796585040088380e-01
+5.548111410812709e+00 +9.093914656273301e-01
+5.998289710281483e+00 +1.845859092339620e+00
+5.586468819814069e+00 +2.958893844938457e+00
+6.773322394657046e+00 +4.341573264475908e+00
+6.244801362810740e+00 +3.255339248138667e+00
+5.386058184710604e+00 +2.832977405694789e+00
+3.382447161292124e+00 +5.131303079674516e+00
+4.874337596806267e+00 +2.495954941054408e+00
+4.650503806334480e+00 +3.995285095184848e-01
+5.025471920274055e+00 -5.406637729947061e-01
+3.653106795537211e+00 -2.331886810567701e+00
+3.777550916600270e+00 -3.262576207290882e+00
+4.208193801414156e+00 -5.223790221008551e+00
+4.894433739744199e+00 -6.435884403595750e+00
+3.799772582533073e+00 -8.195765033159091e+00
+1.063593830208735e+00 -7.803367444083015e+00
-1.186440212491861e+00 -8.416040749834128e+00
-1.769357566346293e+00 -8.393133290175719e+00
-3.454943820617372e+00 -7.829608609619289e+00
-4.718419194171268e+00 -6.806453261226547e+00
-7.128929570346833e+00 -5.582933784317277e+00
-6.754145560982441e+00 -4.927567852678047e+00
-7.809944106601641e+00 -3.287103405394836e+00
-7.608490106653687e+00 -2.786103027237490e+00
-6.664925925528791e+00 -2.687368525725410e+00
-6.819564565858922e+00 -2.207441188034236e+00
-6.119656171566637e+00 -2.321958612214557e+00
-5.398012336881704e+00 -6.633452969104143e-01
-6.326849807413460e+00 +6.558845980278348e-01
-6.175272895510815e+00 +1.444858418003883e+00
-7.632034853475458e+00 +3.355390912891568e+00
-6.251375019139898e+00 +4.004939418758159e+00
-6.160033028437569e+00 +4.084610663102694e+00
-5.910891233600980e+00 +3.553568231521559e+00
-7.187232311111987e+00 +4.703244684591716e+00
-8.802150302563509e+00 +8.035950400947264e+00
};
\addplot [only marks, draw=black, fill=black, colormap/viridis]
table{%
x                      y
+8.000000000000000e+00 -9.000000000000000e+00
+8.511205582201939e+00 -9.490847881526234e+00
+8.123691607827535e+00 -8.945490678745058e+00
+8.263822914662137e+00 -7.614276297182030e+00
+7.514100738376864e+00 -8.134440925432358e+00
+7.298345618417102e+00 -7.124767904275431e+00
+5.661620747278767e+00 -6.163981262845379e+00
+4.308129478899167e+00 -4.248895613551244e+00
+4.661943867045995e+00 -3.734081630931301e+00
+6.045638341595694e+00 -3.283722422053446e+00
+4.311200610860274e+00 -3.054620213961065e+00
+3.764155218303413e+00 -1.339286613286621e+00
+4.507407456386630e+00 -6.260285666477190e-01
+6.195941762467750e+00 -1.751868652546797e+00
+5.652298276850433e+00 -1.433432909888064e+00
+6.004475142579373e+00 -1.176455188056848e+00
+7.867024860002735e+00 -1.436677206342497e+00
+6.149524945185612e+00 -1.765396180120760e-02
+7.231136140072594e+00 -2.834317843654784e-01
+5.357245998891423e+00 +1.116871382840182e+00
+4.924762093179862e+00 +2.175740570857628e+00
+6.145247254490910e+00 +1.347818433730404e+00
+5.672732010334618e+00 +2.405018338308785e+00
+4.778200199507558e+00 +4.416354675986674e+00
+3.100685883042153e+00 +4.107914445522056e+00
+3.891391072358141e+00 +2.859356344633611e+00
+3.874517594984206e+00 +2.035479937103637e+00
+5.201506296191360e+00 -2.313046332494439e-01
+5.104339902961216e+00 -1.029190621036462e+00
+5.015533858742012e+00 -2.461605591765912e+00
+3.404566585667478e+00 -3.208436019200696e+00
+4.063604053128744e+00 -4.223179916104084e+00
+5.419276458139430e+00 -6.587904229262779e+00
+3.770997682573817e+00 -8.969356381483692e+00
+2.072826036930491e+00 -7.932486401835083e+00
+4.691549265209518e-01 -8.578316323074933e+00
-3.520651457661232e-01 -7.765184235105794e+00
-2.518265654752808e+00 -6.411080484717628e+00
-3.413207860281108e+00 -6.198922651652184e+00
-3.456088450097421e+00 -6.235552101303306e+00
-4.673123778502697e+00 -5.140669103912209e+00
-5.511881550779163e+00 -4.652643794197237e+00
-6.952415477083844e+00 -3.989299476321714e+00
-6.474737771072504e+00 -3.130571715919226e+00
-7.065234191935879e+00 -2.542851770813614e+00
-7.172383147511338e+00 -1.298336855355604e+00
-5.757402873681920e+00 -1.438398078411823e+00
-6.528591566722448e+00 -6.312978098556827e-01
-5.772384260334218e+00 +1.466319934873163e-01
-6.242081541010274e+00 -1.387082780615503e-01
-5.956475469228075e+00 +4.792766582561609e-01
-5.373508577776141e+00 +2.220560436309609e-01
-6.942001756278491e+00 +1.075019277811085e-02
-5.789985714740869e+00 -1.994507978371182e-01
-8.574179717233536e+00 +6.995122202175492e-01
-8.546440565751025e+00 +2.428342665136468e+00
-6.945052246644407e+00 +3.410292093002863e+00
-5.658246322673215e+00 +3.669916744769946e+00
-6.704820165881237e+00 +4.943999287103432e+00
-6.985289853433318e+00 +6.021483786213237e+00
};
\path [draw=red, fill=red, opacity=0.5] (axis cs:2,-5)
--(axis cs:2,2)
--(axis cs:0,2)
--(axis cs:0,-5)
--cycle;

\path [draw=red, fill=red, opacity=0.5] (axis cs:2,3)
--(axis cs:2,10)
--(axis cs:0,10)
--(axis cs:0,3)
--cycle;

\path [draw=red, fill opacity=0, dashed] (axis cs:3.9,-6.9)
--(axis cs:3.9,3.1)
--(axis cs:-1.9,3.1)
--(axis cs:-1.9,-6.9)
--cycle;

\path [draw=red, fill opacity=0, dashed] (axis cs:3.9,1.1)
--(axis cs:3.9,10)
--(axis cs:-1.9,10)
--(axis cs:-1.9,1.1)
--cycle;

\path [draw=blue, fill=blue, opacity=0.5] (axis cs:-6,6)
--(axis cs:-6,9)
--(axis cs:-9,9)
--(axis cs:-9,6)
--cycle;

\path [draw=blue, fill=blue, opacity=0.5] (axis cs:9,6)
--(axis cs:9,9)
--(axis cs:6,9)
--(axis cs:6,6)
--cycle;

\path [draw=blue, fill opacity=0, dashed] (axis cs:-3.9,4.1)
--(axis cs:-3.9,10)
--(axis cs:-10,10)
--(axis cs:-10,4.1)
--cycle;

\path [draw=blue, fill opacity=0, dashed] (axis cs:10,4.1)
--(axis cs:10,10)
--(axis cs:4.1,10)
--(axis cs:4.1,4.1)
--cycle;

\addplot [semithick, color0, forget plot]
table {%
-8 -9
-8.28179119381128 -6.95327705975694
-6.60994723167334 -7.30475589577704
-6.44742015199926 -6.98425385921906
-5.93002531987688 -6.2962791257631
-5.4761234584845 -6.04986845358412
-4.6803756283779 -4.98251642275633
-3.35660822183672 -4.76744528308626
-3.85033738695941 -3.88295385178112
-3.88851914572252 -3.480146659512
-5.32030597069191 -2.27711953524281
-4.37972262789372 -3.02329738162675
-2.96683104675121 -2.00880575483292
-4.69605852674931 0.101909130673894
-4.62821930665702 1.30010009054161
-5.64156493791999 1.79663514552238
-6.30718422564146 2.28813062545581
-7.21782873378998 3.1439218120813
-7.30191103118801 4.25595660840265
-6.40134495422319 5.27068016690255
-8.17185641988295 6.56308073416163
};
\addplot [semithick, color1, forget plot]
table {%
0 -9
0.0298578066370816 -8.15546327815605
1.07140925473755 -8.20259010554497
2.13057721793068 -8.14703478329177
3.04665399750676 -7.81478231027314
2.99350395072948 -6.73251342678205
3.36498036493772 -5.91960191104774
4.57726726918374 -5.44253105894818
4.64192436444806 -4.27746938968837
4.06652255081224 -2.26107450829668
4.90088413282334 -1.74783922419739
5.24294854681782 -1.65893279099633
4.9516500270208 -0.545868219199684
5.16072707548411 0.0178649981880403
4.95160053114104 0.379658504008838
5.54811141081271 0.90939146562733
5.99828971028148 1.84585909233962
5.58646881981407 2.95889384493846
6.77332239465705 4.34157326447591
6.24480136281074 3.25533924813867
5.3860581847106 2.83297740569479
3.38244716129212 5.13130307967452
4.87433759680627 2.49595494105441
4.65050380633448 0.399528509518485
5.02547192027405 -0.540663772994706
3.65310679553721 -2.3318868105677
3.77755091660027 -3.26257620729088
4.20819380141416 -5.22379022100855
4.8944337397442 -6.43588440359575
3.79977258253307 -8.19576503315909
1.06359383020873 -7.80336744408301
-1.18644021249186 -8.41604074983413
-1.76935756634629 -8.39313329017572
-3.45494382061737 -7.82960860961929
-4.71841919417127 -6.80645326122655
-7.12892957034683 -5.58293378431728
-6.75414556098244 -4.92756785267805
-7.80994410660164 -3.28710340539484
-7.60849010665369 -2.78610302723749
-6.66492592552879 -2.68736852572541
-6.81956456585892 -2.20744118803424
-6.11965617156664 -2.32195861221456
-5.3980123368817 -0.663345296910414
-6.32684980741346 0.655884598027835
-6.17527289551082 1.44485841800388
-7.63203485347546 3.35539091289157
-6.2513750191399 4.00493941875816
-6.16003302843757 4.08461066310269
-5.91089123360098 3.55356823152156
-7.18723231111199 4.70324468459172
-8.80215030256351 8.03595040094726
};
\addplot [semithick, color2, forget plot]
table {%
8 -9
8.51120558220194 -9.49084788152623
8.12369160782753 -8.94549067874506
8.26382291466214 -7.61427629718203
7.51410073837686 -8.13444092543236
7.2983456184171 -7.12476790427543
5.66162074727877 -6.16398126284538
4.30812947889917 -4.24889561355124
4.66194386704599 -3.7340816309313
6.04563834159569 -3.28372242205345
4.31120061086027 -3.05462021396107
3.76415521830341 -1.33928661328662
4.50740745638663 -0.626028566647719
6.19594176246775 -1.7518686525468
5.65229827685043 -1.43343290988806
6.00447514257937 -1.17645518805685
7.86702486000273 -1.4366772063425
6.14952494518561 -0.0176539618012076
7.23113614007259 -0.283431784365478
5.35724599889142 1.11687138284018
4.92476209317986 2.17574057085763
6.14524725449091 1.3478184337304
5.67273201033462 2.40501833830879
4.77820019950756 4.41635467598667
3.10068588304215 4.10791444552206
3.89139107235814 2.85935634463361
3.87451759498421 2.03547993710364
5.20150629619136 -0.231304633249444
5.10433990296122 -1.02919062103646
5.01553385874201 -2.46160559176591
3.40456658566748 -3.2084360192007
4.06360405312874 -4.22317991610408
5.41927645813943 -6.58790422926278
3.77099768257382 -8.96935638148369
2.07282603693049 -7.93248640183508
0.469154926520952 -8.57831632307493
-0.352065145766123 -7.76518423510579
-2.51826565475281 -6.41108048471763
-3.41320786028111 -6.19892265165218
-3.45608845009742 -6.23555210130331
-4.6731237785027 -5.14066910391221
-5.51188155077916 -4.65264379419724
-6.95241547708384 -3.98929947632171
-6.4747377710725 -3.13057171591923
-7.06523419193588 -2.54285177081361
-7.17238314751134 -1.2983368553556
-5.75740287368192 -1.43839807841182
-6.52859156672245 -0.631297809855683
-5.77238426033422 0.146631993487316
-6.24208154101027 -0.13870827806155
-5.95647546922808 0.479276658256161
-5.37350857777614 0.222056043630961
-6.94200175627849 0.0107501927781108
-5.78998571474087 -0.199450797837118
-8.57417971723354 0.699512220217549
-8.54644056575103 2.42834266513647
-6.94505224664441 3.41029209300286
-5.65824632267322 3.66991674476995
-6.70482016588124 4.94399928710343
-6.98528985343332 6.02148378621324
};
\node at (axis cs:-0.5,-3)[
  anchor=base west,
  text=black,
  rotate=0.0
]{ $R_1$};
\node at (axis cs:-0.5,7.5)[
  anchor=base west,
  text=black,
  rotate=0.0
]{ $R_2$};
\node at (axis cs:-9,7.5)[
  anchor=base west,
  text=black,
  rotate=0.0
]{ $T_1$};
\node at (axis cs:6.5,7.5)[
  anchor=base west,
  text=black,
  rotate=0.0
]{ $T_2$};

\end{axis}

\end{tikzpicture}

  \setlength\figureheight{0.3\columnwidth}

  % This file was created by matplotlib2tikz v0.6.14.
\begin{tikzpicture}

\definecolor{color1}{rgb}{1,0.498039215686275,0.0549019607843137}
\definecolor{color0}{rgb}{0.12156862745098,0.466666666666667,0.705882352941177}
\definecolor{color2}{rgb}{0.172549019607843,0.627450980392157,0.172549019607843}

\begin{axis}[
xlabel={$t$},
ylabel={$\mathbf{P}(\psi)$},
xmin=-2.95, xmax=61.95,
ymin=0.115383927583069, ymax=0.98995146727863,
width=\figurewidth,
height=\figureheight,
tick align=outside,
tick pos=left,
x grid style={lightgray!92.026143790849673!black},
y grid style={lightgray!92.026143790849673!black},
axis y line=middle,
axis x line=middle,
ymin = 0,
ymax = 1,
every axis x label/.style={
    at={(ticklabel* cs:1.05)}
},
every axis y label/.style={
    at={(0.2,0.7)},
    anchor=south,
}
]
\addplot [semithick, color0, forget plot]
table {%
0 0.519437323294361
1 0.505905967792163
2 0.535762794709806
3 0.529641473005327
4 0.534202921709494
5 0.538759248959358
6 0.543315961968619
7 0.552430637025909
8 0.556136049552841
9 0.566110510314734
10 0.571484104700144
11 0.582252336732076
12 0.572331976818973
13 0.585099080576503
14 0.606729657381389
15 0.616691491600847
16 0.622103745520534
17 0.627514467660256
18 0.63366955987662
19 0.941425688953114
20 0.950198397292469
};
\addplot [semithick, color1, forget plot]
table {%
0 0.535132433990641
1 0.52361711183953
2 0.535132433990641
3 0.543731008001737
4 0.550703063551738
5 0.558585302195194
6 0.539173994232859
7 0.547458159763064
8 0.562091024973306
9 0.572039360069865
10 0.596097657367583
11 0.596563953011597
12 0.601126871317435
13 0.611092694091856
14 0.616488991909715
15 0.621058196593617
16 0.625615820087544
17 0.634735491594122
18 0.645561482408208
19 0.234045441759929
20 0.243149052412218
21 0.247728706925773
22 0.155136997569231
23 0.186216159094014
24 0.206143993143537
25 0.21610603533654
26 0.233451571592582
27 0.244022547027297
28 0.26598938278977
29 0.279350961751404
30 0.289174863802017
31 0.316179113764902
32 0.328982444064366
33 0.328982444064366
34 0.336315254852049
35 0.347152718032949
36 0.359673571635938
37 0.364229419086882
38 0.379600870657791
39 0.38500440205473
40 0.388712554616358
41 0.394120578102893
42 0.388712554616358
43 0.407787653050549
44 0.418603916546378
45 0.427715165844223
46 0.459361669473757
47 0.453045606826443
48 0.453045606826443
49 0.44763200589673
50 0.950198397288434
};
\addplot [semithick, color2, forget plot]
table {%
0 0.547722366299717
1 0.531583463211636
2 0.527071760546219
3 0.547722366299717
4 0.560103204844214
5 0.555328267951691
6 0.565540804425142
7 0.576360127473417
8 0.592582903342865
9 0.602542902132007
10 0.606253832137332
11 0.60710633658241
12 0.591576171090529
13 0.611824096954832
14 0.600274259142499
15 0.601126871317435
16 0.605680578715834
17 0.603975733220287
18 0.615647313283849
19 0.613942290635706
20 0.625615820087544
21 0.636351488414114
22 0.630168418209078
23 0.640134983716998
24 0.253945007517459
25 0.168083558274324
26 0.180794947372842
27 0.188822613176964
28 0.211591710630682
29 0.220721176339468
30 0.236209066973026
31 0.244022547027297
32 0.255461135796253
33 0.267559897089635
34 0.293610800263249
35 0.311769872918488
36 0.323468852817987
37 0.328941635241611
38 0.350849395838469
39 0.351708421550836
40 0.351708421550836
41 0.362524629192895
42 0.367932683007021
43 0.374193140341984
44 0.383304454416872
45 0.388712554616358
46 0.399528017149535
47 0.397823889661972
48 0.408640059633355
49 0.413195587429733
50 0.413195587429733
51 0.417751484126762
52 0.416899076992805
53 0.414048005317165
54 0.412343194766756
55 0.420173604424889
56 0.449079203299998
57 0.448481084716646
58 0.452180689640779
59 0.950198397288434
};
\end{axis}

\end{tikzpicture}
  \caption{Same as Fig. \ref{fig:exp1} with the difference that both $O_1$ and $O_2$ contain obstacles and must be avoided. When the obstacles are detected (around time step 20) the satisfaction probabilities decrease slightly due to the additional risk of moving around the obstacles.}
  \label{fig:exp2}
\end{figure}


\section{Conclusions}
\label{sec:conclusions}

In this paper,  we have considered correct-by-construction control synthesis of policies that guarantee behaviors expressed in the belief space of a POMDP.  To cope with the inherent computational difficulties, we have pursued an approach based on principled abstractions via a novel labeling-based concept of stochastic simulation relation. In addition, we have used a synthesis procedure that is robust to differences between a concrete system and its abstraction, as well as a way to refine control policies. For the special case of linear systems with Gaussian noise and observation models, we have given a concrete abstraction construction and an associated simulation relation, and applied the theory in a case study motivated by space exploration.

We are interested in cooperative robotics applications where some agents perform exploration while others collect samples, and will therefore conduct more sophisticated case studies. In future work, we  plan to explore additional settings in which abstractions satisfying our novel labeling-based simulation relation can be constructed, and plan to leverage the \red{decomposed nature} of product MDPs to improve computational efficiency.

\bibliography{AliAgha,references}


\end{document}

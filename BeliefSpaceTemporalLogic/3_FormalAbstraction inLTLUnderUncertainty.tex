\documentclass[conference]{IEEEtran}
\usepackage{standalone}
\usepackage{times}
\usepackage{float}
\usepackage{amsmath}

%% Additional packages
\usepackage{times}
\usepackage{amsmath}
\usepackage{amsthm}
\usepackage{amssymb}
\usepackage{amsfonts}
\usepackage{mathtools}
%\usepackage{calc}
\usepackage{subfigure}
\usepackage{graphicx}
\usepackage{color}
\usepackage{url}
%\usepackage{lineno}
%\usepackage{ulem} % for underlining and strike through
%\normalem % reset emph to normal
%\usepackage{setspace} % for line spacing - e.g. 1.5, 2
\usepackage[usenames,x11names]{xcolor}
%\usepackage{xspace} % for inserting a space in TeX commands if needed
%\usepackage{caption}
\usepackage[bookmarks=true]{hyperref}
%\renewcommand{\thesubfigure}{\relax} % no subfig counters
%\let\chapter\section % algorithm2e natbib compatibility
%\usepackage{accents}
%\usepackage[titletoc,toc,title]{appendix}
%\usepackage{longtable}
%\usepackage{setspace}
\usepackage{multicol}

%% Packages
%\usepackage[margin=1in]{geometry}
%\usepackage[abs]{overpic}
\usepackage[linesnumbered,vlined,ruled]{algorithm2e}
%\usepackage{multirow} % for cell tables spanning multiple rows
\usepackage{tikz,pgf}%,tikz-3dplot}
\usetikzlibrary{arrows,automata,shapes,calc,backgrounds,spy,positioning}
%\usetikzlibrary{external}
%\tikzexternalize
%\usepackage[numbers]{natbib}
%\usepackage[sort,round]{natbib}
%\usepackage{lipsum}
\usepackage{epstopdf}

\theoremstyle{plain}
\newtheorem{theorem}{Theorem}
\newtheorem{cor}{Corollary}
\newtheorem{prop}{Proposition}
\newtheorem{lemma}{Lemma}

\theoremstyle{definition}
%\newtheorem{definition}{Definition}
\newtheorem{remark}{Remark}
\newtheorem{cond}{Condition}

\newtheorem{example}{Example}
\newtheorem{problem}{Problem}
%\newtheorem{assumption}{Assumption}


%%% figure path
\graphicspath{{figures/}}

%% Remove footnote mark
%\renewcommand{\footnotemark}{}

%%% Redefine qed symbol
%\renewcommand{\qedsymbol}{$\blacksquare$}

%% Projection symbol
%\newcommand{\project}[1]{\! \upharpoonright_{#1}}


%% Theorems 
%\newtheorem{theorem}{Theorem}[section]
%\newtheorem{proposition}[theorem]{Proposition}
%\newtheorem{corollary}[theorem]{Corollary}
\newtheorem{definition}[theorem]{Definition}
%\newtheorem{lemma}[theorem]{Lemma}
%\newtheorem{remark}[theorem]{Remark}
%\newtheorem{remarks}[theorem]{Remarks}
%\newtheorem{example}[theorem]{Example}
%\newtheorem{algo}[theorem]{Algorithm}
%\newtheorem{problem}[theorem]{Problem}
%\newtheorem{Procedure}[theorem]{Procedure}
%\newcommand{\exampler}[2]{\medskip \hskip -\parindent {\bf Example #1 Revisited.~}{\it #2}\medskip}

%% Percent
%\newcommand\oprocendsymbol{\hbox{$\square$}}
%\newcommand\oprocend{\relax\ifmmode\else\unskip\hfill\fi\oprocendsymbol}
%\def\eqoprocend{\tag*{$\bullet$}}

%% Enumerate environment
%\renewcommand{\labelenumi}{(\roman{enumi})}
%\renewcommand{\labelenumii}{(\alph{enumii})}

%% Breakable comma
%\mathchardef\breakingcomma\mathcode`\,
%{\catcode`,=\active
%  \gdef,{\breakingcomma\discretionary{}{}{}}
%}
%\newcommand{\breqn}[1]{\mathcode`\,=\string"8000 #1}

%% Other Stuff
%\newcommand{\margin}[1]{\marginpar{\tiny\color{blue} #1}}
%%\addtolength{\marginparwidth}{-0.3in}
\newcommand{\todo}[1]{\vskip 0.05in \colorbox{yellow}{$\Box$ \ttfamily\bfseries\small#1}\vskip 0.05in}
%\newcommand{\todo}[1]{}
%\newcommand{\vers}{\operatorname{vers}}

%% Roman, calligraphic, boldface, double barred letters
\newcommand{\RM}[1]{\mathrm{#1}}
\newcommand{\CA}[1]{\mathcal{#1}}
\newcommand{\BF}[1]{\mathbf{#1}}
\newcommand{\IT}[1]{\mathit{#1}}
\newcommand{\BB}[1]{\mathbb{#1}}
\newcommand{\TT}[1]{\mathtt{#1}}
\newcommand{\FK}[1]{\mathfrak{#1}}
\newcommand{\BS}[1]{\boldsymbol{#1}}

%%% Temporal logic symbols
\newcommand{\notltl}{\neg}
\newcommand{\andltl}{\wedge}
\newcommand{\orltl}{\vee}
\newcommand{\Next}{\ensuremath{\bigcirc}}
\newcommand{\Always}{\ensuremath{\ \square\ }}
\newcommand{\Event}{\ensuremath{\ \diamondsuit\ }}
\newcommand{\Until}{\ \CA{U}\ }
\newcommand{\Implies}{\Rightarrow}
\newcommand{\Equiv}{\Leftrightarrow}
%\newcommand{\Not}{\lnot}
\newcommand{\True}{\top}
\newcommand{\False}{\perp}
%\def\prop{\TT{data}}
%\def\popt{\pi}
\newcommand{\AP}{AP}
\newcommand{\pred}{\xi}

\newcommand{\Real}{\BB{R}}

%% Abbreviations
%\def\eg{e.g.\xspace}
%\def\Eg{E.g.\xspace}
%\def\ie{i.e.\xspace}
%\def\Ie{I.e.\xspace}
%\def\etc{etc.\xspace}
%\def\vs{vs.\xspace}
%\def\wrt{w.r.t.\xspace}
%\def\etal{et al.\xspace}

%% Exotic words
\newcommand{\buchi}{B\"uchi\ }

%% Symbols of automata
\newcommand{\PA}{\mathcal{P}}
\newcommand{\BA}{\mathcal{B}}
%\newcommand{\FA}{\mathcal{F}}

\newcommand{\TS}{\mathcal{F}}
\newcommand{\LA}{\mathcal{L}}
\newcommand{\KA}{\mathcal{K}}
\newcommand{\MDP}{\mathcal{M}}
\newcommand{\RA}{\mathcal{R}}
\newcommand{\FSA}{\mathcal{A}}

\newcommand{\TSX}{\BB{V}_\TS}
\newcommand{\TSE}{\BB{E}_\TS}
\newcommand{\TSEE}{\BB{E}}

\newcommand{\DTL}{DTL\xspace}


%% Short macros for arrows
\newcommand{\la}{\leftarrow}
\newcommand{\ra}{\rightarrow}
\newcommand{\ras}[1]{\stackrel{#1}{\rightarrow}}
\newcommand{\asgn}{\la}
\newcommand{\proj}[2]{{#1}{\downharpoonright_{#2}}}

\newcommand{\df}{\xspace\RM{d}}

%% Names of the Algorithms
%\newcommand{\optrun}{\textsc{Optimal-Run}\ }
%\newcommand{\exactmultioptrun}{\textsc{Exact-Multi-Robot-Optimal-Run}\ }
%\newcommand{\constR}{\textsc{Construct-Region-Automaton}\ }
%\newcommand{\constT}{\textsc{Construct-Team-TS}\ }
%\newcommand{\syncT}{\textsc{Sync-Team-TS}\ }
%\newcommand{\boundOpt}{\textsc{Bound-Optimality}\ }

% Custom operators
%\DeclareMathOperator*{\argmin}{arg\,min}
\newcommand{\norm}[1]{\left\| {#1} \right\|}
\newcommand{\norminf}[1]{\left\| {#1} \right\|_{\infty}}
\newcommand{\normeucl}[1]{\left\| {#1} \right\|_{2}}
\newcommand{\abs}[1]{\left| {#1} \right|}
\newcommand{\card}[1]{\left| {#1} \right|}
\newcommand{\spow}[1]{2^{#1}}
%\newcommand{\interior}[1]{\accentset{\smash{\raisebox{-0.12ex}{$\scriptstyle\circ$}}}{#1}\rule{0pt}{2.3ex}}
\newcommand{\interior}[1]{\mathring{#1}}
\DeclareMathOperator{\diag}{diag}
\newcommand{\lift}{\upharpoonright}
%
%% Display a grid to help align images
%\beamertemplategridbackground[1cm]
%\usepackage[style=numeric-comp]{biblatex}
\usepackage{cite}


%\documentclass[letterpaper, 11pt, onecolumn]{TemplateFiles/ieeeconf}
%\IEEEoverridecommandlockouts \overrideIEEEmargins 
%\pagestyle{plain}

%\usepackage[ruled,vlined,linesnumbered,boxruled]{algorithm2e}
%\usepackage{mathrsfs}
%\usepackage{graphicx}
%\usepackage{amsfonts}
%\usepackage{amsmath}
%\usepackage{amssymb}
%\usepackage{array}
%\usepackage{flafter}
%\usepackage{tabu}
%\usepackage{cite}
%\usepackage{subfigure}
%\usepackage{verbatim}
%\usepackage{bbm}
%\usepackage[usenames]{color}
%\usepackage[svgnames]{xcolor}

%\usepackage{balance}

%\usepackage{nicefrac}
%\usepackage{psfrag}
%\usepackage{umoline}
%\usepackage{hyperref}
%\usepackage{appendix} %[2009/09/02 v1.2b extra appendix facilities]
%\let\proof\relax
%\let\endproof\relax
%\usepackage{amsthm}	% This is needed for \newtheorem and proof environment.
% \usepackage{natbib} % for citing the papers with auther-year format in parantheses.
%\usepackage[numbers, sort]{natbib}

%\usepackage{etoolbox}

%\providecommand{\citet}[1]{\citeauthor{#1}\,[\citeyear{#1}]}
%\providecommand{\citep}[1]{\cite{#1}}

%\newtheorem{thm}{Theorem}
%\newtheorem{Def}{Definition}
%\newtheorem{Problem}{Problem}
%\newtheorem{Lem}{Lemma}
%\newtheorem{Cor}{Corollary}
%\newtheorem{assumption}{Assumption}
%\newtheorem{proposition}{Proposition}
%\newtheorem{definition}{Definition}
%\newtheorem{property}{Property}
%\newtheorem{Remark}{Remark}
%\newtheorem{exmp}{Example}[section]
%\newenvironment{proof}[1][Proof]{\begin{trivlist}
%\item[\hskip \labelsep {\bfseries #1}]}{\end{trivlist}}

%\newcommand{\TypeOfDoc}{IJRR} % This either can be TR or Conf or IJRR
%\newcommand{\FinalFlag}{No} % This either can be No or Accepted
%\newtoggle{finalpaper}
%\toggletrue{finalpaper}
%\togglefalse{finalpaper}

\graphicspath{{./figs/}}

%\allowdisplaybreaks[1]

%\newcommand{\kXX}[1]{\color{blue} XX #1 XX \color{black}}
%newcommand{\AXX}[1]{\color{purple} XX #1 XX \color{black}}
\newcommand{\aXX}[1]{\color{orange} #1  \color{black}}
\newcommand{\axx}[1]{\aXX{#1}}


\newcommand{\pr}[1]{\textbf{#1:} }

\newcommand{\codeline}[1]{\par{\ttfamily #1 \par}}  % This line is added because sth like this \newcommand{\initeali}{\verb|InitializeEdge|} does not work. For further details please check out this link % http://tex.stackexchange.com/questions/86071/newcommand-for-verbatim
% please do not delete above comments as it can be really confsing


%%%%%%%%%%%%%%%% symbols
\newcommand{\tv}{\varpi} % tube volume (edge tube volume)
\newcommand{\td}{\Gamma} % tube distance (edge tube distance)

%%%%%%%%%%%%%%%%%%%%%%%%%%%%%%%%%%%%%%%%%%%%%%%%%% To reduce pages
%\textfloatsep = 0pt
%\renewcommand{\baselinestretch}{0.96}
%%%%%%%%%%%%%%%%% Make bibs smaller
%\renewcommand{\IEEEbibitemsep}{0pt plus 2pt}
%\makeatletter
%\IEEEtriggercmd{\reset@font\normalfont\footnotesize}
%\makeatother
%\IEEEtriggeratref{1}

%%%%%%%%%%%%%%%%% margins
%\usepackage{geometry}
% \newcommand{\papermargin}{0.97in} % IEEE asks for 0.75 on all pages, but the first page
% %\newgeometry{top=0.75in,bottom=.75in,right=.75in,left=.75in}
% \newgeometry{top=\papermargin,bottom=\papermargin,right=\papermargin,left=\papermargin}
% \newgeometry{top=1.in,bottom=1.in,right=1.in,left=1.in}

\let\labelindent\relax
\usepackage{enumitem}


\newcommand{\red}[1]{{\color{red} #1}}
\newcommand{\X}{{\mathbb{X}}}
\newcommand{\Y}{{\mathbb{Y}}}

\newcommand{\po}{\mathbb{P}}     % Probability of an event
\newcommand{\p}[1]{\po\left(#1\right)}     % Probability of an event
\newcommand{\pd}[1]{p\left(#1\right)}     % Probability density of an event
\newcommand{\borel}[1]{\mathcal{B}\left(#1\right)}
\newcommand{\rel}{\mathcal{R}}%{A relation}
% numbers option provides compact numerical references in the text. 

\pdfinfo{
   /Author (Homer Simpson)
   /Title  (Robots: Our new overlords)
   /CreationDate (D:20101201120000)
   /Subject (Robots)
   /Keywords (Robots;Overlords)
}

% Table caption wrangling
\usepackage{etoolbox}
\makeatletter
\patchcmd{\@makecaption}
  {\scshape}
  {}
  {}
  {}
\makeatletter
\patchcmd{\@makecaption}
  {\\}
  {.\ }
  {}
  {}
\makeatother



\allowdisplaybreaks[1]

\begin{document}

% paper title
%\title{\large From Mission Specification to Safe Control Logic Under Uncertainty:\\ Application to Mars Copter-Rover Navigation-Coordination}

\title{\huge Formal abstraction in LTL Under Uncertainty}

\author{Sofie, Petter, Cristi, Rohan, Ali Agha, Richard Murray, Aaron Ames}

\maketitle

\begin{abstract}
This work studies ...
\end{abstract}

\IEEEpeerreviewmaketitle

	
%%%%%%%%%%%%%%%%%%%%%%%%%%%%%%%%%%%%%%%%%%%%%%%%
%%%%%%%%%%%%%%%%%%%%%%%%%%%%%%%%%%%%%%%%%%%%%%%%
\section{Introduction} \label{subsec:intro}
In this work, we use a language for specifying behavior under uncertainty using formal methods. 


\section{Partially observable Markov decision Processes}
In this work, we only consider  Borel measurable spaces, i.e., $(\X,\mathcal{B}(\X))$, 
%In this work,
and we restrict our attention to Polish spaces \cite{bogachev2007measure}. 
Together with the measurable space $(\X,\mathcal{B}(\X))$,  a probability measure $\po$ defines the probability space, denoted by $(\X,\mathcal{B}(\X),\po)$ and has realisations  $x\sim \po$.   
Let us further denote the set of all probability measures for a given measurable space $(\X,\mathcal{B}(\X))$ as $\mathcal P (\X,\mathcal{B}(\X))$.

For the sets $A$ and $B$ a relation $\rel\subset A\times B$ is a subset of the Cartesian product $A\times B$. The relation $\rel$ relates $x\in A$ with $y\in B$ if $(x,y)\in\rel$, which is equivalently written as $x\rel y$.
%We use the following notation for the mappings $\rel(\tilde A):=\{y: x\rel y,\  x\in \tilde A\}$ and $\rel^{-1}(\tilde B):=\{x: x\rel y,\  y\in \tilde B\}$  for $\tilde A\subseteq A$ and $\tilde B\subseteq B$.
For a given set $\Y$ a metric or distance function $\mathbf d_\Y$ is a function $\mathbf{d}_\Y: \Y\times \Y\rightarrow \mathbb R_{\ge 0}$ 
satisfying the following conditions: 
$\forall y_1,y_2,y_3\in\Y$:
$\mathbf d_\Y(y_1,y_2)=0$ iff $y_1=y_2$; 
$\mathbf d_\Y(y_1,y_2)=\mathbf d_{\Y}(y_2,y_1)$;  and
$\mathbf d_\Y(y_1,y_3)\leq \mathbf d_\Y(y_1,y_2) +\mathbf d_\Y(y_2,y_3)$. 
 
 
 
 
The reduction of Markov decision problems with incomplete information to problems with complete information has been tackled in
\cite{yushkevich_reduction_1976,rhenius_incomplete_1974}, with respect to Borel states and Borel actions.





\section{Belief Space Temporal Logic}
	\red{Copy of Best orig. paper. Goal is to first see how the temporal logic specifications can be defined with measurability in mind. Then fix this in the original paper, before starting to write this report.\\*}
	In this section, we define a language for specifying behavior under uncertainty using formal methods. This is of particular interest for planning problems in partially-observable environments.
	 We will start by reviewing partially-observable systems and the POMDP problem. Then we will go over the belief-based temporal logic for mission planning.
	
    %%%%%%%%%%%%%%%%%%%%%%%%%%%%%%%%%%%%%%%%%%%%%%%%%%%%%%%%%%
	\subsection{POMDP}\label{sec:POMDP}
	Let us denote the system state, action, and observation at the $k$-th time-step by $x_k, u_k, z_k$, respectively. Let $x_{k+1}=f(x_k,u_k,w_k)$ and $z_k=h(x_k,v_k)$ denote the system dynamics and measurement models, where $w_k\sim p_w(\cdot|x_k)$ and $v_k\sim p_v(\cdot|x_k)$ denote the state-dependent process and observation noise.  \red{Is this by default independent noise?}
	
	
	Due to the observation noise, the best one can infer about the system state is a probability distribution over all possible states, referred to as belief $b_k=p(x_k|z_{0:k})$. Belief space (i.e., the set of all beliefs) is denoted by $\mathbb{B}$. Belief is typically evolved using a recursive filter denoted by $\tau$ as $b_{k+1}=\tau(b_k,u_k,z_{k+1})$. The control policy $\pi$ in a partially observable setting is a mapping from belief space to the action space, i.e., $u_k=\pi(b_k)$. 
    
    
    %%%%%%%%%%%%%%%%%%%%%%%%%%%%%%%%%%%%%%%%%%%%%%%%%%%%%%%%%%
	\subsection{\DTL}\label{sec:DTL}
    To encode mission specifications, and synthesize a controller to satisfy them, we resort to formal methods. To develop a language for formal planning and reasoning over the belief state, we rely on predicate-based logic.
    %\pr{Predicate-based logic vs. propositions-based logic}
    %In this work, we will rely on predicate logic because we explicitly reason about the belief of the system (e.g. covariance matrix) as opposed to proposition-based logic where the state of system is implicit in the definitions of the propositions.
    
    %a predicate temporal logic defined over the space of Gaussian distributions with fixed dimension.
    
    \pr{Notation}
    Let $\Sigma$ be a finite set. The cardinality,
    power set, Kleene- and $\omega$-closures
    of $\Sigma$ are denoted by $\card{\Sigma}$,
    $\spow{\Sigma}$, $\Sigma^*$ and $\Sigma^\omega$,
    respectively.
    $A \subseteq \BB{R}^n$ and $B \subseteq \BB{R}^m$,
    $n, m \geq 0$, we denote by $\CA{M}(A, B)$ the set of
    functions with domain $A$ and co-domain $B$, where $A$ has positive measure with
    respect to the Lebesgue measure of $\BB{R}^n$.
    The set of all positive semi-definite matrices of size
    $n \times n$, $n \geq 1$, is denoted by $S^n$.
    $\BB{E}[\cdot]$ is the expectation operator.
    The $m \times n$ zero matrix and
    the $n \times n$ identity matrix are denoted by
    $\BF{0}_{m, n}$ and $\BF{I}_n$, respectively.
    The supremum and Euclidean norms are denoted by
    $\norminf{\cdot}$ and $\normeucl{\cdot}$, respectively.
    
    \pr{Parametrized Belief space}
    In this work, we assume the belief space $\mathbb{B}$ is  a finite dimensional space and can be parameterized. For example, let $\CA{G}$ denote the Gaussian belief space
    of dimension $n$, i.e. the space of Gaussian
    probability measures over $\BB{R}^n$.
    For brevity, we identify the Gaussian measures
    with their finite parametrization, mean and
    covariance matrix. Thus,
    $\CA{G} =  \BB{R}^n \times  S^n$.
    If $\BF{b} = b^0b^1 \ldots \in \CA{G}^{\omega}$,
    we denote the suffix sequence $b^i b^{i+1} \ldots$ by
    $\BF{b}^i$, $i \geq 0$. Each member of $\Sigma^*$ and $\Sigma^\omega$ is referred to as ``word" or "sequence". 
    
    For the language grammar we will rely on Bakus-Naur form. $\True$ and $\False$ are Boolean constants that respectively describe specifications that are always satisfied or can never be satisfied. Predicate $f\leq 0$ is defined as a function $f:\mathbb{B}\rightarrow \mathbb{R}$ that encodes constraints or properties over belief space. Defining predicates over the belief space allows us to enforce properties directly on the probability distribution of the system (and hence its chance constraints). Some examples are \textbf{(i)} Bounds on determinant or trace of of the covariance matrix (i.e., $det(P)$, $Tr(P)$) to  bound the uncertainty about the system's state. \textbf{(ii)} Bound on projection of covariance matrix $\Pi P$ to bound the uncertainty in a specific direction.
    \textbf{(iii)} Bounds on state mean $\hat{x}$ to specify
    where in the state space the system should be. \textbf{(iv)} Bounds on Mahalanobis distance $\mathcal{M}(\hat{x},P,x) = (\hat{x}-x)^TP^{-1}(\hat{x}-x)$.
    to describe the distance from a point to a Gaussian distribution, when specifying a desired state (or region) in the state space. We then combine these predicates via operators to create specifications. Operators include boolean "and" $\andltl$, "or" $\orltl$, "not" $\notltl$, and temporal operators: "until" $\Until$, "eventually" $\Event$, "always" $\Always$, "next" $\Next$.
    
    %\pr{Structure of the formulae}
    %Parts of the formulae is not part of the syntax but rather ... In particular $|$ defines option. For example, $\phi::=A | B$ means specification $\phi$ can be either $A$ or $B$. Example: an arithmetic expression has the following syntax grammar $expr::= c | x | expr + expr | expr - expr | expr * expr | expr / expr$.
    
    \begin{definition}[\DTL Syntax]
    \label{def:gdtl-syntax}
    The {\em syntax} of \DTL includes the minimum number of operators to define the logic:
    \begin{equation*}
     \phi :=  \True \ |\ f \leq 0 \ |\ \notltl \phi \ |\ \phi_1 \andltl \phi_2 \ |\ \phi_1 \Until \phi_2 \ |\ \Next \phi
    \end{equation*}
    %where $\True$ is the Boolean constant ``True'',
    %$f \leq 0$ is a predicate over $\mathbb{B}$, where
    %$f \in \CA{M}(\mathbb{B}, \BB{R})$.
    %$\notltl$ is negation (``Not''), $\andltl$ is conjunction (``And''),
    %$\Next$ is ``Next'',
    %and $\Until$ is ``Until''.
    \end{definition}

    For convenience, we define the additional operators:
    $\phi_1 \orltl \phi_2 \equiv  \notltl (\notltl \phi_1 \andltl \notltl \phi_2)$,
    $\Event \phi \equiv \True \Until \phi$, and
    $\Always \phi \equiv \notltl \Event \notltl \phi$,
    %\begin{align*}
    %\phi_1 \orltl \phi_2 & \equiv  \notltl (\notltl \phi_1 \andltl \notltl \phi_2) \\
    %\LTLEVENTUALLY \phi & \equiv \True \LTLUNTIL \phi \\
    %\LTLALWAYS \phi & \equiv \notltl \LTLEVENTUALLY \notltl \phi
    %\end{align*}
    where $\equiv$ denotes semantic equivalence. \DTL syntax defines the symbols and their correct ordering to form a formulae. In the following, we define \DTL semantics, i.e., the meaning of those symbols.

    \begin{definition}[\DTL Semantics]
    \label{def:gdtl-semantics}
    Let $\BF{b} = b^0b^1 \ldots \in \mathbb{B}^{\omega}$
    be an infinite sequence of belief states. $\BF{b} \models \phi$ denotes the event that the word $\BF{b}$ satisfies specification $\phi$. Accordingly, the {\em semantics} of \DTL is defined recursively as
    \begin{align*}
    &\BF{b}^i \models  \top  & \\
    &\BF{b}^i \models f \leq 0 & \Equiv\quad & f(b^i) \leq 0\\ % \forall (x,P) \in b^i \\
    &\BF{b}^i \models \notltl \phi & \Equiv\quad & \notltl (\BF{b}^i \models \phi) \\
    &\BF{b}^i \models \phi_1 \andltl  \phi_2  & \Equiv\quad & ( \BF{b}^i \models \phi_1 ) \andltl ( \BF{b}^i \models \phi_2 ) \\
    &\BF{b}^i \models \phi_1 \orltl  \phi_2  & \Equiv\quad & ( \BF{b}^i \models \phi_1 ) \orltl ( \BF{b}^i \models \phi_2 ) \\
    &\BF{b}^i \models  \phi_1 \Until \phi_2 & \Equiv\quad & \exists j \geq i \text{ s.t. } ( \BF{b}^j \models \phi_2 ) \\
    & & & \andltl (\BF{b}^k \models \phi_1, \forall k \in \{i, \ldots j-1\})\\
    &\BF{b}^i \models \Event \phi  & \Equiv\quad & \exists j \geq i \text{ s.t. } \BF{b}^j \models \phi \\
    &\BF{b}^i \models \Always \phi  & \Equiv\quad & \forall j \geq i \text{ s.t. } \BF{b}^j \models \phi
    \end{align*}
    
    \end{definition}


	\bibliographystyle{plain}
	\bibliography{AliAgha,references}

\end{document}


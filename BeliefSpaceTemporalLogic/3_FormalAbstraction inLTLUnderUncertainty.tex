\documentclass{ifacconf}
\usepackage{standalone}
\usepackage{times}
\usepackage{float}
\usepackage{amsmath}
\usepackage{graphicx}      % include this line if your document contains figures
\usepackage{natbib}        % required for bibliography
%===========================================
%\documentclass[letterpaper, 11pt, onecolumn]{TemplateFiles/ieeeconf}
%\IEEEoverridecommandlockouts \overrideIEEEmargins 
%\pagestyle{plain}

%\usepackage[ruled,vlined,linesnumbered,boxruled]{algorithm2e}
%\usepackage{mathrsfs}
%\usepackage{graphicx}
%\usepackage{amsfonts}
%\usepackage{amsmath}
%\usepackage{amssymb}
%\usepackage{array}
%\usepackage{flafter}
%\usepackage{tabu}
%\usepackage{cite}
%\usepackage{subfigure}
%\usepackage{verbatim}
%\usepackage{bbm}
%\usepackage[usenames]{color}
%\usepackage[svgnames]{xcolor}

%\usepackage{balance}

%\usepackage{nicefrac}
%\usepackage{psfrag}
%\usepackage{umoline}
%\usepackage{hyperref}
%\usepackage{appendix} %[2009/09/02 v1.2b extra appendix facilities]
%\let\proof\relax
%\let\endproof\relax
%\usepackage{amsthm}	% This is needed for \newtheorem and proof environment.
% \usepackage{natbib} % for citing the papers with auther-year format in parantheses.
%\usepackage[numbers, sort]{natbib}

%\usepackage{etoolbox}

%\providecommand{\citet}[1]{\citeauthor{#1}\,[\citeyear{#1}]}
%\providecommand{\citep}[1]{\cite{#1}}

%\newtheorem{thm}{Theorem}
%\newtheorem{Def}{Definition}
%\newtheorem{Problem}{Problem}
%\newtheorem{Lem}{Lemma}
%\newtheorem{Cor}{Corollary}
%\newtheorem{assumption}{Assumption}
%\newtheorem{proposition}{Proposition}
%\newtheorem{definition}{Definition}
%\newtheorem{property}{Property}
%\newtheorem{Remark}{Remark}
%\newtheorem{exmp}{Example}[section]
%\newenvironment{proof}[1][Proof]{\begin{trivlist}
%\item[\hskip \labelsep {\bfseries #1}]}{\end{trivlist}}

%\newcommand{\TypeOfDoc}{IJRR} % This either can be TR or Conf or IJRR
%\newcommand{\FinalFlag}{No} % This either can be No or Accepted
%\newtoggle{finalpaper}
%\toggletrue{finalpaper}
%\togglefalse{finalpaper}

\graphicspath{{./figs/}}

%\allowdisplaybreaks[1]

%\newcommand{\kXX}[1]{\color{blue} XX #1 XX \color{black}}
%newcommand{\AXX}[1]{\color{purple} XX #1 XX \color{black}}
\newcommand{\aXX}[1]{\color{orange} #1  \color{black}}
\newcommand{\axx}[1]{\aXX{#1}}


\newcommand{\pr}[1]{\textbf{#1:} }

\newcommand{\codeline}[1]{\par{\ttfamily #1 \par}}  % This line is added because sth like this \newcommand{\initeali}{\verb|InitializeEdge|} does not work. For further details please check out this link % http://tex.stackexchange.com/questions/86071/newcommand-for-verbatim
% please do not delete above comments as it can be really confsing


%%%%%%%%%%%%%%%% symbols
\newcommand{\tv}{\varpi} % tube volume (edge tube volume)
\newcommand{\td}{\Gamma} % tube distance (edge tube distance)

%%%%%%%%%%%%%%%%%%%%%%%%%%%%%%%%%%%%%%%%%%%%%%%%%% To reduce pages
%\textfloatsep = 0pt
%\renewcommand{\baselinestretch}{0.96}
%%%%%%%%%%%%%%%%% Make bibs smaller
%\renewcommand{\IEEEbibitemsep}{0pt plus 2pt}
%\makeatletter
%\IEEEtriggercmd{\reset@font\normalfont\footnotesize}
%\makeatother
%\IEEEtriggeratref{1}

%%%%%%%%%%%%%%%%% margins
%\usepackage{geometry}
% \newcommand{\papermargin}{0.97in} % IEEE asks for 0.75 on all pages, but the first page
% %\newgeometry{top=0.75in,bottom=.75in,right=.75in,left=.75in}
% \newgeometry{top=\papermargin,bottom=\papermargin,right=\papermargin,left=\papermargin}
% \newgeometry{top=1.in,bottom=1.in,right=1.in,left=1.in}

\let\labelindent\relax
\usepackage{enumitem}

%% Additional packages
\usepackage{amsmath,amssymb,amsfonts}
\usepackage{mathtools}
% \usepackage{subfigure}
\usepackage{graphicx}
\usepackage{color}
\usepackage{url}
%\usepackage[usenames,x11names]{xcolor}

\ifdefined\algorithm
% Don't load theorem style
\else
\usepackage[linesnumbered,vlined,ruled]{algorithm2e}
\fi

\usepackage{tikz}
\usepackage{pgfplots} 
\pgfplotsset{compat=1.14}
\usetikzlibrary{arrows,automata,shapes,calc,backgrounds,spy,positioning}
\usetikzlibrary{fit}

\usepackage{epstopdf}


 

%%% figure path
\graphicspath{{figures/}}
 
 
%% Roman, calligraphic, boldface, double barred letters
\newcommand{\RM}[1]{\mathrm{#1}}
\newcommand{\CA}[1]{\mathcal{#1}}
\newcommand{\BF}[1]{\mathbf{#1}}
\newcommand{\IT}[1]{\mathit{#1}}
\newcommand{\BB}[1]{\mathbb{#1}}
\newcommand{\TT}[1]{\mathtt{#1}}
\newcommand{\FK}[1]{\mathfrak{#1}}
\newcommand{\BS}[1]{\boldsymbol{#1}}


%% spaces 
\newcommand{\Real}{\BB{R}}
\newcommand{\Symb}{\mathcal S}
\newcommand{\borel}[1]{\mathcal{B}\left(#1\right)}


%%% Probability
\newcommand{\Ex}{\mathbf{E}}     % Probability of an event

\newcommand{\po}{\mathbf{P}}     % Probability of an event
\newcommand{\p}[1]{\po\left(#1\right)}     % Probability of an event
\renewcommand{\P}{\BF{P}}
\newcommand{\pd}[1]{p\left(#1\right)}     % Probability density  



%% Modelling symbols
%-------------MDP----------------------------
\newcommand{\MDP}{\mathsf{M}}
\newcommand{\POMDP}{\MDP_{\Z}}
\newcommand{\C}{\mathbf{C}}
\newcommand{\MB}{\mathsf{B}}

 
\newcommand{\X}{{\mathbb{X}}}  % State
\newcommand{\Z}{{\mathbb{Z}}}	% Observation space
\newcommand{\A}{{\mathbb{U}}} % Action space
\newcommand{\init}{\rho}
\newcommand{\tr}{t}


% product MDP
\renewcommand{\S}{{\mathbb{S}}}	% Observation space

\newcommand{\polb}{{\boldsymbol{\mu}}}
\renewcommand{\pol}{{\mu}}

\newcommand{\Y}{{\mathbb{Y}}}  % State

%-------------POMDP-----------------------------
\newcommand{\Hist}{{\mathbb{H}}}  %  History
\newcommand{\I}{{\mathbb{I}}}  %  History
\newcommand{\Belief}{b}
\newcommand{\trb}{\tr_\Belief}
\newcommand{\Xb}{\mathbb{B}}  % State
\newcommand{\initb}{\init_\Belief}


% -----------------Refinement relation----------
\newcommand{\InF}{\mathcal{U}_{v}}
\newcommand{\Wt}{\mathbb{W}_{\tr}}

\newcommand{\grid}{{\boldsymbol{\eta}}}

%%% Logic 
%------------------Predicates-----------------------------------
\newcommand{\Fpred}{{\mathcal F}}
\newcommand{\Lab}{\mathsf{L}}
\newcommand{\Labset}{\mathcal{L}}

\newcommand{\alphabeth}{\Sigma}
\newcommand{\word}{{\boldsymbol{\pi}}} % words formed from the alphabeth
\newcommand{\letter}{\pi} % words formed from the alphabeth

\newcommand{\Bel}{{\mathbf {T}}}
\newcommand{\BelR}{{\mathbf {R}}}
\newcommand{\trunc}[2]{\operatorname{trunc}_{#1}\left(#2\right)}


%%% Temporal logic symbols
\newcommand{\notltl}{\neg}
\newcommand{\andltl}{\wedge}
\newcommand{\orltl}{\vee}
\newcommand{\Next}{\ensuremath{\bigcirc}}
\newcommand{\Always}{\ensuremath{\ \square\ }}
\newcommand{\Event}{\ensuremath{\ \diamondsuit\ }}
\newcommand{\Until}{\ \CA{U}\ }
\newcommand{\Implies}{\Rightarrow}
\newcommand{\Equiv}{\Leftrightarrow}
\newcommand{\True}{\top}
\newcommand{\False}{\perp}
\newcommand{\AP}{{AP}}
\newcommand{\pred}{\xi}


\newcommand{\eps}{\epsilon} \newcommand{\rel}{\mathcal{R}} % numbers option provides compact numerical references in the text. 

%----- Exotic words----- 
\newcommand{\buchi}{B\"uchi\ }

%% Symbols of automata
\newcommand{\PA}{\mathcal{P}} 
\newcommand{\BA}{\mathcal{B}}
\newcommand{\TS}{\mathcal{F}}
\newcommand{\Language}{\mathbf{Lang}} % Language?
\newcommand{\KA}{\mathcal{K}}
\newcommand{\RA}{\mathcal{R}}
\newcommand{\FSA}{\mathcal{A}}

\newcommand{\TSX}{\BB{V}_\TS}
\newcommand{\TSE}{\BB{E}_\TS}
\newcommand{\TSEE}{\BB{E}}

\newcommand{\DTL}{DTL~}

 % Custom operators
\newcommand{\norm}[1]{\left\| {#1} \right\|}
\newcommand{\norminf}[1]{\left\| {#1} \right\|_{\infty}}
\newcommand{\normeucl}[1]{\left\| {#1} \right\|_{2}}
\newcommand{\abs}[1]{\left| {#1} \right|} 
\DeclareMathOperator{\diag}{diag}

\DeclareMathOperator*{\argmin}{arg\,min} 
\DeclareMathOperator*{\argmax}{arg\,max} 

\ifdefined\theoremstyle
% Don't load theorem style:
%% There are a number of predefined theorem-like environments in
%% ifacconf.cls:
%%
%% \begin{thm} ... \end{thm}            % Theorem
%% \begin{lem} ... \end{lem}            % Lemma
%% \begin{claim} ... \end{claim}        % Claim
%% \begin{conj} ... \end{conj}          % Conjecture
%% \begin{cor} ... \end{cor}            % Corollary
%% \begin{fact} ... \end{fact}          % Fact
%% \begin{hypo} ... \end{hypo}          % Hypothesis
%% \begin{prop} ... \end{prop}          % Proposition
%% \begin{crit} ... \end{crit}          % Criterion
\newtheorem{theorem}[thm]{Theorem}
\newtheorem{lemma}[thm]{Lemma}
\newtheorem{definition}[thm]{Definition}
\newtheorem{problem}{Problem}


\newtheorem{example}{Example}

\else
\usepackage{amsthm}
\theoremstyle{plain}
\newtheorem{theorem}{Theorem}
\newtheorem{cor}{Corollary}
\newtheorem{prop}{Proposition}
\newtheorem{lemma}{Lemma}
\newtheorem{remark}{Remark}
\newtheorem{cond}{Condition}

\newtheorem{example}{Example}
\newtheorem{theorem}{Theorem}
\newtheorem{definition}[theorem]{Definition}
\fi

 
\pdfinfo{
   /Author (S.Haesaert et al.)
   /Title  (Formal abstraction of POMDPs for Distribution LTL)
   /CreationDate (D:20101201120000)
   /Subject (Formal abstraction)
   /Keywords (abstraction;POMDP)
}

% Table caption wrangling
\usepackage{etoolbox}
 
\usepackage[utf8]{inputenc}
\usepackage{pgfplots}
\usepackage{tikz}


\allowdisplaybreaks[1]
%% commenting
\newcommand{\red}[1]{{\color{red} #1}}
\renewcommand{\axx}[1]{{\color{orange} Ali: #1}}

\newcommand{\new}[1]{{\color{blue}#1}}
\newcommand{\ind}{\mathbf{1}}

\newcommand{\cristi}[1]{{\color{orange}#1}}


\begin{document}


\begin{frontmatter}

\title{\huge Refinement-based temporal logic control of partially observable Markov decision processes }
%\thanks[footnoteinfo]{Sponsor and financial support acknowledgment
%goes here. Paper titles should be written in uppercase and lowercase
%letters, not all uppercase.}

\author[cal]{S. Haesaert} 
\author[cal]{P. Nilsson} 
\author[mit]{C.I. Vasile}
\author[jpl]{R. Thakker}
\author[jpl]{A. Agha}
\author[cal]{A.D.  Ames}
\author[cal]{R. M. Murray}



\address[cal]{California Institute of Technology, 
   Pasadena, CA 91125 USA} % (e-mail: \{haesaert,pettni,ames,murray\}@caltech ).}
\address[mit]{Massachusetts Institute of Technology, 
   Cambridge, MA 02139 USA}% (e-mail:  cvasile@mit.edu)}
\address[jpl]{Jet Propulsion Laboratory, 
   Pasadena, CA 91109 USA}% (e-mail: rohan.a.thakker@jpl.nasa.gov)} 
\maketitle
\begin{abstract}
The stochastic evolution of a partially observable Markov decision process (POMDP) can be modeled by an associated Markov decision process (MDP) in belief space.
In this work, we consider synthesis of controllers guaranteeing  specifications given in linear temporal logic on such belief models. The computational issues associated with correct-by-construction synthesis on continuous state spaces are circumvented via construction of a finite-state abstraction. A control policy is synthesized for the abstract model and refined back to the original model. We leverage the notion of approximate stochastic simulation to quantify the deviation of the approximate model. By compensating {\it a priori} for these deviations in the control synthesis, correctness guarantees are inherited by the refined control policy.
\end{abstract}
\begin{keyword} Belief space models,
correct-by-construction control synthesis, Markov decision processes, partially observable
\end{keyword}

\end{frontmatter}
%%%%%%%%%%%%%%%%%%%%%%%%%%%%%%%%%%%%%%%%%%%%%%%%
%%%%%%%%%%%%%%%%%%%%%%%%%%%%%%%%%%%%%%%%%%%%%%%%
 
\section{Introduction }\label{subsec:intro}

\red{[Start with motivating the big picture (maybe with a picture)]}

Temporal logics have emerged as an appropriate formalism for specification and design of controllers for navigation problems \citep{Murray2009}. To model control and decision making under motion and sensing uncertainty common in navigation problems, we consider systems modeled in the most generic and principled form as a Partially-Observable Markov Decision Process (POMDPs)~\citep{Kaelbling98,Smallwood73}. As in \citep{JonesDTL2013}, we specify the temporal logic properties with respect to their likelihood or uncertainty over the belief space of a POMDP. In this work, we newly  leverage approximate stochastic simulation relations in the synthesis of  the control policy,  
 
Given a specification $\psi$  written in linear temporal logic with propositions over the belief state of a POMDP, we are interested in the design of a policy $\pol$ such that  $\psi$ is satisfied with probability at least $p$. In this paper, we approach the synthesis problem as follows. Firstly, for a given belief MDP $\MB$ a finite-state abstraction  $\tilde \MB$ is computed. Secondly, we compute a labeling-based $\delta$-approximate stochastic  simulation relation between the abstract belief space model $\tilde{\MB}$ and the concrete model $\MB$. Then, a $\delta$-robust policy for the abstract $\tilde{\MB}$ is computed  together with a stationary value function for the associated stochastic optimal control problem. Finally, a policy for the concrete belief space model is obtained as its refinement.

\subsubsection{Literature.}For probabilistic temporal logic properties  over finite state Markov decision processes there exist several tools for policy synthesis and verification such as PRISM \citep{KNP11} and  Storm \citep{dehnert2017storm}. For Markov decision processes over uncountable state spaces, the characterization of properties cannot in general be attained analytically \citep{Abate1}, so an alternative is to approximate these models by simpler processes that are prone to be mathematically analyzed and algorithmically verified, such as finite-state MDPs \citep{soudjani2015faust}, and deterministic transition systems \citep{Zamani2014}. In \citep{haesaert2017verification}, an approximate stochastic simulation relation has been introduced that allows for the use of  formal abstractions for the correct-by-construction control synthesis with respect to probabilistic linear time temporal logic properties \citep{tech_report_TACAS}. 
 
 
Without full-state observations, the formal synthesis of controllers over hidden state models defines a more challenging problem. For finite state POMDP there are results on the verification and policy synthesis for PCTL properties \citep{Norman2017, Chatterjee2014}. Results for specifications defined on the continuous states of a POMDP are rather preliminary and have been focused mainly on reachability and safety problems. In  \citep{ding2013optimal} and \citep{LESSER20141989}, reachability and safety  are defined with respect to the hidden state of the POMDP. In \citep{ding2013optimal}, the optimal control of partially observable systems over safety specifications is analyzed.  The control synthesis solving the reachability problem over a  partially observable stochastic system is solved in \citep{LESSER20141989}.
 
In this paper, the properties of interest are defined directly on the belief space (i.e., the space of probability distributions - belief - over all possible states) as in \citep{Vasile2016,JonesDTL2013}. The properties over belief space express requirements on the estimation quality or uncertainty. We newly solve this type of synthesis problem in a correct-by-construction fashion.  For this we leverage the results in \citep{haesaert2017verification, tech_report_TACAS}, as such we can precede  discretization with model order reductions and simplifications.  Since we work over belief MDPs, this is crucial.  Further, and new in this paper, we also define an approximate stochastic simulation relation via non-determinism in the labeling, and we newly refine the control policy via the abstract value function.
 
In the next section, the POMDPs and their belief models are first introduced together with the temporal logic used to define properties of interest. Afterwards, we define the verification and control synthesis problems for temporal logic properties over belief MDPs. In Section~\ref{sec:refinement}, the synthesis problem is solved robustly by using the proposed labeling-based stochastic simulation relation. In Section~\ref{sec:case}, an illustrative case study is given.

 
\section{POMDPs and temporal logic specifications}

In this section, we give the definition of a Partially Observable Markov Decision Process (POMDP). Then we define a (propositional) linear temporal logic (LTL)  equivalent to the formulation of distribution temporal logic as originally given in \citep{JonesDTL2013} for POMDPs. 

\subsection{Partially Observable Markov  Decision Processes}

For a metric space $\Y$, we denote by $\borel{\Y}$ its Borel $\sigma$-field, i.e., the collection of all sets that can be formed from countable unions and intersections of open sets in $\Y$. We refer to  $(\Y,\borel{\Y})$ as a Borel-measurable space and we denote with $\mathcal P(\Y)$ the set of probability measures (or, equivalently, probability distributions) on $(\Y,\borel{\Y})$. A probability measure $\po \in \mathcal P(\Y)$ together with the Borel-measurable space $(\Y,\borel{\Y})$ defines a probability space $(\Y,\mathcal{B}(\Y),\po)$ with realizations $s{\,\sim\,}\po$. Given a probability measure, the corresponding expectation operator is denoted as $\Ex[\cdot]$. \new{In this work, we restrict attention to the case when $\Y$ is a Polish space, i.e, a complete and separable metric spaces \citep{bogachev2007measure}.}
    
Based on these notions from probability theory, we define a Markov decision process as follows \citep{bertsekas2004stochastic,mt1993,hll1996}. 
\begin{definition}
\label{def:MDP} 
  A discrete-time \textbf{Markov decision process} (MDP) is a tuple $\MDP = (\X, \init, \tr, \A)$ where
  \begin{itemize}
    \item $\X$ is a \new{(Polish)} state space with states $x\in\X$; % as its elements;
    \item $\init \in P(\X)$ is an initial probability distribution;
    \item $\A$ is a \new{(Polish)} input space with inputs $u\in\A$;
    \item $\tr:\X\times\A\times\mathcal B(\X)\rightarrow[0,1]$ is a conditional stochastic kernel that assigns to each state $x\in \X$ and control $u\in \A$ a probability measure $\tr(\cdot\mid x,u)$ over $(\X,\mathcal B(\X))$.
  \end{itemize}
\end{definition}

A path (or execution) of the MDP up to time $K$ is a state-input sequence $(x_0, u_0), (x_1, u_1), \ldots, (x_K, u_K)$ where $x_0 \sim \rho$ and $x_k \sim t(\cdot \mid x_k, u_k)$ for inputs $u_k \in \A$. While MDPs models capture uncertainty in state transitions, full knowledge is assumed about the state of the system. Augmenting MDPs with partial observability results in partially observable MDPs.
 
% Given a string of inputs $u(0),$ $u(1), $ $\ldots, $ $u(N)$, over a finite time horizon $0,1,\ldots, N$ and an initial state $x_0$ sampled from $\rho$, the states $x_{k+1}$ with $k\in \{0,1,\ldots, N\}$ are obtained as a realization of the controlled Borel-measurable stochastic kernel $\tr\left(\cdot{\,\mid\,t} x_k, u_k \right)$ \cristi{C: $\tr\left(\cdot\mid x_k, u_k \right)$ ?} -- these semantics induce paths (or executions) of the MDP.  
  
\begin{definition}
\label{def:POMDP}

A \textbf{partially observable Markov decision process} (POMDP) $\POMDP$ is an MDP $\MDP = (\X, \init, \tr, \A)$ together with an observation model $(\Z, r)$ where
\begin{itemize}
	\item $\Z$ is a (Polish) output space with outputs $z \in \Z$;
  \item $r : \X \times \mathcal B(\Y) \rightarrow [0,1]$ is an observation kernel that to each state $x$ assigns an output $z \in Z$ according to $z\sim r(\cdot|x)$.
\end{itemize}
\end{definition} 

An execution of the POMDP up to time $K$ is
\begin{equation}
\label{eq:history} 
  (x_0,u_0,z_0), (x_1,u_1,z_1), \ldots, (x_K,u_K,z_K),
\end{equation}
where $(x_0, u_0), \ldots, (x_K, u_K)$ is an execution of the associated MDP and $z_k \sim r(\cdot \mid x_k)$. This sequence is also referred to as the \emph{history sequence}. The execution \eqref{eq:history} grows with number of observations $K$ and takes values in the \emph{history space} $\Hist_K = (\X \times \A \times \Z)^{K+1}$.

In a POMDP control actions $u_k$ can be chosen as a function of the history of inputs and observed outputs. For this purpose, define for $k=0,\ldots,K-1,$ the \emph{$k$th information space} as $\I_k = (\Z \times \A)^{k} \times \Z$. Elements $i_k \in \I_k$ are $i_k = (z_0,u_0,z_1,u_1,\ldots,z_{k-1}, u_{k-1}, z_k)$ and are referred to as the \emph{$k$-th information vector}. 

Based on the notion of information, an \emph{observation-based} policy for $\POMDP$ is a sequence $\polb=(\pol_0,\ldots,\pol_{K-1})$ such that, for each $k$, $\pol_k( \mathrm{d} u_k|\init, i_k)$ is a universally measurable stochastic kernel on $\A$  given $\mathcal{P}(\X)\times \I_k$. We say that $\polb$ is \emph{non-randomized} if for all $\init$, $k$, and $i_k$, $\pol_k(\cdot|\init, i_k)$ is a Dirac distribution. Unless otherwise mentioned, we will in the following assume that a policy represents a sequence of stochastic kernels on $\A$. 

Given an observation-based policy $\pol$ and an initial distribution $\init$, the theorem of Ionescu Tulcea \citep{hll1996} implies the existence of a unique probability measure $\P_\pol^\init$ on the canonical space $\Hist_\infty$, or, equivalently, a stochastic process on the probability space $(\Hist_\infty, \borel{\Omega}, \P_\pol^\init)$.  
 
Given an information sequence $i_k$, state knowledge can be expressed as a conditional probability distribution 
\begin{align}
	b_k( \mathrm{d} x )=\P(x_k\in \mathrm{d} x |\init,i_k)\in \mathcal P (\X).
\end{align}
This distribution $b_k$ is called the \emph{belief state} and the set of all beliefs is the \emph{belief space} $\Xb\subset \mathcal P(\X)$. It can be shown that the belief state evolves based on a fixed stochastic kernel
\begin{align}
\label{eq:trb}
	 b_{k+1} \sim \trb(\cdot|b_k,u_k).
\end{align}
At each time step, the belief state can also be computed using a 
recursive filter $\tau$ as
\begin{equation}
\label{eq:filter_dynamics}
  b_{k+1}=\tau(b_k,u_k,z_{k+1}).
\end{equation}

Since \eqref{eq:filter_dynamics} is a completely observable system it follows that a POMDP $\POMDP$ can equivalently be expressed as an MDP over belief space. Given a POMDP $\POMDP$, we denote the corresponding belief space MDP as $\MB(\POMDP) = (\Xb, \init, \trb, \A)$, with $t_b$ given by \eqref{eq:trb}. In the sequel we will omit $\POMDP$ and write simply $\MB = \MB(\POMDP)$.

Since $\MB$ is a completely observable MDP, its information vector at time $k$ is given as $i_k=(b_0, u_0, b_1, u_1, \ldots, b_{k-1}, u_{k-1}, b_k)$. Thus a policy for $\MB$ is a sequence $\polb=(\pol_0,\ldots,\pol_{K-1})$ such that  for all $k$, $\pol_k(du_k|\init, i_k)$ is a universally measurable stochastic kernel on $\A$.	We say that a policy $\pol$ is a \emph{Markov policy} if for each $k$, it depends only on the current state, that is $\pol_k(du_k|\rho, i_k) = \pol_k(du_k| b_k)$. Furthermore $\polb$ is a \emph{stationary Markov policy} if $\pol(\cdot | b)= \pol_k( \cdot |b)$ for all $k$ and $b$.


\section{Linear Temporal Logic for Belief MDPs}

In this work we are interesting in guaranteeing properties of the belief state in order to control state evolution under uncertainty. First we introduce atomic propositions in belief space to give examples of the types of behavior that can be quantified. These propositions are the basic building blocks from which temporal specifications are constructed.
    
\subsection{Atomic propositions in belief space.}
\label{sec:DTL}  

An atomic proposition $p$ is associated with a subset of the belief space $\Xb$. While belief spaces are generally infinite-dimensional, Gaussian distributions are uniquely characterized by the mean and variance. That is, the belief state is $b = (\hat x, P)$ for $\hat x \in \mathbb{R}^n$ and $P \in \S^n$. In this case, examples of atomic propositions in belief space are:
\begin{enumerate}
  \item a position-based proposition $p_1 = \{ (\hat x, P) : \hat x \in A \}$ that evaluates to true when the state mean is in a specific set $A \subset \mathbb{R}^n$; 
  \item an uncertainty-based proposition $p_2 = \{ (\hat x, P) : \det(P) \leq c \}$ that evaluates to true when the determinant of the state variance is less than a given constant $c$; 
  \item a proposition $p_3 = \{ (\hat x, P) : \int_{A} P( \mathrm{d} x \mid \hat x, P) \geq c \}$ which assigns a lower bound $c$ to the probability of an event $A$. 
\end{enumerate}

Several atomic propositions can simultaneously hold at a state $(\hat x, P)$. For example, the mean $\hat x$ can belong to a given set as in (1), while the variance is also low as in (2). Consider a set $AP = \{ p_i \}_i$ of atomic propositions. This set defines an alphabet $\alphabeth := 2^{AP}$ where each letter $\letter$ of the alphabet is defined as a set of atomic propositions. An infinite string of letters is a \emph{word} $\word=\letter_0,\letter_1,\letter_2,\ldots\in\alphabeth^{\mathbb{N}}$.


To analyze the dynamic behavior of a system, we define the word associated to an execution. Consider a labeling function $\Lab:\Xb\rightarrow \alphabeth$ that maps belief states to letters in the alphabet. We require that $\Lab$ is measurable, i.e., that the induced sets $B_p:=\{b|\Lab(b)\vDash p \}\in \borel{\Xb}$ are Borel measurable. Since Borel measurability is preserved by standard linear operations \citep{azoff1974borel} \citep[page 116]{lang1993real}, the examples above are all Borel measurable.
 
Under these assumptions, the words generated by a belief trajectory $b_0, b_1, \ldots$ can be defined as the word $\word:=\Lab(b_0),\Lab(b_1),\Lab(b_2), \ldots$. System properties can now be expressed via temporal logic formulas over the generated words. 
  
\subsection{Linear temporal logic formulas}

Properties are formulas composed of atomic propositions and operators.
\begin{definition}
  \label{def:gdtl-syntax}
  The \textbf{syntax of LTL} is
  \begin{equation*}
   \psi :=  \True \ |\ p \ |\ \notltl \psi \ |\ \psi_1 \andltl \psi_2 \ |\ \psi_1 \Until \psi_2 \ |\ \Next \psi
  \end{equation*} 
  where $p\in \AP$ is an atomic proposition.  

\end{definition}
  
For convenience, the additional operators ``or"  $\psi_1 \orltl \psi_2 \equiv  \notltl (\notltl \psi_1 \andltl \notltl \psi_2)$, ``eventually" $\Event \psi \equiv \True \Until \psi$, and ``always"  $\Always \psi \equiv \notltl \Event \notltl \psi$ can be introduced. 

The LTL syntax defines the symbols and their correct ordering to form a formulae. To define the interpretation of a formula, i.e. the \emph{semantics}, consider a word  $\word$ with the suffix sequence $\word_i:= \letter_i, \letter_{i+1}, \letter_{i+2}, \ldots$.

\begin{definition}
 The \textbf{semantics} of LTL are defined recursively  over $\word_i$ as
    $\word_i \models \True$, 
    $\word_i \models p$ iff $p \in \letter_i$, 
    $\word_i \models \notltl \psi $ iff $\notltl (\word_i \models \psi) $, 
    $\word_i \models \psi_1 \andltl  \psi_2  $ iff $ ( \word_i \models \psi_1 ) \andltl ( \word_i \models \psi_2 ) $, 
    $\word_i \models  \psi_1 \Until \psi_2 $ iff $\exists j \geq i \text{ s.t. } (\word_j \models \psi_2 ) $ and $\word_k \models \psi_1, \forall k \in \{i, \ldots j-1\}$,
    $\word_i \models \Next \psi$ iff $\word_{i+1} \models \psi$.
\end{definition}

Given this definition, we say that a belief trajectory $\mathbf{b} = b_0, b_1, \ldots$ satisfies a specification $\psi$, written $\mathbf{b} \models \psi$ if the generated word $\word$ satisfies $\psi$ at time 0, i.e. $\word_0 \models \psi$.


\subsection{Problem statement}

The probability measure $\P_\init^\polb$ over the executions $b_0, b_1, b_2, \ldots\in \Xb^{\mathbb N}$ also induces a probability measure over the generated words $\word:= \Lab(b_0), \Lab(b_1), \Lab(b_2), \ldots\in \Sigma^{\mathbb N}$. The objective of this work is to design a policy $\polb$ such that a specification $\psi$ is satisfied with a given probability. 

\begin{problem}
  Consider a belief model $\MB = (\Xb, \rho, t, \A)$ generating trajectories $\mathbf{b} = b_0, b_1, \ldots$, and an LTL formula $\psi$ defined over atomic propositions in $\Xb$. Construct a policy $\polb$ such that
  \begin{equation}
    \P_\init^\polb ( \mathbf{b} \models \psi )\geq p,
  \end{equation}
  where $p$ is either given or to be maximized.
\end{problem}
To tackle this problem, we limit ourselves to the syntactically co-safe subset of LTL properties for which the synthesis problem can be solved via reachability. 


\subsection{Exact control synthesis for scLTL formulae}

The syntactically co-safe subset of LTL (scLTL) is given as
\begin{equation}
  \label{eq:scLTL}
  \psi :=  \True \ |\ p \ |\ \notltl p \ |\ \psi_1 \vee\psi_2  \ |\ \psi_1 \andltl \psi_2 \ |\ \psi_1 \Until \psi_2 \ |\ \Next \psi
\end{equation}
where $p\in \AP$.

Remark that based on this syntax, properties that enforce invariance such as $\Always\psi$ and properties that define liveness specifications, such as $\Always\!\lozenge \psi$, cannot be formed. An important property of this fragment is that satisfaction of a formula can be expressed as a reachability property in a deterministic finite-state automaton.

\begin{definition}
  A \textbf{deterministic finite-state automaton} (DFSA) is a tuple
  \begin{equation}
    \FSA = (Q, q_0, \Sigma, \delta_\FSA, Acc),   
  \end{equation}
  where $Q$ is a finite set of states, $q_0 \in S$ is the initial state, $\Sigma$ is the input alphabet with $\sigma\in\Sigma$ being a letter (or event) that triggers the transition between the states of the FSA. $\delta_\FSA : Q \times \Sigma \rightarrow Q$ is the transition function, and $Acc\subset Q$ is a set of accepting states.

  A word $\word = \letter_0, \letter_1,\ldots,$ \textbf{is accepted} by the DFSA if there exists a sequence $q_0,q_1, q_2, .., q_f$ with $q_f\in Acc$, that starts with the start state $q_0$ and for which $q_{i+1}=\delta_{\CA A}(q_i,\letter_i)$. In other words, a sequence of letters is accepted if the corresponding trace in the DFA {\it reaches} the set of accepting states. We denote the set of words accepted by a DFA $\CA A$ as $\Language (\mathcal A)$.
\end{definition}

For every property $\psi$ expressed as a syntactically co-safe LTL formula \eqref{eq:scLTL}, there exists a DFSA  $\FSA_\psi$ that models the same property \citep{Belta2017}. In particular,
\begin{equation}
  \word\models\psi \Leftrightarrow \word\in \Language(\FSA_\psi).
\end{equation}

We can now reason about satisfaction of properties by $\MB$ by analyzing a product system $\MB \otimes \FSA_\psi$. As shown in \citep{tmka2013}, this system is also an MDP.

\begin{definition}
\label{def:product}
  Given an MDP $\MDP = (\X, \init, \tr, \A)$, a finite alphabet $\Sigma$, a labeling function $\Lab:\X\rightarrow\Sigma$, and a DFA  $\FSA_\psi = (Q, q_0, \Sigma, \delta_\FSA, Acc)$, the \textbf{product} between $\MDP$ and $\FSA_\psi$ is another be MDP $\MDP\otimes\FSA_\psi = (\X \times Q, \bar\init, \bar{\tr}, \A)$, where $\bar\init(dx,q) = \init(dx)$ if $q= \delta_\FSA(q_0,\mathsf L(x))$ and $ \bar\init(dx,q) =0$ otherwise, and the transition kernel is similarly given as  
  \begin{equation*}
    \bar{\tr}(d y\times\{q'\}|x,q,u) = \begin{cases}\tr(dy|x,u)& \text{if } q' =\delta(q,\Lab(y)),\\ 0 & \text{otherwise.}  \end{cases}
  \end{equation*} 
\end{definition}


\new{
We say that a trajectory $
\{\bar x(t)\}_{t\geq 0}$  represents an accepted word for specification $\psi$ if it reaches the set of accept states $K:=\X\times Acc$.
This reachability property can be evaluated over a   trajectory with length $N$ as 
\begin{align*}\textstyle
\exists j \in [0,N]: \bar x(j) \in K.\end{align*}
The probability associated to this finite-time reach  event can be characterized as a boolean expression using indicator functions \cite{Abate1},
which leads to 
%its expression as 
an expectation over the state trajectories as 
\begin{align*} 
r_{\bar x }^\polb(K)=\mathbb{E}_{\bar x(0)}^{\bar \polb}\bigg[\sum\limits_{\mathclap{\qquad j\in[0,N]}}\ind_{K}(\bar x(j))\prod_{i=0}^{j-1}\ind_{\bar{\X}\setminus K}(\bar x(i))\bigg],
\end{align*} 
where $\ind_{B}(x) = 1$ if $x \in B$, 
and otherwise it is equal to $0$.  }
\cristi{$r(\cdot \mid \cdot)$ was used previously for the observation kernel of POMDPs.}

\new{
For a given stationary
%property and
policy $\polb$, the time-dependent value function $\!\mathbf V_\polb^N:\X\rightarrow [0,1]$,  is defined as 
\begin{align}\mathbf V^N_\polb(\bar x)=\mathbb{E}^\polb\!\left[\,\sum\limits_{\mathclap{\ \  j\in [1,N]}} \,\, \ind_K(\bar x(j)) \prod\limits_{\mathclap{i=k+1}}^{j-1}\ind_{\bar{\X}\setminus K}(\bar x(i))\bigg|\bar x(k)=\bar x\right]\!.\label{eq:Valfunc}\end{align}
For a given policy $\mathbf V_\polb^N$ 
expresses the probability that a state trajectory $\{\bar x({0}),\ldots, x({N})\}$,
starting from $\bar x({0})$, will reach the target set $K$ within the time horizon $N$,
while staying within the safe set $A$. 
Hence it expresses the probability of reaching $K$ in $N$ time steps.}
%
\new{This function allows expressing the  reachability probability backward recursively,
as follows. 
Denote the Bellman operator
\begin{align}\label{eq:V_recopt_inf}
& \Bel_\pol (\mathbf  V)(x,q) =\!\!\int_{\bar\X}  \left[\mathbf 1_{K}(\bar x')+  \mathbf 1_{\bar \X\setminus K}( \bar x')\mathbf V( \bar x'))\right]\notag\\&\hspace{5cm}\hfill 
\times \bar{\tr}(d \bar x' |\bar x,\pol(\bar x)),
\end{align}
with $K:=\X \times Acc$.
Then it follows that 
\begin{align}
\mathbf V_\polb^{N} = 
 \Bel_\pol \mathbf V_\polb^{N-1}  .\end{align}
Thus if $\mathbf V_\polb^{N-1} $ expresses the probability of reaching the accept states in $N-1$ then $ \Bel_\pol \mathbf V_\polb^{N} $ expresses the probability of reaching them in $N$.
Every recursion combines the probability to reach the accept states in the next transition with the probability of reaching it in the future.
For the product MDP, this Bellman operator reduces to 
\begin{align}\label{eq:V_recopt_inf_mu}
& \Bel_\pol (\mathbf V)(x,q) =\!\!\int_{\bar\X}  \left[\mathbf 1_{Acc}(q')+  \mathbf 1_{Q\setminus Acc}( q')\mathbf V( x', q'))\right]\notag\\&\hspace{3cm}\hfill 
\times \bar{\tr}(d \bar x' |x,q,\pol(x,q))
\end{align}
with $d \bar x'= d x'\times\{q'\}$. Note that the policy is defined for the product MDP, that is, it defines a control action as a function of the state $(q,x)\in\bar\X$.}
 
%\int_{\bar{\X}}\left[\mathbf 1_{K_\S}(\bar s)+ \mathbf 1_{\S\setminus K_\S}(\bar s) V(\bar s)\right]\bar{\tr}(d\bar s|s,\pol(s)),
%\end{align}
\marginpar{\red{[Richard: Give an explanation of what T represents in words]}}

\new{If we now look at the sequential application of $\mathbf T$, which is denoted by $\mathbf T^n$, we can compute the infinite horizon reachability probability based on the convergence of $\lim_{N\rightarrow \infty}\mathbf T_\mu^{N} \mathbf V$.
Then, for a given stationary policy $\mu$, the  probability of reaching $Acc $, that is $\P_\init^\polb(\MDP\vDash\psi) $, is defined as
\begin{align}
&\P_\init^\polb(\MDP\vDash\psi) = \P_\init^\polb
(\word \in\Language(\mathcal A_\psi))\notag= \\ &\lim_{N\rightarrow \infty}\int_{\bar{\X}}\left[\mathbf 1_{Acc}( q)+ \mathbf 1_{Q\setminus Acc}(q) \Bel^N_\mu\left( \mathbf V\right)(x_0,q)\right] {\bar \init} (d \bar x,q),\label{eq:P:prob}
\end{align}
with $\mathbf V=0$. Remark that a stationary policy $\pol$ for the product MDP can be translated to  time-dependent policy $\polb$ for the original MDP. The latter then implicitly includes the memory function of the product MDP.}

\new{Instead of defining the recursions for a given policy, we can also optimize it with respect to the set of deterministic  policies $\pol: \bar X\times Q\rightarrow \A$ with $\mathbf D_{\pol}$ the set of universally measurable deterministic policies.
This yields the optimal Bellman recursion as
\begin{align}\label{eq:V_recopt_inf}
& \Bel_\ast (\mathbf V)(x,q) =\sup_{\pol\in \mathbf D_{\pol}}\int_{\X}\left[\mathbf 1_{Acc}( q')+  \mathbf 1_{Q\setminus Acc}(q')\mathbf V(x',q'))\right]\notag\\&\hspace{3cm}\hfill\times \bar{\tr}(dx'\times\{q'\}|x,q,\pol(x,q)).
\end{align}
From \cite{Abate1}, we know that the optimal policy of the reachability problem is defined as the argument of the supremum and that the optimal policy is a stationary, universally measurable, and deterministic policy.  }

\new{Hence we have that the maximal satisfaction of a scLTL property $\psi$ reduces to finding the solution of stochastic reachability problem. As a consequence  this reachability problem can be written as follows
\begin{align}
&\sup_{\polb}\P_\init^\polb(\MDP\vDash\psi) = \sup_{\polb }\P_\init^\polb
(\word \in\Language(\mathcal A_\psi))=\label{eq:P:prob_opt} \\ &\lim_{N\rightarrow \infty}\int_{\bar{\X}}\left[\mathbf 1_{Acc}( q)+ \mathbf 1_{Q\setminus Acc}(q) \Bel^N_\ast\left( \mathbf V\right)(x_0,q)\right] {\bar \init} (d \bar x,q).\notag
\end{align}}

Even though we can express the probabilistic satisfaction of these scLTL properties as a stochastic reachability property, we can only perform the computation of the recursions  \eqref{eq:V_recopt_inf_mu} and \eqref{eq:V_recopt_inf}  for a limited set of systems. In general for continuous, or more general uncountable spaces, these recursions are intractable and one needs to use  approximate computations.
This is especially the case for high-dimensional belief MDPs. In this work, we will tackle this issue by constructing an approximate model $\tilde \MB$ that is close to the concrete model $\MB$.

\section{Refinement-based control synthesis} 
\label{sec:refinement}

\new{ Let a belief MDP $\MB$ and its abstraction $\tilde \MB$ be given.  
We assume that control synthesis for $\MB$ is intractable, whereas $\tilde \MB$ is computationally amiable for the control synthesis.
In this section, we show  how we can leverage computations performed on the abstract model to verify and control the concrete model. For this we first quantify the accuracy of the abstraction. Then we design a control policy for the abstract model that satisfies the probabilistic scLTL property robustly. Finally, we show that this control policy can be used to construct a control policy for the concrete belief MDP. We will refer to this construction as control refinement. In the next section, we will detail the abstraction for Gaussian  LTI models with partial observations.}



We approach  the control synthesis of the belief MDP in three steps:
\begin{enumerate}
\item We quantify the similarity of the models. For this we introduce the concept of a labeling based $\delta$-stochastic  simulation relation between  the abstract $\tilde{\MB}$ and the concrete $\MB$. Here, $\delta$ quantifies the difference in the stochastic transitions of the two MDPs.
\item Given a specification $\psi$, we are interested in  the $\delta$-robust evaluation of the probability that a specification $\psi$ is satisfied for  $\tilde\MB$.  More precisely, levering the $\delta$-stochastic simulation relations we will use the fact that there exists  a $\delta$-robust evaluation, denoted
$\mathbf{P_{\! rob}}_{\tilde \init}^{\!\ast}(\tilde \MB\vDash\psi; \tilde\Labset,\delta) %\leq  \mathbf P _{\init}^{\polb}(\MDP\vDash\psi).
 %\mathbf R _{\tilde \initb}^{\tilde \polb}(\tilde \MB\vDash\psi),
$ 
such that any control policy ${\tilde \polb}$ for $\tilde \MB$ can be refined to a control policy   ${ \polb}$ for $  \MB$ such that 
 \begin{align}
\mathbf{P_{\! rob}}_{\tilde \init}^{\!\ast}(\tilde \MB\vDash\psi; \tilde\Labset,\delta) 
 %\mathbf R _{\tilde \initb}^{\tilde \polb}(\tilde \MB\vDash\psi)
 \leq  \mathbf P _{\initb}^{\polb}(\MB\vDash\psi).
\end{align}
In the above notation $\tilde\Labset,\delta$ define, respectively, a relaxed labeling for the abstract system and $\delta$ and the deviation in stochastic transitions both introduced in the following subsections.
\item We give a construction of this refinement that hinges on the robust value function computed in (2).
\end{enumerate}




%In this section, we will first introduce the labeling based $\delta$-stochastic simulation relation. Then, in the next subsection we will give a computation which is robust for models in this $\delta$-stochastic simulation relations.  In subsection \ref{sec:control}, we introduce the new concept of value-based control refinement.
%%%%%%%%%%%%%%%%%%%%%%%%%%%%%%%%%%%%%%%%%%%%
\subsection{Lifting based simulation  and approximate similarity}
%\marginpar{\axx{I think in this paper what can help a lot is that:
%
%In each section, start by saying the puprpose of the section, like, In this section, we will ....
%
%And then at the end of the section, in a couple of lines re-emphaize the findings of the section and motivate the next section...
%
%}}
\new{In this subsection, we introduce a similarity relation between MDPs, that is amiable for belief MDPs. 
Let us anticipated this with the introduction of the concept of a relation between two spaces. }
For the sets $A$ and $B$ a relation $\rel\subset A\times B$ is a subset of the Cartesian product $A\times B$. The relation $\rel$ relates $x\in A$ with $y\in B$ if $(x,y)\in\rel$, which is equivalently written as $x\rel y$. 
We use the following notation for the mappings  $\rel(\widetilde A):=\{y: x\rel y, x\in \widetilde A\}$ and  $\rel^{-1}( \widetilde B):=\{x: x\rel y, y\in \widetilde B\}$ for $\widetilde A\subseteq A$ and $\widetilde B \subseteq B$.

%\begin{definition}[$\delta$-lifting for general state spaces]\label{def:del_lifting}
	Let $\X_1,\X_2$ be two sets with associated measurable spaces $(\X_1,\mathcal B(\X_1)),$ $(\X_2,\mathcal B(\X_2))$,
	and let   $\Delta\in \mathcal{P}(\X_1,\mathcal B(\X_1)) $ and  $\Theta\in \mathcal{P}(\X_2,\mathcal B(\X_2)) $ be two probability measures.  \new{Suppose that $\rel$ defines a set which \cristi{captures pairwise similarity between $x_1 \in \X_1$ and $x_2 \in \X_2$,}
% 	the values $(x_1,x_2)$ are pairwise `similar' in a certain way,
	then we can now quantify the similarity of these two measures as in \cite{haesaert2017verification}.}
	%	We denote by\[\bar\rel_\delta\subseteq \mathcal{P}(\X_1,\mathcal B(\X_1))\times \mathcal{P}(\X_2,\mathcal B(\X_2))\]
%	
\begin{definition}[$\delta$-lifting]\label{def:dellifting}
For a given 
	$\rel\subseteq \X_1\times \X_2$ with $\rel\in \mathcal B(\X_1\times \X_2)$ \cristi{($\po(x_1 \rel x_2) \in \mathcal B(\X_1 \times \X_2)$ ?)}, we say that  $\Delta$ and $ \Theta$ are in the corresponding $\delta$-lifted relation, denoted $\Delta \bar \rel_\delta \Theta$  if there exists a probability distribution $\mathbb W$ for the measure space $(\X_1\times \X_2,\mathcal B(\X_1\times \X_2),)$
	satisfying { \setlength{\parskip}{-1pt}\setlength{\parsep}{0pt}
		\begin{description}
			\item[\textbf{L1.}] for all $X_1\in \mathcal{B}(\X_1)$: $\mathbb W(X_1\times \X_2)=\Delta(X_1)$;
			\item [\textbf{L2.}] for all $X_2\in \mathcal{B}(\X_2)$:  $\mathbb W(\X_1\times X_2)=\Theta(X_2)$;
			\item[\textbf{L3.}] for the probability space  $(\X_1\times \X_2,\mathcal B(\X_1\times \X_2), \mathbb W)$ it holds that
			$x_1\rel x_2$ with probability at least $1-\delta$, or equivalently that $\mathbb{W}\left(\rel\right)\geq1-\delta$.
	\end{description}}%
	
We refer to  $\mathbb W$ as the lifting. %We define 
%\end{definition}
The set of related, or equivalently, $\delta$-lifted probabilities is defined as 
	\[\bar\rel_\delta\subseteq \mathcal{P}(\X_1,\mathcal B(\X_1))\times \mathcal{P}(\X_2,\mathcal B(\X_2)). \hfill \mbox{ }\qed\] 

\end{definition}

Hence based on \textbf{L1-3.} in Definition \ref{def:dellifting}, we can quantify the difference between two probability distributions with respect to a relation $\rel$.  \new{Examples of relations are the diagonal $\rel_{\Delta}=\{(x_1,x_2){\mid} x_1=x_2\}$, or a norm based relations $\rel_{\|\cdot\|}=\{(x_1,x_2){\mid} \|Px_1-x_2\|_2\leq \epsilon\}$ with  $P$  a projection matrix and $\epsilon$ a quantification of the precision. }
%\axx{at this point: it is not clear to a reader like me that why this relation has been defined. And how it really quantify the different between to distributions. Maybe one "example relation" can make the definion more intuitive.}
 
 Of interest is the quantification of similarity between two belief MDPs, for this we introduce an approximate probabilistic simulation relation similar to the one in \citep{haesaert2017verification}.
   First we analyze the relation $\rel$ over the (belief) state spaces.
\color{blue} Consider two belief MDPs, the concrete belief MDP $\MB= (\Xb, \initb, \trb, \A)$  and its abstraction $\tilde \MB=(\tilde\Xb, \tilde\initb, \tilde\trb, \tilde\A)$. 
We now want to quantify the similarity between these two stochastic processes.

  For the moment we will assume that a relation $\rel\subset \tilde \Xb \times \Xb$ is given, and we give the requirements for the two systems to be approximately stochastically simulated with respect to this relation. In the next section, we will give an example of such a relation for LTI Gaussian systems. 

As a first requirement, we need that the states in $\rel$ are pair-wise similar, that is, if a certain atomic proposition holds for the concrete model, it should also hold for the abstract model. To this end, we enable set-valued labelings for the abstract model denoted as  $ \tilde{\Labset}(\tilde x):\Xb\rightarrow2^\alphabeth$ and we require that  \begin{equation}\label{req:lab}
  \forall (\tilde x,x )\in \rel:  \Lab(x)\in \tilde{\Labset}(\tilde x)
  \end{equation} with $\Lab:\Xb\rightarrow\alphabeth$ the labeling function of the concrete belief MDP.
Consider 
 the trivial set-valued extension of  the labeling function, that is,  $\Labset:\cristi{\Xb}\rightarrow2^\alphabeth$ with
 $\Labset(x)=\{\Lab(x)\}$.
Requirement \eqref{req:lab} can now be written as  \begin{equation}
  \forall (\tilde x,x )\in \rel:  \Labset(x)\subset \tilde{\Labset}(\tilde x).
  \end{equation} 
  
Secondly, we require that there exists a $\delta$-lifting for the initial distributions $\initb$ and $\tilde\init$, that is,
\begin{equation}
\tilde\init \bar \rel_\delta \init.
	\tag{\textbf{SR 1.}}
\end{equation}

Finally, we also require that the stochastic transitions of the belief MDPs are approximately similar.  We require that any control action of the abstract MDP can be \emph{refined} to the concrete MDP such that the stochastic transitions are approximately similar. Levering the $\delta$-lifting  we can  express this as follows $\forall (\tilde x, x)\in \rel, \forall \tilde u \in \tilde A, \exists u \in \A$ such that 
\begin{equation}\label{req:translifting}
	\tilde \trb(\cdot| \tilde x, \tilde u)\ \bar \rel_\delta \  \trb(\cdot| x, u).
\end{equation}

To make sure that requirement \eqref{req:translifting} is constructive for the control refinement, we need a more strict condition on the measurability.  Hence, we require that 
  there exists a Borel measurable stochastic kernel $\Wt(\,\cdot\,{\mid} \tilde u,\tilde x,x)$ on $\tilde \Xb\times\Xb$ such that $\forall (\tilde x,x)\in \rel$, $\forall \tilde u\in\tilde \A$:
\begin{equation}\tilde \trb(\cdot| \tilde x, \tilde u)\ \bar \rel_\delta \  \trb(\cdot| x, \InF(\tilde u,\tilde x,x)),\tag{\textbf{SR 2.}}\end{equation} with respect to $\Wt$ and with $\InF$, a given Borel measurable  interface function, mapping the action and state pair to their refined actions
\begin{align*}\InF: \tilde \A\times \tilde \Xb\times\Xb \rightarrow \mathcal{P}(\A,\mathcal B(\A)). \end{align*}

Based on these requirements we are now ready to define a labeling-based $\delta$-stochastic simulation relation as follows. 

\begin{definition}[labeling-based $\delta$-stochastic simulation relation]\label{def:apbsim}
Consider a concrete MDP $\MDP$ and an abstract  MDP $\tilde\MDP$, with labeling maps $\Labset$ and  $\tilde{\Labset}$.   
We say that	$\tilde\MDP$ is $\delta$-stochastically simulated by $\MDP$, that is $$\tilde{\MDP}\preceq^{\delta}_{\tilde{\Labset},\Labset}\MDP,$$ with respect to $(\tilde{\Labset},\Labset)$  if there exists an interface function $\InF$ and
	a Borel measurable relation $\rel\subseteq \tilde \X\times \X$, for which \textbf{SR.1-2} hold and for which 	\begin{equation}
	  \forall (\tilde x,x)\in \rel:  \Labset(x)\subset \tilde{\Labset}(\tilde x)
\tag{\textbf{SR$\,\boldsymbol{\CA{L}}$.}}
	\end{equation} 
holds. \hfill\mbox{ }\qed
\end{definition}

 \color{black}
%  The former notion is defined as follows. 
%Consider a concrete MDP $\MDP$ and an abstract  MDP $\tilde\MDP$, with mappings $h_i$ to a shared {metric} output space  $(\Y,\mathbf{d}_\Y)$. \axx{Define $\Y$, and mention that $\mathbf{d}_\Y$ is the distance metric on $\Y$. When definition $\Y$ explain a bit more the notion of mapping to output space. Are you mapping the latent variable of the MDP? i.e., $y=h(x)\in\Y$. You can state it in equations.} 
%Given a relation over the state spaces of the MDPs we require that the initial distributions $\init$ and $\tilde\init$ can be lifted with $\delta$-accuracy \axx{under which relation? any relation?}
%\begin{equation}
%\tilde\init \bar \rel_\delta \init.
%	\tag{\textbf{SR 1.}}
%\end{equation}
%  
%Furthermore, we require that there exists a Borel measurable stochastic kernel $\Wt(\,\cdot\,{\mid} \tilde u,\tilde x,x)$ on $\tilde \X\times\X$ such that $\forall (\tilde x,x)\in \rel$, $\forall \tilde u\in\tilde \A$:
%\begin{equation}\tilde \tr(\cdot| \tilde x, \tilde u)\ \bar \rel_\delta \  \tr(\cdot| x, \InF(\tilde u,\tilde x,x)),\tag{\textbf{SR 2.}}\end{equation} with $\InF$, a given interface function mapping the action and state pair to a refined action for $\MDP$
%\begin{align*}\InF: \tilde \A\times \tilde \X\times\X \rightarrow \mathcal{P}(\A,\mathcal B(\A)). \end{align*}
%
%Thus the interface function implements (or refines) any control action synthesized over the abstract model to an action for the concrete model.
%\axx{I am lost here :) i do not understand how the interface funciton refines the control action. Even more basic question: what precisely "control action refinement" in this context mean?. Even more-more basic question: why do we need control action refinement? I think the answer to this last question is how we should start writing this part  (i.e., by pointing out an issue and menioning that we need refinement to solve that issue). then propose the interface function.}
%

In \citet{haesaert2017verification}, the difference between the models is quantified based on both a probabilistic error and metric error in an output space of interest. To allow for working with belief space models, we have changed the standard definition such that it now incorporates non-determinism  in the labeling instead of a metric error.
 %In this work we will newly use the computed value function to obtain a refinement.
%\red{[I propose to delete the next definition]}
%\axx{If you end up not deleting it, we should start with some intuition. Why do we need this simulation relation? what does it mean? Is it good for to MDP to be stochastically simulated?}
%\begin{definition}[$\epsilon,\delta$-stochastic simulation relation]\label{def:apbsim}
%Consider a concrete MDP $\MDP$ and an abstract  MDP $\tilde\MDP$, with mappings $h$ and  $\tilde h$  to a shared {metric} output space  $(\Y,\mathbf{d}_\Y)$.   
%	$\tilde\MDP$ is $(\epsilon,\delta)$-stochastically simulated by $\MDP$, denoted by $\tilde\MDP\preceq^{\delta}_\eps\MDP$,  if there exists an interface function $\InF$ and
%	a Borel measurable relation $\rel\subseteq \tilde \X\times \X$, for which \textbf{SR.1-2} hold and for which 
%	\begin{equation}
%		\forall (\tilde x,x)\in \rel:\mathbf{d}_\Y(\tilde h(\tilde x),h(x))  \leq \epsilon.\tag{\textbf{SR\,$\boldsymbol{\epsilon}$}.}
%	\end{equation} 
%\mbox{ }	\hfill\mbox{ }\qed
%\end{definition}
%Condition \textbf{SR$\epsilon$.} is overly conservative for the purpose of this paper. \axx{why?}
%Consider the trivial set-valued extension of  the labeling function $\Lab:\X\rightarrow\alphabeth$, that is  $\Labset:\X\rightarrow2^\alphabeth$ with
% $\Labset(x)=\{\Lab(x)\}$.
%% {\color{blue}A little confusing to use $\Labset$ for both language and labeling map}  => fixed
% 
%We now require that \begin{equation}
%  \forall (\tilde x,x )\in \rel:  \Labset(x)\subset \tilde{\Labset}(\tilde x).
%  \end{equation} 
%  \axx{Is this requirement the result of SR epsilon equation? how? We were talking about relations... what is the connection between "relations" and the "labelling" function? Is labelling function being treated as a mapping to the output space? If yes, I was thinking a bit wrong... when I read output space, I thought we are referring to measurement (e.g., sensory measurement) space. Like the common notion of "output" in basic control theory. Can you just call it "label space" instead of "output space" ?}
%
%\axx{And I still don't get what is the relation yet? it is quite abstract? an example earlier will help so that the reader can project the abstract definition in his/her mind to a concrete example.}
%  Consider the case that $b$ is defined by $\hat x$ and $P$. 
%The choice of $\tilde f =f $  for $f(\cdot):=\det(\cdot)+c_1$, this can be of interest if $\tilde P\succeq P$ implies that
% $f(\tilde b)\leq 0\rightarrow f( b)\leq 0$, then it suffices to show that
% for every state pair  $ (\tilde b,b)\in \rel$ it holds that $\tilde P\succeq P$.
 %Instead, pseudo norms can be leveraged as well as preorders over $b$.  
 
% 
% 
%\begin{definition}[labeling-based $\delta$-stochastic simulation relation]\label{def:apbsim}
%Consider a concrete MDP $\MDP$ and an abstract  MDP $\tilde\MDP$, with labeling maps $\Labset$ and  $\tilde{\Labset}$.   
%We say that	$\tilde\MDP$ is $\delta$-stochastically simulated by $\MDP$, that is $\tilde{\MDP}\preceq^{\delta}_{\tilde{\Labset},\Labset}\MDP$, with respect to $(\tilde{\Labset},\Labset)$  if there exists an interface function $\InF$ and
%	a Borel measurable relation $\rel\subseteq \tilde \X\times \X$, for which \textbf{SR.1-2} hold and for which 	\begin{equation}
%	  \forall (\tilde x,x)\in \rel:  \Labset(x)\subset \tilde{\Labset}(\tilde x)
%\tag{\textbf{SR$\,\boldsymbol{\CA{L}}$.}}
%	\end{equation} 
%holds. \hfill\mbox{ }\qed
%\end{definition}
%

%\axx{summarize the findings of the section, adn say how it connects to the next section...}

%%%%%%%%%%%%%%%%%%%%%%%%%%%%%%%%%%%%%%%%%%%%%%%%%%%%%%%%%%
\subsection{Robust control computation}

In this section, we define a robust quantification of the probability that a scLTL property is satisfied.   
We introduce robust computations,  based on
 \citet{tech_report_TACAS}, for which the policies are robust to the approximation errors. More precisely,   levering the $\delta$-approximate simulation relation these policies can be refined to the original MDPs', while preserving the satisfaction guarantees.

\new{To robustify the Bellman operator \eqref{eq:V_recopt_inf_mu}, we resolve the non-determinism caused by the set-valued labeling by minimizing over the associated values as}
\begin{align}\label{eq:V_recopt_inf}
& \Bel_\pol^\Labset (\mathbf V)(x,q) \!\!=\!\!\int_{\X} 
\min_{q' \in \delta_\FSA(q, \Labset( x'))}\left[\mathbf 1_{Acc}(q')\right.\notag\\ &\left.\hspace{1.25cm}+  \mathbf 1_{Q\setminus Acc}(q')\mathbf V( x',q'))\right] {\tr}(d x'|x,\pol(x,q)),
\end{align}
and the optimal Bellman 
\begin{align}
& \Bel_\ast^\Labset (\mathbf  V)(x,q) =\sup_\pol\int_{\X}\min_{ q' \in \delta_\FSA(q, \Labset(\bar x))}\left[\mathbf 1_{Acc}( q')\right.\hspace{1.25cm}\notag\\&\left.\hspace{1.5cm}+  \mathbf 1_{Q\setminus Acc}( q')\mathbf V( x', q'))\right]{\tr}(d x'
|x,\pol(x,q)).\label{eq:V_recopt_inf}
\end{align}
\marginpar{\tiny \axx{Here is a good place to contrast the above operators with the ones intriduced in the previous section, to remind the reader about concepts and create a coherent stroy.}}
To create a robust controller, we introduce a robust Bellman operator  $\BelR$. This operator is a  composition of the Bellman operator together with a correction factor $\delta$ and  a truncation operation, that is for $\mathbf V:\X\rightarrow [0,1]$
\begin{align}\label{eq:V_recopt_inf}
& \BelR_\pol^{(\Labset,\delta)} (\mathbf V)(x,q) = \trunc{[0,1]}{\Bel_\pol^\Labset (V)(x,q)-\delta}.
\end{align}
\new{Remark that the truncation clips negative values of the now robustified probabilistic reachability property to zero. }
%\axx{provide more intution about the meaning of R. Why do we have trunctation here?}
Similarly, the robust version of the optimal Bellman operator is given as 
\begin{align}\label{eq:V_recopt_inf}
& \BelR_\ast^{(\Labset,\delta)} (\mathbf V)(x,q)= \trunc{[0,1]}{\Bel_\ast^\Labset (V)(x,q)-\delta}
\end{align}

%The ``robustified'' optimal probability of reaching $Acc_\FSA$ is given as
%\axx{why?}
%\begin{align}
%	V^\infty_\ast & =\lim_{N\rightarrow \infty}\BelR_\ast^N (V) \\
%	\pol^\infty_\ast(x,q)  &= \max_\pol \BelR_\pol (V_\ast^\infty)(x,q) 
%\end{align}
%with $V=0$.
%\axx{what do you mean by $V=0$ ?}

The robustified optimal probability of reaching $Acc$ and hence satisfying $\psi$ is given as
\begin{align}\label{eq:prob:robust_optimal}
&\mathbf{P_{\! rob}}_\init^{\!\ast}(\MDP\vDash\psi; \Labset,\delta) =\\ & \int_{\X}\min_{\bar q \in \delta_\FSA(q_0,\Labset(x_0))}\left[\mathbf 1_{Acc}(\bar q)+ \mathbf 1_{Q \setminus Acc}(\bar q) \mathbf W^\infty_\ast(x_0,\bar q) \right]\notag 
\init (d x_0)-\delta\\
&\mbox{ with }\ \mathbf W^\infty_\ast :=  \lim_{N\rightarrow\infty}[\BelR_\ast^{(\Labset,\delta)}]^N (\mathbf W) \mbox{ and } \mathbf W=0.
\end{align} 
Remark the the robust optimal probability is parameterized with the set-valued labeling and the $\delta$ error in the stochastic transitions.


\begin{prop}\label{prop:1}
Suppose that $\tilde \MDP$  is  $\delta$-stochastically simulated by $\MDP$, that is  $\tilde{\MDP}\preceq^{\delta}_{\tilde{\Labset},\Labset}\MDP,$ with respect to  $({\tilde{\Labset},\Labset})$. 
Then, any control policy ${\tilde \polb}^*$ for $\tilde \MDP$ can be refined to a control policy   ${ \polb}$ for $  \MDP$ such that 
 \begin{align}\label{eq:prob_result}
\mathbf{P_{\! rob}}_{\tilde \init}^{\!\ast}(\tilde \MDP\vDash\psi; \tilde\Labset,\delta)\leq  \mathbf P _{\init}^{\polb}(\MDP\vDash\psi).  \qed
\end{align}
 \end{prop}
The proof of this set of operators and of the given proposition can be derived based on the the proofs in \citep{tech_report_TACAS} constructed for $(\eps,\delta)$- stochastic simulation relations.


\red{[Can be removed:]}
\begin{cor}\label{cor:2} Under the same conditions as in Prop. \ref{prop:1}, for 
any stationary optimal control policy ${\tilde \polb}^*$ for $\tilde \MDP$,  a control policy   ${ \polb}$ for $  \MDP$ such that 
 \begin{align}
\mathbf W^{\tilde \polb*}_{N-1} (\tilde x, q)\leq  \mathbf V^\polb_{N-1} (x,q)
\end{align} can be computed if the states are related, that is,
for $(\tilde x, x ) \in \rel$ and with $\mathbf V^\polb_{N-1} (x)$ computed with equation \eqref{eq:Valfunc} for $\MDP$, and  with $\mathbf W^{\tilde \polb}_{N-1} (x)$ computed with the robust operator $\mathbf R^{(\tilde \Labset, \delta)}_{\tilde\pol^\ast}$.  \qed
 \end{cor}


%
%
%
%\begin{figure}[htp]
%	\centering
%
%\begin{tikzpicture}
%
%\tikzset{model/.style={
%  rectangle,
%  inner sep=0pt,
%  text width=25mm,
%  align=center,
%  draw=black, fill=white,
%  minimum height = 10mm
%  }}
%  
%  \node[model] (filthat) at (4,0.75) {\textbf{apprx.  Filter}}; 
%\node[model] (POMDP) at (0,.75) {\textbf{POMDP}}; 
%
%%\node[label=below:$u$](bl) at (-2,.75) {};
%%\node[label=below:$x$](br) at (2,.75) {};
%\node(be) at (.75,1.25) {};
%\node(bw) at (-.75,1.25) {};
%
%%\path[draw,<-] (be.center) -- node[label=right:$\tilde u$]{} (.75,1.75);
%\path[draw,->] (POMDP) -- node[above] {$y$} (filthat);
%%\path[draw,<-] (bl.center)--(Bhat);
%%\path[draw,<-] (Bhat)--(br);
%\node[model, fill=white](m) at (0,-2) {\textbf{POMDP}};
%\node[model, fill=white](filt) at (4,-2) {\textbf{Filter}};
%
%\node[](ml) at (-.75,-1.5) {};
%\node[label=above:$b$](mr) at (6,-2) {};
%\node[label=above:$\hat b$](filthatr) at (6,.75) {};
%\path[draw,->] (-2.2,.75) -- node[above] {$ u$}(POMDP);
%\path[draw,->] (-1.8,.75) |-  (m);
%\path[draw,->] (filt)--(mr);
%\path[draw,->] (filthat)--(filthatr);
%\path[draw,->] (m)-- node[above] {$y$}(filt);
%\begin{scope}[on background layer]
%\node[model, fit=(ml) (filt) (m),inner sep=2.5mm,label=below:$\mathbf B$, fill=gray!30](B) {};
%\node[model, fit=(filthat) (POMDP),inner sep=2.5mm,label=below:$\hat{\mathbf{B}}$, fill=gray!30](B) {};
%\end{scope}			
%\end{tikzpicture}
%\caption{Control synthesis}
%\end{figure}

%%%%%%%%%%%%%%%%%%%%%%%%%%%%%%%%%%%%%%%%%%%%%%%%%%%%%
\subsection{Control refinement}\label{sec:control}
Based on proposition \ref{prop:1}, we already know we can construct a refined policy for the concrete model. In \cite{tech_report_TACAS}, this refined policy is constructed via the composition of the conditional lifting of the $\delta$-stochastic simulation relation and based on the abstract policy $\tilde  \pol^\ast$. 
In this paper, we give an improvement of this construction that is independent of the exact lifting used in the simulation relation. Instead, we use the robust  value function $\mathbf  W^\infty_\ast$ computed for the abstract system.
Given this robust value function  
	and abstract policy  $\tilde  \pol^\ast$. We can compute a refined  concrete policy over the product MDP $\MB\otimes\FSA_\psi$ as
	\begin{align}\label{eq:Valuebased_refined}
		\pol_\ast(x,q):=\, &\InF( \tilde{\pol}(\tilde x, q), \tilde x, x) \\ \mbox{ with } &\tilde x:=\arg\max_{ \tilde x\in
		\rel^{-1}(x)} \mathbf  W^\infty_\ast(\tilde x,q).\notag
	\end{align}
When  $\mathbf  W^\infty_\ast$ has a finite domain, then  the maximization in equation \eqref{eq:Valuebased_refined} is performed over a finite set and does not introduce any measurability issues. 
Remark that as a corollary of Proposition \ref{prop:1}, it also follows that the probability that the scLTL property is satisfied is still lower bounded as in equation \eqref{eq:prob_result}.
 
%%%%%%%%%%%%%%%%%%%%%%%%%%%%%%%%%%%%%%%%%%%%%%%%%
\section{Gaussian LTI system with partial observations}\label{sec:case}
In this section, we derive an approximate model for a POMDP that can be modeled as a Linear Time Invariant (LTI) model.  We first show that the belief MDP of this LTI model can be given based on the Kalman filter. Then, we introduce a formal abstraction of this model.

Consider an LTI system
 \begin{align} \label{eq:LTI} \begin{aligned}
x_{k+1}&=A x_{k} + B u_k+ w_k\\
z_k&=Cx_k+v_k\end{aligned} \end{align}
with $x\in \mathbb{R}^n$ with stochastic disturbances $w_t\sim \mathcal N(0,\CA W)$, and $v_t\sim \mathcal N(0,\CA V)$. 
Equation  \eqref{eq:LTI}  defines an MDP with state space $\X\subseteq\mathbb R^n$,  initial distribution  $\init:=\mathcal N(x_\init,P_\init)$,  control inputs $u_t\in\mathbb R^m$, and a transition kernel $t$. %defined based on difference eqution  \eqref{eq:LTI}. 
This is a partially observable MDP that can only be observed via  $z_t\in\mathbb R^q$.
 
 
%At $k=0$, we know $x_0\sim \init$ with $\init:=\mathcal N(x_\init,P_\init)$.
Before receiving a measurement $z_0$, the initial state is distributed  as $\CA N(\hat x_{0|-}, P_{0|-})$, with $\hat x_{0|-}:= x_\init$ and $P_{0|-}:= P_{\init}$.
After receiving the measurement $z_0$, this is updated to \begin{align*}
	\hat x_{0|0}&:= x_\init+ L_0 (z_0-Cx_\init),\\
	P_{0|0}&:=(I-L_0 C) P_{\init}(I-L_0 C)^T+L_0\CA V L_0^T,\\
	& \mbox{ with } L_0=P_{\init}C^T\left(CP_{\init}C^T+\mathcal V\right)^{-1},
\end{align*}
with $\po(x_t\in \cdot\,|\,\rho,z_0):=\CA N(\hat x_{0|0}, P_{0|0})$.
This probability distribution defines a belief state as $b_0:=(\hat x_{0|0}, P_{0|0})\in\mathbb R^n\times \mathbb S^n$. The belief space $\Xb$ is  a finite dimensional space and can be parameterized. For example, let $\CA{G}$ denote the Gaussian belief space
    of dimension $n$, i.e. the space of Gaussian
    probability measures over $\BB{R}^n$.
    For brevity, we identify the Gaussian measures
    with their finite parametrization, mean and
    covariance matrix.
     Thus,
    $\Xb \subseteq  \BB{R}^n \times  \Symb^n$.


The dynamics of  $b_k:=(\hat x_{k|k}, P_{k|k})$ are defined via the 
 Kalman filter, that is
	\begin{align*}
	&\text{\it predict: }&\hat x_{k|k-1}&=A\hat x_{k-1|k-1}+Bu_{k-1} \\
	&&P_{k|k-1}&=AP_{k-1|k-1}A^T+\mathcal W,
\\
	%&&\textbf{Update} %\  \qquad e_{k}&=z_k-C \hat x_{k|k-1}\\
	%&&S_k&=CP_{k|k-1}C^T+\mathcal V\\
	%&&
	&\text{\it update: }&\hat x_{k|k}&=\hat x_{k|k-1}+L_k\left(z_k-C \hat x_{k|k-1}\right)\\
	&&P_{k|k}&=(I-L_kC)P_{k|k-1}
	\end{align*}
	with  $L_{k}=P_{k|k-1}C^T\left(CP_{k|k-1}C^T+\mathcal V\right)^{-1}$.
 
This defines a belief MDP $\MB{(\POMDP)}$ with stochastic transitions of the belief state given as 
\begin{align}
	&&\hat x_{k|k}&=A\hat x_{k-1|k-1}+Bu_{k-1}+P_{k|k-1}C^Ts_k\label{eq:beliefx}\\
	&&P_{k|k}&=f(P_{k-1|k-1})\label{eq:beliefP}
\end{align}
with $s_k\sim \mathcal N (0, S_k )$ and  $S_k=\left(CP_{k|k-1}C^T+\mathcal V\right)$. 
\new{The computation of the backwards recursions \eqref{eq:V_recopt_inf} is intractable over the continuous space of this system. Hence the objective is to find a close finite state description of $\MB{(\POMDP)}$.  As a first step, we choose to remove the updates of the covariance matrix $P$.  }
% We define an abstraction of $\MB{(\POMDP)}$  as $\hat\MB$ with state space $\mathbb R^n$ and stochastic transitions
Furthermore in this LTI model,  we replace the stochastic transitions  in equation \eqref{eq:beliefx} by
\begin{align}  
		&&\hat x_k &=A\hat x_{k-1} +B\hat u_{k-1} + \bar P  C^T  \hat{s}_k,\label{eq:abstract} 
\end{align}
with $ \hat{s}_k\sim \CA N (0,\hat{S}_{inv})$ and $\hat{S}_{inv}\preceq S_k^{-1}$ for all $k$ \new{where $\bar P$ defines the steady state $P_{k|k-1}$, i.e., the solution of the Kalman equations. We say that the above system has a state space $\Xb_{x}$.}
These stochastic transitions \eqref{eq:abstract} can then be further abstracted to a finite state model $\hat\MB_{grid}$ by gridding the LTI model.
As in Figure \ref{fig:grid}, model  $\hat\MB_{grid}$ has a finite set of states $ \hat {\X}_{bx}$, denoted $x_s\in \hat {\X}_{bx}\subset  {\X}_{bx}$,  which are representative states that are distributed  equidistantly over the state space $  {\X}_{bx}$.   
Each finite state is the representative point $x_s\in  \hat {\X}_{bx}$ for a cell\footnote{$\oplus$ defines the  Minkowski sum also called the set sum. $\prod$ defines the cartesian product of sets.}
 $\Delta_{x_s}=\{x_s\} \oplus \prod_n [-\grid, \grid]$ such that the whole space is covered by these sets, that is, ${\X}_{bx} \subseteq \bigcup_{x_s\in \hat {\X}_{bx}}\Delta_{x_s}$.


Further $\hat\MB_{grid}$  has transitions 
\begin{align}\label{eq:tgrid}
t_{grid}(s'|s,u)=\hat t \left(\Delta_{s'}\mid x_s, u\right)
\end{align}where $\hat t$ is the stochastic transition kernel associated with difference equation \eqref{eq:abstract}.

\begin{figure}[htp]
\centering
	\includegraphics[width = .6\columnwidth]{figs/grid}
	\caption{Depiction of the gridded state space.The mean state of the concrete belief MDP $\bullet$ and the  representative state of the abstract MDP $\boldsymbol{+}$ are given together with an illustration of a stochastic transition.  \red{Feel free to improve[}}\label{fig:grid}
\end{figure}


 
Consider a simulation relation defined as 
	\begin{align}\label{eq:rel}
\rel := \left\{(x_s,b)| (\hat x-x_s)^T M(\hat x-x_s)\leq \eps^2, \right.\\\qquad\left.  P^-\preceq P \preceq   P^+ \mbox{ with } b=(\hat x, P ) \right\},\notag
	\end{align}
and an interface 
\[\InF(\tilde u,  x_s, \hat x):=K( \hat x - x_s)+\tilde u\]
\marginpar{\axx{interface deserve more explanation and intuition. It is a very key equation/example.}\red{Any suggestion?}}
for some matrices $M, K,P^+,P^-$.  
\new{We now derive the conditions under which $\rel$ and $\InF$ define a labeling-based $\delta$-stochastic simulation relation (c.f. Theorem \ref{def:apbsim}) between $\hat \MB_{grid}$ and $\MB$. That is, we  show that for the relation \eqref{eq:rel}, we can find a set-valued labeling  $\hat \Labset:\hat\X_{bx}\rightarrow 2^\alphabeth$ and that we can compute a $\delta$ such that $\hat \MB_{grid}\preceq^\delta_{\hat\Labset,\Labset}\MB$.}



\subsection{Labeling requirement {\bf SR $\boldsymbol{\Labset}.$}}
We show that  the simulation relation \eqref{eq:rel} enables the use of a set-valued labelling function $\hat \Labset:\hat\X_{bx}\rightarrow 2^\alphabeth$  for the atomic propositions introduced in Section \eqref{sec:DTL}.
To construct this labeling function, we require that for all $x_s$ if there exists
$ b \in \rel(x_s) $ then $ \Lab(b) \in \hat\Labset(x_s)$.

For a position-based proposition $p$ consider, without loss of generality, a labeling $\Lab_p: \Xb \rightarrow \{\{p\},\emptyset\}$ for the concrete belief MDP, denoted
$p\in\Lab_p(b)  \Leftrightarrow \hat x \in A$ with $b=(x,P)$.  The set-valued extension to $\{ \{\{p\},\emptyset\},\{\emptyset\}, \{\{p\}\} \}$
 for the abstract MDP is defined as
 \begin{align}
 	\hat \Labset_p(x_s) =\left\{\begin{array}{ll} \{\{p\}\} & \mbox{ if } \forall \hat x \in \rel (x_s):p\in\Lab_p(b), \\
 	  \{\emptyset \} & \mbox{ if } \forall \hat x \in \rel (x_s):p\not\in\Lab_p(b),\\
 	  \{\{p\},\emptyset\}&\mbox{ else.}\end{array} \right.
 \end{align}
This can be easily computed by shrinking, respectively, expanding the set $A$.  The extension towards more atomic propositions follows naturally.
Similarly, for propositions on the variance of the current belief state any atomic property that is monotonic in $P$ can be mapped to the abstract model.
Finally, for propositions that include probability, one should be more care-full as the quantification of probability of an event is in general not monotonic with the variance.

 
  
  
\subsection{Requirements {\bf SR 1.} and {\bf SR 2.}}
For  {\bf SR 1.}, the initial condition  for the concrete system is given deterministically. as $b_0:=(\hat x_{0|0}, P_{0|0})$. Hence we need to show that there exists an initial state $x_{s,0}$ such that the initial states $(x_{s,0},b_0)\in \rel$.  Thus we require that 
$P^-\preceq  P_{0|0} $ and  $P_{0|0}\preceq P^+$.  Additionally, we choose $x_{s,0}\in \rel^{-1}(\hat x_{0|0})$.    To make sure that the latter is not an empty set, it is sufficient to require that for all $\mathbf r \in \prod_n[-\boldsymbol \delta, \boldsymbol \delta]$ it hols that $ \mathbf r ^T M \mathbf r \leq \eps^2$.
 
 For  {\bf SR 2.},  we need to show that there exists a $\delta$-lifting. 
First we require that $P^+$, and $P^-$ are an upper, respectively, lower bound for $P_{k|k}$ of the belief MDP \eqref{eq:beliefx}-\eqref{eq:beliefP}.  We say that $P^-$ is a lower bound if it is a lower bound for the intial condition (see above) and   if it is monotonically increasing with respect to the Riccati equations see \citep{bitmead1985monotonicity}. For the upper bound, we require, mutatis mutandis, a monotonically decreasing $P^+$. 



We can quantify the difference between $\MB$ and $\hat\MB$ via equation \eqref{eq:rel} by verifying that for all  $(\hat x_k,\hat x_{k|k})\in \rel$ with probability at least $1-\delta$ it holds that $(\hat x_{k+1},\hat x_{k+1|k+1})\in \rel$. 
Consider a choice for the lifted stochastic  transitions  for  dynamics \eqref{eq:abstract} and \eqref{eq:beliefx},  denoted 
	$ \mathbb W_{x}((\hat x_k, \hat x_{k|k})\in \cdot| \hat u_{k-1}, \hat x_{k-1}, \hat x_{k-1|k-1})$, based on the combined stochastic difference equation given as
\begin{align*}
		&&\hat x_{k+1} &=A\hat x_{k} +B\hat u_{k} + \bar P  C^T  \hat{s}_{k+1},\\%\label{eq:abstract3} \\
	&&\hat x_{k+1|k+1}&=A\hat x_{k|k}+Bu_{k}+  \bar P   C^T(  \hat{s}_{k+1}+s^\Delta_{k+1})\notag\\&&&\qquad+\Delta_{k+1}( \hat{s}_{k+1}+ s^\Delta_{k+1})%\label{eq:beliefx3}
\end{align*}
 with $\Delta_k:=(P_{k|k-1}C^T-  \bar P   C^T)$ and with $ \hat{s}_k\sim \CA N (0,\hat{S}_{inv})$ and $ s^\Delta_k\sim  \CA N (0,\  S_k^{-1}-\hat{S}_{inv}). $


We can now choose the lifted stochastic transition kernel 	$\Wt$ for the concrete belief MDP $\MB$ and the abstracted finite MDP $\hat\MB$ as follows.
Denote $b=(\hat x, P)$ and $b_+=(\hat x_+ P_+)$, then $\Wt$ is computed as 
 \begin{align*}
 &	\Wt((x_{s,+},b_+)\in \cdot\,| \tilde u, x_s ,b)\\&:= \left\{\begin{array}{ll} \mathbb W_{x}((\Delta_{x_{s,+}}, \hat x_{+})\in \cdot\,|  \tilde u,x_s , \hat x) &\text{ for }  P_+=f(P)\\
 	0 & \text{ else } \end{array}\right.
 \end{align*}

For this choice of  	$\mathbb W_x$, the difference expression in \eqref{eq:rel} evolves   as 
\begin{align}
 \hat x_{k+1|k+1}-	\hat x_{k+1}=(A+BK)(\hat x_{k|k}-\hat x_{k-1})\qquad \quad\notag\\+  \bar P   C^T s^\Delta_{k+1} +\Delta_{k+1}( \hat{s}_{k+1}+ s^\Delta_{k+1})
\end{align}
 with $\Delta_{k+1}:=(P_{k+1|k}C^T-  \bar P   C^T)$, and with $ \hat{s}_{k+1}\sim \CA N (0,\hat{S}_{inv})$ and $ s^\Delta_{k+1}\sim  \CA N (0,\  S_{k+1}^{-1}-\hat{S}_{inv}). $
For all $ \hat x_{k+1}$, there exists $\mathbf  r \in \prod_n[-\grid ,\grid]$ such that   $\hat x_{k+1}-\mathbf r \in \hat \X_{bx}$. Therefore we can write the update of the difference expression as  \begin{align}
 \hat x_{+|+}-	\hat x_{s_+}=(A+BK)(\hat x_{\,|\,}-\hat x_s)+\mathbf r\qquad \quad\notag\\+  \bar P   C^T s^\Delta_{k+1} +\Delta_{k+1}( \hat{s}_{k+1}+ s^\Delta_{k+1}).
\end{align}
Given that $(\hat x_{\,|\,}-\hat x_s)$ and  $\mathbf r$ belongs to a bounded set, we can bound the influence of the noise terms $s^\Delta_{k+1}$ and $ \hat{s}_{k+1}$ with respect to a probability at least $1-\delta$ for which the update is always in $\rel$ see equation \eqref{eq:rel}.




%    
%Consider the linear time invariant system with Gaussian disturbance, given as 
%\begin{align}\begin{aligned}
%	x^m_{k+1} &= A x^m_{k}+B^mu_{k}+ F^m w_{k}\\
%	y^m_{k}&=C^m x^m_{k}+D^m u_{k}+E^m v_{k}\end{aligned}
%\end{align}
%with matrices $A,B,C,D$ and matrices $F,E$. 
%The measurement noise signal $v$ is zero-mean, independently and identically  distributed noise, i.e, $v_k\sim \mathcal{N}(0,I)$.
%% \noindent\textbf{Wind disturbance.}
%The state transitions are affected by the wind $w_k$. 
%We can model the dynamic variations of the wind as filtered noise. 
%There are two commonly used models  \citep{richardson2013quantifying},  this includes the von Karman power spectral density,
%and the Dryden model.
%The former model matches experiment data more than the 
% Dryden model, but the Dryden model can be represented by  a lower order filter.
%Consider a filter model to be given as
%\begin{align}
%	\begin{aligned}
%	x_{t+1}^w &= A^w x_{t}^w+ F^w e_{t},\\
%	w_{t}&=C^w x_{t}^w+E^w e_{t}.
%	\end{aligned}
%\end{align}
%
%%\textbf{Full model \& Belief space model.}
%The full model is given as follows
%%\begin{align} 
%%	\begin{bmatrix}
%%	x^m_{t+1}	\\x^w_{t+1}
%%	\end{bmatrix}
%% &= \begin{bmatrix}
%% 	A^m 	& F^mC^w\\
%% 	0 & A^w
%% \end{bmatrix}
%%\begin{bmatrix}
%%	x^m_{t}	\\x^w_{t}
%%	\end{bmatrix}+\begin{bmatrix} B^m \\ 0~ \end{bmatrix} u_{t}+  \begin{bmatrix}
%%	F^m E^w \\
%%	F^w
%%	\end{bmatrix}e_{t}\notag\\
%%	y^m_{t}&=\begin{bmatrix} C^m& 0 \end{bmatrix}\begin{bmatrix}
%%	x^m_{t}	\\x^w_{t}
%%	\end{bmatrix}+D^m u_{t}+E^m v_{t}.
%%\end{align}
%%This can be written as 
% \begin{align}  \begin{aligned}
%x_{t+1}&=A x_{t} + B u_t+ Fe_t\\
%y_t&=Cx_t+Du_t+Ev_t\end{aligned} \end{align}
%\begin{align}& \mbox{ with }  x_t	= \begin{bmatrix}
%	x^m_{t}	\\x^w_{t}
%	\end{bmatrix},
%A = \begin{bsmallmatrix}
% 	A^m 	& F^mC^w\\
% 	0 & A^w
% \end{bsmallmatrix}\!, \ 
%B = \begin{bsmallmatrix} B^m \\ 0~ \end{bsmallmatrix}\!,\notag \\
%&F=\begin{bsmallmatrix}
%	F^m E^w \\
%	F^w
%	\end{bsmallmatrix}\!,\ 
%C = \begin{bmatrix} C^m& 0 \end{bmatrix}\!, \ 
%D= D^m, \  E = E^m.\notag
%\end{align}
% 
% \noindent{\textbf{}}
%
%    
  
    
    \section{Case study}
    %!TEX root = 3_FormalAbstraction inLTLUnderUncertainty.tex

Motivated by space exploration applications we consider an autonomous rover tasked with identifying and collecting scientific samples.


\textbf{Rover model.} Without loss of generality, we consider a simple rover model as a point mass affected by stochastic disturbances on the state transitions. We assume that the position of the rover cannot be measured exactly. 
We model its dynamics as
\begin{align}
	x_{k+1}&=\begin{bmatrix}
		1&0\\
		0&1
	\end{bmatrix} x_{k} + \begin{bmatrix}
		1&0\\
		0&1
	\end{bmatrix} u_k+ w_k\\
z_k&=\begin{bmatrix}
		1&0\\
		0&1
	\end{bmatrix}x_k+v_k
\end{align}
with $w_k\sim \CA N \left(0,\begin{bmatrix}.4&-.2\\-.2&.4\end{bmatrix}\right)$ and $v_k\sim \CA N \left(0,\begin{bmatrix}1&.1\\.1&1\end{bmatrix}\right)$

A finite abstraction $M_{syst}$ of this model is constructed as outlined in the previous section by discretizing the input space $[-1,1]^2$ with nine discrete inputs, and the state space $[-10, 10]^2$ with grid points separated by $(0.76481, 0.64426)$.

\begin{figure}
	% This file was created by matplotlib2tikz v0.6.14.
\centering
\begin{tikzpicture}
\newlength\figurewidth
\newlength\figureheight
\setlength\figurewidth{\columnwidth}
\setlength\figureheight{.4\columnwidth}
\definecolor{color0}{rgb}{0.12156862745098,0.466666666666667,0.705882352941177}

\begin{axis}[scale =.8,
xlabel={$\delta$},
ylabel={$\epsilon$},
xmin=-0.01395, xmax=0.25,
ymin=0.8, ymax=1.45222764450879,
width=\figurewidth,
height=\figureheight,
% tick align=outside,
% tick pos=left,
x grid style={lightgray!92.026143790849673!black},
y grid style={lightgray!92.026143790849673!black},
axis y line=left,
axis x line=middle,
every axis x label/.style={
    at={(ticklabel* cs:1.05)}
}
]
\addplot [semithick, color0, forget plot]
table {%
0.001 1.4272384844875
0.0167368421052632 1.22339465219428
0.0324736842105263 1.16617411196529
0.0482105263157895 1.13191279574885
0.0639473684210526 1.10601589042324
0.0796842105263158 1.08400668945514
0.0954210526315789 1.06721503359466
0.111157894736842 1.05265893349788
0.126894736842105 1.04157018327801
0.142631578947368 1.02417176675944
0.158368421052632 1.01254639660019
0.174105263157895 1.00120158889137
0.189842105263158 0.991557539021032
0.205578947368421 0.982277989301703
0.221315789473684 0.97353496742902
0.237052631578947 0.965239656823681
0.252789473684211 0.960158902624749
0.268526315789474 0.951512372393061
0.284263157894737 0.927455284061691
0.3 0.935781247731614
};
\end{axis}

\end{tikzpicture}
	\caption{Trade-off between $\epsilon$ and $\delta$.  }
\end{figure}

\textbf{Environment model.} We consider exploration in a partially unknown environment. In particular, there are two potential target regions $T_1$ and $T_2$ where the probability of encountering a sample has been assessed as 0.5 and 0.6, respectively. In addition, there are two potential obstacle regions $O_1$ and $O_2$ where estimates based on satellite imagery give probabilities 0.1 and 0.3 that the rover can not traverse the obstacle safely. For both target and obstacle regions, we assume that the true nature of the region can be measured when the rover is within a certain distance of the regions. The regions are illustrated in Figs. \ref{fig:exp1} and \ref{fig:exp2}.

This environment model can be modeled as a finite-state MDP $M_{env}$. Furthermore, we model a failure probability of 0.01 at each step with an additional MDP $M_{fail}$.

\textbf{Specification.} The objective is to collect a sample while avoiding unsafe regions. To this end, we consider the specification
\begin{equation}
	\varphi = \lozenge \texttt{sample} \land \left( \lnot \texttt{fail} \; \mathcal {U} \; \texttt{sample} \right),
\end{equation}
where the atomic proposition \texttt{sample} is true if the rover is in a target region that contains a sample, and the atomic proposition \texttt{fail} is defined as being in an obstacle region that contains an obstacle, or that the system is in failure mode.

\textbf{Results.} We consider the aggregate system $M_{syst} \times M_{env} \times M_{fail}$ which is an MDP with 9 inputs and 277182 states. The specification $\varphi$ can be converted into a deterministic finite automaton. Via \eqref{eq:prob:robust_optimal} we can synthesize a control policy. Figs. \ref{fig:exp1} and \ref{fig:exp2} depict trajectories generated by the control policies, and the estimated probability that the specification can be satisfied, under two different resolutions of the unknown environment. In both experiments only $T_2$ contains a sample, and $O_1$ contains an obstacle. The difference is that in Fig. \ref{fig:exp1} also $O_2$ contains an obstacle but in Fig. \ref{fig:exp2} this region can be safely transversed.

\begin{figure}
	\footnotesize
	\setlength\figurewidth{\columnwidth} 
	\setlength\figureheight{0.6\columnwidth} 

	% This file was created by matplotlib2tikz v0.6.14.
\begin{tikzpicture}

\definecolor{color1}{rgb}{1,0.498039215686275,0.0549019607843137}
\definecolor{color0}{rgb}{0.12156862745098,0.466666666666667,0.705882352941177}
\definecolor{color2}{rgb}{0.172549019607843,0.627450980392157,0.172549019607843}

\begin{axis}[
xmin=-11, xmax=11,
ymin=-10.3867095347535, ymax=10.9707956921311,
width=\figurewidth,
height=\figureheight,
mark size = 0.5,
tick align=outside,
tick pos=left,
ticks=none,
x grid style={lightgray!92.026143790849673!black},
y grid style={lightgray!92.026143790849673!black}
]
\addplot [only marks, draw=black, fill=black, colormap/viridis]
table{%
x                      y
-8.000000000000000e+00 -9.000000000000000e+00
-8.281791193811282e+00 -6.953277059756937e+00
-6.609947231673338e+00 -7.304755895777043e+00
-6.447420151999264e+00 -6.984253859219061e+00
-5.930025319876885e+00 -6.296279125763103e+00
-5.476123458484502e+00 -6.049868453584115e+00
-4.680375628377898e+00 -4.982516422756326e+00
-3.356608221836718e+00 -4.767445283086258e+00
-3.850337386959414e+00 -3.882953851781119e+00
-3.888519145722516e+00 -3.480146659512000e+00
-5.320305970691908e+00 -2.277119535242806e+00
-4.379722627893722e+00 -3.023297381626749e+00
-2.966831046751210e+00 -2.008805754832921e+00
-4.696058526749307e+00 +1.019091306738940e-01
-4.628219306657025e+00 +1.300100090541607e+00
-5.641564937919989e+00 +1.796635145522382e+00
-6.307184225641469e+00 +2.288130625455811e+00
-7.217828733789998e+00 +3.143921812081303e+00
-7.301911031188041e+00 +4.255956608402661e+00
-6.401344954223247e+00 +5.270680166902570e+00
-8.171856419883056e+00 +6.563080734161671e+00
};
\addplot [only marks, draw=black, fill=black, colormap/viridis]
table{%
x                      y
+0.000000000000000e+00 -9.000000000000000e+00
+2.985780663708154e-02 -8.155463278156050e+00
+1.071409254737555e+00 -8.202590105544974e+00
+2.130577217930679e+00 -8.147034783291767e+00
+3.046653997506764e+00 -7.814782310273142e+00
+2.993503950729483e+00 -6.732513426782055e+00
+3.364980364937721e+00 -5.919601911047740e+00
+4.577267269183741e+00 -5.442531058948178e+00
+4.641924364448064e+00 -4.277469389688367e+00
+4.066522550812239e+00 -2.261074508296681e+00
+4.900884132823335e+00 -1.747839224197386e+00
+5.242948546817821e+00 -1.658932790996332e+00
+4.951650027020791e+00 -5.458682191996806e-01
+5.160727075484096e+00 +1.786499818804538e-02
+4.951600531141017e+00 +3.796585040088483e-01
+5.548111410812660e+00 +9.093914656273503e-01
+5.998289710281388e+00 +1.845859092339660e+00
+5.586468819813884e+00 +2.958893844938534e+00
+6.773322394656686e+00 +4.341573264476058e+00
+6.244801362810041e+00 +3.255339248138960e+00
+5.386058184709245e+00 +2.832977405695359e+00
+3.382447161289477e+00 +5.131303079675625e+00
+2.874337596801116e+00 +4.495954941056568e+00
+8.693708078920757e-01 +4.402472806183647e+00
-1.833677935774903e-01 +4.671535259512195e+00
-2.290906405270350e+00 +4.881507277565658e+00
-4.280927629432187e+00 +6.338912021995005e+00
-4.626650566950111e+00 +6.233071174567763e+00
-5.024226278687982e+00 +7.396015589088654e+00
-6.083092777439494e+00 +7.089329608744675e+00
};
\addplot [only marks, draw=black, fill=black, colormap/viridis]
table{%
x                      y
+8.000000000000000e+00 -9.000000000000000e+00
+7.712479407981174e+00 -9.068560218314534e+00
+7.684857446710697e+00 -9.261842777911905e+00
+9.091886701481306e+00 -9.116022915545141e+00
+7.920264049554019e+00 -8.627351053042807e+00
+7.836372046718704e+00 -7.993013882387533e+00
+6.924714511084391e+00 -7.711316244642425e+00
+6.270580096654860e+00 -7.113203236528710e+00
+4.261157395109652e+00 -5.405610734506721e+00
+4.549567306091522e+00 -4.854291172967754e+00
+5.603414469156846e+00 -4.801669493342254e+00
+5.663344946347504e+00 -4.411676549805895e+00
+6.055644152177520e+00 -4.710990344437167e+00
+5.399991387428658e+00 -2.248963208448632e+00
+4.079080783625948e+00 -3.641899917871495e-01
+3.920155202311550e+00 +8.691698572170817e-01
+3.370749568978111e+00 +3.132663534104976e+00
+3.568671650336753e+00 +3.232289113753480e+00
+4.510831005190408e+00 +3.354928317636965e+00
+5.435609201437062e+00 +3.518835438870448e+00
+5.534818929272653e+00 +4.406944686880900e+00
+3.207427357074954e+00 +6.678357019964662e+00
+1.772408135280159e+00 +5.221828658975668e+00
-2.171849729529804e-01 +5.918511271530562e+00
-1.407705408136741e+00 +7.142892522473158e+00
-2.578687393117541e+00 +5.885635198531411e+00
-4.001703186615487e+00 +7.335332717542842e+00
-4.767609190082132e+00 +7.291363002593066e+00
-6.458371658930611e+00 +8.009045427964324e+00
};
\path [draw=red, fill=red, opacity=0.5] (axis cs:2,-5)
--(axis cs:2,2)
--(axis cs:0,2)
--(axis cs:0,-5)
--cycle;

\path [draw=red, fill=red, opacity=0.5] (axis cs:2,3)
--(axis cs:2,10)
--(axis cs:0,10)
--(axis cs:0,3)
--cycle;

\path [draw=red, fill opacity=0, dashed] (axis cs:3.9,-6.9)
--(axis cs:3.9,3.1)
--(axis cs:-1.9,3.1)
--(axis cs:-1.9,-6.9)
--cycle;

\path [draw=red, fill opacity=0, dashed] (axis cs:3.9,1.1)
--(axis cs:3.9,10)
--(axis cs:-1.9,10)
--(axis cs:-1.9,1.1)
--cycle;

\path [draw=blue, fill=blue, opacity=0.5] (axis cs:-6,6)
--(axis cs:-6,9)
--(axis cs:-9,9)
--(axis cs:-9,6)
--cycle;

\path [draw=blue, fill=blue, opacity=0.5] (axis cs:9,6)
--(axis cs:9,9)
--(axis cs:6,9)
--(axis cs:6,6)
--cycle;

\path [draw=blue, fill opacity=0, dashed] (axis cs:-3.9,4.1)
--(axis cs:-3.9,10)
--(axis cs:-10,10)
--(axis cs:-10,4.1)
--cycle;

\path [draw=blue, fill opacity=0, dashed] (axis cs:10,4.1)
--(axis cs:10,10)
--(axis cs:4.1,10)
--(axis cs:4.1,4.1)
--cycle;

\addplot [semithick, color0, forget plot]
table {%
-8 -9
-8.28179119381128 -6.95327705975694
-6.60994723167334 -7.30475589577704
-6.44742015199926 -6.98425385921906
-5.93002531987688 -6.2962791257631
-5.4761234584845 -6.04986845358412
-4.6803756283779 -4.98251642275633
-3.35660822183672 -4.76744528308626
-3.85033738695941 -3.88295385178112
-3.88851914572252 -3.480146659512
-5.32030597069191 -2.27711953524281
-4.37972262789372 -3.02329738162675
-2.96683104675121 -2.00880575483292
-4.69605852674931 0.101909130673894
-4.62821930665702 1.30010009054161
-5.64156493791999 1.79663514552238
-6.30718422564147 2.28813062545581
-7.21782873379 3.1439218120813
-7.30191103118804 4.25595660840266
-6.40134495422325 5.27068016690257
-8.17185641988306 6.56308073416167
};
\addplot [semithick, color1, forget plot]
table {%
0 -9
0.0298578066370815 -8.15546327815605
1.07140925473755 -8.20259010554497
2.13057721793068 -8.14703478329177
3.04665399750676 -7.81478231027314
2.99350395072948 -6.73251342678205
3.36498036493772 -5.91960191104774
4.57726726918374 -5.44253105894818
4.64192436444806 -4.27746938968837
4.06652255081224 -2.26107450829668
4.90088413282334 -1.74783922419739
5.24294854681782 -1.65893279099633
4.95165002702079 -0.545868219199681
5.1607270754841 0.0178649981880454
4.95160053114102 0.379658504008848
5.54811141081266 0.90939146562735
5.99828971028139 1.84585909233966
5.58646881981388 2.95889384493853
6.77332239465669 4.34157326447606
6.24480136281004 3.25533924813896
5.38605818470924 2.83297740569536
3.38244716128948 5.13130307967563
2.87433759680112 4.49595494105657
0.869370807892076 4.40247280618365
-0.18336779357749 4.6715352595122
-2.29090640527035 4.88150727756566
-4.28092762943219 6.33891202199501
-4.62665056695011 6.23307117456776
-5.02422627868798 7.39601558908865
-6.08309277743949 7.08932960874468
};
\addplot [semithick, color2, forget plot]
table {%
8 -9
7.71247940798117 -9.06856021831453
7.6848574467107 -9.26184277791191
9.09188670148131 -9.11602291554514
7.92026404955402 -8.62735105304281
7.8363720467187 -7.99301388238753
6.92471451108439 -7.71131624464243
6.27058009665486 -7.11320323652871
4.26115739510965 -5.40561073450672
4.54956730609152 -4.85429117296775
5.60341446915685 -4.80166949334225
5.6633449463475 -4.41167654980589
6.05564415217752 -4.71099034443717
5.39999138742866 -2.24896320844863
4.07908078362595 -0.36418999178715
3.92015520231155 0.869169857217082
3.37074956897811 3.13266353410498
3.56867165033675 3.23228911375348
4.51083100519041 3.35492831763697
5.43560920143706 3.51883543887045
5.53481892927265 4.4069446868809
3.20742735707495 6.67835701996466
1.77240813528016 5.22182865897567
-0.21718497295298 5.91851127153056
-1.40770540813674 7.14289252247316
-2.57868739311754 5.88563519853141
-4.00170318661549 7.33533271754284
-4.76760919008213 7.29136300259307
-6.45837165893061 8.00904542796432
};
\node at (axis cs:-0.5,-3)[
  anchor=base west,
  text=black,
  rotate=0.0
]{ $R_1$};
\node at (axis cs:-0.5,7.5)[
  anchor=base west,
  text=black,
  rotate=0.0
]{ $R_2$};
\node at (axis cs:-9,7.5)[
  anchor=base west,
  text=black,
  rotate=0.0
]{ $T_1$};
\node at (axis cs:6.5,7.5)[
  anchor=base west,
  text=black,
  rotate=0.0
]{ $T_2$};
\end{axis}

\end{tikzpicture}
	\setlength\figureheight{0.4\columnwidth} 

	% This file was created by matplotlib2tikz v0.6.14.
\begin{tikzpicture}

\definecolor{color1}{rgb}{1,0.498039215686275,0.0549019607843137}
\definecolor{color0}{rgb}{0.12156862745098,0.466666666666667,0.705882352941177}
\definecolor{color3}{rgb}{0.83921568627451,0.152941176470588,0.156862745098039}
\definecolor{color2}{rgb}{0.172549019607843,0.627450980392157,0.172549019607843}
\definecolor{color5}{rgb}{0.549019607843137,0.337254901960784,0.294117647058824}
\definecolor{color4}{rgb}{0.580392156862745,0.403921568627451,0.741176470588235}

\begin{axis}[
xlabel={$t$},
ylabel={$\mathbf{P}(\varphi)$},
xmin=-1.8, xmax=37.8,
ymin=0.38853456067633, ymax=1.02911740187259,
width=\figurewidth,
height=\figureheight,
tick align=outside,
tick pos=left,
x grid style={lightgray!92.026143790849673!black},
y grid style={lightgray!92.026143790849673!black}
]
\addplot [semithick, color0, forget plot]
table {%
0 0.566291945691099
1 0.566291945691099
2 0.605882976959095
3 0.612354132971019
4 0.615146535917152
5 0.620313388350838
6 0.623169341037894
7 0.632266923722546
8 0.640795190330476
9 0.64428856744584
10 0.650176429635556
11 0.660779548866506
12 0.664373678411201
13 0.667953572290617
14 0.421487160337076
15 0.419564608493884
16 0.417651962548887
17 0.424169566255033
18 0.426345484650421
19 0.431108756242508
20 0.430751120445197
21 0.433029533909929
22 0.436207218324974
23 0.436207218324974
24 0.448504360104309
25 0.462109503911388
26 0.466372172152465
27 0.468495540003818
28 0.474972821467856
29 0.480216364807438
30 0.979868029536619
31 0.988089397434187
32 0.983895865110246
33 1.00000000000003
};
\addplot [semithick, color1, forget plot]
table {%
0 0.598873106893363
1 0.598873106893363
2 0.611938797531012
3 0.620313388350838
4 0.632266923722546
5 0.641338529177884
6 0.651286575651904
7 0.653667661427169
8 0.664373678411201
9 0.667389245077004
10 0.663807726695736
11 0.670472509073791
12 0.671058796741362
13 0.680876876002895
14 0.423299602729996
15 0.428005120445258
16 0.430751120445197
17 0.431108756242508
18 0.436564779737749
19 0.448714044552832
20 0.455361601380584
21 0.454662990626655
22 0.459229983640598
23 0.463841336213989
24 0.463444874596063
25 0.471062027085172
26 0.482793941967823
27 0.479801326577619
28 0.969453775583137
29 0.983895865110246
30 1.00000000000003
};
\addplot [semithick, color2, forget plot]
table {%
0 0.601199049439663
1 0.601199049439663
2 0.611373258769312
3 0.610825977206025
4 0.61950569968835
5 0.61950569968835
6 0.628810791270583
7 0.637985226843664
8 0.641338529177884
9 0.64428856744584
10 0.64428856744584
11 0.65073129659484
12 0.657238925992285
13 0.673118973708607
14 0.680502261669169
15 0.423299602729996
16 0.425630605759241
17 0.428005120445258
18 0.432928711351226
19 0.444768945274215
20 0.442429168485533
21 0.445763301335716
22 0.452612154776499
23 0.455361601380584
24 0.459622816160484
25 0.458418097195606
26 0.460938662424354
27 0.458418097195606
28 0.461333021386082
29 0.468094526343635
30 0.475377924614331
31 0.480216364807438
32 0.970297252430312
33 0.963929119062114
34 0.979868029536619
35 0.984943443987629
36 1.00000000000003
};
\addplot [semithick, color3, forget plot]
table {%
0 0.60221316758681
1 0.60221316758681
2 0.603095656982768
3 0.612754055717539
4 0.624538495801648
5 0.62791668893401
6 0.633081343456916
7 0.639922816020793
8 0.646852485781285
9 0.652756759496805
10 0.658726606328146
11 0.655977449212481
12 0.665293062315518
13 0.671265524930602
14 0.673823637438148
15 0.974478764836217
16 1.00000000000003
};
\addplot [semithick, color4, forget plot]
table {%
0 0.601395515122941
1 0.601395515122941
2 0.61392554430982
3 0.630267741952442
4 0.630807082939585
5 0.633670305343681
6 0.639922816020793
7 0.649819559965179
8 0.659285200283324
9 0.652182261074434
10 0.658796362546444
11 0.661342844708322
12 0.673823637438148
13 0.969766480712513
14 0.979000636609888
15 0.984943431279268
16 1.00000000000003
};
\addplot [semithick, color5, forget plot]
table {%
0 0.511047337255817
1 0.511047337255817
2 0.612978700250407
3 0.620063167202848
4 0.626830204231395
5 0.633081343456916
6 0.640464676396021
7 0.649268206152165
8 0.649819559965179
9 0.659844019602223
10 0.665920200487818
11 0.67605034228333
12 0.681786379389656
13 0.684269702304902
14 0.974797775257462
15 1.00000000000003
};
\end{axis}

\end{tikzpicture}
	\caption{Above: six trajectories starting at different initial conditions. Below: estimated probability to satisfy the specification over time for the same trajectories. No sample is found in $T_1$, and $O_2$ can be safely transversed. Jumps occur in the probabilities when (non-)existence of samples are measured when the rover is close to the regions.}
	\label{fig:exp1}
\end{figure}

\begin{figure}
	\footnotesize
	\setlength\figurewidth{\columnwidth} 
	\setlength\figureheight{0.6\columnwidth} 

	% This file was created by matplotlib2tikz v0.6.14.
\begin{tikzpicture}

\definecolor{color1}{rgb}{1,0.498039215686275,0.0549019607843137}
\definecolor{color0}{rgb}{0.12156862745098,0.466666666666667,0.705882352941177}
\definecolor{color2}{rgb}{0.172549019607843,0.627450980392157,0.172549019607843}

\begin{axis}[
mark size = 0.5,
ticks=none,
xmin=-11, xmax=11,
ymin=-10.6290882408405, ymax=10.9823375352781,
width=\figurewidth,
height=\figureheight,
tick align=outside,
tick pos=left,
x grid style={lightgray!92.026143790849673!black},
y grid style={lightgray!92.026143790849673!black}
]
\addplot [only marks, draw=black, fill=black, colormap/viridis]
table{%
x                      y
-8.000000000000000e+00 -9.000000000000000e+00
-8.281791193811282e+00 -6.953277059756937e+00
-6.609947231673338e+00 -7.304755895777043e+00
-6.447420151999264e+00 -6.984253859219061e+00
-5.930025319876885e+00 -6.296279125763103e+00
-5.476123458484502e+00 -6.049868453584115e+00
-4.680375628377898e+00 -4.982516422756326e+00
-3.356608221836718e+00 -4.767445283086258e+00
-3.850337386959414e+00 -3.882953851781119e+00
-3.888519145722516e+00 -3.480146659512000e+00
-5.320305970691908e+00 -2.277119535242806e+00
-4.379722627893722e+00 -3.023297381626749e+00
-2.966831046751209e+00 -2.008805754832921e+00
-4.696058526749306e+00 +1.019091306738935e-01
-4.628219306657023e+00 +1.300100090541606e+00
-5.641564937919985e+00 +1.796635145522380e+00
-6.307184225641461e+00 +2.288130625455808e+00
-7.217828733789983e+00 +3.143921812081298e+00
-7.301911031188013e+00 +4.255956608402649e+00
-6.401344954223192e+00 +5.270680166902547e+00
-8.171856419882948e+00 +6.563080734161627e+00
};
\addplot [only marks, draw=black, fill=black, colormap/viridis]
table{%
x                      y
+0.000000000000000e+00 -9.000000000000000e+00
+2.985780663708160e-02 -8.155463278156050e+00
+1.071409254737555e+00 -8.202590105544974e+00
+2.130577217930679e+00 -8.147034783291767e+00
+3.046653997506764e+00 -7.814782310273142e+00
+2.993503950729483e+00 -6.732513426782055e+00
+3.364980364937721e+00 -5.919601911047740e+00
+4.577267269183741e+00 -5.442531058948178e+00
+4.641924364448064e+00 -4.277469389688367e+00
+4.066522550812240e+00 -2.261074508296681e+00
+4.900884132823337e+00 -1.747839224197387e+00
+5.242948546817824e+00 -1.658932790996334e+00
+4.951650027020797e+00 -5.458682191996835e-01
+5.160727075484108e+00 +1.786499818804027e-02
+4.951600531141041e+00 +3.796585040088380e-01
+5.548111410812709e+00 +9.093914656273301e-01
+5.998289710281483e+00 +1.845859092339620e+00
+5.586468819814069e+00 +2.958893844938457e+00
+6.773322394657046e+00 +4.341573264475908e+00
+6.244801362810740e+00 +3.255339248138667e+00
+5.386058184710604e+00 +2.832977405694789e+00
+3.382447161292124e+00 +5.131303079674516e+00
+4.874337596806267e+00 +2.495954941054408e+00
+4.650503806334480e+00 +3.995285095184848e-01
+5.025471920274055e+00 -5.406637729947061e-01
+3.653106795537211e+00 -2.331886810567701e+00
+3.777550916600270e+00 -3.262576207290882e+00
+4.208193801414156e+00 -5.223790221008551e+00
+4.894433739744199e+00 -6.435884403595750e+00
+3.799772582533073e+00 -8.195765033159091e+00
+1.063593830208735e+00 -7.803367444083015e+00
-1.186440212491861e+00 -8.416040749834128e+00
-1.769357566346293e+00 -8.393133290175719e+00
-3.454943820617372e+00 -7.829608609619289e+00
-4.718419194171268e+00 -6.806453261226547e+00
-7.128929570346833e+00 -5.582933784317277e+00
-6.754145560982441e+00 -4.927567852678047e+00
-7.809944106601641e+00 -3.287103405394836e+00
-7.608490106653687e+00 -2.786103027237490e+00
-6.664925925528791e+00 -2.687368525725410e+00
-6.819564565858922e+00 -2.207441188034236e+00
-6.119656171566637e+00 -2.321958612214557e+00
-5.398012336881704e+00 -6.633452969104143e-01
-6.326849807413460e+00 +6.558845980278348e-01
-6.175272895510815e+00 +1.444858418003883e+00
-7.632034853475458e+00 +3.355390912891568e+00
-6.251375019139898e+00 +4.004939418758159e+00
-6.160033028437569e+00 +4.084610663102694e+00
-5.910891233600980e+00 +3.553568231521559e+00
-7.187232311111987e+00 +4.703244684591716e+00
-8.802150302563509e+00 +8.035950400947264e+00
};
\addplot [only marks, draw=black, fill=black, colormap/viridis]
table{%
x                      y
+8.000000000000000e+00 -9.000000000000000e+00
+8.511205582201939e+00 -9.490847881526234e+00
+8.123691607827535e+00 -8.945490678745058e+00
+8.263822914662137e+00 -7.614276297182030e+00
+7.514100738376864e+00 -8.134440925432358e+00
+7.298345618417102e+00 -7.124767904275431e+00
+5.661620747278767e+00 -6.163981262845379e+00
+4.308129478899167e+00 -4.248895613551244e+00
+4.661943867045995e+00 -3.734081630931301e+00
+6.045638341595694e+00 -3.283722422053446e+00
+4.311200610860274e+00 -3.054620213961065e+00
+3.764155218303413e+00 -1.339286613286621e+00
+4.507407456386630e+00 -6.260285666477190e-01
+6.195941762467750e+00 -1.751868652546797e+00
+5.652298276850433e+00 -1.433432909888064e+00
+6.004475142579373e+00 -1.176455188056848e+00
+7.867024860002735e+00 -1.436677206342497e+00
+6.149524945185612e+00 -1.765396180120760e-02
+7.231136140072594e+00 -2.834317843654784e-01
+5.357245998891423e+00 +1.116871382840182e+00
+4.924762093179862e+00 +2.175740570857628e+00
+6.145247254490910e+00 +1.347818433730404e+00
+5.672732010334618e+00 +2.405018338308785e+00
+4.778200199507558e+00 +4.416354675986674e+00
+3.100685883042153e+00 +4.107914445522056e+00
+3.891391072358141e+00 +2.859356344633611e+00
+3.874517594984206e+00 +2.035479937103637e+00
+5.201506296191360e+00 -2.313046332494439e-01
+5.104339902961216e+00 -1.029190621036462e+00
+5.015533858742012e+00 -2.461605591765912e+00
+3.404566585667478e+00 -3.208436019200696e+00
+4.063604053128744e+00 -4.223179916104084e+00
+5.419276458139430e+00 -6.587904229262779e+00
+3.770997682573817e+00 -8.969356381483692e+00
+2.072826036930491e+00 -7.932486401835083e+00
+4.691549265209518e-01 -8.578316323074933e+00
-3.520651457661232e-01 -7.765184235105794e+00
-2.518265654752808e+00 -6.411080484717628e+00
-3.413207860281108e+00 -6.198922651652184e+00
-3.456088450097421e+00 -6.235552101303306e+00
-4.673123778502697e+00 -5.140669103912209e+00
-5.511881550779163e+00 -4.652643794197237e+00
-6.952415477083844e+00 -3.989299476321714e+00
-6.474737771072504e+00 -3.130571715919226e+00
-7.065234191935879e+00 -2.542851770813614e+00
-7.172383147511338e+00 -1.298336855355604e+00
-5.757402873681920e+00 -1.438398078411823e+00
-6.528591566722448e+00 -6.312978098556827e-01
-5.772384260334218e+00 +1.466319934873163e-01
-6.242081541010274e+00 -1.387082780615503e-01
-5.956475469228075e+00 +4.792766582561609e-01
-5.373508577776141e+00 +2.220560436309609e-01
-6.942001756278491e+00 +1.075019277811085e-02
-5.789985714740869e+00 -1.994507978371182e-01
-8.574179717233536e+00 +6.995122202175492e-01
-8.546440565751025e+00 +2.428342665136468e+00
-6.945052246644407e+00 +3.410292093002863e+00
-5.658246322673215e+00 +3.669916744769946e+00
-6.704820165881237e+00 +4.943999287103432e+00
-6.985289853433318e+00 +6.021483786213237e+00
};
\path [draw=red, fill=red, opacity=0.5] (axis cs:2,-5)
--(axis cs:2,2)
--(axis cs:0,2)
--(axis cs:0,-5)
--cycle;

\path [draw=red, fill=red, opacity=0.5] (axis cs:2,3)
--(axis cs:2,10)
--(axis cs:0,10)
--(axis cs:0,3)
--cycle;

\path [draw=red, fill opacity=0, dashed] (axis cs:3.9,-6.9)
--(axis cs:3.9,3.1)
--(axis cs:-1.9,3.1)
--(axis cs:-1.9,-6.9)
--cycle;

\path [draw=red, fill opacity=0, dashed] (axis cs:3.9,1.1)
--(axis cs:3.9,10)
--(axis cs:-1.9,10)
--(axis cs:-1.9,1.1)
--cycle;

\path [draw=blue, fill=blue, opacity=0.5] (axis cs:-6,6)
--(axis cs:-6,9)
--(axis cs:-9,9)
--(axis cs:-9,6)
--cycle;

\path [draw=blue, fill=blue, opacity=0.5] (axis cs:9,6)
--(axis cs:9,9)
--(axis cs:6,9)
--(axis cs:6,6)
--cycle;

\path [draw=blue, fill opacity=0, dashed] (axis cs:-3.9,4.1)
--(axis cs:-3.9,10)
--(axis cs:-10,10)
--(axis cs:-10,4.1)
--cycle;

\path [draw=blue, fill opacity=0, dashed] (axis cs:10,4.1)
--(axis cs:10,10)
--(axis cs:4.1,10)
--(axis cs:4.1,4.1)
--cycle;

\addplot [semithick, color0, forget plot]
table {%
-8 -9
-8.28179119381128 -6.95327705975694
-6.60994723167334 -7.30475589577704
-6.44742015199926 -6.98425385921906
-5.93002531987688 -6.2962791257631
-5.4761234584845 -6.04986845358412
-4.6803756283779 -4.98251642275633
-3.35660822183672 -4.76744528308626
-3.85033738695941 -3.88295385178112
-3.88851914572252 -3.480146659512
-5.32030597069191 -2.27711953524281
-4.37972262789372 -3.02329738162675
-2.96683104675121 -2.00880575483292
-4.69605852674931 0.101909130673894
-4.62821930665702 1.30010009054161
-5.64156493791999 1.79663514552238
-6.30718422564146 2.28813062545581
-7.21782873378998 3.1439218120813
-7.30191103118801 4.25595660840265
-6.40134495422319 5.27068016690255
-8.17185641988295 6.56308073416163
};
\addplot [semithick, color1, forget plot]
table {%
0 -9
0.0298578066370816 -8.15546327815605
1.07140925473755 -8.20259010554497
2.13057721793068 -8.14703478329177
3.04665399750676 -7.81478231027314
2.99350395072948 -6.73251342678205
3.36498036493772 -5.91960191104774
4.57726726918374 -5.44253105894818
4.64192436444806 -4.27746938968837
4.06652255081224 -2.26107450829668
4.90088413282334 -1.74783922419739
5.24294854681782 -1.65893279099633
4.9516500270208 -0.545868219199684
5.16072707548411 0.0178649981880403
4.95160053114104 0.379658504008838
5.54811141081271 0.90939146562733
5.99828971028148 1.84585909233962
5.58646881981407 2.95889384493846
6.77332239465705 4.34157326447591
6.24480136281074 3.25533924813867
5.3860581847106 2.83297740569479
3.38244716129212 5.13130307967452
4.87433759680627 2.49595494105441
4.65050380633448 0.399528509518485
5.02547192027405 -0.540663772994706
3.65310679553721 -2.3318868105677
3.77755091660027 -3.26257620729088
4.20819380141416 -5.22379022100855
4.8944337397442 -6.43588440359575
3.79977258253307 -8.19576503315909
1.06359383020873 -7.80336744408301
-1.18644021249186 -8.41604074983413
-1.76935756634629 -8.39313329017572
-3.45494382061737 -7.82960860961929
-4.71841919417127 -6.80645326122655
-7.12892957034683 -5.58293378431728
-6.75414556098244 -4.92756785267805
-7.80994410660164 -3.28710340539484
-7.60849010665369 -2.78610302723749
-6.66492592552879 -2.68736852572541
-6.81956456585892 -2.20744118803424
-6.11965617156664 -2.32195861221456
-5.3980123368817 -0.663345296910414
-6.32684980741346 0.655884598027835
-6.17527289551082 1.44485841800388
-7.63203485347546 3.35539091289157
-6.2513750191399 4.00493941875816
-6.16003302843757 4.08461066310269
-5.91089123360098 3.55356823152156
-7.18723231111199 4.70324468459172
-8.80215030256351 8.03595040094726
};
\addplot [semithick, color2, forget plot]
table {%
8 -9
8.51120558220194 -9.49084788152623
8.12369160782753 -8.94549067874506
8.26382291466214 -7.61427629718203
7.51410073837686 -8.13444092543236
7.2983456184171 -7.12476790427543
5.66162074727877 -6.16398126284538
4.30812947889917 -4.24889561355124
4.66194386704599 -3.7340816309313
6.04563834159569 -3.28372242205345
4.31120061086027 -3.05462021396107
3.76415521830341 -1.33928661328662
4.50740745638663 -0.626028566647719
6.19594176246775 -1.7518686525468
5.65229827685043 -1.43343290988806
6.00447514257937 -1.17645518805685
7.86702486000273 -1.4366772063425
6.14952494518561 -0.0176539618012076
7.23113614007259 -0.283431784365478
5.35724599889142 1.11687138284018
4.92476209317986 2.17574057085763
6.14524725449091 1.3478184337304
5.67273201033462 2.40501833830879
4.77820019950756 4.41635467598667
3.10068588304215 4.10791444552206
3.89139107235814 2.85935634463361
3.87451759498421 2.03547993710364
5.20150629619136 -0.231304633249444
5.10433990296122 -1.02919062103646
5.01553385874201 -2.46160559176591
3.40456658566748 -3.2084360192007
4.06360405312874 -4.22317991610408
5.41927645813943 -6.58790422926278
3.77099768257382 -8.96935638148369
2.07282603693049 -7.93248640183508
0.469154926520952 -8.57831632307493
-0.352065145766123 -7.76518423510579
-2.51826565475281 -6.41108048471763
-3.41320786028111 -6.19892265165218
-3.45608845009742 -6.23555210130331
-4.6731237785027 -5.14066910391221
-5.51188155077916 -4.65264379419724
-6.95241547708384 -3.98929947632171
-6.4747377710725 -3.13057171591923
-7.06523419193588 -2.54285177081361
-7.17238314751134 -1.2983368553556
-5.75740287368192 -1.43839807841182
-6.52859156672245 -0.631297809855683
-5.77238426033422 0.146631993487316
-6.24208154101027 -0.13870827806155
-5.95647546922808 0.479276658256161
-5.37350857777614 0.222056043630961
-6.94200175627849 0.0107501927781108
-5.78998571474087 -0.199450797837118
-8.57417971723354 0.699512220217549
-8.54644056575103 2.42834266513647
-6.94505224664441 3.41029209300286
-5.65824632267322 3.66991674476995
-6.70482016588124 4.94399928710343
-6.98528985343332 6.02148378621324
};
\node at (axis cs:-0.5,-3)[
  anchor=base west,
  text=black,
  rotate=0.0
]{ $R_1$};
\node at (axis cs:-0.5,7.5)[
  anchor=base west,
  text=black,
  rotate=0.0
]{ $R_2$};
\node at (axis cs:-9,7.5)[
  anchor=base west,
  text=black,
  rotate=0.0
]{ $T_1$};
\node at (axis cs:6.5,7.5)[
  anchor=base west,
  text=black,
  rotate=0.0
]{ $T_2$};

\end{axis}

\end{tikzpicture}

	\setlength\figureheight{0.4\columnwidth} 

	% This file was created by matplotlib2tikz v0.6.14.
\begin{tikzpicture}

\definecolor{color1}{rgb}{1,0.498039215686275,0.0549019607843137}
\definecolor{color0}{rgb}{0.12156862745098,0.466666666666667,0.705882352941177}
\definecolor{color2}{rgb}{0.172549019607843,0.627450980392157,0.172549019607843}

\begin{axis}[
xlabel={$t$},
ylabel={$\mathbf{P}(\psi)$},
xmin=-2.95, xmax=61.95,
ymin=0.115383927583069, ymax=0.98995146727863,
width=\figurewidth,
height=\figureheight,
tick align=outside,
tick pos=left,
x grid style={lightgray!92.026143790849673!black},
y grid style={lightgray!92.026143790849673!black},
axis y line=middle,
axis x line=middle,
ymin = 0,
ymax = 1,
every axis x label/.style={
    at={(ticklabel* cs:1.05)}
},
every axis y label/.style={
    at={(0.2,0.7)},
    anchor=south,
}
]
\addplot [semithick, color0, forget plot]
table {%
0 0.519437323294361
1 0.505905967792163
2 0.535762794709806
3 0.529641473005327
4 0.534202921709494
5 0.538759248959358
6 0.543315961968619
7 0.552430637025909
8 0.556136049552841
9 0.566110510314734
10 0.571484104700144
11 0.582252336732076
12 0.572331976818973
13 0.585099080576503
14 0.606729657381389
15 0.616691491600847
16 0.622103745520534
17 0.627514467660256
18 0.63366955987662
19 0.941425688953114
20 0.950198397292469
};
\addplot [semithick, color1, forget plot]
table {%
0 0.535132433990641
1 0.52361711183953
2 0.535132433990641
3 0.543731008001737
4 0.550703063551738
5 0.558585302195194
6 0.539173994232859
7 0.547458159763064
8 0.562091024973306
9 0.572039360069865
10 0.596097657367583
11 0.596563953011597
12 0.601126871317435
13 0.611092694091856
14 0.616488991909715
15 0.621058196593617
16 0.625615820087544
17 0.634735491594122
18 0.645561482408208
19 0.234045441759929
20 0.243149052412218
21 0.247728706925773
22 0.155136997569231
23 0.186216159094014
24 0.206143993143537
25 0.21610603533654
26 0.233451571592582
27 0.244022547027297
28 0.26598938278977
29 0.279350961751404
30 0.289174863802017
31 0.316179113764902
32 0.328982444064366
33 0.328982444064366
34 0.336315254852049
35 0.347152718032949
36 0.359673571635938
37 0.364229419086882
38 0.379600870657791
39 0.38500440205473
40 0.388712554616358
41 0.394120578102893
42 0.388712554616358
43 0.407787653050549
44 0.418603916546378
45 0.427715165844223
46 0.459361669473757
47 0.453045606826443
48 0.453045606826443
49 0.44763200589673
50 0.950198397288434
};
\addplot [semithick, color2, forget plot]
table {%
0 0.547722366299717
1 0.531583463211636
2 0.527071760546219
3 0.547722366299717
4 0.560103204844214
5 0.555328267951691
6 0.565540804425142
7 0.576360127473417
8 0.592582903342865
9 0.602542902132007
10 0.606253832137332
11 0.60710633658241
12 0.591576171090529
13 0.611824096954832
14 0.600274259142499
15 0.601126871317435
16 0.605680578715834
17 0.603975733220287
18 0.615647313283849
19 0.613942290635706
20 0.625615820087544
21 0.636351488414114
22 0.630168418209078
23 0.640134983716998
24 0.253945007517459
25 0.168083558274324
26 0.180794947372842
27 0.188822613176964
28 0.211591710630682
29 0.220721176339468
30 0.236209066973026
31 0.244022547027297
32 0.255461135796253
33 0.267559897089635
34 0.293610800263249
35 0.311769872918488
36 0.323468852817987
37 0.328941635241611
38 0.350849395838469
39 0.351708421550836
40 0.351708421550836
41 0.362524629192895
42 0.367932683007021
43 0.374193140341984
44 0.383304454416872
45 0.388712554616358
46 0.399528017149535
47 0.397823889661972
48 0.408640059633355
49 0.413195587429733
50 0.413195587429733
51 0.417751484126762
52 0.416899076992805
53 0.414048005317165
54 0.412343194766756
55 0.420173604424889
56 0.449079203299998
57 0.448481084716646
58 0.452180689640779
59 0.950198397288434
};
\end{axis}

\end{tikzpicture}
	\caption{Same as Fig. \ref{fig:exp1} with the difference that both $O_1$ and $O_2$ contain obstacles.}
	\label{fig:exp2}
\end{figure}
  \section{Conclusions}


%% Bibiliography %%%%%%%%%%%%%%%%%%%%%%%%
\bibliography{AliAgha,references}

%%%%%%%%%%%%%%%%%%%%%%%%%%%%%%%%%%

%
%\appendix
%
%\section{Kalman filtering}
%Consider a Gaussian LTI system:
% \begin{align}  \begin{aligned}
%x_{k+1}&=A x_{k} + B u_t+ w_k,\\
%z_k&=Cx_k+Du_k+v_k.\end{aligned} \end{align}
%with $w_k\sim \mathcal N(0, \mathcal W)$ and $v_k\sim \mathcal N (0,\mathcal V)$.
%
%At $k=0$, we know $x_0\sim \init$ with $\init:=\mathcal N(x_\init,P_\init)$.
%Thus,  before receiving a measurement $z_0$, the distribution of the belief is defined as $\CA N(x_{0|-}, P_{0|-})$
%\begin{align}
%	\hat x_{0|-}&:= x_\init\\
%	P_{0|-}&:= P_{\init}
%\end{align}
%After receiving the measurement $z_0$, this is updated to $\CA N(\hat x_{0|0}, P_{0|0})$
%\begin{align}
%	\hat x_{0|0}&:= x_\init+ L_0 (z_0-Cx_\init)\\
%	P_{0|0}&:=(I-L_0 C) P_{\init}(I-L_0 C)^T+L_0\CA V L_0^T\\
%	& \mbox{ with } L_0=P_{\init}C^T\left(CP_{\init}C^T+\mathcal V\right)^{-1}
%\end{align}
%We represent the belief state  $\CA N(\hat x_{0|0}, P_{0|0})$ by $b_0:=(\hat x_{0|0}, P_{0|0})\in\mathbb R^n\times \mathbb S^n$.
%
%The dynamics of the Kalman filter are given as
%	\begin{align*}
%	&&\textbf{Predict} \qquad \hat x_{k|k-1}&=A\hat x_{k-1|k-1}+Bu_{k-1}\\
%	&&P_{k|k-1}&=AP_{k-1|k-1}A^T+\mathcal W
%\\
%	&&\textbf{Update} \  \qquad e_{k}&=z_k-C \hat x_{k|k-1}\\
%	&&S_k&=CP_{k|k-1}C^T+\mathcal V\\
%	&&L_{k}&=P_{k|k-1}C^TS_k^{-1}\\
%	&&\hat x_{k|k}&=\hat x_{k|k-1}+L_ke_k\\
%	&&P_{k|k}&=(I-L_kC)P_{k|k-1}\\
%	\end{align*}
%	\mbox{Joseph Formula  }
%	\begin{align*}
%	&&P_{k|k}&=(I-L_kC)P_{k|k-1}(I-L_kC_k)^T+L_k\mathcal V_kL_k^T\\
%		\end{align*}
%\mbox{Observability based }
%	\begin{align*}
%	&& P_{k|k}^{-1}&=P_{k|k-1}^{-1}+C_k^T \mathcal  V_k^{-1}C_k
%	\end{align*}
%
%Though the covariance of the belief state is defined as 
%	\begin{align*}
%	&&P_{k|k}&=(I-L_kC)P_{k|k-1}(I-L_kC_k)^T+L_k\mathcal V_kL_k^T, \\
%		\end{align*}
%		The update equations for $P_{k|k-1}$ are more well know:
%			\begin{align*}
%	&&P_{k+1|k}&=(A-K_kC)P_{k|k-1}(A-K_kC_k)^T+K_k\mathcal V_kK_k^T +\CA W
%		\end{align*}
%		with $K_k=AL_k$.
%		
%Hence, the belief state is updated as
%\begin{align}
%	&&\hat x_{k|k}&=A\hat x_{k-1|k-1}+Bu_{k-1}+L_ke_k\\
%	&&P_{k|k}&=f(P_{k-1|k-1})
%\end{align}
%We now want to model the random variable $s_k=L_ke_k$. We know that $s_k$ evolves as a zero mean Gaussian distributed stochastic process.
%Further 
%\begin{align*}
%	\Ex [s_k]=0\\
%	\Ex [s_ks_k^T]=L_k	\Ex [e_ke_k^T]L_k^T, \mbox{ and } \Ex [e_ke_k^T]=S_k \\
%	e_k = C\left(x_k- \hat x_{k|k-1}\right)+v_k\\
%	\Ex [e_ke_k^T] = C P_{k|k-1} C^T + \CA V\\
%		\Ex [s_ks_k^T]=L_k S_k L_k^T,\\
%				\Ex [s_ks_k^T]= P_{k|k-1} C^T S_k^{-1} C P_{k|k-1},\\
%								\Ex [s_ks_k^T]= P_{k|k-1} C^T \left(CP_{k|k-1}C^T+\mathcal V\right)^{-1} C P_{k|k-1},\notag\\
%\Ex [s_ks_k^T]= P_{k|k-1}-P_{k|k}\notag
%\end{align*}
%
%
%\begin{align}
%	&\text{concrete:}\left\{\begin{aligned}
%		&&\hat x_{k|k}&=A\hat x_{k-1|k-1}+Bu_{k-1}+ P_{k|k-1} C^T  \bar{s}_k\notag \\
%		&&&\mbox{with } \bar{s}_k\sim \CA N (0,S_k^{-1}) \notag 
%	\end{aligned}\right.	\\
%&		\text{abstract 1:}\left\{\begin{aligned}
%		&&\hat x_k &=A\hat x_{k-1} +B\hat u_{k-1} + \bar P  C^T  \hat{s}_k\notag \\
%		&&&\mbox{with } \hat{s}_k\sim \CA N (0,\hat{S}_{inv}) \notag 
%	\end{aligned}\right.	
%\end{align}
%Define  $\Delta_k:=P_{k|k-1} -\bar P $,  then  $S_k=\left(C\bar PC^T+C\Delta_kC^T+\mathcal V\right)$.
%Find maximal $\hat{S}_{inv}$ such that $\hat{S}_{inv}\preceq S_k^{-1}$
%


\end{document}


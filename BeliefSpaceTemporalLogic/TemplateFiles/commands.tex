%% Additional packages
\usepackage{amsmath,amssymb,amsfonts}
\usepackage{mathtools}
\usepackage{subfigure}
\usepackage{graphicx}
\usepackage{color}
\usepackage{url}
%\usepackage[usenames,x11names]{xcolor}

\ifdefined\algorithm
% Don't load theorem style
\else
\usepackage[linesnumbered,vlined,ruled]{algorithm2e}
\fi

\usepackage{tikz,pgf} 
\usetikzlibrary{arrows,automata,shapes,calc,backgrounds,spy,positioning}
\usetikzlibrary{fit}

\usepackage{epstopdf}


 

%%% figure path
\graphicspath{{figures/}}
 
 
%% Roman, calligraphic, boldface, double barred letters
\newcommand{\RM}[1]{\mathrm{#1}}
\newcommand{\CA}[1]{\mathcal{#1}}
\newcommand{\BF}[1]{\mathbf{#1}}
\newcommand{\IT}[1]{\mathit{#1}}
\newcommand{\BB}[1]{\mathbb{#1}}
\newcommand{\TT}[1]{\mathtt{#1}}
\newcommand{\FK}[1]{\mathfrak{#1}}
\newcommand{\BS}[1]{\boldsymbol{#1}}


%% spaces 
\newcommand{\Real}{\BB{R}}
\newcommand{\Symb}{\mathcal S}
\newcommand{\borel}[1]{\mathcal{B}\left(#1\right)}


%%% Probability
\newcommand{\Ex}{\mathbf{E}}     % Probability of an event

\newcommand{\po}{\mathbf{P}}     % Probability of an event
\newcommand{\p}[1]{\po\left(#1\right)}     % Probability of an event
\renewcommand{\P}{\BF{P}}
\newcommand{\pd}[1]{p\left(#1\right)}     % Probability density  



%% Modelling symbols
%-------------MDP----------------------------
\newcommand{\MDP}{\mathsf{M}}
\newcommand{\POMDP}{\MDP_{\Z}}
\newcommand{\C}{\mathbf{C}}
\newcommand{\MB}{\mathsf{B}}

 
\newcommand{\X}{{\mathbb{X}}}  % State
\newcommand{\Z}{{\mathbb{Z}}}	% Observation space
\newcommand{\A}{{\mathbb{U}}} % Action space
\newcommand{\init}{\rho}
\newcommand{\tr}{t}


% product MDP
\renewcommand{\S}{{\mathbb{S}}}	% Observation space

\newcommand{\polb}{{\boldsymbol{\mu}}}
\renewcommand{\pol}{{\mu}}

\newcommand{\Y}{{\mathbb{Y}}}  % State

%-------------POMDP-----------------------------
\newcommand{\Hist}{{\mathsf{H}}}  %  History
\newcommand{\I}{{\mathsf{I}}}  %  History
\newcommand{\Belief}{b}
\newcommand{\trb}{\tr_\Belief}
\newcommand{\Xb}{{\X_\Belief}}  % State
\newcommand{\initb}{\init_\Belief}


% -----------------Refinement relation----------
\newcommand{\InF}{\mathcal{U}_{v}}
\newcommand{\Wt}{\mathbb{W}_{\tr}}

\newcommand{\grid}{\boldsymbol{\delta}}

%%% Logic 
%------------------Predicates-----------------------------------
\newcommand{\Fpred}{{\mathcal F}}
\newcommand{\Lab}{\mathsf{L}}
\newcommand{\Labset}{\mathcal{L}}

\newcommand{\alphabeth}{\Sigma}
\newcommand{\word}{{\boldsymbol{\pi}}} % words formed from the alphabeth
\newcommand{\letter}{\pi} % words formed from the alphabeth

\newcommand{\Bel}{{\mathbf {T}}}
\newcommand{\BelR}{{\mathbf {R}}}
\newcommand{\trunc}[2]{\operatorname{trunc}_{#1}\left(#2\right)}


%%% Temporal logic symbols
\newcommand{\notltl}{\neg}
\newcommand{\andltl}{\wedge}
\newcommand{\orltl}{\vee}
\newcommand{\Next}{\ensuremath{\bigcirc}}
\newcommand{\Always}{\ensuremath{\ \square\ }}
\newcommand{\Event}{\ensuremath{\ \diamondsuit\ }}
\newcommand{\Until}{\ \CA{U}\ }
\newcommand{\Implies}{\Rightarrow}
\newcommand{\Equiv}{\Leftrightarrow}
\newcommand{\True}{\top}
\newcommand{\False}{\perp}
\newcommand{\AP}{{AP}}
\newcommand{\pred}{\xi}


\newcommand{\eps}{\epsilon} \newcommand{\rel}{\mathcal{R}} % numbers option provides compact numerical references in the text. 

%----- Exotic words----- 
\newcommand{\buchi}{B\"uchi\ }

%% Symbols of automata
\newcommand{\PA}{\mathcal{P}} 
\newcommand{\BA}{\mathcal{B}}
\newcommand{\TS}{\mathcal{F}}
\newcommand{\Language}{\mathbf{Lang}} % Language?
\newcommand{\KA}{\mathcal{K}}
\newcommand{\RA}{\mathcal{R}}
\newcommand{\FSA}{\mathcal{A}}

\newcommand{\TSX}{\BB{V}_\TS}
\newcommand{\TSE}{\BB{E}_\TS}
\newcommand{\TSEE}{\BB{E}}

\newcommand{\DTL}{DTL~}

 % Custom operators
\newcommand{\norm}[1]{\left\| {#1} \right\|}
\newcommand{\norminf}[1]{\left\| {#1} \right\|_{\infty}}
\newcommand{\normeucl}[1]{\left\| {#1} \right\|_{2}}
\newcommand{\abs}[1]{\left| {#1} \right|} 
\DeclareMathOperator{\diag}{diag}



\ifdefined\theoremstyle
% Don't load theorem style:
%% There are a number of predefined theorem-like environments in
%% ifacconf.cls:
%%
%% \begin{thm} ... \end{thm}            % Theorem
%% \begin{lem} ... \end{lem}            % Lemma
%% \begin{claim} ... \end{claim}        % Claim
%% \begin{conj} ... \end{conj}          % Conjecture
%% \begin{cor} ... \end{cor}            % Corollary
%% \begin{fact} ... \end{fact}          % Fact
%% \begin{hypo} ... \end{hypo}          % Hypothesis
%% \begin{prop} ... \end{prop}          % Proposition
%% \begin{crit} ... \end{crit}          % Criterion
\newtheorem{theorem}[thm]{Theorem}
\newtheorem{lemma}[thm]{Lemma}
\newtheorem{definition}[thm]{Definition}


\newtheorem{example}{Example}

\else
\usepackage{amsthm}
\theoremstyle{plain}
\newtheorem{theorem}{Theorem}
\newtheorem{cor}{Corollary}
\newtheorem{prop}{Proposition}
\newtheorem{lemma}{Lemma}
\newtheorem{remark}{Remark}
\newtheorem{cond}{Condition}

\newtheorem{example}{Example}
\newtheorem{problem}{Problem}
\newtheorem{theorem}{Theorem}
\newtheorem{definition}[theorem]{Definition}
\fi


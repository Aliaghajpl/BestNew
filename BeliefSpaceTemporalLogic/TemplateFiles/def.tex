%% Additional packages
\usepackage{times}
\usepackage{amsmath}


\usepackage{amssymb}
\usepackage{amsfonts}
\usepackage{mathtools}
%\usepackage{calc}
\usepackage{subfigure}
\usepackage{graphicx}
\usepackage{color}
\usepackage{url}
%\usepackage{lineno}
%\usepackage{ulem} % for underlining and strike through
%\normalem % reset emph to normal
%\usepackage{setspace} % for line spacing - e.g. 1.5, 2
\usepackage[usenames,x11names]{xcolor}
%\usepackage{xspace} % for inserting a space in TeX commands if needed
%\usepackage{caption}
%\usepackage[bookmarks=true]{hyperref}
%\renewcommand{\thesubfigure}{\relax} % no subfig counters
%\let\chapter\section % algorithm2e natbib compatibility
%\usepackage{accents}
%\usepackage[titletoc,toc,title]{appendix}
%\usepackage{longtable}
%\usepackage{setspace}
\usepackage{multicol}

%% Packages
%\usepackage[margin=1in]{geometry}
%\usepackage[abs]{overpic}

\ifdefined\algorithm
% Dont load theorem style
\else
\usepackage[linesnumbered,vlined,ruled]{algorithm2e}
\fi

%\usepackage{multirow} % for cell tables spanning multiple rows
\usepackage{tikz,pgf}%,tikz-3dplot}
\usetikzlibrary{arrows,automata,shapes,calc,backgrounds,spy,positioning}
%\usetikzlibrary{external}
%\tikzexternalize
%\usepackage[numbers]{natbib}
%\usepackage[sort,round]{natbib}
%\usepackage{lipsum}
\usepackage{epstopdf}


%\newtheorem{assumption}{Assumption}


%%% figure path
\graphicspath{{figures/}}

%% Remove footnote mark
%\renewcommand{\footnotemark}{}

%%% Redefine qed symbol
%\renewcommand{\qedsymbol}{$\blacksquare$}

%% Projection symbol
%\newcommand{\project}[1]{\! \upharpoonright_{#1}}


%% Theorems  => Patched to use with different cls style files
%\newtheorem{theorem}{Theorem}[section]
%\newtheorem{proposition}[theorem]{Proposition}
%\newtheorem{corollary}[theorem]{Corollary}

%\newtheorem{lemma}[theorem]{Lemma}
%\newtheorem{remark}[theorem]{Remark}
%\newtheorem{remarks}[theorem]{Remarks}
%\newtheorem{example}[theorem]{Example}
%\newtheorem{algo}[theorem]{Algorithm}
%\newtheorem{problem}[theorem]{Problem}
%\newtheorem{Procedure}[theorem]{Procedure}
%\newcommand{\exampler}[2]{\medskip \hskip -\parindent {\bf Example #1 Revisited.~}{\it #2}\medskip}

%% Percent
%\newcommand\oprocendsymbol{\hbox{$\square$}}
%\newcommand\oprocend{\relax\ifmmode\else\unskip\hfill\fi\oprocendsymbol}
%\def\eqoprocend{\tag*{$\bullet$}}

%% Enumerate environment
%\renewcommand{\labelenumi}{(\roman{enumi})}
%\renewcommand{\labelenumii}{(\alph{enumii})}

%% Breakable comma
%\mathchardef\breakingcomma\mathcode`\,
%{\catcode`,=\active
%  \gdef,{\breakingcomma\discretionary{}{}{}}
%}
%\newcommand{\breqn}[1]{\mathcode`\,=\string"8000 #1}

%% Other Stuff
%\newcommand{\margin}[1]{\marginpar{\tiny\color{blue} #1}}
%%\addtolength{\marginparwidth}{-0.3in}
\newcommand{\todo}[1]{\vskip 0.05in \colorbox{yellow}{$\Box$ \ttfamily\bfseries\small#1}\vskip 0.05in}
%\newcommand{\todo}[1]{}
%\newcommand{\vers}{\operatorname{vers}}

%% Roman, calligraphic, boldface, double barred letters
\newcommand{\RM}[1]{\mathrm{#1}}
\newcommand{\CA}[1]{\mathcal{#1}}
\newcommand{\BF}[1]{\mathbf{#1}}
\newcommand{\IT}[1]{\mathit{#1}}
\newcommand{\BB}[1]{\mathbb{#1}}
\newcommand{\TT}[1]{\mathtt{#1}}
\newcommand{\FK}[1]{\mathfrak{#1}}
\newcommand{\BS}[1]{\boldsymbol{#1}}

%%% Temporal logic symbols
\newcommand{\notltl}{\neg}
\newcommand{\andltl}{\wedge}
\newcommand{\orltl}{\vee}
\newcommand{\Next}{\ensuremath{\bigcirc}}
\newcommand{\Always}{\ensuremath{\ \square\ }}
\newcommand{\Event}{\ensuremath{\ \diamondsuit\ }}
\newcommand{\Until}{\ \CA{U}\ }
\newcommand{\Implies}{\Rightarrow}
\newcommand{\Equiv}{\Leftrightarrow}
%\newcommand{\Not}{\lnot}
\newcommand{\True}{\top}
\newcommand{\False}{\perp}
%\def\prop{\TT{data}}
%\def\popt{\pi}
\newcommand{\AP}{AP}
\newcommand{\pred}{\xi}

\newcommand{\Real}{\BB{R}}

%% Abbreviations
%\def\eg{e.g.\xspace}
%\def\Eg{E.g.\xspace}
%\def\ie{i.e.\xspace}
%\def\Ie{I.e.\xspace}
%\def\etc{etc.\xspace}
%\def\vs{vs.\xspace}
%\def\wrt{w.r.t.\xspace}
%\def\etal{et al.\xspace}

%% Exotic words
\newcommand{\buchi}{B\"uchi\ }

%% Symbols of automata
\newcommand{\PA}{\mathcal{P}}
\newcommand{\BA}{\mathcal{B}}
%\newcommand{\FA}{\mathcal{F}}

\newcommand{\TS}{\mathcal{F}}
\newcommand{\LA}{\mathcal{L}}
\newcommand{\KA}{\mathcal{K}}
\newcommand{\MDP}{\mathcal{M}}
\newcommand{\RA}{\mathcal{R}}
\newcommand{\FSA}{\mathcal{A}}

\newcommand{\TSX}{\BB{V}_\TS}
\newcommand{\TSE}{\BB{E}_\TS}
\newcommand{\TSEE}{\BB{E}}

\newcommand{\DTL}{DTL~}


%% Short macros for arrows
\newcommand{\la}{\leftarrow}
\newcommand{\ra}{\rightarrow}
\newcommand{\ras}[1]{\stackrel{#1}{\rightarrow}}
\newcommand{\asgn}{\la}
\newcommand{\proj}[2]{{#1}{\downharpoonright_{#2}}}

\newcommand{\df}{\xspace\RM{d}}

%% Names of the Algorithms
%\newcommand{\optrun}{\textsc{Optimal-Run}\ }
%\newcommand{\exactmultioptrun}{\textsc{Exact-Multi-Robot-Optimal-Run}\ }
%\newcommand{\constR}{\textsc{Construct-Region-Automaton}\ }
%\newcommand{\constT}{\textsc{Construct-Team-TS}\ }
%\newcommand{\syncT}{\textsc{Sync-Team-TS}\ }
%\newcommand{\boundOpt}{\textsc{Bound-Optimality}\ }

% Custom operators
%\DeclareMathOperator*{\argmin}{arg\,min}
\newcommand{\norm}[1]{\left\| {#1} \right\|}
\newcommand{\norminf}[1]{\left\| {#1} \right\|_{\infty}}
\newcommand{\normeucl}[1]{\left\| {#1} \right\|_{2}}
\newcommand{\abs}[1]{\left| {#1} \right|}
\newcommand{\card}[1]{\left| {#1} \right|}
\newcommand{\spow}[1]{2^{#1}}
%\newcommand{\interior}[1]{\accentset{\smash{\raisebox{-0.12ex}{$\scriptstyle\circ$}}}{#1}\rule{0pt}{2.3ex}}
\newcommand{\interior}[1]{\mathring{#1}}
\DeclareMathOperator{\diag}{diag}
\newcommand{\lift}{\upharpoonright}
%
%% Display a grid to help align images
%\beamertemplategridbackground[1cm]
%\usepackage[style=numeric-comp]{biblatex}
%\usepackage{cite}
\ifdefined\theoremstyle
% Dont load theorem style:
%% There are a number of predefined theorem-like environments in
%% ifacconf.cls:
%%
%% \begin{thm} ... \end{thm}            % Theorem
%% \begin{lem} ... \end{lem}            % Lemma
%% \begin{claim} ... \end{claim}        % Claim
%% \begin{conj} ... \end{conj}          % Conjecture
%% \begin{cor} ... \end{cor}            % Corollary
%% \begin{fact} ... \end{fact}          % Fact
%% \begin{hypo} ... \end{hypo}          % Hypothesis
%% \begin{prop} ... \end{prop}          % Proposition
%% \begin{crit} ... \end{crit}          % Criterion
\newtheorem{theorem}[thm]{Theorem}
\newtheorem{lemma}[thm]{Lemma}
\newtheorem{definition}[thm]{Definition}

\else
\usepackage{amsthm}
\theoremstyle{plain}
\newtheorem{theorem}{Theorem}
\newtheorem{cor}{Corollary}
\newtheorem{prop}{Proposition}
\newtheorem{lemma}{Lemma}

\theoremstyle{definition}
%\newtheorem{definition}{Definition}
\newtheorem{remark}{Remark}
\newtheorem{cond}{Condition}

\newtheorem{example}{Example}
\newtheorem{problem}{Problem}
\newtheorem{theorem}{Theorem}
\newtheorem{definition}[theorem]{Definition}
\fi


\documentclass[conference]{IEEEtran}

\usepackage{amsmath, amsthm, amssymb, dsfont, accents}

\usepackage{hyperref, booktabs, color}
\usepackage{biblatex}
\addbibresource{5_refs.bib}


\pdfinfo{
   /Author (Homer Simpson)
   /Title  (Robots: Our new overlords)
   /CreationDate (D:20101201120000)
   /Subject (Robots)
   /Keywords (Robots;Overlords)
}

\newtheorem{problem}{Problem}
\newtheorem{remark}{Remark}
\newtheorem{prop}{Proposition}

\begin{document}

\title{\huge Provably Safe Flight }

\author{Petter, Andrew, Thomas, Rohan, Ames, etc}

\maketitle

\begin{abstract}
In this work we develop a method for provably safe quadrotor flight. Using analytical control barrier functions as a means of enforcing invariance, we guarantee trajectory tracking within a quantified error bound. We account for both input bounds and disturbance magnitude. The effectiveness of the method is demonstrated via simulations and real-world tests in a wind tunnel. 
\end{abstract}


\IEEEpeerreviewmaketitle

	
%%%%%%%%%%%%%%%%%%%%%%%%%%%%%%%%%%%%%%%%%%%%%%%%
%%%%%%%%%%%%%%%%%%%%%%%%%%%%%%%%%%%%%%%%%%%%%%%%

Initial targets:

\begin{itemize}
  \item Provably safe means
  \begin{itemize}
    \item Reach target within given time
    \item Do not exceed input bounds
    \item Be robust to given disturbance (wind, modeling error)
  \end{itemize}
  \item Provably safe A-B flight
  \begin{itemize}
    \item Trajectory $\rightarrow$ barrier function $\rightarrow$ controller $\rightarrow$ safety
    \item Trajectory+controller $\rightarrow$ (outer approximation of) smallest invariant set $\rightarrow$ safety
    \item Re-parametrization of time to limit inputs?
  \end{itemize}
  \item Provably safe landing
\end{itemize}

Potential later steps:
\begin{itemize}
  \item Extend to probabilistic framework with an uncertain state estimate
  \item Recovery from failure: stabilize a falling quadrotor
\end{itemize}

Open questions:
\begin{itemize}
  \item Can barrier functions with input bound be designed with backstepping?
\end{itemize}

\section{Introduction} \label{subsec:intro}

The scientific tasks of the 2020 NASA Mars mission mission will be carried out by a ground rover operating in a largely unknown environment. The rover moves slowly and has limited capabilities of surveying its environment, therefore, there are plans to also send a helicopter that can assist the rover by more efficiently mapping the surroundings. With a more accurate environment map the rover can optimize routing to execute its tasks quicker.

Helicopter flight is safety-critical: while the rover can always stop to be in a safe state, the helicopter needs to perform a maneuver to land in a safe location. Making matters worse, the helicopter will have just 2-3 minutes of battery life each day. Combined with a communication delay of 8 minutes between Earth and Mars, the helicopter needs to operate autonomously in a highly safety-critical manner.

With these requirements in mind, in this paper we develop a provably safe controller that does X Y and Z. As a proxy for the Mars helicopter we use a quadrotor to validate our work in simulation and experiment.   


\subsection{Related Work}

Previous work on provably safe quadrotor flight has focused on obstacle avoidance and can be characterized into two categories: analytical approaches and computational.

Starting with computational methods, an early effort towards provably safe flight was reported in \cite{Gillula2010} where the authors used Hamilton-Jacobi-Bellman reachability tools to verify correctness of controllers designed to perform a flip maneuver. Due to the relatively high dimensionality of the quadrotor system, several assumptions were introduced that reduce the analysis to two two-dimensional systems. Avoidance of static obstacles was considered for the full 12-dimensional model as an application of the \emph{funnel library} approach developed by Majumdar et. al. \cite{Majumdar2017}, who suggest constructing a library of motion primitives with associated ``funnels'' that bound the maximal deviation. Funnels can be patched together in a provably correct manner by verifying an inclusion property.

Turning to analytical methods, control barrier functions have been used to guarantee collision-free behavior in teams of quadrotors by introducing a safety barrier around each agent \cite{Wang2017a}. In general, composition of barrier functions does not preserve feasibility, but feasibility was proven in \cite{Wang2017} under the assumption that angular velocity (as opposed to angular acceleration) is directly controllable. Avoidance of environmental obstacles via analytical control barrier functions was considered in \cite{Wu2016}.

\subsection{Notation}

\begin{itemize}
  \item $\mathcal B_p(x,\epsilon)$ is the $\epsilon$-ball in $p$-norm centered at $x$.
\end{itemize}

\section{Problem Statement}

A standard 12-dimensional model for a quadrotor is obtained from force-balance equations in a rotating reference frame (e.g. \cite{Mellinger2011,Zhou2014}). Let the 12-dimensional state be $\mathbf{x} = (\mathbf{r}, \mathbf{v}, \xi, \mathbf{\omega})$, where $\mathbf{r}$ and $\mathbf{v}$ are position and velocity in $\mathbb{R}^3$, $\xi = (\phi, \theta, \psi)$ are roll, pitch and yaw angles, and $\mathbf{\omega} \in \mathbb{R}^3$ are angular velocities in the quadrotor body frame. Then
\begin{equation}
\label{eq:quadrotor_dyn}
\begin{aligned}
  & m \frac{\mathrm{d}^2}{\mathrm{d}t^2} \mathbf{r} = \begin{bmatrix}
    0 \\ 0 \\ -m g
  \end{bmatrix} + u_1 R(\xi) \begin{bmatrix}
    0 \\ 0 \\ 1
  \end{bmatrix} + \mathbf{w}, \\
  & \frac{\mathrm{d}}{\mathrm{d}t} \xi
   = T(\xi) \omega, \\
  & J \frac{\mathrm{d}}{\mathrm{d}t} \mathbf{\omega} = \mathbf{u}_2 - (\mathbf{\omega} \times (J \mathbf{\omega})).
\end{aligned}
\end{equation}
Here $R(\xi)$ is the x-y-z rotation matrix from a body-fixed frame to the world frame, and $T$ the resulting mapping between angular velocities:
\begin{equation}
\label{eq:rotmat}
\begin{aligned}
  R(\xi) & = R_z(\psi) R_y(\theta) R_x(\phi), \\
   T(\xi) & = \begin{bmatrix}
     1 & \sin(\phi)\tan(\theta) & \cos(\phi)\tan(\theta) \\
     0 & \cos(\phi)   &  -\sin(\phi) \\
     0 & \sec(\theta)\sin(\phi)  & \cos(\phi)\sec(\theta)
  \end{bmatrix}.
\end{aligned}
\end{equation}

In \eqref{eq:quadrotor_dyn}, $\mathbf{w}$ is a disturbance force acting on the quadrotor as a result of wind; it is assumed to be bounded as $\mathbf{w} \in \mathcal W$. The controlled inputs are total upwards thrust force $u_1$ and angular torques $\mathbf{u}_2 = (p,q,r)$. Equivalently, the input can be chosen as the angular velocities of the propellers. Let propeller $i$ be located at $r_x^i, r_y^i, r_z^i$ in relation to the copter center of gravity, and assume that it generates a torque $- k_t \omega_i^P |\omega_i^P|$ and an upward force $d_i k_f \omega_i^P |\omega_i^P|$ where $d_i = (-1)^i$ is the direction of rotation. Then the following linear map from squared angular propeller velocities to total force and moment is invertible:
\begin{equation}
\label{eq:omega_to_u}
\begin{bmatrix}
  u_1 \\
  u_2 \\
  u_3 \\
  u_4
\end{bmatrix} =
\begin{bmatrix}
  -k_f  & k_f& -k_f&  k_f \\
    -k_f r_1^y &   k_f r_2^y &  -k_f r_3^y &   k_f r_4^y \\
     k_f r_1^x &  -k_f r_2^x &   k_f r_3^x  & -k_f r_4^x \\
     -k_t  &     -k_t    &   -k_t       & -k_t
\end{bmatrix}
\begin{bmatrix}
  \omega^P_1 |\omega^P_1| \\
  \omega^P_2 |\omega^P_2| \\
  \omega^P_3 |\omega^P_3| \\
  \omega^P_4 |\omega^P_4|
\end{bmatrix}
\end{equation}

This model is differentially flat in the output $(\mathbf{r}, \psi$), where $\psi$ is the yaw angle, meaning that there is a map from any four-times differentiable output trajectory $(\mathbf{r}(t), \psi(t))$ to the remaining system states and model inputs.

We can further incorporate motors dynamics by modeling motors that generate the angular propeller velocities $\omega_i^P$ from an input voltage $V_i$. We use a first-order model for the voltage-torque relationship in the motors, and the reaction torques $-k_t \omega_i^P |\omega_i^P|$ from above to obtain:
\begin{equation}
\label{eq:propdyn}
  I_{prop} \dot \omega_i^P = \frac{k_m}{ R} \left( d_i V_i - k_m \omega_i^P \right) - k_t |\omega_i^P| \omega_i^P.
\end{equation}
Here, $R$ is the resistance and $k_m$ is a motor constant. Combined, the dynamics consist of equations \eqref{eq:quadrotor_dyn}, \eqref{eq:rotmat}, \eqref{eq:omega_to_u}, and \eqref{eq:propdyn} with inputs $V_i$ and states $\mathbf{x}, \omega_1^P, \omega_2^P, \omega_3^P, \omega_4^P$.

\textcolor{red}{For now, this disregards the addition of mass and inertia from the propellers to the overall system.}

\begin{problem}
  \label{prob:main}
  Given an initial state $x_0$, a target state $x_1$, bounds $\epsilon_0, \epsilon_1$ and a time horizon $T$, construct a controller such that any trajectory starting in $\mathcal B(x_0, \epsilon_0)$ will reach $\mathcal B(x_1, \epsilon_1)$ after time at most $T$, or determine that no such controller exists.
\end{problem}

An algorithm that solves this problem can serve as feedback to a high-level planning framework. If the planner requests a feasible trajectory the quadrotor executes it in a safe manner, whereas if a safe trajectory can not be found this is reported to the planner, potentially along with information quantifying the infeasibility.

\begin{remark}
  Problem \ref{prob:main} is at least NP-hard and potentially undecidable \cite{Bell2010} so we can not expect to solve it exactly: for a solution there will typically be a gap between those problem instances for which a safe trajectory can be found, and those instances for which it can be proved that no safe trajectory exists. 
\end{remark}

\section{Approach: SOS-Verified Barrier Function}

Here an initial approach based on polynomial barrier functions is described. For simplicity, we consider the problem of stabilizing the quadrotor in hovering mode under external disturbance, with the intention to extend to trajectory tracking.

The dynamics \eqref{eq:quadrotor_dyn} can be written with the Euler angles replaced by a quaternion state $q$:
\begin{equation}
\label{eq:quadrotor_dyn_quat}
\begin{aligned}
  & \frac{\mathrm{d}^2}{\mathrm{d}t^2} \mathbf{r} =  -g \mathbf{e}_3 + \frac{1}{m} (q * \mathbf{e}_3 * q^{-1}) u_1, \\
  & \frac{\mathrm{d}}{\mathrm{d}t} q
   = \frac{1}{2} (q * \omega), \\
  & J \frac{\mathrm{d}}{\mathrm{d}t} \mathbf{\omega} = \mathbf{u}_2 - (\mathbf{\omega} \times (J \mathbf{\omega})).
\end{aligned}
\end{equation}
Here, $*$ denotes the quaternion product; the product between a quaternion $q$ and a vector $v$ is the quaternion product of $q$ and $q_v = v_x \mathbf{i} + v_y \mathbf{j} + v_z \mathbf{k}$.

The most cited advantage of the quaternion representation over Euler angles is the absence of singularities (or gimbal lock), however, another important difference is that \eqref{eq:quadrotor_dyn_quat} is a polynomial system which makes sums-of-squares approach more applicable\footnote{Sums-of-squares can be applied also to trigonometric terms by introducing additional variables.}. A third alternative formulation is to use the full rotation matrix $R$ as a state in the dynamics, which also results in polynomial dynamics without trigonometric terms. However, a rotation matrix requires nine states as opposed to the four quaternion states. As such, a quaternion representation is the most parsimonious representation that renders the dynamics polynomial.

\subsection{Invariance of a Polynomial Level Function}

In the following, we develop sufficient conditions for a polynomial $h(x)$ to define a controlled-invariant set $\mathcal C = \{ x : h(x) \leq \epsilon \}$. In particular, we want to verify whether the following is true:
\begin{equation}
\label{eq:invariancecond}
   \forall x \in \partial \mathcal C, \;  \forall d \in \mathcal D, \; \exists u \in \mathcal U, \; \frac{\mathrm{d}}{\mathrm{d}t} h(x,u,d) \leq 0.
 \end{equation} 
\begin{remark}
  We could equivalently replace the invariance condition with the standard control barrier function condition
  \begin{equation}
  \label{eq:cbf_condition}
    \forall x \in \mathcal C, \;  \forall d \in \mathcal D, \; \exists u \in \mathcal U, \; \frac{\mathrm{d}}{\mathrm{d}t} h(x,u,d) + h(x) \leq 0.
  \end{equation}
  without changing the fundamentals of the derivations below. Obviously, \eqref{eq:cbf_condition} implies \eqref{eq:invariancecond}.
\end{remark}
 For reasons that will be made clear below, we assume that the set of admissible inputs $\mathcal U$ is rectangular:
 \begin{equation}
   \mathcal U = \prod_{i=1}^4 [-\bar u_i, \bar u_i].
 \end{equation}
 We remark that any rectangular input set can be converted to this symmetric form by a change of input variables. In addition, we consider semi-algebraic disturbance sets:
\begin{equation}
  \mathcal D = \left\{ d : h_d \leq 0 \right\},
\end{equation}
where $h_d$ is potentially vector-valued in which case the inequality is interpreted component-wise.

The invariance expression \eqref{eq:invariancecond} can be restated as follows:
\begin{equation}
  \max_{x : h(x) = \epsilon} \max_{d \in \mathcal D} \min_{u \in \mathcal U} \mathcal L_f h(x) + \mathcal L_{g_u} h(x) u + \mathcal L_{g_d}(x) d \leq 0.
\end{equation}
\begin{remark}
  Since the expression is affine in both $d$ and $u$, the quantification over these two variables can be interchanged for a fixed $x$. In other words, there is no ``first mover advantage'' in the game between $u$ and $d$.
\end{remark}
The inner minimization problem is linear in $u$ and can thus be solved in a bang-bang fashion:
\begin{equation*}
\begin{aligned}
    & \min_{u \in \mathcal U} \mathcal L_{g_u} h(x) u = \sum_{i=1}^4 \min \left([\mathcal L_{g_u} h(x)]_i \bar u_i, [\mathcal L_{g_u} h(x)]_i (- \bar u_i) \right) \\
    & = - \sum_{i=1}^4 \left| [\mathcal L_{g_u} h(x)]_i \right|  \bar u_i.
\end{aligned}
\end{equation*} 
Hence, we are left with a condition for invariance quantified only over $x$ and $d$.
\begin{equation}
  \max_{x : h(x) = \epsilon} \max_{d \in \mathcal D} \mathcal L_f h (x) + \mathcal L_{g_d} h (x) d \leq \sum_{i=1}^4 \left| \left[ \mathcal L_{g_u} h(x) \right]_i \right|  \bar u_i.
\end{equation}
For signs $s \in \{ -1,1 \}^4$ the problem can be decomposed into eight problems
\begin{equation}
\begin{aligned}
  & \max_{x : \substack{h(x) = \epsilon \\ s_i \left[ \mathcal  L_{g_u} h(x) \right]_i \geq 0 }} \max_{d \in \mathcal D} \mathcal L_f h (x) + \mathcal L_{g_d} h (x) d \\
  & \quad \leq \sum_{i=1}^4 s_i \left[ \mathcal L_{g_u} h(x) \right]_i \bar u_i
\end{aligned}
\end{equation}
that verify the condition in different parts of the state space depending on the sign of $\mathcal L_{g_u} h(x)$.

Consider the optimization problem constructed using the S procedure:
\begin{equation}
\label{eq:tmin}
\begin{aligned}
  \min_{t, \sigma_1, \sigma_2} & \quad t \\
  \text{s.t} &  \quad  t - \left[ \mathcal L_f h(x) + \mathcal L_{g_d} h(x) d -\sum_{i=1}^4 s_i \left[\mathcal L_{g_u} h(x) \right]_i \bar u_i \right] \\
             & \quad - \sigma_1(x,d) \left( h(x) - \epsilon \right) - \sigma_{2,i}(x,d) \left( s_i \left[ \mathcal L_{g_u} h(x) \right]_i \right) \\
             & \quad + \sigma_3(x,d) \left( h_d(d) \right) \geq 0, \\
             & \quad  \sigma_{2,i}(x) \geq 0, \sigma_3(x) \geq 0.
\end{aligned}
\end{equation}

\textcolor{red}{Can pack all constraints into one problem and solve bilinearly by alternating multipliers and V..}

\textcolor{red}{Can optimize volume of invariant set.}

\begin{prop}
  If the minimizing $t$ is less than or equal to zero for all eight combinations of $s_i$, then $\mathcal C$ is robustly controlled invariant for disturbances $d \in \mathcal D$.
\end{prop}

Upper bounds to the problem \eqref{eq:tmin} can be found by converting the inequality constraints to SOS constraints and solving the resulting semi-definite program for increasing polynomial degrees.

\subsection{Finding a Polynomial}

Given the invariance verification above, the problem is finding a candidate polynomial $h$. Below potential approaches are outlined:
\begin{itemize}
   \item Construct a feedback controller (e.g LQR) and use a Lyapunov function as a candidate
   \item Impose symmetry arguments based on symmetries in dynamics to reduce search space and guess a function.
   \item Solve \eqref{eq:tmin} as bi-linear SOS.
   \item Impose structural constraints on $h$ such that sign of $\mathcal L_g h$ only depends on $x$, which makes problem linear in barrier-function setting $\dot h + h \geq 0$ when analysis confined to a box.
   \begin{itemize}
     \item This should be sufficient: $h = h_1(r,q) + \alpha \| \dot r \|^2_2 + \beta \| \omega \|_2^2$ for $\alpha, \beta > 0$.
   \end{itemize} 
\end{itemize} 


\printbibliography 

\end{document}


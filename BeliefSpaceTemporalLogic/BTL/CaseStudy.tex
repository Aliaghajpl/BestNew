\section{Case Studies}
\label{sec:caseStudy}

In this section, we apply our algorithm to control a unicycle robot
moving in a bounded planar environment.
To deal with the non-linear nature of the robot model,
we locally approximate the robot's dynamics using LTI systems
with Gaussian noise around samples in the workspace.
This heuristic is very common, since the non-linear and
non-Gaussian cases yield recursive filters that do not
in general admit finite parametrization.
Moreover, the control policy is constrained to satisfy a
rich temporal specification.
The proposed sampling-based solution opvercomes %may overcome
these difficulties due to its randomized and incremental
nature.
As the size of the GDTL-FIRM increases, we expect
the algorithm to return a policy, if one exists, with
increasing satisfaction probability.
Since it is very difficult to obtain analytical bounds on
the satisfaction probability, we demonstrate the
performance of our solution in experimental trials.

\subsubsection{Motion model}

%The motion model for a unicycle is
%\begin{equation}
%\label{eq:unicycle-mm-cont}
%\begin{bmatrix} \dot{p}^x \\ \dot{p}^y \\ \dot{\theta} \end{bmatrix} =
%\begin{bmatrix} \cos(\theta) & 0 \\ \sin(\theta) & 0 \\ 0 & 1 \end{bmatrix} \cdot
% \begin{bmatrix} v \\ \omega \end{bmatrix} +
%\RM{d}\,w^{\CA{X}},
%\end{equation}
%where $p^x$, $p^y$ and $\theta$ are the
%position and orientation of the robot
%in the global reference frame,
%$v$ is the forward velocity,
%$\omega$ is the angular velocity
%and $\RM{d}\,w^{\CA{X}}$ and $\RM{d}\,w^{\CA{U}}$
%are Gaussian white noise process with
%covariance matrices $Q_\CA{X}$ and $Q_\CA{U}$,
%respectively.

The motion model for our system is a unicycle.
We discretize the system dynamics
using Euler's approximation. The motion
model becomes:
{\small
\begin{align}
\label{eq:unicycle-mm-disc}
x_{k+1} &= f(x_k, u_k, w_k) = x_k +
\begin{bmatrix} \cos(\theta_k) & 0 \\ \sin(\theta_k) & 0 \\ 0 & 1 \end{bmatrix} \cdot
u_k + w_k
\end{align}
}%
where
$x_k = \begin{bmatrix} p^x_k\  p^y_k\ \theta_k \end{bmatrix}^T$,
$p^x_k$, $p^y_k$ and $\theta_k$ are the position and orientation of the robot
in a global reference frame,
$u_k = \begin{bmatrix} v'_k\ \omega'_k \end{bmatrix}^T = \Delta t \begin{bmatrix} v_k \ \omega_k \end{bmatrix}^T$,
$v_k$ and $\omega_k$ are the linear and rotation velocities
of the robot,
$\Delta t$ is the discretization step, and
$w_k$ is a zero-mean Gaussian process with covariance matrix
%$Q = \sqrt{\Delta t} Q_{\CA{X}}$, where $Q_{\CA{X}}\in\mathbb{R}^{3\times 3}$.
$Q \in \BB{R}^{3\times 3}$.
%
Next, we linearize the system around a nominal
operating point $(x^d, u^d)$ without noise,
\begin{equation}
\label{eq:unycycle-lti}
x_{k+1}  =  f(x^d, u^d, 0) + A \ (x_k - x^d) + B \ (u_k - u^d) + w_k,
\end{equation}
where $A = \frac{\partial f}{\partial x_k} (x^d, u^d, 0)$
and $B = \frac{\partial f}{\partial u_k} (x^d, u^d, 0)$
are the process and control Jacobians,
$x^d = \begin{bmatrix} p^{x\,d} \  p^{y\,d}\ \theta^d \end{bmatrix}^T$,
and $u^d = \begin{bmatrix} v'^d_k\ \omega'^d_k \end{bmatrix}^T$.
%\begin{align}
%A &= \frac{\partial f}{\partial x_k} (x^d, u^d, 0) =
%\begin{bmatrix} 1 & 0 & - v'^d \cos(\theta^d) \\ 0 & 1 & v'^d \sin(\theta^d) \\ 0 & 0 & 1 \end{bmatrix}\\
%B &= \frac{\partial f}{\partial u_k} (x^d, u^d, 0) =
%\begin{bmatrix} \cos(\theta^d) & 0 \\ \sin(\theta^d) & 0 \\ 0 & 1 \end{bmatrix}\\
%G &= \frac{\partial f}{\partial w_k} (x^d, u^d, 0) =
%\begin{bmatrix} 1 & 0 & 0 \\ 0 & 1 & 0 \\ 0 & 0 & 1 \end{bmatrix}
%\end{align}

In our framework, we associate with each belief node
$B_g \in \TSX$ centered at $(\hat{x}^g, P)$ an LTI
system obtained by linearization~\eqref{eq:unycycle-lti}
about $(\hat{x}^g, u^g)$, where $u^g = [0.1, \, 0]^T$
corresponds to 0.1\;m/s linear velocity and 0
angular velocity.

%%%%%%%%%%
\subsubsection{Observation Model}
\label{sec:bayes}
%%%%%%%%%%%%%%%%%%%%%%%%%%%%%%%%%%%%%%%%%%%%%%%%%%%%%%%%%%%%%%
% Observation Model - Last Modified: 2016/01/25 by Eric Cristofalo
%%%%%%%%%%%%%%%%%%%%%%%%%%%%%%%%%%%%%%%%%%%%%%%%%%%%%%%%%%%%%%

%{\color{blue} [To be expanded by Eric]}
We localize the robot with a multiple camera network. 
This reflects the real world constraints of sensor networks, e.g. finite coverage, finite resolution, and improved accuracy with the addition of more sensors. 
The network was implemented using four TRENDnet Internet Protocol (IP) cameras with known pose with respect to the global coordinate frame of the experimental space. 
Each $640\times400$ RGB image is acquired and segmented, yielding multiple pixel locations that correspond to a known pattern on the robot. 
%The pixel locations and camera poses are used to solve common least squares problem from the Structure from Motion community that estimates the planar position and orientation of the robot in the global frame~\cite{YM-SS-JK-SSS:04}.
The estimation of the planar position and orientation of the robot in the global frame is formulated as a least squares problem
({\em structure from motion})~\cite{YM-SS-JK-SSS:04}.
%as it is commonly done in the {\em structure from motion} community~\cite{YM-SS-JK-SSS:04}.
The measurement, $y_k \in \CA{Y}$, is given by the discrete observation model:
$y_k = Cx_k+v_k$.
%\begin{equation}
%\label{eq:unicycle-on-disc}
%z_k = Cx_k+v_k
%\, .
%\end{equation}
The measurement error covariance matrix is defined as $R = \mathrm{diag}(r_x,r_y,r_{\theta})$, where the value of each scalar is inversely proportional to the number of cameras used in the estimation, i.e. the number of camera views that identify the robot. These values are generated from a camera coverage map (Fig.~\ref{fig:simple-arena-cover}) of the experimental space.  


%The measurement, $z_k \in \mathbb{R}^3$, is calculated by solving a least squares problem given $m$ cameras, their known relative positions and orientations, and the pixel positions of identifiable points on the robot in each camera's frame~\cite{YM-SS-JK-SSS:04}. 
%The resulting measurement includes the position and orientation of the robot on the ground plane with respect to the global coordinate system. 

%%%The continuous time observation model of the system is the following:
%%%\begin{equation}
%%%\label{eq:unicycle-om-cont}
%%%z = C
%%%\begin{bmatrix} p^x \\ p^y \\ \theta \end{bmatrix} +
%%%\RM{d}\,v,
%%%\end{equation}
%%%where $C = \begin{bmatrix} 1 & 0 & 0 \\ 0 & 1 & 0 \\ 0 & 0 & 1 \end{bmatrix}$
%%%is the observation matrix corresponding to the camera
%%%localization system, and
%%%$\RM{d}\,v$ is a Gaussian white noise process
%%%with covariance matrix $R$.
%%%
%%The discrete 
%%%version of the 
%%observation model is
%%\begin{equation}
%%\label{eq:unicycle-om-disc}
%%z_k = Cx_k+
%%%\begin{bmatrix} p^x_k \\ p^y_k \\ \theta_k \end{bmatrix} +
%%v_k.
%%\end{equation}
%%%where $C\in\mathbb{R}^{3\times3}$ is the observation matrix corresponding to the camera localization system, and $v_k$ is a zero-%mean Gaussian process with covariance matrix $R\in\mathbb{R}^{3\times 3}$.
%%%Note that 
%%The robot's orientation is not directly measured and its estimation is left for the filter. The measurement error covariance matrix, $R$, is inversely proportional to the number of cameras used in the robot's pose estimation. 
%%% and was estimated via sample covariance with multiple robot ground truth measurements. 

%The observation model of the system is the following:
%\begin{equation}
%\label{eq:unicycle-om-cont}
%z = C
%\begin{bmatrix} p^x \\ p^y \\ \theta \end{bmatrix} +
%\RM{d}\,v,
%\end{equation}
%where $C = \begin{bmatrix} 1 & 0 & 0 \\ 0 & 1 & 0 \end{bmatrix}$
%is the observation matrix corresponding to the camera
%localization system, and
%$\RM{d}\,v$ is a Gaussian white noise process
%with covariance matrix $R$.
%%
%The discrete version of the observation model is
%\begin{equation}
%\label{eq:unicycle-om-disc}
%z_k = C
%\begin{bmatrix} p^x_k \\ p^y_k \\ \theta_k \end{bmatrix} +
%v_k,
%\end{equation}
%where $v_k$ is a zero-mean Gaussian process
%with covariance matrix $R$.




%%%%%%%%%%
%\subsubsection{Observation model}
%{\color{blue} [To be expanded by Eric]}
%
%The observation model of the system is the following:
%\begin{equation}
%\label{eq:unicycle-om-cont}
%z = C
%\begin{bmatrix} p^x \\ p^y \\ \theta \end{bmatrix} +
%\RM{d}\,v,
%\end{equation}
%where $C = \begin{bmatrix} 1 & 0 & 0 \\ 0 & 1 & 0 \end{bmatrix}$
%is the observation matrix corresponding to the camera
%localization system, and
%$\RM{d}\,v$ is a Gaussian white noise process
%with covariance matrix $R$.
%%
%The discrete version of the observation model is
%\begin{equation}
%\label{eq:unicycle-om-disc}
%z_k = C
%\begin{bmatrix} p^x_k \\ p^y_k \\ \theta_k \end{bmatrix} +
%v_k,
%\end{equation}
%where $v_k$ is a zero-mean Gaussian process
%with covariance matrix $R$.

\subsubsection{Specification}
The specification is given over belief states associated
with the measurement $y$ of the robot as follows:
``Visit regions $A$ and $B$ infinitely many times.
If region $A$ is visited, then only corridor $D_1$
may be used to cross to the right side of the environment.
Similarly, if region $B$ is visited, then only corridor
$D_2$ may be used to cross to the left side of the
environment. The obstacle $Obs$ in the center must
always be avoided. The uncertainty must always
be less than $0.9$. When passing through the
corridors $D_1$ and $D_2$ the uncertainty must
be at most $0.6$.''

The corresponding GDTL formula is:
\begin{align}
\label{eq:simple-spec}
\phi_1 =\ & \phi_{avoid}  \andltl \phi_{reach}  \andltl \phi_{u,1} \andltl \phi_{u,2} \andltl \phi_{bounds} \\
\phi_{avoid} = & \LTLALWAYS \neg \phi_{Obs} \nonumber\\
\phi_{reach} = & \LTLALWAYS \big( \LTLEVENTUALLY (\phi_A \andltl \notltl \phi_{D_2} \LTLUNTIL \phi_B) \LTLEVENTUALLY (\phi_B \andltl \notltl \phi_{D_1} \LTLUNTIL \phi_A) \big) \nonumber\\
% &\qquad \andltl \LTLEVENTUALLY (\phi_B \andltl \notltl \phi_{D_1} \LTLUNTIL \phi_A) \big)\\
\phi_{u,1} = & \LTLALWAYS ( tr(P) \leq 0.9) \nonumber\\
\phi_{u,2} =&  \LTLALWAYS \big( ( \phi_{D_1} \orltl \phi_{D_2}) \Implies ( tr(P) \leq 0.6 ) \big) \nonumber\\
\phi_{bounds} =& \LTLALWAYS (box(\hat{x},x_c,a)\leq 1) , \nonumber
\end{align}
where $(\hat{x}, P)$ is a belief state associated with $y$,
%\begin{equation*}
$
a = \begin{bmatrix}
\frac{2}{l} & \frac{2}{w} & 0
\end{bmatrix} $
%\end{equation*}
%where
so that
$\hat{x}$ must remain within a rectangular $l\times w$ region
with center $x_c= \begin{bmatrix}
\frac{l}{2} & \frac{w}{2} & 0
\end{bmatrix}$, $l=4.13\,m$ and $w=3.54\,m$.
The 5 regions in the environment are defined by GDTL predicate
formulae $\phi_{Reg} = (box(\hat{x},x_{Reg},r_{Reg})\leq 1)$,
%as follows:
%\begin{align*}
%\phi_A &=& (\CA{M}(\hat{x},P,x_A) \leq r_A) \\
%\phi_B &=& (\CA{M}(\hat{x},P,x_B) \leq r_B) \\
%\phi_{D_1} &=& (\CA{M}(\hat{x},P,x_A) \leq r_{D_1}) \\
%\phi_{D_2} &=& (\CA{M}(\hat{x},P,x_A) \leq r_{D_2}) \\
%\phi_{Obs} &=& (\CA{M}(\hat{x},P,x_{Obs}) \leq r_{Obs}),
%\end{align*}
%\begin{equation*}
%\phi_{Reg} = (box(\hat{x},x_{Reg},r_{Reg})\leq 1),
%\end{equation*}
where $x_{Reg}$ and $r_{Reg}$ are the center and the dimensions
of region $Reg \in \{A, B, D_1, D_2, Obs\}$, respectively.
%Fig.~\ref{fig:simple-arena} shows the planar environment
%and the 5 regions.

%The LTL formula $\varphi_1$ corresponding to $\phi_1$ is:
%\begin{align*}
%\phi_1 = & \LTLALWAYS \big( \LTLEVENTUALLY (\pi_A \andltl \notltl \pi_{D_2} \LTLUNTIL \pi_B)
%\andltl \LTLEVENTUALLY (\pi_B \andltl \notltl \pi_{D_1} \LTLUNTIL \pi_A) \big)\\
%& \andltl \LTLALWAYS \neg \pi_{Obs} \andltl \LTLALWAYS \pi_{u,1}
%\andltl \LTLALWAYS \big( ( \pi_{D_1} \orltl \pi_{D_2}) \Implies \pi_{u,2} \big),
%\end{align*}
%where $\AP=\{ \pi_A, \pi_B, \pi_{D_1}, \pi_{D_2}, \pi_{Obs}, \pi_{u,1}, \pi_{u,2} \}$,
%and the map from the predicates of $\phi_1$ to $\AP$ is
%$\widetilde{\phi_{Reg}} = \pi_{Reg}$, $Reg \in \{A, B, D_1, D_2, Obs\}$,
%$\widetilde{tr(P) \leq 0.5} = \pi_{u,1}$ and
%$\widetilde{tr(P) \leq 0.3} = \pi_{u,2}$.

\begin{figure}[!htb]
\centering
\subfigure[Environment]{
%  \resizebox{0.46\columnwidth}{!}{\arena{1}{4}}
  \includegraphics[width=0.45\columnwidth]{exp_setup_1}
  \label{fig:simple-arena}
}
\subfigure[Camera coverage map]{
%  \resizebox{0.46\columnwidth}{!}{\arenacams{1}{4}}
  \includegraphics[width=0.45\columnwidth]{exp_setup_2}
  \label{fig:simple-arena-cover}
}
\subfigure[Pose estimation]{
%  \resizebox{0.46\columnwidth}{!}{\arenacams{1}{4}}
  \includegraphics[width=0.45\columnwidth]{exp_setup_3}
  \label{fig:simple-arena-pose}
}
\subfigure[Transition system ]{
%  \resizebox{0.46\columnwidth}{!}{\arenaprm{1}{4}}
  \includegraphics[width=0.45\columnwidth]{exp_setup_4}
  \label{fig:simple-arena-ts}
}
\caption{Fig.~\subref{fig:simple-arena}
shows an environment with
two regions $A$ and $B$, two corridors
$D_1$ and $D_2$ and an obstacle $Obs$.
%The magenta disk is the initial position of
%the robot.
Fig.~\subref{fig:simple-arena-cover}
shows the coverage of the cameras.
Fig.~\subref{fig:simple-arena-pose}
shows the pose of the robot computed
from the images taken by the 4 cameras.
Fig.~\subref{fig:simple-arena-ts}
shows the transition system computed
by Alg.~\ref{alg:compute-mdp}.
%Fig.~\subref{fig:simple-arena-cover}
%shows the overall coverage of the four
%cameras placed at the corners of the
%arena.
%Fig.~\subref{fig:simple-arena-prm}
%shows a possible MDP obtained by
%a run of the FIRM-based algorithm.
}
\label{fig:simple-case}
\end{figure}


\subsubsection{Local controllers}
We used the following simple switching controller
to drive the robot towards belief nodes:
\begin{equation*}
\resizebox{0.95\columnwidth}{!}{$u_{k+1} = \begin{cases}
\begin{bmatrix}k_D \normeucl{\alpha^T(x^g - \hat{x}_k)} &  k_\theta (\theta^{los}_k - \hat{\theta}_k) \end{bmatrix}^T & \mbox{if } \abs{\theta^{los}_k - \hat{\theta}_k} < \frac{\pi}{12}\\
\begin{bmatrix}0 & k_\theta (\theta^{los}_k - \hat{\theta}_k) \end{bmatrix}^T, & \mbox{otherwise}
\end{cases}$,}
\end{equation*}
where $k_D>0$ and $k_\theta>0$ are proportional scalar gains,
$x^g$ is the goal position, $\theta^{los}_k$ is the line-of-sight angle
and $\alpha=[1\  1\  0]^T$.
We assume, as in~\cite{Agha14}, that the controller
is able to stabilize the system state and uncertainty
around the goal belief state $(x^g, P^\infty)$,
where $P^\infty$ is the stationary covariance matrix.

\subsubsection{Experiments}

The algorithms in this paper were implemented in Python2.7 using
{\em LOMAP}~\cite{ulusoy-ijrr2013} and {\em networkx}~\cite{nx} libraries.
The {\em ltl2star} tool~\cite{klein2006} was used to convert the LTL specification
into a Rabin automaton.
All computation was performed on a Ubuntu 14.04 machine with
an Intel Core i7 CPU at 2.4 Ghz and 8GB RAM.

\begin{figure}[!htb]
\centering
\includegraphics[width=0.8\columnwidth]{experiment_trajectories}
\caption{The figure shows the trajectory of the robot over 10 surveillance cycles.
At each time step, the pose of the robot is marked by an arrow.
The true trajectory of the robot is shown in green. The trajectory obtained from
the camera network is shown in yellow, while the trajectory estimated by the
Kalman filter is shown in black.}
\label{fig:exp-traj}
\end{figure}

A switched feedback policy was computed for the ground
robot described by~\eqref{eq:unicycle-mm-disc} operating
in the environment shown in Fig.~\ref{fig:simple-arena}
with mission specification~\eqref{eq:simple-spec} using
Alg.~\ref{alg:compute-mdp}.
The overall computation time to generate the policy was
32.739 seconds and generated a transition system and product
MDP of sizes (23, 90) and (144, 538), respectively.
The Rabin automaton obtained from the GDTL formula has
7 states and 23 transitions operating over a set of atomic
propositions of size 8.
The most computationally intensive operation in
Alg.~\ref{alg:compute-mdp} is the computation of the
transition and intersection probabilities.
To speed up the execution, we generated trajectories for
each transition of the TS and reused them whenever
Alg.~\ref{alg:compute-prob} is called for a transition
of the product MDP.
The mean execution time for the probability computation
was 0.389 seconds for each transition of $\TS$.

We executed the computed policy on the ground vehicle
over 9 experimental trials for a total of 24 surveillance cycles.
The specification was met in all of surveillance cycles.
A trajectory of the ground robot over 10 surveillance cycles
(continuous operation) is shown in Fig.~\ref{fig:exp-traj}.
